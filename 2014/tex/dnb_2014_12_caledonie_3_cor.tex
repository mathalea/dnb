
\begin{minipage}{4.2cm}
\medskip 
\begin{enumerate}
\item 
~ \\
\vspace{0.7cm}
\psset{xunit=1.0cm,yunit=1.0cm,algebraic=true,dimen=middle,dotstyle=o,dotsize=3pt 0,linewidth=0.8pt,arrowsize=3pt 2,arrowinset=0.25}
\hspace{-0.7cm}\begin{pspicture*}(-1,-0.48)(2.68,5.46)
\psline(0.,0.)(2.,0.)
\psline(0.93,0.09)(0.93,-0.09)
\psline(1.,0.09)(1.,-0.09)
\psline(1.07,0.09)(1.07,-0.09)
\parametricplot{1.5707963267948966}{1.9825193257874667}{1.*5.*cos(t)+0.*5.*sin(t)+2.|0.*5.*cos(t)+1.*5.*sin(t)+0.}
\parametricplot{1.1590733278023266}{1.5707963267948966}{1.*4.99763944278*cos(t)+0.*4.99763944278*sin(t)+0.|0.*4.99763944278*cos(t)+1.*4.99763944278*sin(t)+0.}
\psline(0.,0.)(0.9941,4.89777145138)
\psline(0.401886486091,2.43248738369)(0.578289540242,2.39668287997)
\psline(0.415810459758,2.50108857141)(0.592213513909,2.4652840677)
\psline(0.9941,4.89777145138)(2.,0.)
\psline(1.57816858612,2.50127632585)(1.40184881388,2.46506392585)
\psline(1.59225118612,2.43270752553)(1.41593141388,2.39649512553)
\parametricplot{1.039072259536091}{1.6938822864899026}{1.*2.*cos(t)+0.*2.*sin(t)+0.|0.*2.*cos(t)+1.*2.*sin(t)+0.}
\psplot{-0.52}{2.68}{(--3.92006787001-0.*x)/2.}
\psline(0.,0.)(0.397827819066,1.96003393501)
\psline(0.0967884087906,0.929318031636)(0.273191462941,0.89351352792)
\psline(0.110712382458,0.997919219361)(0.287115436608,0.962114715645)
\psline(0.124636356125,1.06652040709)(0.301039410276,1.03071590337)
\rput[bl](-0.32,-0.12){\darkgray{$B$}}
\rput[bl](2.08,-0.12){\darkgray{$C$}}
\rput[bl](0.8,-0.42){2 cm}
\rput[bl](0.9,5.02){\darkgray{$A$}}
\rput[bl](-0.5,2.4){5 cm}
\rput[bl](0,2.08){\darkgray{$M$}}
\rput[bl](1.68,2.08){\darkgray{$N$}}
\end{pspicture*}
\end{enumerate}
\end{minipage}	
\begin{minipage}{9.8cm}
\begin{enumerate}
\item[\textbf{2.}]
 Dans le triangle $ABC$, \\
$M$ appartient à $[AB]$ et $N$ appartient à $[AC]$, \\
Les droites $(MN)$ et $(BC)$ sont parallèles, \\
donc d'après le théorème de Thalès : \\[2mm]
$\dfrac{AM}{AB}=\dfrac{AN}{AC}=\dfrac{MN}{BC}$. \\[2mm]
$M$ appartient à $[AB]$, donc $AM=5-2=3$ cm. \\[2mm]
$\dfrac{3}{5}=\dfrac{AN}{5}=\dfrac{MN}{2}$ \\[2mm]
On a : $\dfrac{3}{5}=\dfrac{AN}{5}$, donc $AN=3$ cm. \\
$N$ appartient à $[AC]$, donc $NC=5-3=2$ cm.\\
On a : $\dfrac{3}{5}=\dfrac{MN}{2}$, donc $MN=\dfrac{3\times2}{5}=1,2$ cm.

\item[\textbf{3.}] 
Périmètre de $AMN=3+3+1,2=7,2$ cm. \\
Périmètre de $BMNC=2+1,2+2+2=7,2$ cm \\
Les périmètres du triangle $AMN$ et du \linebreak quadrilatère $BMNC$ sont égaux. 
\end{enumerate}
\end{minipage}	

\vspace{0.5cm}

