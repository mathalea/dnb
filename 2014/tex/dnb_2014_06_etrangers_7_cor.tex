
\medskip
 
Il existe différentes unités de mesure de la température : en France on utilise le degré Celsius (\degres C), aux États-Unis on utilise le degré Fahrenheit (\degres~F).

\medskip
 
Pour passer des degrés Celsius aux degrés Fahrenheit, on multiplie le nombre de départ par $1,8$ et on ajoute 32 au résultat.

\medskip
 
\begin{enumerate}
\item Qu'indiquerait un thermomètre en degrés Fahrenheit si on le plongeait dans une casserole d'eau qui gèle ? On rappelle que l'eau gèle à 0~\degres C. 

\textit{Il indiquerait} $1,8\times 0+32=32$ \degres~F
\item Qu'indiquerait un thermomètre en degrés Celsius si on le plongeait dans une 
casserole d'eau portée à $212$~\degres F ? Que se passe t-il ?

\textit{Il indiquerait} $\dfrac{212-32}{1,8}=100$\degres~C. L'eau bout.
\item 
	\begin{enumerate}
		\item \textit{Si l'on note $x$ la température en degré Celsius et $f(x)$ la température en degré Fahrenheit, alors} \fbox{$f(x)=1,8x+32$}
		\item \textit{C'est une fonction \fbox{affine}}
		\item \textit{L'image de $5$ par la fonction $f$ est} $f(5)=1,8\times 5+32=$\fbox{$41$} 
		\item \textit{L'antécédent $x$ de $5$ par la fonction $f$ est la solution de l'équation $18x+32=5$ soit} $x=\dfrac{5-32}{1,8}=$\fbox{$-15$}
		\item \textit{En terme de conversion de température la relation $f(10) =$\fbox{$ 50$} signifie que $10$\degres~C correspondent à $50$\degres~F.}
	\end{enumerate} 
\end{enumerate} 
