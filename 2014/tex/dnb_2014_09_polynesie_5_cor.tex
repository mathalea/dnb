
\bigskip
 
%Pour son anniversaire, Julien a reà§u un coffret de tir à  l'arc.
% 
%Il tire une flèche. La trajectoire de la pointe de cette flèche est représentée ci-dessous.
% 
%La courbe donne la hauteur en mètres (m) en fonction de la distance horizontale en mètres (m) parcourue par la flèche. 
%
%\begin{center}
%\psset{unit=1cm}
%\begin{pspicture}(-0.5,-1)(11,5)
%\psgrid[gridlabels=0,subgriddiv=2,gridcolor=cyan,subgridcolor=cyan](0,0)(11,5)
%\psaxes[linewidth=1.25pt]{->}(0,0)(11,5)
%\uput[d](10.8,0){$x$}
%\uput[l](0,4.6){$y$}
%\uput[d](9,-0.5){Distance horizontale (m)}
%\rput(1,4.8){Hauteur (m)}
%\psplot[plotpoints=5000,linewidth=1.25pt,linecolor=blue]{0}{10}{1 0.9 x mul add x dup mul 0.1 mul sub}
%\end{pspicture}
%\end{center} 

\begin{enumerate}
\item %Dans cette partie, les réponses seront données grà¢ce à  des \textbf{lectures graphiques}. Aucune justification n'est attendue sur la copie. 
	\begin{enumerate}
		\item %De quelle hauteur la flèche est-elle tirée ?
		La flèche est tirée à  la hauteur 1~m. 
		\item %à€ quelle distance de Julien la flèche retombe-t-elle au sol?
		La flèche retombe à  10~m de Julien. 
		\item %Quelle est la hauteur maximale atteinte par la flèche ?
		La flèche monte au plus haut à  3~m. (approximativement)
	\end{enumerate} 
\item %Dans cette partie, les réponses seront justifiées par des \textbf{calculs} :
 
%La courbe ci-dessus représente la fonction $f$ définie par 
%
%$f(x) = - 0,1 x^2 + 0,9x + 1$. 
	\begin{enumerate}
		\item %Calculer $f(5)$.
$f(5) = - 0,1 \times 5^2 + 0,9 \times 5 + 1 = - 2,5 + 4,5 + 1 = 3$. 
		\item %La flèche s'élève-t-elle à  plus de 3 m de hauteur ?
Quand la flèche est à  5~m de Julien il ne semble pas que la hauteur soit  maximale car elle est déjà  retombée. 

On a $f(4) = -0,1 \times 4^2 + 0.9 \times 4 -2 = - 1,6 + 3,6 + 1 = 3$ et en ce point la flèche semble monter.

Il est difficile d’envisager une partie de trajectoire horizontale, donc la flèche doit s'élever à  plus de 3~m.

\emph{Remarque} :  En fait $f(4,5) = - 0,1 \times 4,5^2 + 0,9 \times 0,45 + 1 = 3,025$~(m) semble être la hauteur maximale.
	\end{enumerate} 
\end{enumerate} 

\bigskip
 
