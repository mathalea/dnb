
\medskip

%En utilisant le codage et les données, dans chacune des figures, est-il vrai que les droites (AB) et (CD) sont parallèles ? Justifier vos affirmations. 

%\begin{center}
%
%\textbf{Figure 1}
%
%\psset{unit=1cm}
%\begin{pspicture}(9,4)
%%\psgrid
%\psdots[dotstyle=+,dotangle=45](0.5,0.5)(5.7,0.8)(1.7,2.8)(6.9,3)
%\uput[dl](0.5,0.5){D} \uput[dr](5.7,0.8){C} \uput[ul](1.7,2.8){A} \uput[ur](6.9,3){B}\uput[d](3.75,1.8){O}
%\pspolygon(0.5,0.5)(5.7,0.8)(1.7,2.8)(6.9,3)
%\psline(2.7,2.15)(2.8,2.4)\psline(4.6,1.2)(4.75,1.4)
%\psline(2,1)(1.9,1.15)\psline(2.1,1.1)(2,1.25)
%\psline(5.3,2.5)(5.4,2.3)\psline(5.4,2.6)(5.5,2.4)   
%\end{pspicture} 
% 
%O, A, C sont alignés et O, B, D sont alignés 
%
%\medskip
%
%\textbf{Figure 2}
%
%\psset{unit=1cm}
%\begin{pspicture}(8,4.5)
%%\psgrid 
%\pscircle(1.8,1.6){1.4}
%\psline(0.5,1.1)(1.4,2.94)(7.8,0)
%\psline(0.5,1.1)(7.7,4.3)
%\psline(6.6,4.4)(4.4,0)
%\rput{68}(5.05,1.25){\psframe(0,0)(0.3,0.3)}
%\uput[dl](0.5,1.1){A} \uput[u](1.4,2.94){B} \uput[d](5.1,1.3){C} \uput[dr](6.3,3.6){D} \uput[ur](3.1,2.25){E}\uput[d](1.8,1.67){O}
%\psdots[dotstyle=+,dotangle=45](1.8,1.67)
%\psline(1.1,1.5)(1.22,1.3)
%\psline(2.3,2)(2.42,1.8)   
%\end{pspicture}
%
%A, B, E appartiennent au cercle de centre O
% 
%8, E et C sont alignés ; A, O, E et D sont alignés
%
%\end{center}
\textbf{Figure 1} Le quadrilatère ABCD a ses diagonales qui ont le même milieu O : c’est donc un parallélogramme et pa conséquent les côtés opposés sont parallèles et (AB) et (CD) sont parallèles.

\textbf{Figure 2} (ABE) est un triangle inscrit dans un cercle dont un des diamètres est l’un de ses côtés : il est donc rectangle en B.

Les droites (AB) et (CD) sont perpendiculaires à la même droite (BC) : elles sont donc parallèles. 
\bigskip

