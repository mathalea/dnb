Un sac contient 20 jetons qui sont soit jaunes, soit verts, soit rouges, soit bleus. On considère l'expérience suivante: tirer au hasard un jeton, noter sa couleur et remettre le jeton dans le sac. \textcolor{blue}{Chaque jeton a la même probabilité d'être tirer}.
\begin{enumerate}
\item Le professeur, qui connaît la composition du sac, a simulé un grand nombre de fois l'expérience avec un tableur.

D'après le graphe:
\begin{itemize}
\item \textcolor{yellow}{La fréquence d'apparition d'un jeton jaune semble être 0,5};
\item \textcolor{green}{la fréquence d'apparition d'un jeton vert semble être 0,25};
\item \textcolor{red}{la fréquence d'apparition d'un jeton rouge semble être 0,2};
\item \textcolor{blue}{la fréquence d'apparition d'un jeton bleu semble être 0,05}.
\end{itemize} 
\begin{enumerate}
\item La couleur est la plus présente dans le sac est le jaune.
\item Le professeur a construit une feuille de calcul:

La formule a-t-il saisie dans la cellule \texttt{C2} avant de la recopier vers le bas est: \texttt{B2/A2}.
\end{enumerate}
\item La probabilité de tirer un jeton rouge est de $\dfrac{1}{5}=\dfrac{4}{520}$.

Il y a équiprobabilité (\textcolor{blue}{Chaque jeton a la même probabilité d'être tirer}), le nombre de jetons rouges dans le sac est:
\[
\frac{\text{nombre de jetons rouges}}{\text{nombre de jetons total}}=\frac{4}{520}\Longrightarrow \text{nombre de jetons rouges}=4
\]
\end{enumerate}

\vspace{0,5cm}

