Léa a besoin de nouveaux cahiers. pour les acheter au meilleurs prix, elle étudie les offres promotionnelles de trois magasins. Dans ces trois magasins, le modèle de cahier dont elle a besoin a le même prix avant promotion.
\begin{multicols}{3}
\begin{bclogo}[couleur= gray!10,arrondi =0.1,cadreTitre=true]{Magasin A}
\begin{center}
Cahier à l'unité 
ou 
lot de 3 cahiers pour le prix de deux
\end{center}
\end{bclogo}

\begin{bclogo}[couleur= gray!10,arrondi =0.1,cadreTitre=true]{Magasin B}
\begin{center}
Pour un cahier acheté, le deuxième à moitié prix.
\end{center}
\end{bclogo}

\begin{bclogo}[couleur= gray!10,arrondi =0.1,cadreTitre=true]{Magasin C}
\begin{center}
30\%{} de réduction sur chaque cahier acheté.
\end{center}
\end{bclogo}
\end{multicols}

\begin{enumerate}
\item Seul le magasin C est propose une réduction de 30\%{} sur chaque cahier acheté, donc sur le premier. Si on achète qu'un seul cahier, c'est le magasin C qui est le plus intéressant.
\item Pour plusieurs cahiers de prix que nous nommerons $x,\ x > 0$:
\begin{enumerate}
\item deux cahiers: 
\begin{description}
\item[A] Prix de deux cahiers: $p_A(2)=2x$;
\item[B] Prix de deux cahiers: $p_B(2)=x+\dfrac{1}{2}x=\dfrac{3}{2}x=1,5x$;
\item[C] Prix de deux cahiers: $p_C(2)=2\times\left(1-\dfrac{30}{100}\right)x=2\times 0,7x=1,4x$.
\end{description}
On a: $p_A(2) > p_B(2) > p_C(2)$ ;

Si on achète deux cahiers, c'est le magasin C qui est le plus intéressant.
\item trois cahiers:
\begin{description}
\item[A] Prix de trois cahiers: $p_A(3)=2x$;
\item[B] Prix de trois cahiers: $p_B(3)=x+\dfrac{1}{2}x+x==\dfrac{5}{2}x=2,5x$;
\item[C] Prix de trois cahiers: $p_C(2)=3\times\left(1-\dfrac{30}{100}\right)x=3\times 0,7x=2,1x$.
\end{description}
On a: $p_B(3)>p_C(3)>p_A(3)$;

Si on achète trois cahiers, c'est le magasin A qui est le plus intéressant.
\end{enumerate}
\item La carte de fidélité du magasin C permet d'obtenir 10\%{} de réduction sur le ticket de caisse, y compris sur les articles ayant déjà bénéficié d'une première réduction.
\[
p'_C(1)=\left(1-\dfrac{30}{100}\right)\times\left(1-\dfrac{10}{100}\right)x=0,7\times 0,9=0,61-0,37
\]
Elle obtient donc une réduction de 37\%.
\end{enumerate}

\vspace{0,5cm}

