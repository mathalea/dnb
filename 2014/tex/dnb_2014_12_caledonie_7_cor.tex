
\begin{enumerate}
\item La fonction $f$ correspond à  la formule saisie dans la cellule B2 car \linebreak $f(0)=2\times 0=0$ alors que $g(0)=-2\times 0+8=8$. 
\item Dans la cellule B5, on saisit \quad $=-2*$B4$+8$ 
\item La fonction $f$ est représentée dans le repère de l'annexe 2 car, par exemple $f(0) = 0$ alors que $g(0) = 8$. 
\item \textbf{ANNEXE 2 - Exercice 7}

\vspace{1cm} 

\psset{xunit=0.7cm,yunit=0.7cm,algebraic=true,dimen=middle,dotstyle=o,dotsize=3pt 0,linewidth=0.8pt,arrowsize=3pt 2,arrowinset=0.25}
\begin{pspicture*}(-0.75,-0.75)(12.5,11.5)
\psgrid[gridlabels=0,subgriddiv=1,gridcolor=cyan](13,12)
\psaxes[xAxis=true,yAxis=true,Dx=1.,Dy=1.,ticksize=-2pt 0,subticks=2]{->}(0,0)(-2.02,-0.74)(12.5,11.5)
\psplot{-2.02}{6.86}{(-0.--4.*x)/2.}
\psplot{-2.02}{6.86}{(--16.-4.*x)/2.}
\rput[bl](-0.32,-0.3){0}
\end{pspicture*}
\vspace{1cm}
\item À partir du tableau, pour $x=2$, l'image est 4 pour les deux fonctions. Donc la solution de l'équation : $2x = - 2x + 8$ est 2. \\
Graphiquement, la solution de l'équation est l'abscisse du point \linebreak d'intersection des deux droites. \\
On peut aussi résoudre l'équation : 
\begin{eqnarray*}
2x & = & - 2x + 8 \\
2x+2x & = & 8 \\
4x & = & 8 \\
x &=&2
\end{eqnarray*}
\end{enumerate}

\vspace{0,5cm}

