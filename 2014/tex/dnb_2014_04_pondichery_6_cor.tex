
\medskip

\begin{enumerate}
\item Dans la cellule O2, on a saisi la formule : = SOMME(B1 : N1)   
\item 
	\begin{enumerate}
	\item $\overline{x} = \dfrac{8\times 1 + 8\times 2 + 2 \times 3 + 2 \times 4 + 1 \times 5 + 3 \times 6 + 1\times 11 + 2 \times 13 + 1\times 14 + 1 \times 15 + 1 \times 18 + 1\times 32 + 1 \times 40}{26}$
	
$\overline{x} = \dfrac{205}{26} \approx  8$. 

La moyenne de cette série est égale \`{a} environ 8 médailles.

	\item On calcule $\dfrac{N}{2}= \dfrac{26}{2} = 13$. 
	
La médiane de cette série est comprise entre la 13\up{e} valeur et la 
  14\up{e} de la série rangée dans l'ordre croissant.
  
On cumule les effectifs jusqu'\`{a} dépasser 13 :  $8 + 2 + 2= 12$. La 13\up{e} valeur est 4 et la 14\up{e} valeur est 4.

Donc la médiane de cette série est égale \`{a} 4 médailles.
	\item Les valeurs de la moyenne et de la médiane sont différentes car l'étendue de la série est très grande : $40 - 1 = 39$. Les valeurs sont alors très dispersées.
		\end{enumerate}
\item  Soit $x$ le nombre de pays médaillés.

70\,\% des pays médaillés ont obtenu au moins une médaille d'or ; ainsi, $\dfrac{70}{100} \times x = 26$, c'est-\`{a}-dire $0,7\times x = 26$.

Par suite, $x = \dfrac{26}{0,7} \approx 37$ et $37 - 26 =11$.

Par conséquent, $11$ pays n'ont obtenu que des médailles d'argent ou de bronze.
\end{enumerate}
