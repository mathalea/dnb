
\medskip

%Les appareils de la maison consomment de l'énergie même quand ils sont en veille.
% 
%La feuille de calcul ci-dessous donne la consommation en kilowattheures (kwh) des appareils en veille d'une famille pour une année et les dépenses correspondantes en euros :
% 
%\begin{center}
%\begin{tabularx}{\linewidth}{|c|c|*{4}{>{\centering \arraybackslash}X|}}\hline
%	&A	&B	&C	&D	&E\\ \hline
%1	&\footnotesize Appareil&\footnotesize Nombre d'appareils&\footnotesize Consom\-mation en veille par an pour un appareil (en kWh)&\footnotesize Prix du kilowattheure (en \euro)&\footnotesize Dépenses (en \euro)\\ \hline
%2	&Téléviseur&3&77&0,13&30,03\\ \hline
%3	&Ordinateur&1 &209& 0,13& 27,17\\ \hline
%4	&Parabole&2&131&  0,13 &34,06 \\ \hline
%5	&Four	&1&86&  0,13 &11,18\\ \hline 
%6	&Démodulateur satellite&3&59 &0,13& 23,01\\ \hline
%7	&Lecteur DVD &2&58&  0,13 &15,08 \\ \hline
%8	&Machine à laver&1&51  &0,13 &6,63\\ \hline
%9	&Console de jeu&1&42&  0,13 &5,46\\ \hline 
%10	&Four à micro-ondes&1&25  &0,13 &3,25\\ \hline 
%11	&Téléphone sans fil&1&25 &0,13 &3,25\\ \hline
%12	&Lave-vaisselle&1&17 &0,13 &2,21\\ \hline 
%13	&Chargeur batterie&4&13& 0,13 &6,76\\ \hline
%14	&&&\multicolumn{2}{|r|}{\textbf{Dépense Totale}}&168,09\\ \hline
%\multicolumn{6}{r}{\emph{Données extraites du site de l'ADEME}}\\ 
%\end{tabularx}
%\end{center}
 
\begin{enumerate}
\item 
	\begin{enumerate}
		\item %Quel calcul permet de vérifier le résultat $34,06$ affiché dans la cellule E4 ?
Il y a deux paraboles. On a $2 \times 131 \times 0,13 = 131 \times 0,26 = 34,06~$\euro. 
		\item %Quelle formule a-t-on saisie dans la cellule E2 avant de la recopier vers le bas ?
Dans le cellule E2 on a écrit   :  B2 * C2 * D2.		 
		\item %Une des quatre formules ci-dessous a été saisie dans la cellule E14 pour obtenir le montant total des dépenses dues aux veilles. Recopier sur la copie cette formule.
	\end{enumerate}
			
\medskip
On a écrit dans la cellule E14 : = SOMME(E2 : E3)
%\begin{small}
%\hspace{-1cm}\begin{tabularx}{1.1\linewidth}{*{4}{X}} 
%\fbox{$=\text{SOMME(E}2:\text{E}13)$}&\fbox{$= \text{E}2:\text{E}13$}&\fbox{$=\text{E}2+\text{E}13$}&\fbox{$=\text{SOMME(E}2:\text{E}14)$}\\ 
%\end{tabularx}
%\end{small}
%\medskip
 
\item %Dans une pièce de cette maison, les appareils qui sont en veille sont : 

%\medskip
%\begin{tabularx}{\linewidth}{*{2}{X}}
%$\bullet~~$un téléviseur& $\bullet~~$une console de jeu\\
%$\bullet~~$un ordinateur& $\bullet~~$un lecteur DVD
%\end{tabularx}
%\medskip
 
%La consommation de l'ordinateur représente-t-elle plus de la moitié de la consommation totale des appareils de cette pièce ?
Consommation de l'ordinateur : 209 sur un total de $77 + 209 + 42 + 58 = 386$ et la moitié de 386 est égale à 193 qui est inférieur ) 209.

La consommation de l'ordinateur représente plus de la moitié de la consommation totale des appareils de cette pièce. 
\end{enumerate}

\bigskip

