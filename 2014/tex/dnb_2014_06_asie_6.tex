
\medskip

Une association décide d'organiser une tombola pour financer entièrement une sortie pour ses adhérents d'un montant de \np{2660}~\euro.
 
Le 1\up{er} ticket tiré au sort fera remporter le gros lot d'une valeur de 300~\euro, 

Les 10 tickets suivants tirés au sort feront remporter un lot d'une valeur de 25~\euro{} chacun. 

Les 20 tickets suivants tirés au sort feront remporter un lot d'une valeur de 5~\euro{} chacun.
 
\textbf{L'association finance entièrement les lots.}

\smallskip
 
Chaque ticket de tombola est vendu 2~\euro{} et les tickets sont vendus durant 6 jours.
 
On a représenté ci-dessous le diagramme des ventes des tickets durant ces 6 jours. 

\begin{center}
\psset{xunit=1.3cm,yunit=0.012cm}
\begin{pspicture}(-1,-50)(6,500)
\multido{\n=0+25}{21}{\psline(0,\n)(6,\n)}
\psframe[fillstyle=solid,fillcolor=lightgray](0.333,0)(0.666,350)
\psframe[fillstyle=solid,fillcolor=lightgray](1.333,0)(1.666,225)
\psframe[fillstyle=solid,fillcolor=lightgray](2.333,0)(2.666,400)
\psframe[fillstyle=solid,fillcolor=lightgray](3.333,0)(3.666,125)
\psframe[fillstyle=solid,fillcolor=lightgray](4.333,0)(4.666,325)
\psframe[fillstyle=solid,fillcolor=lightgray](5.333,0)(5.666,475)
\uput[d](0.5,0){Lundi} \uput[d](1.5,0){Mardi} \uput[d](2.5,0){Mercredi} 
\uput[d](3.5,0){Jeudi} \uput[d](4.5,0){Vendredi} \uput[d](5.5,0){Samedi}
\psaxes[linewidth=1.25pt,Dx=10,Dy=25](0,0)(6,500)
\rput{90}(-1.,250){Nombre de tickets vendus}
\end{pspicture}
\end{center}

\begin{enumerate}
\item  L'association pourra-t-elle financer entièrement cette sortie ? 
\item  Pour le même nombre de tickets vendus, proposer un prix de ticket de tombola permettant de financer un voyage d'une valeur de \np{10000}~\euro{} ? 

Quel serait le prix minimal ? 
\item  Le gros lot a été déjà tiré. Quelle est la probabilité de tirer un autre ticket gagnant ? (donner le résultat sous la forme fractionnaire) 
\end{enumerate}  

\bigskip

