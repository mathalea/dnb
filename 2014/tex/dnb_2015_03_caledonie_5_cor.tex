
\medskip

Martin va en vacances durant une semaine chez sa grand-mère
au bord de la mer.

Les crabes se mesurent dans leur plus grande largeur (sans les pinces).

Voici les différentes tailles en centimètres des crabes qu'il a
pêchés au cours de la semaine :

\[23 - 9 - 10 - 10 - 23 - 22 - 18 - 16 - 13 - 8 - 8 - 16 - 18 - 10 - 12\]

\begin{enumerate}
\item% Quelle est la moyenne de cette série ?
Martin a mesuré 15 crabes. La moyenne de cette série est:

$\dfrac{23 + 9 + 10 + 10 + 23 + 22 + 18 + 16 + 13 + 8 + 8 + 16 + 18 + 10 + 12}{15} = \dfrac{216}{15}=14,4 $

\item% Quelle est la médiane de cette série ?
Pour déterminer la médiane, on écrit les tailles des crabes en ordre croissant:

\[ 8 - 8 - 9 - 10 - 10 - 10 - 12 - \fbox{13} - 16 - 16 - 18 - 18 - 22 - 23 - 23\]

La médiane de cette série est la valeur du nombre situé \og{} au milieu\fg{} de cette série, soit le $8\ieme$ nombre qui est 13.

\item Les crabes de moins de 14~cm dans leur plus grande largeur sont interdits à la pêche. 

%Quelle proportion de crabes a-t-il dû remettre en liberté pour protéger l'espèce ?

Il y a 8 crabes ayant une largeur inférieure à 14~cm; il faut donc remettre en liberté une proportion de crabes égale à $\dfrac{8}{15}$.

\end{enumerate}

\vspace{0,5cm}

