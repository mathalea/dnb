
\bigskip
 
On considère ces deux programmes de calcul :

\begin{center}
\begin{tabularx}{\linewidth}{X p{0.2cm}X} 
\textbf{Programme A :}& &\textbf{Programme B :}\\
\psframebox[framearc=0.25]{\parbox{4.9cm}{Choisir un nombre\\Soustraire 0,5\\Multiplier le résultat par le double\\du nombre choisi au départ}}
&& 
\psframebox[framearc=0.25]{\parbox{4.9cm}{
Choisir un nombre\\
Calculer son carré\\
Multiplier le résultat par 2\\
Soustraire \`a ce nouveau résultat\\
le nombre choisi au départ}}\\
\end{tabularx}
\end{center} 
 
\begin{enumerate}
\item 
	\begin{enumerate}
		\item Montrer que si on applique le programme A au nombre 10, le résultat est 190.
		\item Appliquer le programme B au nombre 10.
	\end{enumerate} 
\item On a utilisé un tableur pour calculer des résultats de ces deux programmes. Voici ce qu'on a obtenu : 

\begin{center}
\begin{tabularx}{0.8\linewidth}{|c|*{3}{>{\centering \arraybackslash}X|}}\hline
&A& B& C\\ \hline 
1& Nombre choisi& Programme A& Programme B\\ \hline 
2 &1 &1 &1\\ \hline 
3 &2 &6 &6 \\ \hline 
4 &3 &15 &15 \\ \hline 
5 &4 &28 &28 \\ \hline 
6 &5 &45 &45 \\ \hline 
7& 6 &66 &66\\ \hline 
\end{tabularx}
\end{center} 
	\begin{enumerate}
		\item Quelle formule a-t-on saisie dans la cellule C2 puis recopiée vers le bas ? 
		\item Quelle conjecture peut-on faire à la lecture de ce tableau? 
		\item Prouver cette conjecture. 
	\end{enumerate}
\item Quels sont les deux nombres à choisir au départ pour obtenir 0 à l'issue de ces programmes ? 
\end{enumerate}

\bigskip
 
