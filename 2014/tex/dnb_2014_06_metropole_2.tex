
\medskip

Léa a besoin de nouveaux cahiers. Pour les acheter au meilleur prix, elle étudie les offres promotionnelles de trois magasins. Dans ces trois magasins, le modèle de cahier dont elle a besoin a le même prix avant promotion. 

\begin{center}
\begin{tabularx}{\linewidth}{|>{\centering\arraybackslash}X|m{0.1cm}|>{\centering\arraybackslash}X|m{0.1cm}|>{\centering\arraybackslash}X|}\cline{1-1}\cline{3-3}\cline{5-5}
\textbf{Magasin A} &&\textbf{Magasin B} 
&&\textbf{Magasin C}\\ 
Cahier à l'unité ou lot de 3 cahiers pour le prix de 2. &&Pour un cahier acheté, le deuxième à moitié prix.&&30\,\% de réduction sur chaque cahier acheté.\\  \cline{1-1}\cline{3-3}\cline{5-5}
\end{tabularx}
\end{center}

\begin{enumerate}
\item Expliquer pourquoi le magasin C est plus intéressant si elle n'achète qu'un cahier. 
\item Quel magasin doit-elle choisir si elle veut acheter : 
\begin{enumerate}
\item deux cahiers ? 
\item trois cahiers ? 
\end{enumerate}
\item La carte de fidélité du magasin C permet d'obtenir 10\,\% de réduction sur le ticket de caisse, y compris sur les articles ayant déjà bénéficié d'une première réduction. 

Léa possède cette carte de fidélité, elle l'utilise pour acheter un cahier. Quel pourcentage de réduction totale va-t-elle obtenir ? 
\end{enumerate}

\vspace{0,5cm} 

