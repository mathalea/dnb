
\medskip

Dans ce questionnaire à choix multiple, pour chaque question, des réponses sont proposées, une seule est exacte. Pour chacune des questions, écrire le numéro de la question et recopier la bonne réponse. Aucune justification n'est attendue. 

\medskip

\begin{tabularx}{\linewidth}{|p{10.2cm}|X|}\hline
Questions&   Propositions \\ \hline  
\textbf{Question 1}&  \textbf{a.~~} 2\\   
Quand on double le rayon d'une boule, son volume est  par :&   \textbf{b.~~} 4\\   
multiplié & \textbf{c.~~} 6\\   
&  \textbf{d.~~} 8   \\ \hline
\textbf{Question 2}&   \textbf{a.~~} 10 m.s$^{- 1}$\\   
Une vitesse égale à 36 km.h$^{- 1}$ correspond à:&  \textbf{b.~~} 60 m.s$^{- 1}$ \\   
&\textbf{c.~~} 100 m.s$^{- 1}$    \\
&\textbf{d.~~} 360 m.s$^{- 1}$   \\ \hline
\textbf{Question 3}&  \textbf{a.~~} $21 \sqrt{5}$\\   
Quand on divise $\sqrt{525}$  par $5$, on obtient:   &\textbf{b.~~} $5\sqrt{21}$\\   
&\textbf{c.~~} $\sqrt{21}$\\  
&\textbf{d.~~}  $\sqrt{105}$\\ \hline   
\textbf{Question 4}&  \textbf{a.~~}  25\\   
On donne : 1To (téraoctet) = $10^{12}$ octets et 1 Go (gigaoctet) $= 10^9$ octets.&   \textbf{b.~~} \np{1 000}\\   
On partage un disque dur de 1,5 To en dossiers de 60 Go chacun. & \textbf{c.~~} $4 \times 10^{22}$\\   
Le nombre de dossiers obtenus est égal à :  & \textbf{d.~~}  $2,5 \times  10^{19}$\\ \hline
\end{tabularx}

\smallskip

\vspace{0,5cm} 

