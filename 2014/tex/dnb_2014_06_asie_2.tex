
\medskip

Une corde de guitare est soumise à une tension $T$, exprimée en Newton (N), 
qui permet d'obtenir un son quand la corde est pincée.
  
Ce son plus ou moins aigu est caractérisé par une fréquence $f$exprimée en Hertz (Hz). 

\medskip

\parbox{0.34\linewidth}{La fonction qui à une tension $T$ associe 
sa fréquence  est définie par la relation : 
$f(T) = 20\sqrt{T}$.
 
On donne ci-contre la représentation 
graphique de cette fonction.}\hfill
\parbox{0.65\linewidth}{
\psset{unit=0.005cm}
%\def\pshlabel#1{\tiny#1}
%\def\psvlabel#1{\scriptsize#1}
\begin{pspicture}(-150,-100)(1700,1300) 
\uput[r](0,1250){\small Fréquence $f$ en Hz}
\uput[u](1400,0){\small Tension $T$ en N } 
\multido{\n=0+100}{18}{\psline[linewidth=0.3pt,linecolor=orange](\n,0)(\n,1300)}
\multido{\n=0+100}{14}{\psline[linewidth=0.3pt,linecolor=orange](0,\n)(1700,\n)}
\psaxes[linewidth=1.25pt,Dx=1700,Dy=1500]{->}(0,0)(0,0)(1700,1300)
\multido{\n=0+100}{17}{\uput[d](\n,0){\scriptsize \np{\n}}}
\multido{\n=0+100}{13}{\uput[l](0,\n){\scriptsize \np{\n}}}
\psplot[plotpoints=4000,linewidth=1.25pt,linecolor=cyan]{0}{1700}{x sqrt 20 mul}
\end{pspicture}}

\medskip

\textbf{Tableau des fréquences (en Hertz) de différentes notes de musique}

\medskip
\begin{tabularx}{\linewidth}{|m{1.7cm}|*{14}{>{\footnotesize \centering \arraybackslash}X|}}\hline
 Notes& Do2& Ré2& Mi2& Fa2& Sol2 &La2& Si2& Do3& Ré3& Mi3& Fa3& Sol3& La3& Si3\\\hline 
Fréquences (en Hz)&132& 148,5& 165& 176& 198& 220& 247,5& 264 &297 &330& 352& 396& 440& 495\\\hline
\end{tabularx}
\medskip
 
Déterminer graphiquement une valeur approchée de la tension à appliquer sur la corde pour obtenir un \og La3 \fg. 

Déterminer par le calcul la note obtenue si on pince la corde avec une tension de 220~N environ.
 
La corde casse lorsque la tension est supérieure à 900~N.
 
Quelle fréquence maximale peut-elle émettre avant de casser? 
Page 2 sur 6 

\bigskip

