
\bigskip

%ABC est un triangle tel que AB = 5 cm, BC = 7,6 cm et AC = 9,2 cm. 
%
%\medskip

\begin{enumerate}
\item %Tracer ce triangle en vraie grandeur.
~
\begin{center}
\psset{unit=0.8cm}
%\psgrid
\begin{pspicture}(9.5,5)
\pspolygon(0.25,0.25)(9.45,0.25)(3.04,4.39)
\psarc(0.25,0.25){5}{50}{65}\psarc(9.42,0.25){7.6}{140}{155}
\uput[dl](0.25,0.25){A}\uput[dr](9.45,0.25){C}\uput[ur](3.1,4.2){B}
\end{pspicture}
\end{center} 
\item %ABC est-il un triangle rectangle ?
Il suffit de vérifier si AC$^2 = \text{AB}^2 + \text{BC}^2$.

Or  AC$^2 = 9,2^2 = ...4$ et $\text{AB}^2 + \text{BC}^2 = 5^2 + 7,6^2 = ... 6$.

L'égalité ci-dessus ne peut être vraie : le triangle ABC n'est pas rectangle.
\item ~

%\parbox{0.4\linewidth}{Avec un logiciel, on a construit ce triangle, puis : 
%
%- on a placé un point P mobile sur le cà´té [AC] ;
% 
%- on a tracé les triangles ABP et BPC ;
% 
%- on a affiché le périmètre de ces deux triangles.}\hfill
%\parbox{0.55\linewidth}{\psset{unit=0.5cm}
%\begin{pspicture}(11.7,6)
%\pspolygon(0.5,0.5)(11.2,0.5)(4,5.3)%ACB
%\psline(4,5.3)(5.1,0.5)%BP
%\uput[dl](0.5,0.5){A} \uput[ur](4,5.3){B} \uput[dr](11.2,0.5){C} \uput[d](5.1,0.5){P}
%\rput(2.8,1){\scriptsize Périmètre de ABP = 13,29}
%\rput(8.15,1){\scriptsize Périmètre de BPC = 17,09} 
%\end{pspicture}} 
% 
%\medskip 
	\begin{enumerate}
		\item %On déplace le point P sur le segment [AC].
		 
%O๠faut-il le placer pour que la distance BP soit la plus petite possible ? 
La distance BP est la plus petite quand P est le pied de la hauteur issue de B.

On peut construire ce point comme intersection du cercle de diamètre [BC] (ou [AB]) avec le côté [AC].
		\item %On place maintenant le point P à  5~cm de A. 

%Lequel des triangles ABP et BPC a le plus grand périmètre ?
Périmètre de ABP : $5 + 5 + \text{BP} = 10 + \text{BP}$ ;

Périmètre de BPC : $7,6 + (9,2 - 5) + \text{BP} = 11,8 + \text{BP}$ : c'est BPC qui a le plus grand périmètre. 
		\item %On déplace à  nouveau le point P sur le segment [AC]. 

%Où faut-il le placer pour que les deux triangles ABP et BPC aient le même périmètre ?
Soit $x = \text{AP}$. On a :

Périmètre de ABP : $5 + x + \text{BP}$ ;

Périmètre de BPC : $7,6 + (9,2 - x) + \text{BP} = 16,8  - x + \text{BP}$.

On doit donc avoir :

$5 + x + \text{BP} = 16,8  - x + \text{BP}$ soit $5 + x  = 16,8 - x$ ou encore $2x = 11,8$ d'où $x = 5,9$.
	\end{enumerate} 
\end{enumerate} 

\bigskip
 
