
\medskip  

%Voici les résultats du DNB blanc de deux classes de 3\up{e} d'un collège de Nouméa. 
%
%Pour la 3\up{e} A, on a : 8~;~7~;~12~;~15~;~15~;~12~;~18~;~18~;~11~;~7~;~8~;~11~;~7~;~13~;~10~;~10~;~6 et 11. 
%
%Pour la 3\up{e} B, on a : 7~;~8~;~7~;~9~;~8~;~13~;~8~;~13~;~13~;~8~;~19~;~13~;~7~;~16~;~18~;~ 12 et 9.
%
%\medskip 

\begin{enumerate}
\item %Calculer la moyenne de chaque classe, arrondie au dixième. Que constate-t-on ?
$m_{\text{A}} = \dfrac{8 + 7 + 12 + \cdots + 11}{18} = \dfrac{199}{18} \approx 11,1$ ;

$m_{\text{B}} = \dfrac{7 + 8 + 7 + \cdots + 9}{18} = \dfrac{188}{17} \approx 11,1$.

On a $m_{\text{A}} \approx m_{\text{B}}$. les deux classes ont sensiblement la même moyenne. 
\item %Calculer ensuite leurs médianes.
La médiane de 6 ; 7 ; 7 ; 7 ; 8 ; 8 ; 10 ; 10 ; 11 ; 11 ; \ldots est 11.

La médiane de 7 ; 7 ; 7 ; 8 ; 8 ; 8 ; 8 ; 9 ; 12 ; \ldots est 8.
\item %Quelle est, d'après les calculs, la classe ayant le mieux assimilé les leçons ? Justifier la réponse.
La médiane de la classe A est la plus grande : la moitié de la classe B a moins ou juste 9 : la classe A a mieux assimilé les leçons..
\item %Deux des graphiques donnés ci-dessous représentent la répartition des notes des classes précédentes. 

%Attribuer à chaque classe le graphique qui lui correspond. 
Le graphique 3 est à éliminer puisqu'il n'y a pas de notes de 0 à 5.

Considérons les notes de la tranche [5~;~10[ : dans la classe A il y en a 6 sur 18 ce qui représente un secteur de 60\degres sur 180\degres ou encore 120\degres sur 360\degres. Ceci  correspond au graphique 2.

Dans la classe B il y a 9 notes sur 17 dans la tranche [5~;~10]  ; or $\dfrac{9}{17} = \dfrac{x}{360}$ soit $17x = 9 \times 360$ ou $x = \dfrac{3240}{17} \approx 190,6$\degres. Ceci correspond au graphique 1.
\end{enumerate}

%\bigskip
%
%\begin{tabularx}{\linewidth}{|*{3}{>{\centering \arraybackslash}X|}m{1.75cm}|}\hline
%Graphique 1 &Graphique 2&Graphique 3 &Légende  \\ \hline 
%\psset{unit=1cm}
%\begin{pspicture}(-1.6,-1.6)(1.6,1.6)
%\pswedge[fillstyle=solid,fillcolor=gray!20](0,0){1.4}{-100}{90}
%\pswedge[fillstyle=hlines](0,0){1.4}{90}{150}
%\pswedge[fillstyle=vlines](0,0){1.4}{150}{-100}
%\end{pspicture}
%&
%\psset{unit=1cm}
%\begin{pspicture}(-1.6,-1.6)(1.6,1.6)
%\pswedge[fillstyle=solid,fillcolor=gray!20](0,0){1.4}{-30}{90}
%\pswedge[fillstyle=hlines](0,0){1.4}{90}{160}
%\pswedge[fillstyle=vlines](0,0){1.4}{160}{-30}
%\end{pspicture}
%&
%\psset{unit=1cm}
%\begin{pspicture}(-1.6,-1.6)(1.6,1.6)
%\pswedge[fillstyle=solid,fillcolor=gray!20](0,0){1.4}{-170}{-15}
%\pswedge[fillstyle=hlines](0,0){1.4}{90}{110}
%\pswedge[fillstyle=vlines](0,0){1.4}{110}{190}
%\pswedge[fillstyle=solid,fillcolor=gray](0,0){1.4}{-15}{90}
%\end{pspicture}
%&
%\psset{unit=1cm}
%\begin{pspicture}(-1,0)(1,1)
%\rput(-1,0.5){\psframe[fillstyle=hlines](0.5,0.5)}
%\rput(0.2,0.75){[15~;~20]}
%\rput(-1,1){\psframe[fillstyle=vlines](0.5,0.5)}
%\rput(0.2,1.25){[10~;~15]}
%\rput(-1,1.5){\psframe[fillstyle=solid,fillcolor=gray!20](0.5,0.5)}
%\rput(0.2,1.75){[5~;~10]}
%\rput(-1,2){\psframe[fillstyle=solid,fillcolor=gray](0.5,0.5)}
%\rput(0.2,2.25){[0~;~5]}
%\end{pspicture}
%\\ \hline
%\end{tabularx}

\vspace{0,5cm}

