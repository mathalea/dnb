
\medskip

Le premier juillet 2018, la vitesse maximale autorisée sur les routes à double sens de circulation, sans séparateur central, a été abaissée de $90$ km/h à $80$ km/h.

En 2016, \np{1911} personnes ont été tuées sur les routes à double sens de circulation, sans séparateur central, ce qui représente environ 55\,\% des décès sur l'ensemble des routes en France.

\emph{Source : www.securite-routiere.gouv.fr}

\medskip

\begin{enumerate}
\item 
	\begin{enumerate}
		\item Montrer qu'en 2016, il y a eu environ \np{3475} décès sur l'ensemble des routes en France.
		\item Des experts ont estimé que la baisse de la vitesse à $80$ km/h aurait permis de sauver $400$ vies en 2016. 
		
De quel pourcentage le nombre de morts sur l'ensemble des routes de France
aurait-il baissé ? Donner une valeur approchée à $0,1$\,\% près.
	\end{enumerate}
\item  En septembre 2018, des gendarmes ont effectué une série de contrôles sur une route dont la vitesse maximale autorisée est $80$ km/h. Les résultats ont été entrés dans un tableur dans l'ordre croissant des vitesses. Malheureusement, les données de la colonne B ont été effacées.
	
\begin{center}
	\begin{tabularx}{\linewidth}{|c|l|*{9}{>{\centering \arraybackslash}X|}c|}\hline
	&A 						&B 	&C 	&D 	&E 	&F 	&G 	&H 	&I 	&J &K\\ \hline
1 	&vitesse relevée (km/h)	&	&72 &77 &79 &82 &86 &90 &91 &97& TOTAL\\ \hline
2 	&nombre d'automobilistes&	& 2 &10 &6 	&1 	&7 	&4 	&3 	&6	&\\ \hline
\end{tabularx}
\end{center}

	\begin{enumerate}
		\item Calculer la moyenne des vitesses des automobilistes contrôlés qui ont dépassé la vitesse maximale autorisée. Donner une valeur approchée à $0,1$ km/h près.
		\item Sachant que l'étendue des vitesses relevées est égale à $27$ km/h et que la médiane est égale à $82$ km/h, quelles sont les données manquantes dans la colonne B ?
		\item Quelle formule doit-on saisir dans la cellule K2 pour obtenir le nombre total d'automobilistes contrôlés?
	\end{enumerate}
\end{enumerate}

\bigskip

