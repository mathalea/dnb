
\medskip

\parbox{0.77\linewidth}{Les questions 1 et 2 sont indépendantes.

\medskip

Un sablier est composé de 
\begin{itemize}
\item Deux cylindres $C_1$ et $C_2$ de hauteur $4,2$~cm et de diamètre $1,5$~cm
\item Un cylindre $C_3$
\item Deux demi-sphères $S_1$ et $S_2$ de diamètre $1,5$~cm
\end{itemize}

\medskip

On rappelle le volume $V$ d'un cylindre d'aire de base $B$ et de hauteur $h$ :
\[V = B \times h.\]}\hfill
\parbox{0.2\linewidth}{\psset{unit=1cm}
\begin{pspicture}(2.4,6)
%\psgrid
\psline(0.2,3.3)(0.2,5.4)(1.2,5.4)(1.2,3.3)
\psarc(0.7,3.3){0.5}{-180}{-98}
\psarc(0.7,3.3){0.5}{-82}{0}
\psline[linestyle=dashed](0.2,3.3)(1.2,3.3)
\psline(0.6,2.6)(0.6,2.8)\psline(0.8,2.6)(0.8,2.8)
\psarc(0.7,2.1){0.5}{98}{180}
\psarc(0.7,2.1){0.5}{0}{78}
\psline[linestyle=dashed](0.2,2.1)(1.2,2.1)
\psline(0.2,2.2)(0.2,0)(1.2,0)(1.2,2.2)
\rput(1.8,1){sable}\rput(1.5,3.1){$S_1$}\psline(1.3,3.1)(0.8,3.1)
\psframe[fillstyle=solid,fillcolor=lightgray](0.2,0)(1.2,1.7)
\rput(1.5,2.7){$C_3$}\psline(1.3,2.7)(0.8,2.7)
\rput(1.5,2.3){$S_2$}\psline(1.3,2.3)(0.8,2.3)
\rput(0.7,4){$C_1$}\rput(0.7,1.9){$C_2$}
\end{pspicture}
}

\medskip

\begin{enumerate}
\item 
	\begin{enumerate}
		\item Au départ, le sable remplît le cylindre $C_2$ aux deux tiers. Montrer que le volume du sable est environ $4,95$ cm$^3$.
		\item On retourne le sablier. En supposant que le débit d'écoulement du sable est constant et égal à 1,98 cm$^3$/min, calculer le temps en minutes et secondes que va mettre le sable à s'écouler dans le cylindre inférieur.
	\end{enumerate}
\item En réalité, le débit d'écoulement d'un même sablier n'est pas constant.
	
Dans une usine où on fabrique des sabliers comme celui-ci, on prend un sablier au hasard et on teste plusieurs fois le temps d'écoulement de ce sablier. 
	
Voici les différents temps récapitulés dans le tableau suivant: 

\begin{center}
\begin{tabularx}{\linewidth}{|m{1.25cm}|*{7}{>{\centering\arraybackslash\footnotesize}X|}}\hline
Temps mesuré&2 min 22 s&2 min 24 s&2 min 26 s&2 min 27 s& 2 min 28 s& 2 min 29 s&2 min 30 s\\ \hline
Nombre de tests &1 &1 & 2&6&3&7&6\\ \hline
\end{tabularx}
\end{center}
\begin{center}
\begin{tabularx}{\linewidth}{|m{1.25cm}|*{7}{>{\centering\arraybackslash\footnotesize }X|}}\hline
Temps mesuré	&2 min 31 s&2 min 32 s&2 min 33 s&2 min 34 s &2 min 35 s& 2 min 38 s\\ \hline
Nombre de tests	&3			& 1		& 2		& 3		& 2		&3\\ \hline
\end{tabularx}
\end{center}

	\begin{enumerate}
		\item Combien de tests ont été réalisés au total ?
		\item Un sablier est mis en vente s'il vérifie les trois conditions ci-dessous, sinon il est éliminé :
		
\begin{itemize}[leftmargin=10mm]
\item[$\bullet~~$] L'étendue des temps est inférieure à 20 s.
\item[$\bullet~~$] La médiane des temps est comprise entre 2 min 29 s et 2 min 31 s.
\item[$\bullet~~$] La moyenne des temps est comprise entre 2 min 28 s et 2 min 32 s.
\end{itemize}
\setlength\parindent{0cm}
 	\end{enumerate}
	
Le sablier testé sera-t-il éliminé ?
\end{enumerate}


