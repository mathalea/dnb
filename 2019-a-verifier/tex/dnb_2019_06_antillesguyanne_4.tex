
\medskip

Leila est en visite à Paris. Aujourd'hui, elle est au Champ de Mars où l'on peut voir la tour Eiffel dont la hauteur totale BH est $324$~m.

Elle pose son appareil photo au sol à une distance AB = 600 m du monument et le programme pour prendre une photo (voir le dessin ci-dessous).

\medskip

\begin{enumerate}
\item Quelle est la mesure, au degré près, de l'angle $\widehat{\text{HAB}}$ ?
\item Sachant que Leila mesure $1,70$ m, à quelle distance AL de son appareil doit-elle se placer pour paraître aussi grande que la tour Eiffel sur sa photo ?

Donner une valeur approchée du résultat au centimètre près.
\end{enumerate}

\begin{center}
\psset{unit=0.9cm,arrowsize=2pt 3}
\begin{pspicture}(10,5)
\pspolygon(1,1)(9,1)(9,4.5)%ABL
\uput[dl](1,1){A}\uput[d](9,1){B}\uput[ur](9,4.5){H}
\uput[u](1,1){appareil photo}\uput[u](3.,1){Leila}\rput{90}(9.5,2){Tour Eiffel}
\uput[d](3,1){L}
\psline(3,1)(3,1.88)
\rput(5,-0.2){Le dessin n'est pas à l'échelle}
\uput[u](4.5,0.4){AB  = 600 m}\rput{90}(9.5,3.75){324~m}
\psline[linewidth=0.3pt]{<->}(1,0.4)(9,0.4)
\end{pspicture}
\end{center}

\bigskip

