
\medskip

%Dans ce questionnaire à choix multiples, pour chaque question des réponses sont proposées,
%une seule est exacte. Sur la copie, écrire le numéro de la question et recopier la bonne réponse.
%Aucune justification n'est attendue.

%\begin{center}
%\begin{tabularx}{\linewidth}{|m{6cm}|*{3}{>{\centering \arraybackslash}X|}}\hline
%Questions &A &B &C\\ \hline
%\textbf{1.~~} Le nombre $(- 2)^4$ est égal à :& 16 &$- 8$ &\np{20000}\\ \hline
%\textbf{2.~~} Une vitesse de $90$ km/h est égale à : &$0,025$ m/s &\np{25000} m/s &25 m/s\\ \hline
%\textbf{3.~~} La décomposition en produit de facteurs premiers de $24$ est: &$2\times3\times4$ &$2\times2\times2\times3$ &$2\times2\times6$\\ \hline
%\textbf{4.~~} Soit $f$ la fonction affine définie par
%
%$f : x \longmapsto 2x + 5$ 
%
%L'image de $- 1$ par la fonction $f$ est:&3 &6 &$- 7$\\ \hline
%\textbf{5.~~} Si on multiplie par 3 toutes les dimensions d'un rectangle, son aire est multipliée par:	&3	&6	&9\\ \hline
%\end{tabularx}
%\end{center}
\begin{enumerate}
\item $(- 2)^4 = 2^4 = 16$.
\item 90 km parcourus en 1 h soit \np{90000}~m en \np{3600}~s, soit $\dfrac{\np{90000}}{\np{3600}} = \dfrac{900}{36} = \dfrac{9 \times 100}{9 \times 4}=  \dfrac{4 \times 25}{4} = 25$~m/s.
\item $24 = 8 \times 3 = 2^3 \times 3$.
\item $f(- 1) = 2\times (- 1) + 5 = - 2 + 5 = 3$.
\item Si les dimensions du rectangle sont $\ell$ et $L$, son aire est $\ell \times L$.

En triplant les dimensions celles-ci deviennent $3\ell$ et $3L$, donc l'aire est égale à $3\ell \times 3L = 9\ell L$. L'aire a donc été multipliée par 9.
\end{enumerate}


\bigskip

