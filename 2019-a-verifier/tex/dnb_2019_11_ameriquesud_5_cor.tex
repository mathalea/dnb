
\medskip

%Une entreprise rembourse à ses employés le coût de leurs déplacements professionnels, quand les employés utilisent leur véhicule personnel.
%
%Pour calculer le montant de ces remboursements, elle utilise la formule et le tableau d'équivalence ci-dessous proposés par le gestionnaire:
%
%\begin{center}
%\begin{tabularx}{\linewidth}{|p{5cm}|X|}\hline
%\multicolumn{2}{|c|}{\textbf{Document 1}}\\
%Formule 		&Tableau\\
%Montant du remboursement:
%
%\qquad $a + b\times d$
%
%où :
%
%$\bullet~~$ $a$ est un prix (en euros) qui ne dépend que de la longueur du trajet;
%
%$\bullet~~$ $b$ est le prix payé (en euros) par kilomètre parcouru;
%
%$\bullet~~$ $d$ est la longueur en kilomètres du \og trajet aller \fg.&
%\begin{tabular}{|m{3cm}|c| c|}\hline
%%\begin{tabularx}{\linewidth}{|m{3cm}|*{2}{>{\centering \arraybackslash}X|}}\hline
%Longueur $d$ du \og trajet aller\fg	&Prix $a$&Prix $b$ par kilomètre\\ \hline
% De 1 km à 16 km						&\np{0,7781}		&\np{0,1944}\\ \hline
% De 17 km à 32 km 						&\np{0,2503}		&\np{0,2165}\\ \hline
% De 33 km à 64 km 						&\np{2,0706}		&\np{0,1597}\\ \hline
% De 65 km à 109 km 						&\np{2,8891}		&\np{0,1489}\\ \hline
% De 110 km à 149 km 					&\np{4,0864}		&\np{0,1425}\\ \hline
% De 150 km à 199 km 					&\np{8,0871}		&\np{0,1193}\\ \hline
% De 200 km à 300 km 					&\np{7,7577}		&\np{0,1209}\\ \hline
% De 301 km à 499 km 					&\np{13,6514}	&\np{0,1030}\\ \hline
% De 500 km à 799 km 					&\np{18,4449}	&\np{0,0921}\\ \hline
% De 800 km à \np{9999}~km				&\np{32,2041}	&\np{0,0755}\\ \hline
%\end{tabular} \\ \hline
%\end{tabularx}
%\end{center}
%\medskip

\begin{enumerate}
\item %Pour un \og trajet aller\fg{} de 30~km, vérifier que le montant du remboursement est environ $6,75$~\euro.
Pour un \og trajet aller\fg{} de 30~km le montant du remboursement est égal à :

$\np{0,2503} + 30 \times \np{0,2165} = \np{6,7453} \approx 6,75$~\euro{} au centime près.
\item  %Dans le cadre de son travail, un employé de cette entreprise effectue un déplacement à Paris. Il choisit de prendre sa voiture et il trouve les informations ci-dessous sur un site internet.

%\begin{center}
%\begin{tabularx}{\linewidth}{|X|}\hline
%\multicolumn{1}{|c|}{\textbf{Document 2}}\\
%Distance Nantes - Paris : 386 km\\
%Coût du péage entre Nantes et Paris: 37~\euro\\
%Consommation moyenne de la voiture de l'employé: $6,2$ litres d'essence aux $100$~km\\
%Prix du litre d'essence: 1,52~\euro\\\hline
%\end{tabularx}
%\end{center}
$\bullet$~~la dépense en essence s'élève à $\dfrac{368}{100} \times 6,2 \times 1,52 = \np{36,3766} \approx 36,38$~\euro ;

$\bullet~~$le coût du péage s'élève à 37~\euro.

La dépense totale sera donc de : $36,38 + 37 = 73,38$~\euro.

Le remboursement sera égal à :

$\np{13,6514} + 386 \times 0,103 = \np{53,4094} \approx 53,41$~\euro. 

L'employé perdra environ 20~\euro{} sur ce déplacement.

\textbf{À l'aide des documents 1 et 2, répondre à la question suivante:}

\og Le montant du remboursement sera-t-il suffisant pour couvrir les dépenses de cet
employé pour effectuer le \og trajet aller\fg{} de Nantes à Paris ? \fg
\end{enumerate}

\vspace{0,5cm}

