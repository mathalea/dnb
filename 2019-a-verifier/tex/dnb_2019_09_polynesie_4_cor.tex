
\medskip

%\begin{minipage}{.55\linewidth}On s'intéresse aux ailes d'un moulin à vent
%décoratif de jardin. Elles sont représentées
%par la figure ci-contre:
%
%On donne :
%
%\begin{itemize}
%\item BCDE, FGHI, JKLM et PQRS sont
%des rectangles superposables.
%\item C, B, A, J, K d'une part et
%
%G, F, A, P, Q d'autre part sont alignés.
%\item AB = AF = AJ = AP
%\end{itemize}
%\end{minipage}\hfill \begin{minipage}{.42\linewidth}
%\psset{unit=1cm}
%\begin{pspicture}(-3,-3)(3,3)
%\def\pale{\psline(0,0)(2.5,0)\psframe(1,0)(2.5,1)}
%\psframe(0.3,0.3)
%\multido{\n=0+90}{4}{\rput{\n}(0,0){\pale}}
%\uput[dl](0,0){A} \uput[d](1,0){P} \uput[d](2.5,0){Q} \uput[u](2.5,1){R} 
%\uput[u](1,1){S} \uput[r](0,1){B} \uput[r](0,2.5){C} \uput[l](-1,2.5){D} 
%\uput[l](-1,1){E} \uput[u](-1,0){F} \uput[l](-2.5,0){G} \uput[l](-2.5,-1){H} 
%\uput[r](-1,-1){I} \uput[l](0,-1){J} \uput[l](0,-2.5){K} \uput[r](1,-2.5){L}
%\uput[r](1,-1){M}  
%\end{pspicture}
%\end{minipage}
%
%\medskip

\begin{enumerate}
\item %Quelle transformation permet de passer du rectangle FGHI au rectangle PQRS ?
Soit la symétrie centrale par rapport au point A, soit la rotation de centre A et d'angle 180\degres .
\item %Quelle est l'image du rectangle FGHI par la rotation de centre A d'angle $90\degres$ dans le sens
%inverse des aiguilles d'une montre?
L'image est le rectangle JKLM.
%\end{enumerate}

%\begin{minipage}{.6\linewidth}
%\begin{enumerate}[resume]
%\item Soit V un point de [EB] tel que BV = 4 cm.
%
%On donne:
%
%AB = 10 cm et AC = 30 cm.
%
%\emph{Attention la figure n'est pas construite à la taille réelle}.
\item
	\begin{enumerate}
		\item %Justifier que (DC) et (VB) sont parallèles.
BCDE est un rectangle, ses côtés opposés (BE) et (CD) sont parallèles et puisque V est un point de [BE], (DC) et (VB) sont parallèles.
		\item %Calculer DC.
		D'après la question précédente on a une configuration de Thalès, on a donc :
		
$\dfrac{\text{BV}}{\text{CD}} = \dfrac{\text{AB}}{\text{AC}}$ ou $\dfrac{4}{\text{CD}} = \dfrac{10}{30}$, d'où CD $ = \dfrac{4 \times 30}{10} = 12$~(cm).
		\item %Déterminer la mesure de l'angle $\widehat{\text{DAC}}$. Arrondir au degré près.
		Dans le triangle ACD rectangle en C, on a :
		
		$\tan \widehat{\text{DAC}} = \dfrac{\text{CD}}{\text{AC}} = \dfrac{12}{30} = \dfrac{2}{5} = 0,4$.
		
		La calculatrice donne $\widehat{\text{DAC}} \approx 21,8$, soit 22\degres{} au degré près.
	\end{enumerate}
\end{enumerate}
%\end{minipage}\hfill \begin{minipage}{.38\linewidth}
%\psset{unit=1cm}
%\begin{pspicture}(-3,-3)(3,3)
%\def\pale{\psline(0,0)(2.5,0)\psframe(1,0)(2.5,1)}
%\multido{\n=0+90}{4}{\rput{\n}(0,0){\pale}}
%\psframe(0.3,0.3)
%\psline[linestyle=dashed](-1,2.5)
%\uput[dl](0,0){A} \uput[d](1,0){P} \uput[d](2.5,0){Q} \uput[u](2.5,1){R} 
%\uput[u](1,1){S} \uput[r](0,1){B} \uput[r](0,2.5){C} \uput[l](-1,2.5){D} 
%\uput[l](-1,1){E} \uput[u](-1,0){F} \uput[l](-2.5,0){G} \uput[l](-2.5,-1){H} 
%\uput[r](-1,-1){I} \uput[l](0,-1){J} \uput[l](0,-2.5){K} \uput[r](1,-2.5){L}
%\uput[r](1,-1){M}\uput[dl](-0.4,1){V}  
%\end{pspicture}
%\end{minipage}
\bigskip

