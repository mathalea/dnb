
\medskip

Void deux programmes de calcul :

\begin{center}
\psset{unit=1cm}
\begin{pspicture}(12.5,4)
%\psgrid
\uput[r](0,3){Choisir un nombre}
\uput[r](0,2){Le multiplier par 3}
\uput[r](0,1){Ajouter 1}
\psframe(0,0.5)(3.1,3.3)
\rput(2,3.5){\textbf{Programme 1}}
\rput(8.8,3.5){\textbf{Programme 2}}
\rput(9.3,3){Choisir un nombre}
\psframe(7.8,2.8)(10.8,3.2)
\rput(7.3,2){Soustraire 1}\psframe(6.,1.8)(8.2,2.2)
\rput(11.5,2){Ajouter 2}\psframe(10.7,1.8)(12.3,2.2)
\rput(9.3,1){Multiplier les}
\rput(9.3,0.5){deux nombres obtenus}\psframe(7.5,0.2)(11.1,1.3)
\psline{->}(8.8,2.8)(7.1,2.2)
\psline{->}(9.8,2.8)(11.5,2.2)
\psline{->}(7.1,1.8)(8.4,1.3)
\psline{->}(11.5,1.8)(10.4,1.3)
\end{pspicture}
\end{center}

\medskip

\begin{enumerate}
\item Vérifier que si on choisit 5 comme nombre de départ.

\setlength\parindent{9mm}
\begin{itemize}
\item[$\bullet~~$]le résultat du programme 1 vaut $16$.
\item[$\bullet~~$]le résultat du programme 2 vaut $28$. 
\end{itemize}
\setlength\parindent{0mm}

On appelle $A(x)$ le résultat du programme 1 en fonction du nombre $x$ choisi au départ.

La fonction $B$  :  $x \longmapsto  (x - 1)(x + 2)$ donne le résultat du programme 2 en fonction du nombre $x$ choisi au départ.
\item
	\begin{enumerate}
		\item Exprimer $A(x)$ en fonction de $x$.
		\item Déterminer le nombre que l'on doit choisir au départ pour obtenir 0 comme résultat du
programme 1.
	\end{enumerate}
\item  Développer et réduire l'expression :
	
\[B(x) = (x - 1)(x + 2).\]
	
\item
	\begin{enumerate}
		\item Montrer que $B(x) - A(x) = (x + 1)(x - 3)$.
		\item  Quels nombres doit-on choisir au départ pour que le programme 1 et le programme 2 donnent le même résultat ? Expliquer la démarche.
	\end{enumerate}
\end{enumerate}
