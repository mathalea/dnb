
\medskip

\begin{enumerate}
\item Avec le programme 1, on a :

$5 \to  3 \times 5 = 15 \to 15 + 1 = 16$

Le résultat du programme 1 vaut 16.
 
Avec le programme 1, on a :

$5 \to 5 - 1 = 4$ (à gauche) et  $5 + 2 = 7 $(à droite) $\to  4 \times 7 = 28$.

Le résultat du programme 2 vaut $28$.
\item 
	\begin{enumerate}
		\item Pour le programme 1, on a $x \to  3x \to  3x + 1$, donc on a
$A(x) = 3x + 1$.
		\item  On veut $A(x) = 0$, ce qui donne successivement :
		
$3x + 1 = 0 ;\: 3x = 0 - 1 ; \:3x = -1  ;\: x = - \dfrac{1}{3}$.

On doit choisir $- \dfrac{1}{3}$ au départ pour obtenir $0$ comme résultat du programme 1.
	\end{enumerate} 
\item  $B(x) = (x - 1)(x + 2) = x^2 + 2x – x – 2 = x^2 + x – 2$.
\item 
	\begin{enumerate}
		\item On a : 
		
$B(x) – A(x) = x^2 + x – 2 - (3x + 1) = x^2 + x - 2 - 3x – 1 = x^2 - 2x - 3$
et $(x + 1)(x - 3) = x^2 - 3x + x - 3 = x^2 - 2x - 3$.

On a bien $B(x) – A(x) = (x + 1)(x - 3)$.
		\item  On veut $B(x) = A(x)$, soit $B(x) – A(x) = 0$ ou encore
$(x + 1)(x - 3) = 0$, soit $x + 1 = 0$ ou $x – 3 = 0$.

On a donc $x = - 1$ ou $x = 3$.

Il faut choisir $- 1$ ou $3$ au départ pour que le programme 1 et le programme 2 donnent le même résultat. 
	\end{enumerate}
\end{enumerate}
