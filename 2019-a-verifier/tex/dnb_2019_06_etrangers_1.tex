
\medskip

Cet exercice est un questionnaire à choix multiples (QCM). Pour chaque question, \textbf{une seule} des trois réponses proposées est exacte. Sur la copie, indiquer le numéro de la question et recopier, sans justifier, la réponse choisie. Une bonne réponse rapporte $3$ points; aucun point ne sera enlevé en cas de mauvaise réponse.

\medskip

\begin{tabularx}{\linewidth}{|m{7cm}|*{3}{>{\centering \arraybackslash}X|}}\hline
\multicolumn{1}{|c|}{\textbf{Questions}}& \textbf{Réponse A} &\textbf{Réponse B} &\textbf{Réponse C}\\ \hline
\textbf{1.~~}Quelle est la décomposition en produit de facteurs premiers de $28$ ?&$4 \times 7$& 
$2 \times 14$&$2^2 \times 7$ \\ \hline
\textbf{2.~~}Un pantalon coûte $58$~\euro. Quel est son prix en \euro{} après une réduction de 20\,\% ? &38 &46,40& 57,80\\ \hline
\textbf{3.~~} Quelle est la longueur en m du côté [AC], arrondie au dixième près ?

\psset{unit=1cm}
\begin{pspicture}(6.5,2.25)
\pspolygon(0.5,0.5)(6,0.5)(6,1.75)%BAC
\uput[l](0.5,0.5){B} \uput[dr](6,0.5){A} \uput[u](6,1.75){C} \uput[r](1.75,0.7){15\degres}
\psframe(5.75,0.5)(6,0.75)
\psarc(0.5,0.5){1cm}{0}{14} 
\uput[d](3.25,0.5){25~m}
\end{pspicture}&6,5& 6,7&24,1\\ \hline
\textbf{4.~~} Quelle est la médiane de la série statistique suivante ? 

2~;~5~;~3~;~12~;~8~;~6.&5,5 &6 &10\\ \hline
\textbf{5.~~}Quel est le rapport de l'homothétie qui transforme le carré A en carré B ?

\psset{unit=1cm}
\begin{pspicture}(-2.5,0)(3,3.35)
\psframe(1,1)(3,3)\psframe(0,0)(1,1)
\rput(2,2){carré A}\rput(0.5,0.5){carré B}
\end{pspicture} &$-0,5$& 0,5& 2\\ \hline
\end{tabularx}

\vspace{0,5cm}

