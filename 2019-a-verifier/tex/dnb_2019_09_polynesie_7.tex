
\medskip

On donne le programme ci-dessous où on considère $2$ lutins. Pour chaque lutin, on a écrit un
script correspondant à un programme de calcul différent.

\begin{center}
\begin{tabularx}{\linewidth}{|X|c|}\hline
Lutin \no 1 &Numéro d'instruction\\ \hline
\begin{scratch}
\blockinit{Quand \greenflag est cliqué}\end{scratch}&\raisebox{12pt}{1}\\
\begin{scratch}\blocksensing{demander \txtbox{Saisir un nombre} et attendre}\end{scratch}&\raisebox{12pt}{2}\\
\begin{scratch}\blockvariable{mettre \ovalvariable{x} à {\ovaloperator{\ovalvariable{réponse} + \ovalnum{5}}}}\end{scratch}&\raisebox{12pt}{3}\\
\begin{scratch}\blockvariable{mettre \ovalvariable{x} à {\ovaloperator{\ovalvariable{x} * \ovalnum{2}}}}\end{scratch}&\raisebox{12pt}{4}\\
\begin{scratch} \blockvariable{mettre \ovalvariable{x} à \ovaloperator{\ovalvariable{x} - \ovalvariable{réponse}}} \end{scratch}&\raisebox{12pt}{5}\\
\begin{scratch}\blocklook{dire \txtbox{regroupe} Le programme de calcul donne{\ovalvariable{x}}}\end{scratch}&\raisebox{12pt}{6}\\ 
\hline
\end{tabularx}
\end{center}

\medskip

\begin{flushleft}
\begin{tabularx}{0.75\linewidth}{|X|}
\hline
Lutin \no 2\\ 
\hline
\begin{scratch}\blockinit{Quand je reçois \ovalvariable{nombre saisi}}\end{scratch}\\
\begin{scratch}\blockvariable{mettre \ovalvariable{x} \`a {\ovaloperator{\ovalnum{7} * \ovalvariable{réponse}}}}\end{scratch}\\
\begin{scratch}\blockvariable{mettre \ovalvariable{x} \`a {\ovalvariable{x} - \ovalnum{8}}}\end{scratch}\\
\begin{scratch}\blocklook{dire \txtbox{regroupe} Le programme de calcul donne{\ovalvariable{x}}}\end{scratch}\\ 
\hline
\end{tabularx}
\end{flushleft}

\begin{enumerate}
\item Vérifier que si on saisit $7$ comme nombre, le lutin \no 1 affiche comme résultat $17$ et le lutin
\no 2 affiche $41$.
\item Quel résultat affiche le lutin \no 2 si on saisit le nombre $- 4$ ?
\item 
	\begin{enumerate}
		\item Si on appelle $x$ le nombre saisi, écrire en fonction de $x$ les expressions qui traduisent le programme de calcul du lutin \no 1, à chaque étape (instructions 3 à 5).
		\item Montrer que cette expression peut s'écrire $x + 10$.
	\end{enumerate}
\item Célia affirme que plusieurs instructions dans le script du lutin \no 1 peuvent être supprimées et remplacées 
par celle ci-contre.
\begin{minipage}[c][1cm][c]{6cm}
\hfill\begin{scratch}
\blockvariable{mettre \ovalvariable{x} à {\ovaloperator{\ovalvariable{réponse} + \ovalnum{10}}}}
\end{scratch}
\end{minipage}

Indiquer, sur la copie, les numéros des instructions qui sont alors inutiles.
\item  Paul a saisi un nombre pour lequel les lutins \no 1 et \no 2 affichent le même résultat. Quel
est ce nombre ?
\end{enumerate}
