
\medskip

\begin{enumerate}
\item Tableau complété :

\begin{center}
\begin{tabularx}{\linewidth}{|*{4}{>{\centering \arraybackslash}X|}}\hline
Modèle	&Pour la ville	&Pour le sport	& Total\\ \hline
Noir	&\blue 15		&\black 5		&\black 20\\ \hline
Blanc	&\black 7		&\blue 10		&\blue 17\\ \hline
Marron	&\blue 5		&\black 3		&\blue 8\\ \hline
Total	&\black 27		&\blue 18		&\black 45\\ \hline
\end{tabularx}
\end{center}

\item 
	\begin{enumerate}
		\item La probabilité de choisir un modèle de couleur noire est égale à $\dfrac{20}{45} = \dfrac{5 \times 4}{5 \times 9} = \dfrac{4}{9}$.
		\item La probabilité de choisir un modèle pour le sport est égale à $\dfrac{18}{45} = \dfrac{9 \times 2}{9 \times 5}  = \dfrac{2}{5} = \dfrac{4}{10} = 0,4$.
		\item La probabilité de choisit un modèle pour la ville de couleur marron est égale à $\dfrac{5}{45} = \dfrac{5 \times 1}{5 \times 9} = \dfrac{1}{9}$.	
	\end{enumerate}
\item Dans le magasin B la probabilité de choisir un modèle de couleur noire est égale à $\dfrac{30}{54} = \dfrac{6 \times 5}{6 \times 9} = \dfrac{5}{9}$. 

Comme $\dfrac{5}{9} > \dfrac{4}{9}$ on a plus de chance d'obtenir un modèle de couleur noire dans le magasin B.
\end{enumerate}

\vspace{0,5cm}

