
\medskip
\textbf{Partie I}

\medskip

\emph{Dans cette partie, toutes les longueurs sont exprimées en centimètre}.

On considère les deux figures ci-dessous, un triangle équilatéral
et un rectangle, où $x$ représente un nombre positif quelconque.

\begin{center}
\begin{tabularx}{\linewidth}{*{2}{>{\centering\arraybackslash}X}}
\begin{pspicture}(-2,-1)(2,2)
\pspolygon(2;-30)(2;90)(2;210)
\psline(0.9;30)(1.1;30)
\psline(0.9;150)(1.1;150)
\psline(0.9;-90)(1.1;-90)
\uput[d](1.1;-90){$4x + 1$}
\end{pspicture}&\psset{unit=0.9cm}\begin{pspicture}(-2,-1.5)(2,1.5)
\psframe(-2,-1.5)(2.5,1)
\psframe(-2,-1.5)(-1.75,-1.25)\psframe(-2,0.75)(-1.75,1)
\psframe(2.5,-1.5)(2.25,-1.25)\psframe(2.5,0.75)(2.25,1)
\uput[d](0.25,-1.5){$4x + 1,5$}\uput[r](2.5,-0.25){$2x$}
\end{pspicture}\\
\end{tabularx}
\end{center}
\bigskip

\begin{enumerate}
\item Construire le triangle équilatéral pour $x = 2$.
\item  
	\begin{enumerate}
		\item Démontrer que le périmètre du rectangle en fonction de $x$ peut s'écrire $12 x + 3$.
		\item Pour quelle valeur de $x$ le périmètre du rectangle est-il égal à $18$~cm ?
 	\end{enumerate}
\item  Est-il vrai que les deux figures ont le même périmètre pour toutes les valeurs de $x$ ?
Justifier.
\end{enumerate}

\bigskip

\textbf{Partie II}

\medskip

\begin{tabularx}{\linewidth}{p{3.75cm}X X}
On a créé les scripts (ci-contre) sur Scratch qui, après avoir
demandé la valeur de $x$ à l'utilisateur, construisent les deux figures de la partie I.

Dans ces deux scripts, les lettres A, B, C et D remplacent des nombres.

Donner des valeurs à A, B, C et D pour que ces deux scripts permettent de construire les figures
de la partie 1 et préciser alors la figure associée à chacun des scripts.&\small{\setscratch{scale=0.8}\begin{scratch}

\initmoreblocks{définir \namemoreblocks{script 1}}
\blocksensing{demander \ovalnum{Donner une valeur} et attendre}
\blockpen{stylo en position d’écriture}
\blockrepeat{répéter \ovalnum{A} fois}
{
\blockmove{avancer de \ovalnum{4} * réponse + \ovalnum{1,5}}
\blockmove{tourner \turnleft{} de \ovalnum{B} degrés}
\blockmove{avancer de \ovalnum{2} * réponse}
\blockmove{tourner \turnleft{} de \ovalnum{90} degrés}
}
\blockpen{relever le stylo}
\end{scratch}}&\small{\setscratch{scale=0.8}\begin{scratch}
\initmoreblocks{définir \namemoreblocks{script 2}}
\blocksensing{demander \ovalnum{Donner une valeur} et attendre}
\blockpen{stylo en position d’écriture}
\blockrepeat{répéter \ovalnum{C} fois}
{
\blockmove{avancer de \ovalnum{4} * réponse + \ovalnum{1}}
\blockmove{tourner \turnleft{} de \ovalnum{D} degrés}
}
\blockpen{relever le stylo}
\end{scratch}}\\
\end{tabularx}

\vspace{0,5cm}

