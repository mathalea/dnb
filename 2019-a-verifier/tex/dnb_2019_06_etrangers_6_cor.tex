
\medskip

\begin{enumerate}
\item Les points du graphique ne sont pas alignés. Il ne s'agit donc pas d'une situation de proportionnalité.
\item 
	\begin{enumerate}
		\item La randonnée a duré  $7$ heures.
		\item La famille a parcouru 20~km.
		\item Le point d'abscisse 6 a une ordonnée de 18 : au bout de six heures la famille a parcouru 18~km.
		\item Le point d'ordonnée 8 a pour abscisse 3 : la famille a parcouru 8 km en 3 heures.
		\item Entre la 4\up{e} et la 5\up{e} heure la distance parcourue n'a pas augmenté : ceci signifie que la famille s'est arrêtée.		
	\end{enumerate}
\item Un randonneur expérimenté parcourt $7 \times 4 = 28$~km en 7 heures. La famille n'en a fait que 20 : elle n'est pas expérimentée.
\end{enumerate}

\vspace{0,5cm}

