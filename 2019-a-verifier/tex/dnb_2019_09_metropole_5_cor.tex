
\medskip

%On donne le programme de calcul suivant :
%
%\begin{center}
%\begin{tabularx}{0.6\linewidth}{|l|X|}\hline
%Étape 1 :& Choisir un nombre de départ\\
%Étape 2 :& Ajouter 6 au nombre de départ\\
%Étape 3 :& Retrancher 5 au nombre de départ\\
%Étape 4 :& Multiplier les résultats des étapes 2 et 3\\
%Étape 5 :& Ajouter 30 à ce produit\\
%Étape 6 :& Donner le résultat\\ \hline
%\end{tabularx}
%\end{center}

\begin{enumerate}
\item 
	\begin{enumerate}
		\item %Montrer que si le nombre choisi est $4$, le résultat est $20$.
On obtient successivement :
		
$4 \to 10 \to 10 \times (4 - 5) = -10 \to 20$.
		\item %Quel est le résultat quand on applique ce programme de calcul au nombre $-3$ ?
On obtient successivement :
		
$- 3 \to 3 \to 3 \times (-3 - 5) = -24 \to 6$.
	\end{enumerate}
\item %Zoé pense qu'un nombre de départ étant choisi, le résultat est égal à la somme de ce nombre et de
%son carré.
	\begin{enumerate}
		\item %Vérifier qu'elle a raison quand le nombre choisi au départ vaut $4$, et aussi quand on choisit $- 3$.
On a effectivement $4 + 4^2 = 4 + 16 = 20$ et $- 3 + (- 3)^2 = - 3 + 9 = 6$ trouvés précédemment.
		\item  %Ismaël décide d'utiliser un tableur pour vérifier l'affirmation de Zoé sur quelques exemples.

%\begin{center}
%\begin{tabularx}{\linewidth}{|c|m{3cm}|*{5}{>{\centering \arraybackslash}X|}}\hline
%\multicolumn{2}{|l}{B6}&& \multicolumn{4}{|l|}{$=\text{B}1+\text{B}1\text{\verb+^+}2$}\\\hline
%&A& B&C&D&E&F\\ \hline
%1& Étape 1			& 2 	&5 	&7 	&10 	&20\\ \hline
%2& Étape 2			& 8 	&11 &13 &16 	&26\\ \hline
%3& Étape 3			& $-3$	& 0 &2 	&5 		&15\\ \hline
%4& Étape 4			& $-24$	& 0 &26 &80 	&390\\ \hline
%5& Étape 5 (résultat)&  6 	&30 &56 &110 	&420\\ \hline
%6&Somme du nombre et de son carré &6&30 &56 	&110 &420\\ \hline
%\end{tabularx}
%\end{center}
%
%Il a écrit des formules en B2 et B3 pour exécuter automatiquement les étapes 2 et 3 du
%programme de calcul.
%
%Quelle formule à recopier vers la droite a-t-il écrite dans la cellule B4 pour exécuter l'étape 4 ?
=B2*B3
		\item  %Zoé observe les résultats, puis confirme que pour tout nombre $x$ choisi, le résultat du
%programme de calcul est bien $x^2 + x$. Démontrer sa réponse.
En partant de $x$ on obtient :

$x \to x + 6 \to (x + 6)(x - 5) \to (x + 6)(x - 5) + 30 = x^2 - 5x + 6x - 30 + 30 = x^2 + x$. 
		\item  %Déterminer tous les nombres pour lesquels le résultat du programme est $0$.
Il faut résoudre l'équation :
		
$x + x^2 = 0$ ou $x(1 + x) = 0$ soit $\left\{\begin{array}{l c l}
x&=&0\\
1 + x&=&0
\end{array}\right.$ soit enfin  $\left\{\begin{array}{l c l}
x&=&0\\
x&=&- 1
\end{array}\right.$

$0$ et $- 1$ donnent $0$ par le programme de calcul.
	\end{enumerate}
\end{enumerate}

\bigskip

