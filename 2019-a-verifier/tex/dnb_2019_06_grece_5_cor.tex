
\medskip

%On donne le programme de calcul suivant : 
%
%\begin{center}
%\begin{tabularx}{0.6\linewidth}{|X|}\hline
%$\bullet~~$Choisir un nombre\\
%$\bullet~~$Ajouter 1 \\
%$\bullet~~$Élever le résultat au carré\\
%$\bullet~~$Soustraire au résultat le carré du nombre de départ\\ \hline
%\end{tabularx}
%\end{center}

\smallskip

\begin{enumerate}
\item %Montrer que lorsqu'on choisit le nombre 2 au départ, on obtient le nombre 5 au final.
On obtient successivement :

$2 \to 2 + 1 = 3 \to 3^2 = 9 \to 9 - 2^2 = 9 - 4 = 5$.
\item %Quel résultat obtient-on lorsqu'on choisit au départ le nombre $-3$ ? 
En patant de $- 3$, on obtient :

$- 3 \to - 3 + 1 = - 2 \to (- 2)^2 = 4 \to 4 - (- 3)^2 = 4 - 9 = - 5$.
\item %On définit une fonction $f$ qui, à tout nombre $x$ choisi à l'entrée du programme, associe le résultat obtenu à la fin de ce programme. 

\[\text{Ainsi, pour tout }\:x, \text{on obtient } \:f(x) = (x + 1)^2 - x^2\]

%Montrer que $f(x) = 2x + 1$. 
$f(x) = (x + 1)^2 - x^2 = x^2 + 2x + 1 - x^2 = 2x + 1$.
 \item %Cette question est un questionnaire à choix multiples (QCM). 

%Dans chaque cas, une seule réponse est correcte. Pour chacune des questions, écrire sur la copie le numéro de la question et la bonne réponse. 

%Aucune justification n'est demandée. 

%\begin{center}
%\begin{tabularx}{\linewidth}{|m{4cm}|*{3}{>{\centering \arraybackslash}X|}}\hline
%\textbf{Question}&   \textbf{Réponse A}&   \textbf{Réponse B}&   \textbf{Réponse C}\\ \hline   
%\textbf{1.~~} La représentation graphique de la fonction $f$ est:&\small  La représentation A&\small La représentation B&\small La représentation C\\ \hline
%\textbf{2.~~} En utilisant la représentation A, l'image de 1 par la fonction  représentée est:& 4   &$-2$&0\\ \hline
%\textbf{3.~~} En utilisant la représentation B, l'antécédent de 3 par la fonction représentée est :&$-1$   &$-5$&   2\\ \hline
%\end{tabularx}
%\end{center}
\begin{itemize}
\item La représentation graphique de la fonction $f$ est la représentation C ;
\item L'image de 1 par la fonction  représentée est $3$ ;
\item En utilisant la représentation B, l'antécédent de 3 par la fonction représentée est $- 1$.
\end{itemize}

%\parbox{0.33\linewidth}{\begin{center}
%Représentation A : 
%
%\psset{unit=0.6cm}
%\begin{pspicture*}(-4,-2)(3,6.5)
%\psgrid[gridlabels=0pt,subgriddiv=1,gridwidth=0.2pt]
%\psaxes[linewidth=1.25pt,labelFontSize=\scriptstyle]{->}(0,0)(-4,-2)(3,6.5)
%\psplot[plotpoints=2000,linewidth=1.25pt]{-3.5}{2}{ x 1 add dup mul}
%\end{pspicture*}
%\end{center}}
%\hfill
%\parbox{0.32\linewidth}{\begin{center}
%Représentation B :
%
%\psset{unit=0.6cm}
%\begin{pspicture*}(-1.5,-5)(3.5,3.5)
%\psgrid[gridlabels=0pt,subgriddiv=1,gridwidth=0.2pt]
%\psaxes[linewidth=1.25pt,labelFontSize=\scriptstyle]{->}(0,0)(-1.5,-5)(3.5,3.5)
%\psplot[plotpoints=2000,linewidth=1.25pt]{-3.5}{4}{  1 2 x  mul sub}
%\end{pspicture*}
%\end{center}} 
%\hfill
%\parbox{0.33\linewidth}{\begin{center}
%Représentation C : 
%
%\psset{unit=0.6cm}
%\begin{pspicture*}(-3,-2.5)(3.5,6)
%\psgrid[gridlabels=0pt,subgriddiv=1,gridwidth=0.2pt]
%\psaxes[linewidth=1.25pt,labelFontSize=\scriptstyle]{->}(0,0)(-3,-2.5)(3.5,6)
%\psplot[plotpoints=2000,linewidth=1.25pt]{-3.5}{4}{ 1 2 x  mul add}
%\end{pspicture*}
%\end{center}}

\end{enumerate}

\vspace{0.5cm}

