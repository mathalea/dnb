
\medskip

%Le tableau ci-dessous a été réalisé à l'aire d'un \textbf{tableur}.
%
%\smallskip
%
%Il indique le nombre d'abonnements Internet à haut débit et à très haut débit entre 2014 et 2016, sur réseau fixe, en France. (Sources : Arcep et Statistica).

%\smallskip
%
%\begin{tabularx}{\linewidth}{|c | >{\centering \arraybackslash}p{7,5cm} |*{3}{>{\centering \arraybackslash} X|}} \cline{2-5}
%\multicolumn{1}{c|}{~}  &\textsf{A}&\textsf{B}&\textsf{C}&\textsf{D}\\ \hline
%\textsf{1}&&\textbf{2014}&\textbf{2015}&\textbf{2016}\\ \hline
%\textsf{2}&Nombre d'abonnements Internet à haut débit (en millions)&22,855&22,63&22,238 \\ \hline
%\textsf{3}&Nombre d'abonnements Internet à très haut débit (en millions)&3,113&4,237&5,446 \\ \hline
%\textsf{4}&Total (en millions)&25,968&26,867&27,684 \\ \hline
%\end{tabularx}
%
%\smallskip
\begin{enumerate}
	\item %Combien d'abonnements Internet à très haut débit, en millions, ont été comptabilisés pour l'année 2016 ?
	En 2016, il y avait 5,446 millions d'abonnements Internet à très haut débit.
	\item %Vérifier qu'en 2016, il y avait \np{817000} abonnements Internet à haut débit et à très haut débit de plus qu'en 2015.
	On a $27,684 - 26,867 = 0,817$ million soit environ \np{817000} abonnements Internet à haut débit et à très haut débit de plus qu'en 2015.
	\item %Quelle formule a-t-on pu saisir dans la cellule \textsf{B4} avant de la recopier vers la droite, jusqu'à la cellule \textsf{D4} ?
On a saisi dans la cellule \textsf{B4} : $=\text{B}2 + \text{B}3$.
	\item %En 2015, seulement 5,6~\% des abonnements Internet à très haut débit utilisaient la fibre optique.
	
%Quel nombre d'abonnements Internet à très haut débit cela représentait-il?
On a $4,237 \times \dfrac{5,6}{100} = \np{0,237272}$ million d'abonnés soit \np{234272} qui utilisaient la fibre optique.
\end{enumerate}

\vspace{0,5cm}

