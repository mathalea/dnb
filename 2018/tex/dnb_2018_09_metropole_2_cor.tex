
\medskip

\begin{enumerate}
\item Le nombre $588$ peut se décomposer sous la forme $588 = 2^2 \times  3 \times 7^2$.

%Quels sont ses diviseurs premiers, c'est-à-dire les nombres qui sont à la fois des nombres premiers et des diviseurs de $588$ ?
Les diviseurs premiers de 588 sont 2~;~3 et 7.
\item 
	\begin{enumerate}
		\item %Déterminer la décomposition en facteurs premiers de \np{27000000}.
$\np{27000000} = 27 \times \np{1000000} = 3^3 \times 10^6 = 3^3 \times (2 \times 5)^{6} = 3^3 \times 2^6 \times 5^6 = 2^6 \times 3^3 \times 5^6$.
		\item %Quels sont ses diviseurs premiers ?
		Les diviseurs premiers de \np{27000000} sont 2~;~3 et 5
	\end{enumerate}
\item %Déterminer le plus petit nombre entier positif impair qui admet trois diviseurs premiers différents. Expliquer votre raisonnement.
Les premiers nombres impairs premiers sont 3~;~5 et 7, donc le plus petit entier impair admettant trois diviseurs premiers différents est $3 \times 5 \times 7 = 105$.
\end{enumerate}

\vspace{0,5cm}

