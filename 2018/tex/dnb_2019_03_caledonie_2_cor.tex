
\medskip

\begin{enumerate}
	\item Calculons le volume d'un moule, en prenant la moitié du volume d'une boule de rayon $r = \np[cm]{3}$. On a :
	
$\mathcal{V}_\text{moule} = \dfrac{1}{2} \times \dfrac{4}{3}\times \pi \times r^3 = \dfrac{2}{3} \times \pi \times 3^3 = 18\pi \approx \np[cm^3]{56,5}$.
	
	\item Remplir un moule dont le volume est de \np[cm^3]{57} au trois quarts signifie que chaque moule contiendra $\dfrac{3}{4}\times 57 = \np[cm^3]{42,75}$ de pâte, soit \np[dm^3]{0,04275} de pâte, et donc \np[L]{0,04275}.
	
$\dfrac{1}{\np{0,04275}} \approx 23,4$.
	
Jade a assez de pâte pour préparer 23 takoyakis.
	
\end{enumerate}



