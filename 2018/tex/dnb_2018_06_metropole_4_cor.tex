
\medskip

\begin{enumerate}
\item Le triangle CBD est rectangle en B. Le théorème de Pythagore s'écrit  :
$\text{CD}^2 = \text{DB}^2 + \text{CB}^2$, soit $\text{DB}^2 = \text{CD}^2 - \text{CB}^2 = 8,5^2 - 7,5^2 = (8,5 + 7,5)(8,5 - 7,5) = 6 \times 1 = 16 = 4^2$.

DB $ = 4$~(cm).
\item Deux triangles semblables ont les mesures de leurs côtés proportionnelles.

Or $\dfrac{6}{7,5} = 0,8$, \quad $\dfrac{3,2}{4} = 0,8$ et $\dfrac{6,8}{8,5} = 0,8$

Par conséquent les triangles CBD et BFE sont semblables.
\item Vérifions que le triangle BFE est rectangle :

$\bullet~~$$\text{BE}^2 = 6,8^2 = 46,24$, \quad $\text{BF}^2 = 6^2 = 36$ et $\text{FE}^2 : 3,2^2 = 10,24$.

$\text{BF}^2 + \text{FE}^2 = 36 + 10,24 = 46,24$.

Donc $\text{BE}^2 = \text{BF}^2 + \text{FE}^2$ et par la réciproque de Pythagore le triangle BEF est rectangle en F.

$\bullet~~$Plus rapide : les triangles CBD et BFE étant semblables, on a $\widehat{\text{CBD}} = \widehat{\text{BFE}} = 90\degres$ puisque le triangle CBD est rectangle en B.
\item  Calculons l'angle $\widehat{\text{DCB}}$ par son cosinus dans le triangle rectangle DCB :

$\cos \widehat{\text{DCB}} = \dfrac{\text{CB}}{\text{CD}}  = \dfrac{7,5}{8,5} = \dfrac{75}{85} = \dfrac{15}{17}$. La calculatrice donne $\cos ^{-1} \dfrac{15}{17} \approx 28$\degres.

Or : $28 + 61  = 89 \neq 90$ : l'angle $\widehat{\text{ACD}}$ n'est pas droit.
\end{enumerate}

\bigskip

