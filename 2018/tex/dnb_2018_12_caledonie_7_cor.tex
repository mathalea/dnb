
\medskip

%Pour chacune des affirmations suivantes, indiquer si elle est \textbf{VRAIE} ou \textbf{FAUSSE} et justifier la réponse.
%
%\medskip

\textbf{Affirmation 1 :} %l'aire de la partie grise de la figure ci-dessous est $36 - x^2$.
L'aire du grand carré est : $6^2 = 36$ et l'aire du petit carré est $x^2$, donc l'aire de la surface grise est : $36 - x^2$. Affirmation vraie.

%\begin{center}
%\psset{unit=1cm}
%\begin{pspicture}(3.2,3.2)%\psgrid
%\psframe[fillstyle=solid,fillcolor=lightgray](0.2,0.2)(3,3)
%\psframe[fillstyle=solid,fillcolor=white](1.2,1.2)(2.2,2.2)
%\psline{<->}(3.2,0.2)(3.2,3)\uput[r](3.2,1.7){6}
%\psline{<->}(1.2,1.4)(2.2,1.4)\uput[u](1.7,1.4){$x$}
%\psframe(0.2,3)(0.4,2.8)\psframe(1.2,2.2)(1.35,2.05)
%\psline(0.1,1.55)(0.3,1.55)\psline(0.1,1.65)(0.3,1.65)
%\psline(2.9,1.55)(3.1,1.55)\psline(2.9,1.65)(3.1,1.65)
%\psline(1.55,2.9)(1.55,3.1)\psline(1.65,2.9)(1.65,3.1)
%\psline(1.55,0.1)(1.55,0.3)\psline(1.65,0.1)(1.65,0.3)
%\psline(1.1,1.7)(1.3,1.7)\psline(2.1,1.7)(2.3,1.7)
%\psline(1.65,2.1)(1.65,2.3)\psline(1.75,1.1)(1.75,1.3)
%\end{pspicture}
%\end{center}

\textbf{Affirmation 2 :} %le chiffre 8 est écrit $20$ fois lorsque j'écris tous les nombres entiers de 1 à 100.
Il y a tous les nombres se terminant par 8 : 8, 18 ; 28 ; etc. : 10 nombres, donc 10 chiffres 8 ;

Il y a tous les nombres commençant par 8,  soit 10 chiffres 8. On a donc utilisé en tout $10 + 10 = 20$ chiffres 8. L'affirmation est vraie.

\vspace{0,5cm}

