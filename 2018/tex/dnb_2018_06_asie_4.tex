
\medskip

\parbox{0.65\linewidth}{Arthur doit écrire un programme avec Scratch
pour dessiner une étoile comme le dessin représenté ci-contre.

Il manque dans son programme le nombre de répétitions.}\hfill
\parbox{0.33\linewidth}{\psset{unit=0.35cm}
\begin{pspicture}(-6,-6)(6,5)
\rput{34}(0,0){\pspolygon[linecolor=blue](2.05;-16)(5.51;20)(2.05;56)(5.51;92)(2.05;128)(5.51;164)(2.05;200)(5.51;236)(2.05;272)(5.51;308)}
\rput(3,-4){Point de départ}\rput(3,-5){du tracé}
\psline{->}(3,-3.4)(1.3,-1.7)
\end{pspicture}}

\begin{tabularx}{\linewidth}{l X}
\begin{tabular}{|l|}\hline
\multicolumn{1}{|c|}{Programme commencé par Arthur}\\ \hline
{\small{\begin{scratch}
\blockinit{quand \greenflag est cliqué}
\blockmove{s'orienter à \ovalnum{90\selectarrownum}}
\blockpen{effacer tout}
\blockpen{stylo en position d'écriture}
\blockrepeat{répéter \ovalnum{} fois}
     {\blockmove{avancer de \ovalnum{80}}
      \blockmove{tourner \turnleft{} de \ovalnum{144} degrés}
     \blockmove{avancer de \ovalnum{80}}
      \blockmove{tourner \turnright{} de \ovalnum{72} degrés}
    }
\blockpen{relever le stylo}
\end{scratch}}}\\ \hline
\end{tabular}&\hspace{1.5cm}\begin{tabular}{|c|}\hline
Information\\
L'instruction\\
\small{\begin{scratch}
\blockmove{s'orienter à \ovalnum{90\selectarrownum}}
\end{scratch}}\\
signifie qu'on se dirige\\
vers la droite.\\ \hline
\end{tabular}\\ 
\end{tabularx}

\medskip

\begin{enumerate}
\item Quel nombre doit-il saisir dans la boucle
\og répéter \fg{} pour obtenir l'étoile?
\item Déterminer le périmètre de cette étoile.
\end{enumerate}

\parbox{0.58\linewidth}{\begin{enumerate}[resume]
\item Arthur souhaite agrandir cette étoile pour obtenir une étoile dont le périmètre serait le double, en modifiant son programme.

Recopier la partie du programme ci-contre sur
la copie en modifiant les valeurs nécessaires
pour obtenir cette nouvelle étoile.
\end{enumerate}}\hfill\parbox{0.38\linewidth}{\begin{scratch}
\blockrepeat{répéter \ovalnum{} fois}
     { \blockmove{avancer de \ovalnum{80}}
      \blockmove{tourner \turnleft{} de \ovalnum{144} degrés}
     \blockmove{avancer de \ovalnum{80}}
      \blockmove{tourner \turnright{} de \ovalnum{72} degrés}
    }
\end{scratch}}

\bigskip

