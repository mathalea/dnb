
\medskip

\textbf{1\up{re} partie}

%\smallskip
%
%\parbox{0.45\linewidth}{\psset{unit=1cm}
%\begin{pspicture}(7.4,3.6)
%%\psgrid
%\psline[linewidth=1.5pt](0,1.3)(5.3,1.3)(7,1.6)(7.4,1.8)
%\psline[linewidth=1.5pt,linestyle=dashed](6.9,1.9)(7.4,1.8)
%\psline[linewidth=1.5pt](0.6,1.7)(1.2,2)(6.5,2)(6.9,1.9)
%\psline[linewidth=1.5pt,linestyle=dashed](0,1.3)(0.6,1.7)
%\pspolygon(0,1.65)(5.3,1.65)(7.1,1.9)(7.4,2)(6.5,2.3)(1.15,2.3)
%\psline(0,0)(5.3,0)(7.1,0.2)(7.4,0.5)
%\psline[linestyle=dashed](7.4,0.5)(6.4,1.1)(1.2,1.1)(0,0)
%\psline[linestyle=dashed](6.4,1.2)(6.4,2.3)
%\psline[linestyle=dashed](1.2,1.1)(1.2,2)
%\psline(1.2,2)(1.2,2.3)
%\psline(0,1.7)(0,0)\psline(5.3,0)(5.3,1.7)\psline(7.1,0.2)(7.1,1.9)
%\psline(7.4,0.5)(7.4,2)
%\rput(6,3.3){Frise de la piscine}\psline{->}(6,3.1)(5.5,2.2)
%\end{pspicture}}\hfill
%\parbox{0.45\linewidth}{Une personne possède une piscine. 
%
%Elle veut coller une frise en carrelage au niveau de la ligne d'eau.}
%
%\medskip
%
%La piscine vue de haut, est représentée à l'échelle par la partie grisée du schéma ci-après.
%
%\begin{center}
%\psset{unit=1cm}
%\begin{pspicture}(10,4.5)
%\pspolygon[linewidth=1.3pt,fillstyle=solid,fillcolor=lightgray](0.5,0.5)(7.5,0.5)(9.1,1.8)(9.1,2.6)(7.5,4)(0.5,4)%HGEDBA
%\psline(7.5,0.5)(9.1,0.5)(9.1,4)(7.5,4)
%\uput[dl](0.5,0.5){H}\uput[d](7.5,0.5){G}\uput[r](9.1,1.8){E}\uput[r](9.1,2.6){D}
%\uput[u](7.5,4){B}\uput[ul](0.5,4){A}\uput[dr](9.1,0.5){F}\uput[ur](9.1,4){C}
%\rput(8.3,4){|}\rput(8.3,0.5){|}\rput{90}(9.1,3.3){|}\rput{90}(9.1,3.4){|}
%\rput{90}(9.1,1){|}\rput{90}(9.1,1.1){|}
%\end{pspicture}
%\end{center}
%
%\textbf{Données :}
%
%\smallskip
%
%\setlength\parindent{6mm}
%\begin{itemize}
%\item[$\bullet~~$]le quadrilatère ACFH est un rectangle;
%\item[$\bullet~~$]le point B est sur le côté [AC] et le point G est sur le côté [FH] ;
%\item[$\bullet~~$]les points D et E sont sur le côté [CF] ;
%\item[$\bullet~~$]AC $= 10$ m; AH $= 4$ m ; BC = FG $= 2$ m ; CD = EF $= 1,5$ m.
%\end{itemize}
%\setlength\parindent{0mm}

\smallskip

\textbf{Question :}

%Calculer la longueur de la frise.

La longueur de la frise est : AB + BD + DE + EG + GH + HA.

Or BCD et FGH sont des triangles rectangles dont les deux côtés de l'angle droit mesurent 2~m et 1,5~m. Les hypoténuses de ces triangles [BD] et [EG] ont donc d'après le théorème de Pythagore une longueur telle que :

$\text{BD}^2 = \text{EG}^2 = 2^2 + 1,5^2 = 4 + 2,25 = 6,25$.

Donc $\text{BD} = \text{EG} = 2,5$.

La longueur de la frise est donc égale à :

$10 - 2 + 2,5 + 1 + 2,5 + 10 - 2 + 4 = 26$~(m).

\bigskip

\textbf{2\up{e} partie}

%\smallskip
%
%La personne décide d'installer, au-dessus de la piscine, une grande voile d'ombrage
%qui se compose de deux parties détachables reliées par une fermeture éclair comme
%le montre le schéma ci-dessous qui n'est pas à l'échelle.
%
%\begin{center}
%\psset{unit=1cm}
%\begin{pspicture}(12,9)
%\pspolygon[fillstyle=solid,fillcolor=lightgray](-0.2,4.5)(7.2,4.5)(7.9,5.5)(7.9,6.5)(7.2,7.9)(-0.2,7.9)
%\pspolygon[fillstyle=solid,fillcolor=white](0.5,0.5)(8.4,1)(1,8.5)
%\psline[linestyle=dashed,linewidth=1.5pt](0.66,3.9)(5.5,4)
%\uput[dl](0.5,0.5){M}\uput[dr](8.4,1){O}\uput[u](1,8.5){K}
%\uput[l](0.65,3.9){L}\uput[r](5.7,4){N}
%\rput(9.5,7.5){Piscine}\psline{->}(8.8,7.4)(7.5,5.5)
%\rput(9,3){2 parties du voile d'ombrage}
%\psline{->}(6.8,3)(5.7,1.5)
%\psline{->}(6.8,3.2)(3,5)
%\end{pspicture}
%\end{center}
%
%\textbf{Données :}
%
%\smallskip
%
%\setlength\parindent{6mm}
%\begin{itemize}
%\item[$\bullet~~$]la première partie couvrant une partie de la piscine est représentée par le
%triangle KLN ;
%\item[$\bullet~~$]la deuxième partie est représentée par le trapèze LMON de bases [LN] et [MO] ;
%\item[$\bullet~~$]la fermeture éclair est représentée par le segment [LN] ;
%\item[$\bullet~~$]les poteaux, soutenant la voile d'ombrage positionnés sur les points K, L et M, sont alignés;
%\item[$\bullet~~$]les poteaux, soutenant la voile d'ombrage positionnés sur les points K, N et 0,
%sont alignés;
%\item[$\bullet~~$]KL $= 5$ m ; LM $= 3,5$ m ; NO $= 5,25$ m ; MO $= 10,2$ m.
%\end{itemize}
%\setlength\parindent{0mm}
%
%\smallskip
%
%\textbf{Question :}
%
%Calculer la longueur de la fermeture éclair.
LMON étant un trapèze les droites (LN) et (MO) sont parallèles.

Dans le triangle KMO on a donc d'après le théorème de Thalès :

$\dfrac{\text{KL}}{\text{KM}} = \dfrac{\text{KN}}{\text{KO}} = \dfrac{\text{LN}}{\text{MO}}$, soit 

$\dfrac{5}{5 + 3,5} =  \dfrac{\text{LN}}{10,2}$ ou $\dfrac{5}{8,5} =  \dfrac{\text{LN}}{10,2}$ d'où

LN $ = 10,2 \times \dfrac{5}{8,5} = \dfrac{51}{8,5} = 6$~(m).

