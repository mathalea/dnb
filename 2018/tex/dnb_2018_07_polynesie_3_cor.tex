
\medskip

\begin{enumerate}
\item Nombre moyen dans la classe A : $\dfrac{5 + 7 + 12 + 15 + 15 + 16 + 18 + 21 + 34 + 67}{15} = \dfrac{210}{15} = \dfrac{70}{5} = 14$.

Il y a 15 valeurs ; la médiane est donc la 8\up{e} soit 12.
\item En Q3  : =somme(B3\negthinspace:K3)/10

En R3 : =(F3+G3)/2
\item On calcule $\dfrac{15 \times 14 + 10 \times 12}{10 + 15} = \dfrac{210 + 120}{25} = \dfrac{330}{25} = 13,2$.
\item On prend pour nombre médian la treizième valeur : 12.
\end{enumerate}


