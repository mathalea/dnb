
\medskip

Un amateur de football, après l'Euro 2016, décide de s'intéresser à l'historique des
treize dernières rencontres entre la France et le Portugal, regroupées dans le tableau
ci-dessous.

On rappelle la signification des résultats ci-dessous en commentant deux exemples :

\setlength\parindent{6mm}
\begin{itemize}
\item[$\bullet~~$]la rencontre du 3 mars 1973, qui s'est déroulée en France, a vu la victoire du Portugal par 2 buts à 1 ;
\item[$\bullet~~$]la rencontre du 8 mars 1978, qui s'est déroulée en France, a vu la victoire de la France par 2 buts à 0.
\end{itemize}
\setlength\parindent{0mm}

\begin{center}
\begin{tabularx}{\linewidth}{|*{3}{>{\centering \arraybackslash}X|}}
\hline
\multicolumn{3}{|c|}{\textbf{Rencontres de football opposant la France et le Portugal depuis 1973}}\\ \hline
3 mars 1973		& France - Portugal& 1-2\\ \hline
26 avril 1975	& France - Portugal& 0-2\\ \hline
8 mars 1978		& France - Portugal& 2-0\\ \hline
16 février 1983	& Portugal - France& 0-3\\ \hline
23 juin 1984	& France - Portugal& 3-2\\ \hline
24 janvier 1996	& France - Portugal& 3-2\\ \hline
22 janvier 1997	& Portugal - France& 0-2\\ \hline
28 juin 2000	& Portugal - France& 1-2\\ \hline
25 avril 2001	& France - Portugal& 4-0\\ \hline
5 juillet 2006	& Portugal - France& 0-1\\ \hline
11 octobre 2014	& France - Portugal& 2-1\\ \hline
4 septembre 2015& Portugal - France& 0-1\\ \hline
10 juillet 2016	& France - Portugal& 0-1\\ \hline
\end{tabularx}
\end{center}

\begin{enumerate}
\item Depuis 1973, combien de fois la France a-t-elle gagné contre le Portugal ?
\item Calculer le pourcentage du nombre de victoires de la France contre le Portugal
depuis 1973. Arrondir le résultat à l'unité de \%.
\item Le 3 mars 1973, 3 buts ont été marqués au cours du match. Calculer le nombre
moyen de buts par match sur l'ensemble des rencontres. Arrondir le résultat au
dixième.
\end{enumerate}

\bigskip

