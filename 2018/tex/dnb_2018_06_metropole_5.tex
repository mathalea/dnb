
\medskip

Voici un programme de calcul \quad  \begin{tabular}{|l|}\hline
$\bullet~~$Choisir un nombre\\
$\bullet~~$Multiplier ce nombre par 4\\
$\bullet~~$Ajouter 8\\
$\bullet~~$Multiplier le résultat par 2\\ \hline
\end{tabular}

\medskip

\begin{enumerate}
\item Vérifier que si on choisit le nombre $- 1$, ce programme donne 8 comme résultat final.
\item Le programme donne 30 comme résultat final, quel est le nombre choisi au départ ?
\end{enumerate}
\smallskip

Dans la suite de l'exercice, on nomme $x$ le nombre choisi au départ.

\begin{enumerate}[resume]
\item L'expression $A = 2(4x + 8)$ donne le résultat du programme de calcul précédent pour un nombre $x$ donné.

On pose $B = (4 + x)^2 - x^2$.

Prouver que les expressions $A$ et $B$ sont égales pour toutes les valeurs de $x$.
\item Pour chacune des affirmations suivantes, indiquer si elle est vraie ou fausse. On rappelle que les réponses doivent être justifiées.

\smallskip

$\bullet~~$ Affirmation 1 : Ce programme donne un résultat positif pour toutes les valeurs de $x$.

$\bullet~~$ Affirmation 2 : Si le nombre $x$ choisi est un nombre entier, le résultat obtenu est un multiple de $8$.
\end{enumerate}

\bigskip

