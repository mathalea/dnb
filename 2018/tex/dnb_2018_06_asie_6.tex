
\medskip


Avec un logiciel de géométrie dynamique, on a construit la figure A.
En appliquant à la figure A des homothéties de centre O et de rapports différents, on a
ensuite obtenu les autres figures.
\medskip

\begin{center}
\psset{unit=0.65cm}
\begin{pspicture}(-0.5,-0.5)(20.5,9.5)
%\psgrid
\pspolygon(2,1.8)(3.8,0.4)(3.2,0.7)(2.4,1)%figureA
\pspolygon(4,3.6)(7.6,0.8)(6.4,1.4)(4.8,2)%figureB
\pspolygon(6,5.4)(11.4,1.2)(9.6,2.1)(7.2,3)%figureC
\pspolygon(8,7.2)(15.2,1.6)(12.8,2.8)(9.6,4)%figureD
\pspolygon(10,9)(19,2)(16,3.5)(12,5)
\psline(10.5,9.45)(0,0)(19.5,2.1)
\rput(2.6,1.3){figure A}
\rput(5.2,2.6){figure B}
\rput(7.8,3.9){figure C}
\rput(10.4,5.2){figure D}
\rput(13,6.5){figure E}
\uput[dl](0,0){O}\uput[ul](2,1.8){A} \uput[ul](4,3.6){B} 
\uput[ul](6,5.4){C} \uput[ul](8,7.2){D} \uput[ul](10,9){E} 
\psdots[dotstyle=+,dotangle=45,dotscale=1.8](1,0.9)(3,2.7)(5,4.5)(7,6.3)(9,8.1)
\end{pspicture}
\end{center}

\begin{enumerate}
\item Quel est le rapport de l'homothétie de centre O qui permet d'obtenir la figure C à
partir de la figure A ? Aucune justification n'est attendue.
\item On applique l'homothétie de centre O et de rapport $\frac{3}{5}$ à la figure E. Quelle figure obtient-on ?

\emph{Aucune justification n'est attendue.}
\item Quelle figure a une aire quatre fois plus grande que celle de la figure A ?
\end{enumerate}

\bigskip

