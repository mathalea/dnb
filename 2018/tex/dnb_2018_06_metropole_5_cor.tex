
\medskip

\begin{enumerate}
\item Si $n$ est ce nombre on obtient: $2 \times (4n + 8)$.

Avec $n = - 1$ :  $2 \times(-1  \times 4 + 8) = 2 \times 4 = 8$.
\item  On résout l'équation: $8x + 16 = 30$ ou
$8x = 14$ et enfin $x = \dfrac{14}{8} = 1,75$.
\item  Si $A = B$ alors
$8x + 16 = (4 + x)^2 - x^2$ ou encore  $16 + 8x + x^2 - x^2 = 8x + 16$ ; 
les deux expressions sont  effectivement égales.
\item  $16 + 8x > 0$ ou  $8x > - 16$ et enfin  $x > - 2$.

Non, seulement pour les valeurs de $x$ supérieures à $-2$.

Affirmation 2

$A = 16 + 8x = 8(2 + x)$ :  affirmation juste car les résultats sont multiples de 8.
\end{enumerate}

\bigskip

