
\medskip

La géode, située à la Cité des Sciences de la Villette à Paris, est une
structure sphérique.

\medskip

\begin{enumerate}
\item La salle de projection, située à l'intérieur de la géode, est une
demi-sphère de diamètre 26~m.

Une sphère de diamètre 26~m a pour rayon 13~m et donc pour volume
$\dfrac{4}{3}\pi r^3 = \dfrac{4}{3}\pi\times 13^3$.

Le volume de la géode est donc 
$\dfrac{1}{2}\times \dfrac{4}{3}\pi \times 13^3 \approx \np{4601}$~m$^3$

%Calculer le volume de cette salle. Donner la réponse en m$^3$ arrondie à l'unité.

\item La surface extérieure est en partie recouverte de triangles équilatéraux de $120$~cm de côté.

	\begin{enumerate}

		\item% Montrer que la hauteur d'un de ces triangles est $104$~cm (arrondie à l'unité).

\begin{multicols}{2}

Soit ABC un triangle équilatéral de côté 120~cm. On appelle H le pied de la hauteur issue de C.

Dans le triangle ACH rectangle en H:\\ $\sin \widehat{\text{CAH}} = \dfrac{\text{CH}}{\text{AC}}$ donc $\text{CH}= \sin \widehat{\text{CAH}}\times \text{AC}$.

Le triangle ABC est équilatéral de côté 120 donc $\text{AC}=120$ et chaque angle de ce triangle vaut 60° donc $\widehat{\text{CAH}}=60\degres$.

On a donc: 
$\text{CH}= \sin 60\degres\times 120 \approx 104$

La hauteur d'un de ces triangles est approximativement de 104~cm.

\columnbreak

\begin{center}
\psset{unit=0.3cm}
\def\xmin {-1}   \def\xmax {13}
\def\ymin {-1}   \def\ymax {13}
\begin{pspicture}(\xmin,\ymin)(\xmax,\ymax)
%\psgrid[subgriddiv=10]
\pspolygon(0,0)(12,0)(12;60)
\psline(12;60)(6,0)
\uput[-150](0,0){\small A} \uput[-30](12,0){\small B} 
\uput[u](12;60){\small C} 
\uput[d](6,0){\small H}
\psarc(0,0){2}{0}{60} \uput[30](2;30){60°} 
\psline(5,0)(5,1)(6,1)
\end{pspicture}
\end{center}

\end{multicols}

		\item% En déduire que l'aire d'un triangle est d'environ \np{6240}cm$^2$.
L'aire d'un de ces triangles est égale à
$\dfrac{\text{AB}\times \text{CH}}{2} \approx \dfrac{120\times 104}{2} \approx \np{6240}$~cm$^2$.



 	\end{enumerate}

\item Il a fallu \np{6433} triangles pour recouvrir la partie extérieure de la Géode.
	
%Quelle est l'aire de la surface recouverte par ces triangles ? Donner la réponse en m$^2$ arrondie à l'unité.

La surface recouverte par ces triangles est approximativement de
$\np{6433}\times\np{6240} = \np{40141920}$~cm$^2$ soit \np{4014,1920}~m$^2$ ce qui donne en arrondissant au m$^2$: \np{4014}~m$^2$.  

\end{enumerate}

%\medskip
%
%\begin{tabularx}{\linewidth}{|l X|}\hline
%Formulaire :& Volume d'une sphère : $S = \dfrac{4}{3} \times \pi \times r^3$ où $r$ est le rayon de la sphère.\\
%&Aire d'un triangle :  $A = \dfrac{b \times h}{2}$ où $b$ est l'aire d'une base et $h$ sa hauteur associée.\\ \hline
%\end{tabularx}

\vspace{0,5cm}

