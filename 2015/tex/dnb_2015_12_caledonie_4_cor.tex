
\medskip

%Pour répondre à la demande d'un client, un décorateur a besoin de découper des triangles dans du carrelage. Les triangles doivent être rectangles et isocèles avec une hypoténuse de longueur 15~cm. Les carreaux qu'il doit utiliser sont des carrés de 12~cm de côté.
%
%Ces carreaux sont-ils assez grands pour faire deux de ces triangles dans chacun d'eux ? 
%
%Justifier.
%
%\emph{Dans cette question, toute trace de recherche, même incomplète ou non fructueuse, sera prise en compte dans l'évaluation.}
Soit $c$ la longueur de côté du triangle rectangle isocèle d’hypoténuse 15~cm.

D’après le théorème de Pythagore on a $c^2 + c^2 = 15^2$, soit $2c^2 = 225$, donc $c^2 = 112,5$, donc $c = \sqrt{112,5} \approx 10,61$~cm.

Cette longueur étant inférieure à 12~cm on pourra découper les triangles rectangles isocèles dans des carreaux de 12~cm de côté.
\vspace{0,5cm}

