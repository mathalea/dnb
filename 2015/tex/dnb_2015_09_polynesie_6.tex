
\medskip

Germaine souhaite réaliser un escalier pour monter à l'étage de son
appartement.

Elle a besoin pour cela de connaître les dimensions du limon (planche
dans laquelle viendront se fixer les marches de cet escalier).

Elle réalise le croquis ci-dessous.

\begin{center}
\psset{unit=0.6cm,arrowsize=3pt 5}
\begin{pspicture}(0,-1)(18.2,7)
%\psgrid
\psline(0,0.5)(18.2,0.5)
\pspolygon[linewidth=2pt,fillstyle=solid,fillcolor=lightgray](1.3,0.5)(2.6,0.5)(10.1,5.2)(10.1,6.1)
\psline(16.8,5.2)(10.1,5.2)(10.1,6.1)(16.8,6.1)
\psline[linestyle=dotted](10.1,5.2)(10.1,0.5)
\psline[linewidth=0.2pt]{<->}(11.5,6.1)(11.5,5.2)
\psline[linewidth=0.2pt]{<->}(11.5,0.5)(11.5,5.2)
\psline[linewidth=0.2pt]{<->}(1.3,-0.5)(10.1,-0.5)
\uput[u](5.7,-0.5){360 cm}
\rput(15.6,5.7){Épaisseur de la dalle : 20 cm}
\rput(15.6,2.8){Hauteur sous plafond : 250 cm}
\psframe(10.1,0.5)(10.4,0.8)
\uput[d](2.6,0.5){A} \uput[d](10.1,0.5){B} \uput[dr](10.1,5.2){C} 
\uput[u](10.1,6.1){D} \uput[d](1.3,0.5){E} 
\end{pspicture}
\end{center}

Sur ce croquis :

\setlength\parindent{6mm}
\begin{itemize}
\item le limon est représenté par le quadrilatère ACDE.
\item les droites (AC) et (ED) sont parallèles.
\item les points E, A et B sont alignés.
\item les points B, C et D sont alignés.
\end{itemize}
\setlength\parindent{0mm}

\medskip

\begin{enumerate}
\item Prouver que ED = 450 cm.
\item Calculer les deux dimensions AC et AE de cette planche. Arrondir les résultats au centimètre.
\end{enumerate}

\vspace{0,5cm}

