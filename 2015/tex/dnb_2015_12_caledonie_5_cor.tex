
\medskip

%Une boîte \og Chocodor \fg{} contient exactement 10 chocolats au lait, 8 chocolats noirs et 6 chocolats blancs.
%
%Tous les chocolats ont la même forme et sont indiscernables au toucher.
%
%\medskip

\begin{enumerate}
\item %Si l'on prend un chocolat au hasard dans cette boîte, quelle est la probabilité que ce soit un chocolat au lait ?
Il y a 10 chocolats au lait sur un total de 24 chocolats ; la probabilité est donc égale à : $\dfrac{10}{24} = \dfrac{5}{12}$.
\item %Alexis a acheté une boîte \og Chocodor\fg{} et a déjà pris un chocolat de chaque sorte. Par gourmandise, il veut en prendre un quatrième sans regarder. Quelle est la probabilité que ce soit un chocolat noir ?
Il reste 9 chocolats au lait, 7 chocolats noirs et 5 chocolats blancs.

La probabilité de tirer un chocolat noir est donc égale à : $\dfrac{7}{21} = \dfrac{1}{3}$.
\item %Thomas a aussi acheté une boîte identique. Il l'a ouverte et a pris deux chocolats au hasard.

%Quelle est la probabilité qu'il prenne deux chocolats blancs ?
La probabilité de tirer un premier chocolat blanc est égale à $\dfrac{6}{24} = \dfrac{1}{4}$.

Il reste alors 5 chocolats blancs sur 23 chocolats : la probabilité de tirer alors un chocolat blanc est égale à $\dfrac{5}{23}$.

La probabilité d’avoir tiré deux chocolats blancs est donc égale à :

$\dfrac{1}{4} \times \dfrac{5}{23} = \dfrac{5}{92} \approx 0,054$ soit un peu plus de 5\,\%.
\end{enumerate}

\vspace{0,5cm}

