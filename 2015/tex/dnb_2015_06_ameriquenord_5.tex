
\medskip

Pour filmer les étapes d'une course cycliste, les réalisateurs de télévision utilisent des
caméras installées sur deux motos et d'autres dans deux hélicoptères. Un avion relais,
plus haut dans le ciel, recueille les images et joue le rôle d'une antenne relais.
On considère que les deux hélicoptères se situent à la même altitude et que le peloton  des coureurs roule sur une route horizontale. Le schéma ci-dessous illustre cette
situation:

\begin{center}
\psset{unit=0.8cm}
\begin{pspicture}(14,7)
%\psgrid
\psdots[dotstyle=+,dotangle=45](7,5.8)(1.8,0)(11.8,0)(3.8,2.2)(10,2.2)%AMNHJ
\uput[u](7,5.8){A (avion)} \uput[d](1.8,0){M (moto 2)} \uput[d](11.8,0){N (moto 1)} \uput[ul](3.8,2.2){H hélicoptère 2} \uput[ur](10,2.2){L hélicoptère 1} 
%\pspolygon(7,5.8)(1.8,0)(11.8,0)
\psline(0,0)(14,0)
\end{pspicture}
\end{center}

\medskip

L'avion relais (point A), le premier hélicoptère (point L) et la première moto (point N)
sont alignés. 

De la même manière, l'avion relais (point A), le deuxième hélicoptère
(point H) et la deuxième moto (point M) sont également alignés.

On sait que : AM = AN = 1 km ; HL = 270~m et AH = AL = 720~m.

\medskip

\begin{enumerate}
\item Relever la phrase de l'énoncé qui permet d'affirmer que les droites (LH) et (MN) sont
parallèles.
\item  Calculer la distance MN entre les deux motos.
\end{enumerate}
 
\vspace{0,5cm}

