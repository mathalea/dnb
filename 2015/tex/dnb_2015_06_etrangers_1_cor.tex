
\medskip

%\parbox{0.75\linewidth}{\emph{Pour cet exercice, aucune justification n'est attendue.}
%
%
%En appuyant sur un bouton, on allume une des cases de la grille ci-contre 
%au hasard.}\hfill 
%\parbox{0.22\linewidth}{\begin{tabularx}{\linewidth}{|*{3}{>{\centering \arraybackslash}X|}}\hline
%1 &2& 3\\ \hline
%4 &5& 6\\ \hline 
%7 &8 &9\\ \hline
%\end{tabularx}}
%
%\medskip

\begin{enumerate}
\item 
	\begin{enumerate}
		\item %Quelle est la probabilité que la case 1 s'allume?
La probabilité est égale à $\dfrac{1}{9}$.
		\item %Quelle est la probabilité qu'une case marquée d'un chiffre impair s'allume ?
		Il y a sur les 9 nombres, 5 qui sont impairs ; la probabilité est donc égale à $\dfrac{5}{9}$.		
		\item %Pour cette expérience aléatoire, définir un évènement qui aurait pour probabilité $\dfrac{1}{3}$.
Évènements de probabilité  $\dfrac{1}{3}$ : 

\og la case d'un multiple de 3 s'allume \fg ;

\og la case d'un nombre plus petit que 4 s'allume \fg.
	\end{enumerate}
\item %Les cases 1 et 7 sont restées allumées. En appuyant sur un autre bouton, quelle est la
%probabilité que les trois cases allumées soient alignées ?
En supposant que les seules les cases éteintes puissent s'allumer la seule possibilité d'avoir trois cases allumées et alignées est que la case 4 s'allume soit une chance sur 7 cases éteintes : probabilité égale à $\dfrac{1}{7}$.
\end{enumerate}

\vspace{0,5cm}

