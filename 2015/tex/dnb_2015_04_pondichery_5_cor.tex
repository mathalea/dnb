
\medskip

%Un jeu télévisé propose à des candidats deux épreuves :
%
%\setlength\parindent{6mm}
%\begin{itemize}
%\item[$\bullet~~$]Pour la première épreuve, le candidat est face à 5 portes : une seule
%porte donne accès à la salle du trésor alors que les 4 autres s'ouvrent
%sur la salle de consolation.
%\item[$\bullet~~$]Pour la deuxième épreuve, le candidat se retrouve dans une salle face
%à 8 enveloppes.
%
%\textbf{Dans la salle du trésor }: 1 enveloppe contient \np{1000}~\euro, 5 enveloppes
%contiennent 200~\euro. Les autres contiennent 100~\euro.
%
%\textbf{Dans la salle de consolation} : 5 enveloppes contiennent 100~\euro{} et les
%autres sont vides.
%\end{itemize}
%\setlength\parindent{0mm}
%
%Il doit choisir une seule enveloppe et découvre alors le montant qu'il a gagné.
%
%\medskip

\begin{enumerate}
\item %Quelle est la probabilité que le candidat accède à la salle du trésor ?
Il y a une porte sur cinq qui donne accès à la salle du trésor ; la probabilité  d'y accéder est donc égale à $\dfrac{1}{5} = 0,2$.
\item %Un candidat se retrouve dans la salle du trésor.
	\begin{enumerate}
		\item %Représenter par un schéma la situation.
Soit $M$ l'évènement \og le candidat choisit une enveloppe contenant mille euros \fg ; on a $p(M) = \dfrac{1}{8} = 0,125$ ;

Soit $D$ l'évènement \og le candidat choisit une enveloppe contenant deux cents euros \fg ; on a $p(D) = \dfrac{5}{8} = 0,625$ ;

Soit $C$ l'évènement \og le candidat choisit une enveloppe contenant cent euros \fg ; on a $p(C) = \dfrac{2}{8} = 0,250$.

Ce que l'on peut schématiser par :

\begin{center}
\pstree[treemode=R,nodesep=2pt]{\TR{}}
{\TR{$M$}\taput{0,125}
\TR{$D$}\taput{0,625}
\TR{$C$}\tbput{0,250}
}
\end{center}
		\item %Quelle est la probabilité qu'il gagne au moins 200~\euro{} ?
		La probabilité de gagner au moins 200~\euro{} est la probabilité contraire de gagner 100~\euro{} soit :

$1 - 0,250 = 0,75$ ou encore 3 chances sur 4.
	\end{enumerate}
\item %Un autre candidat se retrouve dans la salle de consolation.

%Quelle est la probabilité qu'il ne gagne rien ?
Dans la salle de consolation 3 enveloppes sur 8 ne contiennent rien ; la probabilité de ne rien gagner est donc égale à $\dfrac{3}{8} = 0,375$.
\end{enumerate}

\vspace{0.5cm}

