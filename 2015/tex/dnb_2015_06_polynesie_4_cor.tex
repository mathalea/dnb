
\medskip

\begin{enumerate}
\item ~
La solde est de $80 - 60 = 20$ pour un prix initial de 80, soit une réduction de $\dfrac{20}{80} = \dfrac{1}{4} = \dfrac{25}{100} = 25$\,\%.

Le nombre caché sur l’affiche est 25.
%\parbox{0.62\linewidth}{Quel est le nombre caché par la tache sur cette étiquette ?} \hfill
%\parbox{0.35\linewidth}{\psset{unit=0.8cm}
%\begin{pspicture}(3.8,6)
%%\psgrid
%
%\psframe[framearc=0.3](3.8,6)
%\rput(1.3,5){Ancien prix}
%\rput(1.3,4){80~\euro}
%\rput(1.9,3){\Large Soldes \quad \quad \,\%}
%\rput(1.85,2.5){\pspolygon*(0,0.4)(0.4,0.6)(0.2,0.8)(0.55,0.9)(0.75,1.15)(0.82,.8)(1.1,0.8)(0.9,0.6)(1.1,0.4)(0.8,0.22)(0.7,0)(0.5,0.24)(0.22,0)(0.25,0.3)}
%\rput(1.3,2){Nouveau prix}
%\rput(1.3,1){60~\euro}
%\end{pspicture}}


\item  %\np{2048} est une puissance de 2. Laquelle ?
$2^{10} = \np{1024}$, donc $2^{11} = \np{2048}$.
\item  %En développant l'expression $(2x - 1)^2$, Jules a obtenu $4x^2 - 4x - 1$. A-t-il raison ?
$(2x - 1)^2 = (2x)^2 + 1^2 - 2 \times 2x \times 1 = 4x^2 + 1 - 4x$. Jules n'a pas raison.
\end{enumerate}

\vspace{0,5cm}

