
\medskip

\begin{enumerate}
\item %Le graphique ci-dessous donne le niveau de bruit (en décibels) d'une tondeuse à gazon en marche, en fonction de la distance (en mètres) entre la tondeuse et l'endroit où s'effectue la mesure.
\begin{center}
\psset{unit=0.07cm}
\begin{pspicture}(-8,-5)(150,95)
\psaxes[linewidth=1.25pt,Dx=10,Dy=10]{->}(0,0)(0,0)(150,95)
\uput[r](0,92){\scriptsize Niveau de bruit (en décibels)}
\uput[u](135,0){\scriptsize  Distance (en mètres)}
\psplot[plotpoints=3000,linewidth=1.25pt,linecolor=blue]{1}{150}{2.71828 x 0.04 mul neg exp 50 mul 45 add}
\psset{arrowsize=3pt 4}
\psline[linestyle=dashed,ArrowInside=->](100,0)(100,46)(0,46)
\psline[linestyle=dashed,ArrowInside=->](0,60)(30,60)(30,0)
\end{pspicture}
\end{center}
%
%En utilisant ce graphique, répondre aux deux questions suivantes. \emph{Aucune justification n'est attendue.}

	\begin{enumerate}
		\item %Quel est le niveau de bruit à une distance de $100$ mètres de la tondeuse ?
À une distance de 100 mètres de la tondeuse, le niveau de bruit est d’environ $45$~décibels.
		\item %À quelle distance de la tondeuse se trouve-t-on quand le niveau de bruit est égal à $60$~décibels ?
Le niveau de bruit est de 60 décibels à une distance de 30 mètres de la tondeuse.
	\end{enumerate}
\item  %Voici les graphiques obtenus pour deux machines très bruyantes d'une usine .
À 5 mètres de la machine A, le bruit est de $88$~décibels environ. Pour la machine B, ce 
niveau de bruit est atteint à presque 10 mètres de distance.
\begin{center}
\begin{tabularx}{\linewidth}{X X}
\psset{xunit=0.14cm,yunit=0.07cm}
\begin{pspicture}(-5,-5)(35,105)
\uput[r](0,102){\scriptsize Niveau de bruit (en décibels)}
\uput[u](25,0){\scriptsize  Distance (en mètres)}
\rput(17.5,35){Machine A}
\psaxes[linewidth=1.25pt,Dx=5,Dy=10,labelFontSize=\scriptstyle]{->}(0,0)(0,0)(35,105)
\multido{\n=0+5}{8}{\psline[linestyle=dotted,linewidth=0.2pt](\n,0)(\n,100)}
\multido{\n=0+10}{11}{\psline[linestyle=dotted,linewidth=0.2pt](0,\n)(35,\n)}
\psplot[plotpoints=3000,linewidth=1.25pt]{1}{35}{2.71828 x 0.11 mul neg exp 40 mul 65 add}
\psset{arrowsize=3pt 4}
\psline[linestyle=dashed,ArrowInside=->](5,0)(5,88)(0,88)	
\end{pspicture}&\psset{xunit=0.14cm,yunit=0.07cm}
\begin{pspicture}(-5,-5)(35,105)
\uput[r](0,102){\scriptsize Niveau de bruit (en décibels)}
\uput[u](25,0){\scriptsize  Distance (en mètres)}
\rput(17.5,35){Machine B}
\multido{\n=0+5}{8}{\psline[linestyle=dotted,linewidth=0.2pt](\n,0)(\n,100)}
\multido{\n=0+10}{11}{\psline[linestyle=dotted,linewidth=0.2pt](0,\n)(35,\n)}
\psaxes[linewidth=1.25pt,Dx=5,Dy=10,labelFontSize=\scriptstyle]{->}(0,0)(0,0)(35,105)
\psplot[plotpoints=3000,linewidth=1.25pt]{1}{35}{2.71828 x 0.1 mul neg exp 28 mul 77 add}
\psset{arrowsize=3pt 4}
\psline[linestyle=dashed,ArrowInside=->](0,87.86)(9.3,87.86)(9.3,0)	
\end{pspicture}
\end{tabularx}
\end{center}
%
%Dans l'usine, le port d'un casque antibruit est obligatoire à partir d'un \textbf{même niveau de bruit}.
%
%Pour la machine A, il est obligatoire quand on se trouve à moins de $5$~mètres de la machine. En utilisant ces graphiques, déterminer cette distance pour la machine B.
\end{enumerate}

\vspace{0,5cm}

