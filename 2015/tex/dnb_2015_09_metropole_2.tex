
\medskip

\emph{On considère la figure ci-contre qui n'est pas à l'échelle.}

\medskip

\parbox{0.55\linewidth}{\setlength\parindent{6mm} 
\begin{itemize}
\item[$\bullet~~$] Le triangle JAB est rectangle en A.   
\item[$\bullet~~$] Les droites (MU) et (AB) sont parallèles.         
\item[$\bullet~~$] Les points A, M et J sont alignés.         
\item[$\bullet~~$] Les points C, U et J sont alignés.         
\item[$\bullet~~$] Les points A, C et B sont alignés.         
\item[$\bullet~~$] AB = 7,5 m.         
\item[$\bullet~~$] MU = 3 m.         
\item[$\bullet~~$] JM = 10 m.         
\item[$\bullet~~$] JA = 18 m. 
\end{itemize}       
} \hfill
\parbox{0.45\linewidth}{\psset{unit=0.9cm}
\begin{pspicture}(6,7)
%\psgrid
\pspolygon(0.5,0.5)(0.5,6)(5.5,6)
\psline(0.5,0.5)(3,6)
\psline(0.5,2.6)(1.42,2.6)
\uput[u](0.5,6){A} \uput[u](5.5,6){B} \uput[u](3,6){C} 
\uput[l](0.5,2.6){M} \uput[r](1.42,2.6){U} \uput[d](0.5,0.5){J} 
\psframe(0.5,6)(0.7,5.8)
\end{pspicture}}

\begin{enumerate}
\item Calculer la longueur JB. 
\item Montrer que la longueur AC est égale à 5,4 m. 
\item Calculer l'aire du triangle JCB. 
\end{enumerate}

\vspace{0.5cm}

