
\medskip

Le 14 octobre 2012, Félix Baumgartner, a effectué un saut d'une altitude de \np{38969,3}~mètres.

La première partie de son saut s'est faite en chute libre (parachute fermé).

La seconde partie, s'est faite avec un parachute ouvert.

Son objectif était d'être le premier homme à \textbf{\og dépasser le mur du son \fg}.

\begin{center}\textbf{\og dépasser le mur du son \fg{}} : signifie atteindre une vitesse supérieure ou égale à la vitesse du son, c'est à dire $340$ m.s$^{-1}$.\end{center}

La Fédération Aéronautique Internationale a établi qu'il avait atteint la vitesse maximale de
\np{1357,6} km.h$^{-1}$ au cours de sa chute libre.

\medskip

\begin{enumerate}
\item A-t-il atteint son objectif ? Justifier votre réponse.
\item Voici un tableau donnant quelques informations chiffrées sur ce saut :

\begin{center}
\begin{tabularx}{0.7\linewidth}{|l|X|}\hline
Altitude du saut 					&\np{38969,3} m\\ \hline
Distance parcourue en chute libre	&\np{36529} m\\ \hline
Durée totale du saut				&9 min 3 s\\ \hline
Durée de la chute libre				&4 min 19 s\\ \hline
\end{tabularx}
\end{center}

Calculer la vitesse moyenne de Félix Baumgartner en chute avec parachute ouvert
exprimée en m.s$^{-1}$. On arrondira à l'unité.
\end{enumerate}

\vspace{0,5cm}

