
\medskip

%Louise a téléchargé une liste de lecture sur son lecteur MP4 : 
%
%\medskip
%\begin{tabularx}{\linewidth}{|*{3}{>{\centering \arraybackslash}X|}}\hline  
%\textbf{Titre de la chanson}&   \textbf{Nom de l'interprète}&\textbf{Durée de la chanson en secondes}\\ \hline   
%Mamatéou		&Timaté 	&  232 \\ \hline  
%La différence	&Timaté		&  211 \\ \hline  
%Amazing		&Timaté		&  214 \\ \hline  
%Tes racines	&Timaté		&  175\\ \hline   
%YoungBov   	&Hudad		&  336 \\ \hline  
%La ficelle  	&Maen		&  191\\ \hline   
%Fou fou fou	&Maen		&  184 \\ \hline  
%Nina   		&Maen		&  217 \\ \hline 
%\end{tabularx}
%\medskip

\begin{enumerate}
\item 
	\begin{enumerate}
		\item %Quelle est la durée totale de cette liste ? Exprimer cette durée en minutes et secondes.
$232+211+214+175+336+191+184+217 = \np{1760}$~s soit $\np{1800} - 40$~(s) ou 30~min moins 40 s  soit 29~min 20~s.		 
		\item %Déterminer le pourcentage de chansons dont la durée est supérieure à 3 min 30 s.
3 min 30 s $= 180 + 30 = 210$~(s).

5 chansons sur 8 dépassent la durée, soit 2,5 sur 4 ou en multipliant par 25, 62,5 pour 100.\: (62,5\,\%) 
	\end{enumerate}
\item %Louise décide d'utiliser la fonction \og aléatoire \fg{} de son MP4. Cette fonction choisit au hasard une chanson parmi celles qui sont présentes dans la liste de lecture. Chaque chanson a la même probabilité d'être écoutée. 

%Déterminer la probabilité que Louise écoute une chanson de Maen.
Sur 8 chansons 3 sont interprétées par Maen ; la probabilité est donc égale à $\dfrac{3}{8} = \dfrac{1,5}{4} = \dfrac{37,5}{100} = 0,375 = 37,5\,\%$. 
\item %Elle répète 25 fois l'utilisation de la fonction \og aléatoire \fg{} de son MP4 et note à chaque fois le nom de l'interprète qu'elle a écouté. Les résultats qu'elle obtient sont notés dans le graphique ci-dessous.

%Déterminer la fréquence d'écoute de Hudad. 
%
%\begin{center}
%\psset{xunit=1cm,yunit=0.4cm}
%\begin{pspicture}(-1,-4)(9,17)
%\psaxes[Dx=20,Dy=2](0,0)(9,17)
%\multido{\n=0+2}{9}{\psline(0,\n)(9,\n)}
% \rput{90}(-1,13){Nombre  d'écoutes}  
%\uput[d](6.5,-1){Nom des interprètes }  
%\uput[d](2,0){Timaté} 
%\uput[d](4.5,0){Hudad} 
%\uput[d](7,0){Maen} 
%\psframe[fillstyle=solid,fillcolor=lightgray](1,0)(3,15)
%\psframe[fillstyle=solid,fillcolor=lightgray](3.5,0)(5.5,4)
%\psframe[fillstyle=solid,fillcolor=lightgray](6,0)(8,6)
%\end{pspicture}
%\end{center}
Sur 25 morceaux écoutés 4 étaient interprétées par Hudad : la fréquence d'écoute de cet interprète est donc égale à $\dfrac{4}{25} = \dfrac{16}{100} = 0,16$.
\end{enumerate}

\vspace{0.5cm}

