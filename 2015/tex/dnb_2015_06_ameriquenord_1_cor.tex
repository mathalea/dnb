
\medskip

%Dans ce questionnaire à choix multiple, pour chaque question, des réponses sont
%proposées et une seule est exacte.
%
%Pour chacune des questions, écrire le numéro de la question et recopier la bonne
%réponse.
%
%Aucune justification n'est attendue.
%
%\begin{center}
%\begin{tabularx}{\linewidth}{|m{7.5cm}|*{3}{>{\centering \arraybackslash}X|}}\hline
%Questions&\multicolumn{3}{c|}{Réponses}\\ \hline
%\textbf{1.} Quelle est l'écriture scientifique de $\dfrac{5 \times 10^6 \times 1,2 \times 10^{- 8} }{2,4 \times  10^5}$ ?&$25 \times 10^{- 8}$&$2,5 \times 10^{- 7}$&$2,5 \times 10^{3}$\\ \hline
%\textbf{2.} Pour $x = 20$ et $y = 5$, quelle est la valeur de $R$ dans l'expression
%$\dfrac{1}{R} = \dfrac{1}{x} + \dfrac{1}{y}$ ?&0,25& 4 &25\\ \hline
%\textbf{3.} Un article coûte 120~\euro. Une fois soldé, il coûte 90~\euro.
%Quel est le pourcentage de réduction ?&25\,\% &30\,\% &75\,\%\\ \hline
%\textbf{4.} On considère l'agrandissement de coefficient 2 d'un rectangle
%ayant pour largeur 5~cm et pour longueur 8~cm.
%
%Quelle est l'aire du rectangle obtenu ?&40~cm$^2$&80~cm$^2$&160~cm$^2$\\ \hline
%\end{tabularx}
%\end{center}
\begin{enumerate}
\item $\dfrac{5 \times 10^6 \times 1,2 \times 10^{- 8} }{2,4 \times  10^5} = \dfrac{5 \times 1,2}{2,4} \times \dfrac{10^6 \times 10^{-8}}{10^5} = \dfrac{5}{2}\times \dfrac{10^2}{10^{-5}} = 2,5 \times 10^{- 7}$ : réponse B.
\item Pour $x = 20$ et $y = 5$,\:$\dfrac{1}{R} = \dfrac{1}{20} + \dfrac{1}{5} =  \dfrac{1}{20} +  \dfrac{4}{20}  =  \dfrac{5}{20}  =  \dfrac{1}{4}$, donc $R = 4$ : réponse B.
\item La solde est égale à 120 - 90 = 30~\euro{} pour un prix initial de 120~\euro, soit une réduction de $\dfrac{30}{120} = \dfrac{1}{4} = \dfrac{25}{100} = 25$\,\% : réponse A.
\item Puisque l'agrandissement est de coefficient 2, l'aire est multipliée par $2^2 = 4$. 
	Aire du rectangle avant agrandissement : $5 \times 8 = 40$~cm$^2$ ; 
	$40 \times  4 = 160$~cm$^2$.	L'aire du rectangle obtenu après agrandissement est 160~cm$^2$ :  réponse C.
\end{enumerate}

\vspace{0,5cm}

