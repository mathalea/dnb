
\medskip

%Un vendeur souhaite rendre son magasin plus accessible aux personnes en fauteuil roulant. Pour cela il s'est renseigné sur les normes et a décidé d'installer une rampe avec une pente de 3 degrés comme indiqué sur le schéma suivant.
%
%\begin{center}
%\psset{unit=1cm}
%\begin{pspicture}(10,5)
%\psframe(0,0)(3,4.5)
%\psline(0,1)(3,1)(9.5,0)(3,0)
%\uput[ur](9.5,0){A}\uput[ur](3,1){C} \uput[d](3,0){B}
%\rput(1.5,3.5){Entrée}
%\uput[r](5,4){ABC est un triangle rectangle en B.}
%\uput[r](5,3.5){$\widehat{\text{CAB}}$ mesure 3\degres}
%\uput[r](5,3){BC = 30 cm}
%\end{pspicture}
%\end{center}
%
%Calculer la longueur AB, arrondie au centimètre, pour savoir où la rampe doit commencer.
Dans ABC  triangle rectangle en B, on a $\tan \widehat{\text{CAB}} = \dfrac{\text{CB}}{\text{AB}}$, donc AB $ =\dfrac{\text{CB}}{\tan \widehat{\text{CAB}}} = \dfrac{30}{\tan 3} \approx 572,43$~cm. Il faut donc prendre une longueur AB  au moins égale à 573~cm.
\vspace{0,5cm}

