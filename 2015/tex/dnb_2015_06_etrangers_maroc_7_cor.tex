
\medskip

%\begin{center}
%\psset{unit=0.8cm}
%\begin{pspicture}(15,12.5)
%%\psgrid
%\psframe(1.5,1.2)(8.5,3.1)%EHDA
%\psline(8.5,1.2)(10.5,2.7)(10.5,4.6)(6,11.7)(8.5,3.1)%HGCID
%\psline(6,11.7)(1.5,3.1)%IA
%\psline(10.5,4.6)(8.5,3.1)%CD
%\psline[linestyle=dashed](1.5,3.1)(3.5,4.6)(10.5,4.6)%ABC
%\psline[linestyle=dashed](1.5,1.2)(3.5,2.7)(10.5,2.7)%EFG
%\psline[linestyle=dashed](6,11.7)(3.5,4.6)(3.5,2.7)%IBF
%\psline(3.4,6.7)(7.45,6.7)(8.6,7.6)%RMS
%\psline[linestyle=dashed](8.6,7.6)(4.58,7.6)(3.4,6.7)%STR
%\uput[ul](1.5,3.1){A} \uput[ul](3.5,4.6){B}
%\uput[ur](10.5,4.6){C}\uput[r](8.5,3.1){D}
%\uput[dl](1.5,1.2){E}\uput[ul](3.5,2.7){F}
%\uput[ul](10.5,2.7){G}\uput[ul](8.5,1.2){H}
%\uput[u](6,11.7){I}\uput[dr](7.45,6.7){M}
%\uput[ul](3.4,6.7){R}\uput[ur](8.6,7.6){S}
%\uput[ul](4.58,7.6){T}\uput[r](6,3.85){K$_1$}
%\uput[r](6,7.15){K$_2$}
%\psdots(1.5,3.1)(3.5,4.6)(10.5,4.6)(8.5,3.1)(1.5,1.2)(3.5,2.7)
%(10.5,2.7)(8.5,1.2)(6,11.7)(7.45,6.7)(3.4,6.7)(8.6,7.6)(4.58,7.6)(6,3.85)(6,7.15)
%\psline{->}(12,3.6)(11,3.6)\uput[r](12,3.8){Partie}\uput[r](12,3.4){principale}
%\psline{->}(12,6)(11,6)\uput[r](12,6){Chambres}
%\psline{->}(12,9.6)(11,9.6)\uput[r](12,9.6){Grenier}
%\psline{<->}(2.6,6.7)(2.6,11.7)\uput[l](2.6,9.2){4,5 m}
%\psline{<->}(1,1.2)(1,3.1)\uput[l](1,2.2){3 m}
%\psline{<->}(1,11.7)(1,3.1)\uput[l](1,7.4){6,75 m}
%\psline{<->}(1.5,0.6)(8.5,0.6)\uput[d](5,0.6){12 m}
%\psline{<->}(9,1.2)(11,2.7)\rput(10.5,1.7){9 m}
%\end{pspicture}
%\end{center}
%
%Une maison est composée d'une partie principale qui a la forme d'un pavé droit
%ABCDEFGH surmonté d'une pyramide IABCD de sommet I et de hauteur [IK$_1$]
%$\left[\text{IK}_1\right]$
%perpendiculaire à la base de la pyramide.
%
%Cette pyramide est coupée en deux parties :
%
%\setlength\parindent{8mm}
%\begin{itemize}
%\item[$\bullet~~$] Une partie basse ABCDRTSM destinée aux chambres;
%\item[$\bullet~~$] Une partie haute IRTSM réduction de hauteur $\left[\text{IK}_2\right]$ de la pyramide IABCD correspondant au grenier.
%\end{itemize}
%\setlength\parindent{0mm}
%
%\end{center}
%On a : EH = 12 m ; AE = 3 m ; HG = 9 m ; IK$_1$ = 6,75 m et IK$_2$ = 4,5 m.
%
%\emph{La figure donnée n'est pas à l'échelle.}
%
%\medskip

\begin{enumerate}
\item %Calculer la surface au sol de la maison.
Le sol est un rectangle de 12 m sur 9 m  ; la surface au sol est donc égale à $12 \times 9 = 108$~m$^2$.
\item %Des radiateurs électriques seront installés dans toute la maison, excepté au grenier.

%On cherche le volume à chauffer de la maison.

%On rappelle que le volume d'une pyramide est donné par :
%
%\[V_{\text{pyramide}} = \dfrac{\text{Aire de la Base} \times \text{Hauteur}}{3}\]

	\begin{enumerate}
		\item %Calculer le volume de la partie principale.
La base est un pavé dont on vient de calculer l'aire de la base et de hauteur 3~m ; le volume de la partie principale est donc égal à : $108 \times 3 = 324$~m$^3$.
		\item %Calculer le volume des chambres.
La partie haute (grenier) est une réduction de la pyramide IABCD dans le rapport $\dfrac{4,5}{6,75} = \dfrac{450}{675} = \dfrac{18 \times 25}{27 \times 25} = \dfrac{18}{27}  = \dfrac{2 \times 9}{3 \times 9} = \dfrac{2}{3}$.

Chaque dimension de la petite pyramide étant égale à celle de la grande multipliée par $\dfrac{2}{3}$, son volume est donc égal à celui de la grande multiplié par $\left(\dfrac{2}{3} \right)^3$.

Volume de la grande pyramide :

$\dfrac{108 \times 6,75}{3} = 108 \times 2,25 = 243$~m$^3$.

Volume de la petite pyramide = $243 \times \left(\dfrac{2}{3} \right)^3 = \dfrac{9 \times 27 \times 8}{\times 27} = 72$~m$^3$.

Le volume des chambres est donc égal à $243 - 72 = 171$~m$^3$.
		\item % Montrer que le volume à chauffer est égal à 495~m$^3$.
Le volume total à chauffer est donc égal à : $324 + 171 = 495$.
	\end{enumerate}
\item  %Un expert a estimé qu'il faut dans cette maison une puissance électrique de
%925~Watts pour chauffer 25 mètres cubes.
%	
%Le propriétaire de la maison décide d'acheter des radiateurs qui ont une puissance
%de \np{1800}~watts chacun et qui coûtent 349,90~\euro{} pièce.
%	
%Combien va-t-il devoir dépenser pour rachat des radiateurs ?
Pour chauffer la partie principale et les chambres il faut une puissance de $\dfrac{495}{25} \times 925 = 495 \times 37 = \np{18315}$~Watts.

Il faut donc acheter un nombre de radiateurs égal à : $\dfrac{\np{18315}}{\np{1800}} \approx 10,17$.

IL faut acheter 11 radiateurs à 349,90~euros pièce d'où une dépense de :

\[11 \times 349,90 = \np{3848,90}~(\text{\euro}).\]

\end{enumerate} 
