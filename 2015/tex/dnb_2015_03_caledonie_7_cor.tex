
\medskip

Le club de sport \og Santé et Forme \fg{} propose à ses clients deux
tarifs :

Tarif A: forfait annuel à \np{90000}~F

Tarif B: une adhésion à \np{5000}~F puis un abonnement mensuel à
\np{7900}~F.
%
%
%\parbox{0.58\linewidth}{Le club de sport \og Santé et Forme \fg{} propose à ses clients deux
%tarifs :
%
%Tarif A: forfait annuel à \np{90000}~F
%
%Tarif B: une adhésion à \np{5000}~F puis un abonnement mensuel à
%\np{7900}~F.
%
%\begin{enumerate}
%\item Mathilde est intéressée mais elle ne sait pas quel tarif
%choisir. Pour s'aider elle utilise un tableur (ci-contre).
%
%Mathilde a utilisé une formule pour le calcul du tarif B.
%
%Parmi les quatre propositions suivantes, recopie sur ta feuille
%celle qui correspond à la cellule C4 :
%
%\fbox{20800 + 7900}
%
%\fbox{=5000+A4*7900}
%
%\fbox{=somme (C2:C3)}
%
%\fbox{(C2+C3)/2}
%\end{enumerate}} \hfill
%\parbox{0.4\linewidth}{\begin{tabularx}{\linewidth}{|c|*{3}{>{\centering \arraybackslash}X|}}\hline
%&A&B&C\\ \hline
%1&\footnotesize Nombre de mois&\footnotesize tarif A&\footnotesize tarif B\\ \hline
%2	&1	&\np{90000}&\np{12900}\\ \hline
%3	&2	&\np{90000}&\np{20800}\\ \hline 
%4	&3	&\np{90000}&\np{28700}\\ \hline 
%5	&4	&\np{90000}&\np{36600}\\ \hline 
%6	&5	&\np{90000}&\np{44500}\\ \hline 
%7	&6	&\np{90000}&\np{52400}\\ \hline 
%8	&7	&\np{90000}&\np{60300}\\ \hline 
%9	&8	&\np{90000}&\np{68200}\\ \hline 
%10	&9	&\np{90000}&\np{76100}\\ \hline 
%11	&10	&\np{90000}&\np{84000}\\ \hline 
%12	&11	&\np{90000}&\np{91900}\\ \hline 
%13	&12	&\np{90000}&\np{99800}\\ \hline  
%\end{tabularx}}
%
%\parbox{0.4\linewidth}{
%\begin{enumerate}
%\item[\textbf{2.}] À partir de combien de mois d'abonnement le tarif A devient-il
%plus intéressant que le tarif B ?
%\item[\textbf{3.}] Mathilde construit aussi le graphique correspondant (ci-contre).
%
%Lequel des tarifs A ou B est représenté par la droite $g$ ?
%\end{enumerate}}
%\hfill
%\parbox{0.6\linewidth}{\psset{xunit=0.45cm,yunit=0.000045cm}
%\begin{pspicture}(-3,-10000)(13,120000)
%\multido{\n=0+1}{14}{\psline[linestyle=dotted](\n,0)(\n,120000)}
%\multido{\n=0+10000}{13}{\psline[linestyle=dotted](0,\n)(13,\n)}
%\psaxes[linewidth=1.25pt,Dy=150000,labelFontSize=\scriptstyle]{->}(0,0)(13,120000)
%\psline[linewidth=1.5pt](0,90000)(13,90000)
%\psplot[plotpoints=3000,linewidth=1.5pt]{0}{13}{7900 x mul 5000 add}
%\uput[u](4.8,90000){$h$}
%\uput[u](4.8,44000){$g$}
%\multido{\n=0+10000}{13}{\uput[l](0,\n){\scriptsize \np{\n}}}
%\end{pspicture}
%}

\begin{enumerate}
\item La formule qui doit être dans la cellule C4 est
\fbox{= 5000 + A4*7900}

\item D'après le tableur, le 11\ieme{} mois est la première fois que le tarif B (\np{91900}~F) est supérieur au tarif A (\np{90000}~F); c'est donc à partir du 11\ieme{} mois que le tarif A devient plus intéressant que le tarif B. 

\item La droite $h$ correspond à une fonction constante, donc au prix constant correspondant au tarif A; c'est donc le tarif B qui est représenté par la droite $g$.

\end{enumerate}


\vspace{0,5cm}

