
\medskip

%\parbox{0.5\linewidth}{Un aquarium a la forme d'une sphère de 10~cm de
%rayon, coupée en sa partie haute: c'est une \og calotte
%sphérique \fg.
%
%La hauteur totale de l'aquarium est 18 cm.}\hfill
%\parbox{0.47\linewidth}{\psset{unit=0.9cm}
%\begin{pspicture*}(5.6,4)
%%\psgrid
%\psarc(2.5,2.5){2.5}{144}{36}
%\psline(0.45,4)(4.55,4)
%\psline{<->}(2.5,2.5)(5,2.5)\uput[u](3.75,2.5){$r$}
%\psline{<->}(5.3,0)(5.3,4)\rput{90}(5.45,2){$h$}
%\end{pspicture*}}
%
%\medskip

\begin{enumerate}
\item %Le volume d'une calotte sphérique est donné par la formule :

%\[V \dfrac{\pi}{3} \times h^2 \times (3r - h)\]

%où $r$ est le rayon de la sphère et $h$ est la hauteur de la calotte sphérique.
	\begin{enumerate}
		\item %Prouver que la valeur exacte du volume en cm$^3$ de l'aquarium est $\np{1296}\pi$.
$V = \dfrac{\pi}{3} \times  h^2 \times (3r - h)$ 

$V = \dfrac{\pi}{3} \times 18^2 \times(3 \times 10 - 18)$

$V = \dfrac{\pi}{3} \times 324 \times (30 - 18)$

$V = \dfrac{\pi}{3} \times 324 \times 12$

$V = \dfrac{\np{3888}\pi}{3} \approx \np{1296} \pi$~cm$^3$.
		\item %Donner la valeur approchée du volume de l'aquarium au litre près.
$V = \dfrac{\np{3888}\pi}{3} \approx \np{1296} \pi \approx \np{4072}$~cm$^3$ soit à peu près $4~\ell$.
	\end{enumerate}
\item %On remplit cet aquarium à ras bord, puis on verse la totalité de son contenu dans
%un autre aquarium parallélépipédique. La base du nouvel aquarium est un rectangle
%de $15$~cm par $20$~cm.

%Déterminer la hauteur atteinte par l'eau (on arrondira au cm).

%* Rappel: 1 l = 1 dm$^3 = \np{1000}$ cm$^3$
Soit $h$ la hauteur atteinte par l’eau dans le nouvel aquarium. On a :

$15 \times 20 \times  h = \np{1296}\pi$

$300h = \np{1296}\pi$

$h = \dfrac{\np{1296}\pi}{300}$

$h \approx  14$~cm.

La hauteur atteinte par l’eau est d’environ $14$~cm.
\end{enumerate}
