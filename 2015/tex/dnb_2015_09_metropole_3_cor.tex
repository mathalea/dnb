
\medskip

%\textbf{Document 1 : Principe de fonctionnement d'un radar tronçon}
%
%\medskip
%
% Étape 1 : enregistrement de la plaque d'immatriculation et de l'heure de passage par un premier portique. 
% 
%Étape 2 : enregistrement de la plaque d'immatriculation et de l'heure de passage par un second portique. 
%
%Étape 3 : calcul de la vitesse moyenne du véhicule entre les deux radars par un ordinateur. 
%
%Étape 4 : calcul de la vitesse retenue afin de prendre en compte les erreurs de précisions du radar. 
%
%Étape 5 : si la vitesse retenue est au-dessus de la vitesse limite, l'automobiliste reçoit une contravention. 
%
%\medskip
%
%\textbf{Document 2 : Calcul de la vitesse retenue pour la contravention}
%
%\medskip
%\begin{footnotesize}
%\begin{tabularx}{\linewidth}{|*{3}{>{\centering \arraybackslash}X|}}\hline     
%Vitesse moyenne calculée par  l'ordinateur&  inférieure à 100 km/h&   supérieure à 100 km/h \\ \hline  
%Vitesse retenue&On enlève 5 km/h à la vitesse enregistrée&On diminue la vitesse  enregistrée de 5\,\%\\ \hline   
%Exemples&Vitesse enregistrée : 97 km/h  & Vitesse enregistrée : 125 km/h   \\
%&Vitesse retenue: 92 km/h&Vitesse retenue : 118,75 km/h \\ \hline
%\end{tabularx}
%\end{footnotesize}
%
%\medskip
%
%\textbf{Document 3 : Le radar tronçon du pont d'Oléron}
%
%\medskip 
%
%Le pont d'Oléron est équipé d'un radar tronçon sur une distance de 3,2 km. 
%
%Sur le pont, la vitesse est limitée à 90 km/h. 
%
%\medskip

\begin{enumerate}
\item Les deux personnes suivantes ont reçu une contravention après avoir emprunté le pont d'Oléron. 

Cas 1 : %Madame Surget a été enregistrée à une vitesse moyenne de 107~km/h. Quelle est la vitesse retenue ? 
La vitesse étant supérieure à 100~km/h, on enlève 5\,\% à la vitesse constatée.
La vitesse retenue est donc : $107 - \dfrac{5}{100}\times 107 = \dfrac{95}{100} \times 107 =$

$ 95 \times 1,07 = 101,65$~(km/h).

Cas 2 : %Monsieur Lagarde a mis 2 minutes pour parcourir la distance entre les deux points d'enregistrement. Quelle est la vitesse retenue ? 
La vitesse de M. Lagarde est $\dfrac{3,2}{2} = 1,6$~(km/min), soit $1,6 \times 60 = 96$~(km/h). La vitesse étant inférieure à 100, on enlève 5 à cette vitesse : la vitesse retenue est égale à $96 - 5 = 91$; d'où la contravention.
\item %La plaque d'immatriculation de Monsieur Durand a été enregistrée à 

%13 h~46~min~54~s puis à 13~h~48~min~41~s. 

%A-t-il eu une contravention ?
M. Durand a parcouru les 3,2~km en 13~h~48~min~41~s moins 13 h~46~min~54~s, soit 1~min 47~s, soit 107~s.

Il a donc roulé en moyenne à la vitesse de :

$\dfrac{3,2}{107}~\text{km/s}$, soit $\dfrac{3,2}{107} \times \np{3600}~\text{km/h} \approx 107,664$~(km/h).

La vitesse étant supérieure à 100, on enlève 5\,\% à cette vitesse et la vitesse retenue est égale à :

$107,664 \times \dfrac{95}{100} \approx 102,28$~(km/h). Il y aura contravention.

\emph{Remarque} : M. Durand a roulé plus vite que M. Lagarde : il aura donc une contravention.
\end{enumerate}

\vspace{0.5cm}

