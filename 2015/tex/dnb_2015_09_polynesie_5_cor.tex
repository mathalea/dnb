
\medskip

%Laurent s'installe comme éleveur de chèvres pour produire du lait afin de fabriquer des fromages.
%
%\medskip
%
\textbf{PARTIE 1 : La production de lait}
%
%\medskip
%
%\textbf{Document 1}
%
%\textbf{Chèvre de race alpine :} 
%
%\textbf{Production de lait :} 1,8 litre de lait par jour et par chèvre en moyenne
%
%\textbf{Pâturage :} 12 chèvres maximum par hectare 
%
%\medskip
%
%\textbf{Document 2}
%
%Plan simplifié des surfaces de pâturage.
%
%\begin{center}
%\psset{unit=0.75cm}
%\begin{pspicture}(9.5,7.8)
%\pspolygon(0.2,0.5)(3.8,0.5)(3.8,4.1)(9.4,4.1)(9.4,7.5)(0.2,7.5)
%\psframe(0.2,0.5)(0.6,0.9)
%\psframe(3.8,0.5)(3.4,0.9)
%\psframe(9.4,4.1)(9,4.5)
%\psframe(9.4,7.5)(9,7.1)
%\psframe(0.2,7.5)(0.6,7.1)
%\uput[u](4.8,7.5){620 m}
%\uput[d](2,0.5){240 m}
%\psline(2,0.7)(2,0.3)
%\psline(3.6,2.3)(4,2.3)
%\psline(9.2,5.8)(9.6,5.8)
%\end{pspicture}
%\end{center}
%
%\textbf{Document 3}
%
%1 hectare = \np{10000} m$^2$

\begin{enumerate}
\item %Prouver que Laurent peut posséder au maximum 247 chèvres.
On peut partager la surface de pâturage en deux rectangles, l’un de 240~(m) sur $2 \times 240 = 480$~(m) et l’autre de 240~(m) sur $620 - 240 = 380$~(m).

L’aire totale est égale à $240 \times 480 + 380 \times 240 = \np{206400}$~m$^2$, soit 20,64 ha ; donc on peut y faire paître au maximum :

$20,64 \times 12 = 247,68$, soit un maximum de 247 chèvres.

\emph{Remarque} : Autre méthode : on peut décomposer la surface du pâturage en un rectangle de longueur $620$~m et de largeur $240$~m et un carré de côté $240$~m.

Aire totale : $620 \times 240 + 240^2 = \np{206400}$~m$^2$.
\item %Dans ces conditions, combien de litres de lait peut-il espérer produire par jour en moyenne ?
Les 247 chèvres donneront en moyenne par jour :

$247 \times 1,8 = 444,6$~litres de lait.
\end{enumerate}

\bigskip
 
\textbf{PARTIE 2 : Le stockage du lait}
 
\medskip

%Laurent veut acheter une cuve cylindrique pour stocker le lait de ses
%chèvres.
%
%Il a le choix entre 2 modèles :
%
%\setlength\parindent{6mm}
%\begin{itemize}
%\item[$\bullet~~$] cuve A : contenance 585 litres
%\item[$\bullet~~$] cuve B : diamètre 100 cm, hauteur 76 cm
%\end{itemize}
%\setlength\parindent{0mm} 
% 
%Formule du volume du cylindre : $V = \pi \times  r^2 \times h$
% 
%Conversion : 1 dm$^3$ = 1 L
% 
% \medskip
 
%Il choisit la cuve ayant la plus grande contenance. Laquelle va-t-il acheter ?
Volume de la cuve B : $V_{\text{B}} = \pi \times  5^2 \times 7,6 = 190\pi \approx 596,9$~dm$^3$. 

Il va donc acheter une cuve B.
\vspace{0,5cm}

