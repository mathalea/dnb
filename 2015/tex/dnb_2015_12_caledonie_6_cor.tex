
\medskip

\begin{enumerate}
\item On considère les polygones réguliers suivants:
	\begin{enumerate}
		\item Le carré :
		
%\parbox{0.3\linewidth}{\psset{unit=1cm}
%\begin{pspicture}(-1.2,-1.2)(1.2,1.2)
%\pscircle(0,0){1.2}
%\pspolygon(1.2;-45)(1.2;45)(1.2;135)(1.2;225)
%\psline(1.2;-45)(1.2;135)
%\psline(1.2;45)(1.2;225)
%\uput[ur](1.2;45){\small B}\uput[ul](1.2;135){\small A}
%\uput[dl](1.2;225){\small D}\uput[dr](1.2;-45){\small C}\uput[r](0,0){\small O}
%\psdots[dotstyle=+,dotangle=45](0.85;0)(0.85;90)(0.85;180)(0.85;270)
%\psarc(0,0){0.25cm}{45}{135}
%\end{pspicture}} \hfill\parbox{0.65\linewidth}{Expliquer pourquoi l'angle 
%$\widehat{\text{AOB}}$ mesure 90\degres}.
Par exemple : A et C sont équidistants de B et de D, donc la droite (AC) est la médiatrice de [BD] : donc $\widehat{\text{AOB}}= 90$\degres.
		\item Le pentagone régulier:

%\medskip
%
%\parbox{0.3\linewidth}{\psset{unit=1cm}
%\begin{pspicture}(-1.2,-1.2)(1.2,1.2)
%\pscircle(0,0){1.2}
%\pspolygon(1.2;15)(1.2;87)(1.2;159)(1.2;231)(1.2;303)
%\psline(1.2;15)(0;0)(1.2;87)
%\psline(1.2;159)(0;0)(1.2;231)
%\psline(1.2;231)(0;0)(1.2;303)
%\uput[ur](1.2;15){\small B}\uput[ul](1.2;87){\small A}
%\uput[ul](1.2;159){\small E}\uput[dl](1.2;231){\small D}\uput[dr](1.2;303){\small C} \uput[ur](0,0){\small O}
%\psdots[dotstyle=+,dotangle=45](0.98;51)(0.98;123)(0.98;195)(0.98;267)(0.98;339)
%\pscircle(0,0){0.25cm}
%\multido{\n=51+72}{5}{\psline(0.22;\n)(0.28;\n)}
%\end{pspicture}} 
%\hfill\parbox{0.65\linewidth}{Expliquer pourquoi l'angle $\widehat{\text{AOB}}$ mesure 72\degres.}
%
%\medskip
Les cinq triangles isocèles AOB, BOC, COD, EOF et FOA ont les mêmes dimensions donc les cinq angles au centre ont la même mesure : $\dfrac{360}{5} = 72$~\degres.
		\item L'hexagone régulier :

%\medskip
%\parbox{0.3\linewidth}{\psset{unit=1cm}
%\begin{pspicture}(-1.2,-1.2)(1.2,1.2)
%\pscircle(0,0){1.2}
%\pspolygon(1.2;0)(1.2;60)(1.2;120)(1.2;180)(1.2;240)(1.2;300)
%\multido{\n=0+60}{6}{\psline(0.;0)(1.2;\n)}
%\uput[ur](1.2;0){\small C}\uput[ur](1.2;60){\small B}
%\uput[ul](1.2;120){\small A}\uput[l](1.2;180){\small F}\uput[dl](1.2;240){\small E} \uput[dr](1.2;300){\small D}\uput[ur](0,0){\small O}
%\psdots[dotstyle=+,dotangle=45](1.04;30)(1.04;90)(1.04;150)(1.04;210)(1.04;270)(1.04;330)
%\psarc(0,0){0.25cm}{60}{120}
%\end{pspicture}} 
%\hfill\parbox{0.65\linewidth}{ Calculer la mesure de l'angle $\widehat{\text{AOB}}$.}
%
%\medskip
Comme précédemment chaque angle au centre mesure $\dfrac{360}{6} = 60$~\degres.
	\end{enumerate}
\item ~
%Un polygone régulier a des côtés de longueur 5~cm. Les angles à chaque sommet mesurent 140\degres.
	
%Calculer le périmètre de ce polygone.
\begin{center}
\psset{unit=1cm}
\begin{pspicture}(-3,-3)(3,3)
\pscircle(0,0){3}
\pspolygon(0;0)(3;15)(3;55)
\psline(0;0)(2.819;35)
\rput{-145}(2.819;35){\psframe(0.3,0.3)}
\uput[dl](0;0){O}\uput[dr](3;15){A}\uput[ur](2.819;35){H}
\uput[ul](3;12){70\degres}
\end{pspicture}
\end{center}	
%\emph{Dans cette question, toute trace de recherche, même incomplète ou non fructueuse, sera prise en compte dans l'évaluation.}
On a $\widehat{\text{AOH}} = 90 - 70 = 20$\degres.

Chaque angle au centre mesure donc $2 \times 20 = 40$\degres.

Il y a donc dans ce polygone : $\dfrac{360}{40} = 9$~côtés de 5 cm. Son périmètre est donc de $9 \times 5 = 45$~cm.
\end{enumerate}

\vspace{0,5cm}

