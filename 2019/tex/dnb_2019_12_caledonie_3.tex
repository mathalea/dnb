
\medskip

\parbox{0.62\linewidth}{Le quadrilatère EFGH est un agrandissement de ABCD.

Le schéma ci-contre n'est pas à l'échelle. 

On donne AC $= 80$ cm et GE $= 1$ m

\medskip

\begin{enumerate}
\item Montrer que le coefficient d'agrandissement est 1,25.
\item Calculer GH et EF.
\item On considère que l'aire du quadrilatère ABCD est égale
à \np{1950}~cm$^2$. Calculer l'aire de EFGH en cm$^2$. \emph{Arrondir à l'unité}.
\end{enumerate}}\hfill
\parbox{0.36\linewidth}{\psset{unit=0.8cm}
\begin{pspicture}(7,6.1)
%\psgrid
\pspolygon(0.4,3.2)(1.6,0.6)(2.9,3.2)(1.6,4.9)%DCBA
\uput[l](0.4,3.2){D} \uput[d](1.6,0.6){C} \uput[r](2.9,3.2){B} \uput[u](1.6,4.9){A}
\psline(1.6,4.9)(1.6,0.6)
\psline(1,1.7)(1.15,1.85)\psline(2.2,2)(2.35,1.8)
\psline(0.9,4)(1.1,3.8)\psline(0.9,4.1)(1.1,3.9)\psline(2.3,3.8)(2.5,3.9)\psline(2.3,3.9)(2.5,4)
\pspolygon(3.6,4)(5.3,0.4)(6.8,4)(5.3,5.9)%HGFE
\psline(5.3,5.9)(5.3,0.4)
\rput{90}(1.4,3.2){\small 80 cm}
\uput[l](3.6,4){H} \uput[d](5.3,0.4){G} \uput[r](6.8,4){F} \uput[u](5.3,5.9){E} 
\rput{53}(0.7,4.3){\small 35 cm}\rput{90}(5,3.3){\small 1 m}\rput{-63}(0.8,1.8){\small 60~cm}
\end{pspicture}}

\vspace{0,5cm}

