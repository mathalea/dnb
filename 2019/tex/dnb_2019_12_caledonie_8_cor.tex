
\medskip

%Dans les figures de cet exercice la flèche indique la position et l'orientation du lutin au départ.
%
%\medskip
%
\begin{enumerate}
\item La figure obtenue a six côtés : c'est le dessin \no 1 qui est obtenu.
%Indiquer sur la copie le numéro du dessin correspondant au script ci-dessous.
%
%\begin{center}
%\begin{tabularx}{\linewidth}{l *{3}{X}}
%\begin{pspicture}(3,3.5)
%\rput(1.25,1){{\small \begin{scratch}
%\blockinit{Quand \greenflag est cliqué}
%\blockpen{stylo en position d'écriture}
%	\blockrepeat{répéter \ovalnum{6} fois}
%		{
%		\blockmove{avancer de \ovalnum{50} pas}
%		\blockmove{tourner \turnright{} de \ovalnum{60} degrés}
%		}
%\end{scratch}}}
%\end{pspicture}&\psset{unit=1cm}
%\begin{pspicture}(-1.2,-1.2)(1.2,1.5)
%\pspolygon(1.1;0)(1.1;60)(1.1;120)(1.1;180)(1.1;240)(1.1;300)
%\psline[linewidth=5pt]{->}(-1.2,0.95)(0,0.95)
%\rput(0,1.5){Dessin \no 1}
%\end{pspicture}&\psset{unit=1cm}
%\begin{pspicture}(-1.2,-1.2)(1.2,1.5)
%\pspolygon(1.1;22.5)(1.1;67.5)(1.1;112.5)(1.1;157.5)(1.1;202.5)(1.1;247.5)(1.1;292.5)(1.1;337.5)
%\psline[linewidth=5pt]{->}(-1.2,1)(0,1)
%\rput(0,1.5){Dessin \no 2}
%\end{pspicture}&\psset{unit=1cm}
%\begin{pspicture}(-1.2,-1.2)(1.2,1.5)
%\pspolygon(-0.9,-0.9)(0.9,-0.9)(0.9,0.9)(-0.9,0.9)
%\psline[linewidth=5pt]{->}(-1.2,0.9)(0,0.9)	
%\rput(0,1.5){Dessin \no 3}
%\end{pspicture}
%\end{tabularx}
%\end{center}
%
%\vspace{1.5cm}
\item  Voir l'annexe.
%Sur l'\textbf{annexe 2}, compléter les deux informations manquantes du script qui permet de réaliser la figure ci-dessous

%\begin{center}
%\psset{unit=1cm}
%\begin{pspicture}(-1.2,-1.2)(1.2,1.5)
%\pspolygon(1.1;-90)(1.1;30)(1.1;150)
%\psline[linewidth=5pt]{->}(-1.2,0.55)(0,0.55)	
%\end{pspicture}
%\end{center}
\item  %En ordonnant les instructions proposées en \textbf{annexe 2}, compléter le script permettant de réaliser la figure ci-dessous. On indiquera les numéros des instructions sur l'annexe.
Voir l'annexe.
%\begin{center}
%\psset{unit=0.9cm}
%\begin{pspicture}(-3.5,-3.5)(3.5,3.5)
%\psline(0,0)(0.2,0)(0.4,-0.346)(0.05,-1)(-0.7,-1)(-1.3,0)(-0.7,1.1)(0.9,1.1)(1.7,-0.346)(0.8,-2)(-1.3,-2)(-2.5,0)(-1.3,2.2)(1.5,2.2)(3,-0.346)(1.5,-3)(-1.8,-3)(-3.7,0)(-2.05,3.2)
%%\multido{\n=0.0+0.2,\nc=0.2+\n,\na=0+-60}{18}{\rput{\na}(\n,0){\psline(\n,\nc)}}
%\psline[linewidth=5pt]{->}(-1.2,0)(0,0)	
%\end{pspicture}
%\end{center}
\end{enumerate}
\begin{center}
\textbf{ANNEXES À RENDRE AVEC LA COPIE}

\textbf{Annexe 2} 

\bigskip

\begin{tabularx}{\linewidth}{X|XXX}
\textbf{Question 2} &	&\textbf{Question 3}	&\\
					&\multicolumn{3}{l}{Pour ce script on a créé la variable 
{\scriptsize \ovalvariable{longueur}}}\\
					&	&\multicolumn{2}{r}{Compléter en \pnode{A}mettant les numéros à leur place}\\
{\scriptsize 
\begin{scratch}
\blockinit{Quand \greenflag est cliqué}
\blockpen{stylo en position d'écriture}
	\blockrepeat{répéter \ovalnum{3} fois}
		{
		\blockmove{avancer de \ovalnum{50} pas}
		\blockmove{tourner \turnright{} de \ovalnum{120} degrés}
		}
\end{scratch}}	&\parbox{0.22\linewidth}
{\scriptsize \no 1 

\begin{scratch}
	\blockrepeat{répéter \ovalnum{18} fois}
		\blockspace{0.3}
\end{scratch}

\no 2

\begin{scratch}
\blockmove{tourner \turnright{} de \ovalnum{60} degrés}
\end{scratch}

\no 5

\begin{scratch}
\blockmove{ajouter \ovalnum{10} à longueur}
\end{scratch}}&

{\scriptsize \no 3

 \begin{scratch}
\blockinit{Quand \greenflag est cliqué}
\end{scratch}

\no 4

\begin{scratch}
\blockmove{avancer de \ovalvariable{longueur} pas}
\end{scratch}

\no 7

\begin{scratch}
\blockvariable{mettre longueur à \ovalnum{10}}
\end{scratch}

\no 6

\begin{scratch}
\blockpen{stylo en position d'écriture}
\end{scratch}}

&
{\begin{scratch}[print,fill blocks,fill gray=0.76]
	\blockinit{\no 3}
	\blockpen{\no 7}
	\blockpen{\no 6}
	\blockpen{\no 1}
	\blockpen{\no 4}
	\blockpen{\no 2}
	\blockpen{\no 5}
\end{scratch}}
\end{tabularx}
\end{center}

