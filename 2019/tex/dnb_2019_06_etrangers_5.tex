
\medskip

\emph{Dans l'exercice suivant, les figures ne sont pas à l'échelle.}

\medskip

\begin{tabular}{m{5cm}m{8cm}}
\psset{unit=1cm}
\def\etage{\pspolygon(0,0)(2.8,0)(0,1.9)(0,0)(1.9,1.9)(0,1.9)}
\def\plateau{\psframe[fillstyle=solid,fillcolor=lightgray](0,0)(3.4,0.2)}
\begin{pspicture}(0,-2)(4,8.5)
%\psgrid
\multido{\n=0.20+2.05,\na=0.00+2.05}{4}{\rput(0.4,\n){\etage}\rput(0,\na){\plateau}}
\rput(0,8.25){\plateau}
\rput(2,-0.25){Plateau en bois}
\rput(2,-0.6){d'épaisseur 2 cm}
\rput(3,1){Étage}
\rput(3,3){Étage}
\uput[d](2,-1){Figure 1}
\end{pspicture}
&Un décorateur a dessiné une vue de côté d'un meuble de rangement
composé d'une structure métallique et de plateaux en bois d'épaisseur 2
cm, illustré par la figure 1.

Les étages de la structure métallique de ce meuble de rangement sont
tous identiques et la figure 2 représente l'un d'entre eux.

\psset{unit=2.25cm}
\hspace{0.25cm}\begin{pspicture}(3,2)
\pspolygon(0,0)(2.8,0)(0,1.8)(0,0)(1.9,1.8)(0,1.8)%CDACBA
\uput[ul](0,1.8){A} \uput[ur](1.9,1.8){B} \uput[dl](0,0){C} \uput[dr](2.8,0){D} \uput[d](1.15,1.1){O}
\uput[d](1.5,0){Figure 2} 
\end{pspicture}\\
\end{tabular}

On donne :

$\bullet~~$ OC = 48 cm ; OD = 64 cm ; OB = 27 cm ; OA = 36 cm et CD = 80 cm ;

$\bullet~~$ les droites (AC) et (CD) sont perpendiculaires.

\medskip

\begin{enumerate}
\item Démontrer que les droites (AB) et (CD) sont parallèles.
\item Montrer par le calcul que AB $= 45$ cm.
\item Calculer la hauteur totale du meuble de rangement.
\end{enumerate}

\vspace{0,5cm}

