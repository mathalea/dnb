
\medskip

%\parbox{0.5\linewidth}{\psset{unit=0.8cm}
%\begin{pspicture}(9,10)%\psgrid
%\pspolygon(0.5,9)(0.5,0.5)(9,0.5)(9,6.7)(5.3,0.5)%RTPUS
%\uput[ul](0.5,9){R} \uput[d](0.5,0.5){T} \uput[d](9,0.5){P} \uput[ur](9,6.7){U} \uput[d](5.3,0.5){S}
%\psline(2.6,5.2)(8.1,5.2)%KL
%\uput[ur](2.6,5.2){K}\uput[ul](8.1,5.2){L} 
%\psarc(2.6,5.2){6mm}{-60}{0}\rput(3.6,4.6){60\degres}
%\psarc(9,6.7){4mm}{-120}{-90}\rput(8.69,5.6){30\degres}
%\psframe(0.5,0.5)(0.9,0.9)\psline[linestyle=dashed]{<->}(0.5,0.4)(5.3,0.4)\uput[d](2.9,0.4){14 cm}
%\psframe(8.6,0.9)(9,0.5)\psline[linestyle=dashed]{<->}(5.3,0.4)(9,0.4)\uput[d](7.15,0.4){10,5 cm}
%\psline[linestyle=dashed]{<->}(0.4,9)(5.2,0.5)\rput{-60}(2.6,4.75){28 cm}
%\end{pspicture}
%}\hfill
%\parbox{0.46\linewidth}{\begin{tabularx}{\linewidth}{|X|}\hline
%\textbf{Données }:\\ 
%TSR et SPU sont des triangles rectangles respectivement en T et en P.\\ 
%TS = 14 cm\\
%SP = 10,5 cm\\
%RS = 28 cm \\
%$\widehat{\text{SKL}} = 60$\degres ; $\widehat{\text{SUP}} = 30$\degres \\
%Les points T, S et P sont alignés\\
%Les points R, K et S sont alignés\\
%Les points S, L et U sont alignés\\ \hline
%\end{tabularx}}

\medskip

\begin{enumerate}
\item %Montrer que la mesure de l'angle $\widehat{\text{TSR}}$ est $60$\degres.
Dnas le triangle RST rectangle en T, on a $\cos \widehat{\text{RST}} = \dfrac{\text{ST}}{\text{SR}} = \dfrac{14}{28} = \dfrac{1}{2} = 0,5$. On a donc $\widehat{\text{RST}} = 60$\degres.

\smallskip

\emph{Rem.} STR est un demi-triangle équilatéral obtenu en prenant le symétrique S par rapport à T et les angles d'un triangle équilatéral mesurent \ldots 
\item %Démontrer que les triangles SRT et SUP sont semblables
Le triangle SUP est aussi un demi triangle équilatéral puisque par complément $\widehat{\text{PSU}} = 90 - 30 = 60$\degres, donc SU $ = 2 \text{SP} = 2 \times 10,5 = 21.$

Or $\dfrac{\text{SP}}{\text{ST}} = \dfrac{10,5}{14} = \dfrac{105}{140} = \dfrac{5 \times 21}{5 \times 28} = \dfrac{5 \times 7 \times 3}{5 \times 7 \times 4}
 = \dfrac{3}{4}$ et 
 
$\dfrac{\text{SU}}{\text{SR}} = \dfrac{21}{28} = \dfrac{10,5}{14} 
 = \dfrac{3}{4}$ (d'après le calcul précédent).
 
 Les côtés des triangles rectangles SRT et SUP sont donc proportionnels. 
 \item %Déterminer le coefficient de réduction liant les triangles SRT et SUP.
 On a vu que le triangle SUP est une réduction du triangle SRT de coefficient $\dfrac{3}{4} = 0,75$.
\item %Calculer la longueur SU.
On a déjà vu que SU $ =  21$.
\item %Quelle est la nature du triangle SKL ? A justifier.
On a vu que  $\widehat{\text{PSU}} = \widehat{\text{TSR}} = 60$ donc par supplément :

$\widehat{\text{RSU}} = 180 - 60 - 60 = 60$\degres.

Le triangle SKL a deux angles de 60\degres ; le troisième angle a pour mesure : $180 - 60 - 60 = 60$ : le triangle SKL a donc trois angles de même mesure c'est donc un triangle équilatéral.
\end{enumerate}

\vspace{0.5cm}

