
\medskip

Une famille a effectué une randonnée en montagne. Le graphique ci-dessous donne la distance
parcourue en km en fonction du temps en heures.

\begin{center}
\psset{xunit=1.5cm,yunit=0.3cm}
\begin{pspicture*}(-0.5,-4)(7.4,27)
\uput[r](0,26){Distance en km}
\uput[d](6.5,-2){Temps en heures}
\psgrid[gridlabels=0,subgriddiv=5](0,0)(8,25)
\psaxes[linewidth=1pt,Dy=5](0,0)(0,0)(8,25)
\psline[linewidth=1.25pt](0,0)(1,4)(2,7)(3,8)(4,15)(5,15)(6,18)(7,20)
\end{pspicture*}
\end{center}
\medskip

\begin{enumerate}
\item Ce graphique traduit-il une situation de proportionnalité ? Justifier la réponse.
\item On utilisera le graphique pour répondre aux questions suivantes. Aucune justification n'est
demandée.
	\begin{enumerate}
		\item Quelle est la durée totale de cette randonnée?
		\item Quelle distance cette famille a-t-elle parcourue au total?
		\item Quelle est la distance parcourue au bout de $6$~h de marche?
		\item Au bout de combien de temps ont-ils parcouru les $8$ premiers km ?
		\item Que s'est-il passé entre la $4$\up{e} et la $5$\up{e} heure de randonnée?
	\end{enumerate}
\item  Un randonneur expérimenté marche à une vitesse moyenne de $4$~km/h sur toute la randonnée.
Cette famille est-elle expérimentée? Justifier la réponse.
\end{enumerate}

\vspace{0,5cm}

