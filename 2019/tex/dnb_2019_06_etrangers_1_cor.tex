
\medskip

\begin{enumerate}
\item $28= 4\times 7 = 2^2 \times 7$ : Réponse C 

A et B contiennent des facteurs non premiers.

\item Le nouveau prix est égal à :
$58 \times \left(1 - \dfrac{20}{100}\right)
= 58 \times \dfrac{80}{100} = 58 \times 0,8 = 46,4$, soit 46,40~\euro : 
Réponse B
\item Dans le triangle ABC rectangle en  A, on a 
$\tan 15 = \dfrac{\text{AC}}{\text{AB}} = \dfrac{\text{AC}}{25}$, d'où en multipliant chaque membre par 25 :

$\text{AC} = 25 \times \tan 15 \approx 6,698$ : réponse la plus proche 
\item Rangés dans l'ordre croissant les termes de la série sont : 2 ; 3 ; 5 ; 6 ; 8 ; 12.

Il y a 6 termes, donc la médiane est tout nombre compris entre le  3\up{e} et le 4\up{e} terme, donc en particulier la moyenne des deux nombres soit 5,5 : réponse A.
\item Les dimensions du carré B sont deux fois plus petites que celles du carré A : le rapport d'homothétie est donc égal à $+ 0,5$ ou $- 0,5$.

Avec A comme centre d'homothétie le rapport est égal à $- 0,5$ ; réponse a.

Avec B comme centre d'homothétie le rapport est égal à $0,5$ : réponse b. 

\psset{unit=0.8cm}
\begin{center}
\begin{pspicture}(-2,-2)(3,3.35)
%\psgrid
\psframe(1,1)(3,3)\psframe(0,0)(1,1)
\rput(2,2){\small carré A}\rput(0.5,0.5){\scriptsize carré B}
\psline[linestyle=dotted,dotsep=1.2pt](3,1)(-1,-1)(1,3)(3,3)(-1,-1)
\uput[ul](1,1){A} \uput[dl](-1,-1){B}
\end{pspicture} 
\end{center}
\end{enumerate}
\vspace{0,25cm}

