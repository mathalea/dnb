
%\parbox[t]{8cm}{\medskip
%	\emph{Dans cet exercice, aucune justification n'est attendue}
%	
%\bigskip
%	
%On considère l'hexagone ABCDEF de centre O représenté ci-contre.}
%\hfill 
%\begin{tikzpicture}[x=25mm,y=25mm,baseline={(A)},,line width=1pt]
%\newcommand{\point}[3]{\draw[shift={#1}] (-3pt,-3pt)--(3pt,3pt) (-3pt,3pt)--(3pt,-3pt) (0,0) node[shift={#2}]{#3};}
%	
%\coordinate (A) at (120:1); \point{(A)}{(120:3mm)}{A};
%\coordinate (B) at ( 60:1); \point{(B)}{( 60:3mm)}{B};
%\coordinate (C) at (  0:1); \point{(C)}{(  0:3mm)}{C};
%\coordinate (D) at (300:1); \point{(D)}{(300:3mm)}{D};
%\coordinate (E) at (240:1); \point{(E)}{(240:3mm)}{E};
%\coordinate (F) at (180:1); \point{(F)}{(180:3mm)}{F};
%\coordinate (O) at (0 , 0); \point{(O)}{(270:3mm)}{O};
%
%\draw (A)--(B) node[pos=0.5,sloped]{||}
%	--(C) node[pos=0.5,sloped]{||}
%	--(D) node[pos=0.5,sloped]{||}
%	--(E) node[pos=0.5,sloped]{||}
%	--(F) node[pos=0.5,sloped]{||}
%	--cycle node[pos=0.5,sloped]{||}
%	(A)--(O) node[pos=0.5,sloped]{||}
%	(B)--(O) node[pos=0.5,sloped]{||}
%	(C)--(O) node[pos=0.5,sloped]{||}
%	(D)--(O) node[pos=0.5,sloped]{||}
%	(E)--(O) node[pos=0.5,sloped]{||}
%	(F)--(O) node[pos=0.5,sloped]{||};
%	\end{tikzpicture}

\begin{enumerate}
	\item  %Parmi les propositions suivantes, recopier celle qui correspond à l'image du quadrilatère CDEO par la symétrie de centre O.
	
%\renewcommand{\arraystretch}{1.5}
%\begin{tabularx}{\linewidth}{|*{3}{>{\centering \arraybackslash} X|}} \hline
%		\textbf{Proposition 1}&\textbf{Proposition 2}&\textbf{Proposition 3}\\ \hline
%		FABO & ABCO & FODE\\ \hline
FABO.
%\end{tabularx}
		
		\item %Quelle est l'image du segment [AO] par la symétrie d'axe (CF) ?
		Le segment [EO].
		\item  %On considère la rotation de centre O qui transforme le triangle OAB en le triangle OCD.
		
%Quelle est l'image du triangle BOC par cette rotation ?
La rotation est d'angle 120~\degres{} dans le sens horaire.

L'image du triangle BOC par cette rotation est le triangle DOE.
\item C'est l'hexagone 19.
	\end{enumerate}
%\parbox{8cm}{La figure ci-contre représente un pavage dont le motif de base a la même forme que l'hexagone ci-dessus.
%On a numéroté certains de ces hexagones.
%	\begin{enumerate}[start=4]
%		\item Quelle est l'image de l'hexagone 14 par la translation qui transforme l'hexagone 2 en l'hexagone 12 ?
%\end{enumerate}}\hfill 
%\begin{tikzpicture}[x=0.6cm,y=0.6cm,line width=1pt,baseline={(current bounding box.center)}]
%%\draw[teal,xstep=1,ystep=1] (0,0) grid (10,10);
%\clip[draw] (0.3,0.4) rectangle (10.5,8);
%\foreach \x  in {0,...,3}{
%\foreach \y in {0,...,5}
%\draw[shift={({3*\x},{1.732*\y})}] (0:1)--(60:1)--(120:1)--(180:1)--(240:1)--(300:1)--cycle (0,0);}
%\foreach \x  in {0,...,3}{
%\foreach \y in {0,...,5}
%\draw[shift={({1.5+3*\x},{0.866+1.732*\y})}] (0:1)--(60:1)--(120:1)--(180:1)--(240:1)--(300:1)--cycle;}
%\node at(3,6.928) {1}; \node at(6,6.928) {2}; \node at(9,6.928) {3};
%\node at(3,5.196) {5}; \node at(6,5.196) {7}; \node at(9,5.196) {9};
%\node at(3,3.464) {11}; \node at(6,3.464) {13}; \node at(9,3.464) {15};
%\node at(3,1.732) {17}; \node at(6,1.732) {19}; \node at(9,1.732) {21};
%\node at(1.5,6.062){4}; \node at(4.5,6.062){6}; \node at(7.5,6.062){8};
%\node at(1.5,4.33){10}; \node at(4.5,4.33){12}; \node at(7.5,4.33){14};
%\node at(1.5,2.598){16}; \node at(4.5,2.598){18}; \node at(7.5,2.598){20};
%\node at(1.5,0.866){22}; \node at(4.5,0.866) {23}; \node at(7.5,0.866) {24};
%\end{tikzpicture}
	
\bigskip

