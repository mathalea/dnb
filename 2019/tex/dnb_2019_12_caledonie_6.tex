
\medskip

On veut peindre des murs d'aire inférieure à 100 m$^2$.

Voici les tarifs proposés par trois peintres en fonction de l'aire des murs à peindre en m$^2$ :

\begin{tabular}{|l l|}\hline
\textbf{Peintre A :}& \np{1500} F par m$^2$\\
\textbf{Peintre B :}& \np{1000} F par m$^2$ et \np{10000}~F d'installation de chantier\\ 
\textbf{Peintre C :}& \np{70000} F quelle que soit l'aire inférieure à 100 m$^2$\\ \hline
\end{tabular}

\medskip

\begin{enumerate}
\item Montrer que pour 40 m$^2$, le tarif du peintre A est de \np{60000}~F{}, le tarif du peintre B est de \np{50000}~F et le tarif du peintre C est de \np{70000}~F{}.
\end{enumerate}

Dans la suite de l'exercice, $x$ désigne l'aire des murs à peindre en m$^2$. 

\begin{enumerate}[resume]
\item Écrire, en fonction de $x$, le prix proposé par le peintre B.
\end{enumerate}

Les fonctions donnant les prix proposés par le peintre B et le peintre C sont représentées sur l'\textbf{annexe 1}.

\begin{enumerate}[resume]
\item Soient $A(x)$ et $C(x)$ les expressions des fonctions donnant le prix proposé par les peintres A et C en fonction de $x$.

On a $A(x) = \np{1500}x$ et $C(x)= \np{70000}$.
	\begin{enumerate}
		\item Quelle est la nature de la fonction $A$ ?
		\item Calculer l'image de $60$ par la fonction $A$.
		\item Calculer l'antécédent de \np{30000} par la fonction $A$.
		\item Tracer la représentation graphique de la fonction $A$ sur l'\textbf{annexe 1}.
	\end{enumerate}
\item 
	\begin{enumerate}
		\item Résoudre l'équation $\np{1500} x = \np{1000} x + \np{10000}$. 
		\item Interpréter le résultat de la question \textbf{4. a.}
	\end{enumerate}
\item Lire graphiquement, sur l'\textbf{annexe 1}, les surfaces entre lesquelles le peintre B est le moins cher des trois peintres.
\end{enumerate}

\vspace{0,5cm}

\begin{center}
\textbf{ANNEXES À RENDRE AVEC LA COPIE}

\bigskip

\textbf{Annexe 1}
 
\medskip
 
\psset{xunit=0.145cm,yunit=0.0001cm}
\begin{pspicture}(-5,-8000)(90,100000)
\multido{\n=0+1}{91}{\psline[linewidth=0.2pt](\n,0)(\n,100000)} 
\multido{\n=0+5}{19}{\psline[linewidth=0.7pt](\n,0)(\n,100000)} 
\multido{\n=0+2000}{51}{\psline[linewidth=0.2pt](0,\n)(90,\n)}
\multido{\n=0+10000}{11}{\psline[linewidth=0.7pt](0,\n)(90,\n)}
\psaxes[labels=x,linewidth=1.25pt,Dx=5,Dy=10000]{->}(0,0)(0,0)(95,100000)
\uput[u](84,70000){Peintre C}
\rput{36}(80,94000){Peintre B}
\uput[d](82,-4000){surface en m$^2$}
\uput[r](0,98000){tarif en F}
\psline[linewidth=1.5pt](0,10000)(90,100000)
\psline[linewidth=1.5pt](0,70000)(90,70000)
 \end{pspicture}
\end{center}

