
\medskip

Un silo à grains permet de stocker des céréales. Un ascenseur permet d'acheminer le blé dans le silo. L'ascenseur est soutenu par un pilier.

\begin{center}
\psset{unit=0.9cm}
\begin{pspicture}(0,-1.5)(16,5)
%\psgrid
\psline(11.8,0)
\pscircle(1.8,0.3){0.1}\pscircle(3.6,0.3){0.1}\pscircle(4.3,0.3){0.1}
\pscircle(1.8,0.3){0.3}\pscircle(3.6,0.3){0.3}\pscircle(4.3,0.3){0.3}
\psline(2.5,0.6)(4.6,0.6)
\rput(8,4){Ascenseur à blé}
\rput(4.8,-1){Déversoir}
\psframe(1.5,0.6)(2.6,1.7)
\psframe(1.6,1)(2.3,1.5)
\rput{60}(4.6,0.6){\psframe(0,0)(1,2.1)}
\psline(5.1,0)(4.8,0.6)(5.8,0.6)(5.6,0)\psline[linewidth=2.5pt](3,0.6)(3.4,1.3)
\scalebox{.99}[0.3]{\psarc[linewidth=1.5pt](13.22,0){1.3}{180}{0}}%
\psellipse(13.1,4)(1.3,0.65)
\psline(11.8,0)(11.8,4)\psline(14.4,0)(14.4,4)
\psline(5.75,0.4)(12,4)\psline(5.65,0.2)(12.1,3.875)
\pscurve(12,4)(12.05,3.95)(12.1,3.875)
\pscurve(12.1,3.875)(12.6,3.8)(12.9,3.36)
\pscurve(12,4)(12.6,4)(13.12,3.36)
\pscurve(4.8,0.9)(5,0.8)(5.2,0.6)
\pscurve(4.9,1.1)(5.15,0.95)(5.4,0.6)
\psline{->}(4.8,-0.8)(5.3,0.3)\psline{->}(8,3.7)(9,2.3)
\psline(8.9,0)(8.9,2.07)\psline(9.1,0)(9.1,2.17)
\rput(9,-1){Pilier de soutien}\psline{->}(8.1,-0.8)(8.9,1.4)
\rput(13.2,-1){Silo}\psline{->}(13.2,-0.8)(13.2,1)
\psline(14.4,0)(15.4,0)
\rput(4.5,4){Camion à benne}\psline{->}(4.5,3.8)(4.2,2)
\end{pspicture}
\end{center}

On modélise l'installation par la figure ci-dessous qui n'est pas réalisée à l'échelle :

\parbox{0.46\linewidth}{\begin{itemize}[label=$\bullet~~$]
\item Les points C, E et M sont alignés.
\item Les points C, F{}, H et P sont alignés.
\item Les droites (EF) et (MH) sont perpendiculaires à la droite (CH).
\item CH $=8,50$ m et CF $= 2,50$ m.
\item Hauteur du cylindre: HM $= 20,40$ m.
\item Diamètre du cylindre: HP $= 4,20$ m.
\end{itemize}
}\hfill
\parbox{0.54\linewidth}{\psset{unit=1cm}
\begin{pspicture}(8,5.5)
%\psgrid
\psline(8,0)\psline(0.7,0)(5.5,4.6)%CM
\psline(2.6,0)(2.6,1.8)%FE
\psframe[fillstyle=solid,fillcolor=lightgray](5.5,0)(7.2,4.6)%HPM..
\rput(6.4,2.1){Cylindre}
\uput[d](0.7,0){C} \uput[d](2.6,0){F} \uput[d](5.5,0){H} 
\uput[d](7.2,0){P} \uput[ul](2.6,1.8){E} \uput[ul](5.5,4.6){M}
\psframe(2.6,0)(2.85,0.25)\psframe(5.5,0)(5.75,0.25)
\end{pspicture}
}

\vspace{0,7cm}

\textbf{Les quatre questions suivantes sont indépendantes.}

\medskip

\begin{enumerate}
\item Quelle est la longueur CM de l'ascenseur à blé ?
\item Quelle est la hauteur EF du pilier ?
\item Quelle est la mesure de l'angle $\widehat{\text{HCM}}$ entre le sol et l'ascenseur à blé ? On donnera une valeur approchée au degré près.
\item Un mètre-cube de blé pèse environ $800$~kg.

Quelle masse maximale de blé peut-on stocker dans ce silo ? On donnera la réponse à une tonne près.
\end{enumerate}

\medskip

\begin{tabularx}{\linewidth}{|X|}\hline
Rappels :\\
\hspace{1.2cm}$\bullet~~$1 tonne = \np{1000}~kg\\
\hspace{1.2cm}$\bullet~~$ volume d'un cylindre de rayon $R$ et de hauteur $h$ : $\pi \times R^2 \times h$\\ \hline
\end{tabularx}

\vspace{0,5cm}

