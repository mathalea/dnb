
\medskip

\emph{Dans cet exercice, on donnera, si nécessaire, une valeur approchée des résultats au centième près}.

\smallskip

Pour construire le décor d'une pièce de théâtre (\emph{Figure 1}), Joanna dispose d'une plaque rectangulaire ABCD de 4~m sur 2~m dans laquelle elle doit découper les trois triangles du décor avant de les superposer. Elle propose un découpage de la plaque (\emph{Figure 2}).

\begin{center}
\psset{unit=1cm}
\begin{pspicture}(10,4)
%\psgrid
\pspolygon[fillstyle=solid,fillcolor=lightgray](0,0.6)(1.8,0.6)(0,3.9)
\pspolygon[fillstyle=solid,fillcolor=gray](0,0.6)(1.2,0.6)(0,2.8)
\pspolygon[fillstyle=solid,fillcolor=darkgray](0,0.6)(0.8,0.6)(0,2.1)
\pspolygon[fillstyle=solid,fillcolor=lightgray](5.3,2.6)(5.3,0.6)(8.9,2.6)%ADM
\pspolygon[fillstyle=solid,fillcolor=gray](5.3,0.6)(8.05,2.15)(8.9,0.6)%DPN
\pspolygon[fillstyle=solid,fillcolor=darkgray](8.05,2.15)(8.9,2.6)(8.9,0.6)%PMN
\psline(5.3,0.6)(8.9,2.6)%DM
\psline(8.05,2.15)(8.9,0.6)%NP
\psdots(5.3,0.6)(9.6,0.6)(9.6,2.6)(5.3,2.6)(8.9,0.6)(8.9,2.6)(8.05,2.15)%DCBANMP
\uput[dl](5.3,0.6){D} \uput[dr](9.6,0.6){C} \uput[ur](9.6,2.6){B} 
\uput[ul](5.3,2.6){A} \uput[d](8.9,0.6){N} \uput[u](8.9,2.6){M}
\uput[ul](8.05,2.15){P}
\psframe(5.3,0.6)(8.9,2.6) 
\psframe[fillstyle=crosshatch](8.9,0.6)(9.6,2.6)
\rput(0.9,0.2){\emph{Figure $1$}}\rput(7.4,0.2){\emph{Figure $2$}}
\psarc(5.3,0.6){0.4}{30}{90}
\rput{-150}(8.05,2.15){\psframe(0,0)(0.2,0.2)}
\psframe(8.9,2.6)(8.7,2.4)
\end{pspicture}
\end{center}

\medskip

Le triangle ADM respecte les conditions suivantes :

%\setlength{\parindent}{20mm}
\begin{itemize}[leftmargin=20mm]
\item[$\bullet~~$]le triangle ADM est rectangle en A
\item[$\bullet~~$]AD = 2~m
\item[$\bullet~~$]$\widehat{\text{ADM}} = 60$\degres
\end{itemize}
%\setlength\parindent{0cm}

\medskip

\begin{enumerate}
\item Montrer que [AM] mesure environ $3,46$~m.
\item La partie de la plaque non utilisée est représentée en quadrillé sur la figure 2.

Calculer une valeur approchée au centième de la proportion de la plaque qui n'est pas utilisée.
\item  Pour que la superposition des triangles soit harmonieuse, Joanna veut que les trois triangles AMD, PNM et PDN soient semblables. Démontrer que c'est bien le cas.
\item Joanna aimerait que le coefficient d'agrandissement pour passer du triangle PDN au triangle AMD soit plus petit que $1,5$. Est-ce le cas ? Justifier.
\end{enumerate}

\bigskip

