
\medskip

%Pour servir ses jus de fruits, un restaurateur a le choix entre deux types de verres : un verre cylindrique A de hauteur $10$~cm et de rayon $3$~cm et un verre conique B de hauteur $10$~cm et de rayon $5,2$~cm.
%
%\medskip
%
%\parbox{0.42\linewidth}{\psset{unit=0.9cm}
%\begin{pspicture}(8,6)
%%\psgrid
%\psellipse(1.5,5)(1.1,0.2)\psline{<->}(1.5,5)(2.6,5)\uput[u](2.05,5){\small 3~cm}
%\psellipse(1.5,1)(1.1,0.2)
%\psline(0.4,1)(0.4,5)\psline(2.6,1)(2.6,5)
%\psellipse(5,5)(1.7,0.4)
%\psellipse(5,1)(1.1,0.2)
%\rput(1.5,0){Verre A}\rput(5,0){Verre B}
%\psline{<->}(3,1)(3,5)\uput[r](3,3){\small 10~cm}
%\psline{<->}(5,5)(6.7,5)\uput[u](5.85,4.85){\small 5,2~cm}
%\psline(3.3,4.95)(5,1)(6.7,4.95)
%\end{pspicture}}\hfill 
%\parbox{0.56\linewidth}{\begin{tabularx}{\linewidth}{|X|}\hline
%\textbf{Rappels :}\\
%$\bullet~~$ Volume d'un cylindre de rayon $r$ et de hauteur $h$ :
%
%\[\pi \times  r^2 \times h\]\\
%$\bullet~~$ Volume d'un cône de rayon $r$ et de hauteur $h$ :
%
%\[\dfrac{1}{3} \times  \pi \times r^2 \times h\]\\
%$\bullet~~$ 1 L = 1 dm$^3$\\ \hline
%\end{tabularx}}
%
%Le graphique situé en \textbf{ANNEXE 1.2} représente le volume de jus de fruits dans chacun des verres en fonction de la hauteur de jus de fruits qu'ils contiennent.
%
%\medskip

\begin{enumerate}
\item Répondre aux questions suivantes à l'aide du graphique en \textbf{ANNEXE 1.2} :
	\begin{enumerate}
		\item %Pour quel verre le volume et la hauteur de jus de fruits sont-ils proportionnels ? Justifier.
Le volume est proportionnel à la hauteur pour le verre cylindrique.
		\item %Pour le verre A, quel est le volume de jus de fruits si la hauteur est de $5$~cm ?
On lit approximativement $V \approx 141$~cm$^3$.
		\item %Quelle est la hauteur de jus de fruits si on en verse $50$~cm$^3$ dans le verre B ?
On lit approximativement $h \approx 5,6$~cm.
 	\end{enumerate}
\item  %Montrer, par le calcul, que les deux verres ont le même volume total à $1$ cm$^3$ près.
$V_{\text{A}} = \pi \times 3^2 \times 10 = 90\pi \approx 282,74$~cm$^3$ ;

$V_{\text{B}} = \dfrac{1}{3} \times \pi \times 5,2^2 \times 10 = \dfrac{270,4}{3}\pi \approx 283,16$~cm$^3$.

Les deux verres ont le même volume à 1 cm$^3$ près. 
\item %Calculer la hauteur du jus de fruits servi dans le verre A pour que le volume de jus soit égal à $200$~cm$^3$. Donner une valeur approchée au centimètre près.
On doit avoir $200 = \pi \times 3^2 \times h$, soit $h = \dfrac{200}{9\pi} \approx 7,07$~cm soit environ 7~cm..
\item  %Un restaurateur sert ses verres de telle sorte que la hauteur du jus de fruits dans le verre soit égale à $8$~cm.
	\begin{enumerate}
		\item %Par lecture graphique, déterminer quel type de verre le restaurateur doit choisir pour servir le plus grand nombre possible de verres avec 1 L de jus de fruits.
		Graphiquement o voit qu'avec une hauteur de 8~cm le volume de jus dans le verre B sera d'environ 140~cm$^3$, alors que dans le verre A il y aura plus de 220~cm$^3$. le restaurateur fera davantage de verres en utilisant des verres B.
		\item %Par le calcul, déterminer le nombre maximum de verres A qu'il pourra servir avec 1 L de jus de fruits.
1~L $  = 1~\text{dm}^3 = \np{1000}$~cm$^3$.

Il y aura dans les verres A pour une hauteur de 8~cm : $\pi \times 3^2 \times 8 = 72 \pi$cm$^3$.

Donc avec 1 L il pourra faire $\dfrac{\np{1000}}{72\pi} \approx 4,4$ : il pourra servir donc au plus 4 verres A.
	\end{enumerate}
\end{enumerate}
\begin{center}
\textbf{\large ANNEXE 1 - A rendre avec la copie}

\textbf{ANNEXE 1.2}

\bigskip

\psset{xunit=1cm,yunit=0.025cm}
\begin{pspicture}(-0.5,-25)(10.5,325)
\multido{\n=0.0+0.2}{53}{\psline[linewidth=0.2pt,linecolor=orange](\n,0)(\n,325)}
\multido{\n=0+1}{11}{\psline[linewidth=0.6pt,linecolor=orange](\n,0)(\n,325)}
\multido{\n=0+10}{33}{\psline[linewidth=0.2pt,linecolor=orange](0,\n)(10.5,\n)}
\multido{\n=0+10}{7}{\psline[linewidth=0.6pt,linecolor=orange](0,\n)(10.5,\n)}
\psaxes[linewidth=1.25pt,Dy=50]{->}(0,0)(0,0)(10.5,325)
\uput[r](0,310){\footnotesize Volume de jus de fruits (en cm$^3$)}
\uput[d](8.6,-18){\footnotesize hauteur de jus de fruits (en cm)}
\psplot[plotpoints=2000,linewidth=1.25pt,linecolor=blue]{0}{10}{ x 3 exp 5.2 mul 5.2 mul 3.14159 mul 300 div}
\psline[linestyle=dashed](0,0)(10,283.162)
\uput[dr](7.8,140){\blue Verre B}
\uput[ul](6,170){Verre A}
\end{pspicture}
\end{center}

