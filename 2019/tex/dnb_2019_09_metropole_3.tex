
\medskip

Les activités humaines produisent du dioxyde de carbone (CO$_2$) qui contribue au réchauffement
climatique. Le graphique suivant représente l'évolution de la concentration atmosphérique moyenne
en CO$_2$ (en ppm) en fonction du temps (en année).

\begin{center}
\psset{xunit=0.35cm,yunit=0.05cm}
\begin{pspicture}(45,190)
\psaxes[linewidth=1.25pt,Ox=1980,Dx=5,Oy=300,Dy=20](0,0)(0,0)(35,160)
\multido{\n=0+5}{8}{\psline[linestyle=dashed](\n,0)(\n,160)}
\multido{\n=0+20}{9}{\psline[linestyle=dashed](0,\n)(35,\n)}
\pscurve[linewidth=1.25pt,linecolor=blue](4,42)(7.5,50)(10,54)(12.5,58)(15,60)(20,68)(25,80)(30,88)(33,97)
\rput(17.5,188){\textbf{Concentration de CO$_2$ atmosphérique}}
\rput(17.5,172){\footnotesize (Source: Centre Mondial de Données relatives aux Gaz à Effet de Serre sous l'égide de l'OMM)}
\uput[r](35,150){\scriptsize 450 ppm}
\uput[r](35,145){\scriptsize niveau moyen}
\uput[r](35,140){\scriptsize à ne pas dépasser}
\uput[r](35,135){\scriptsize à l'horizon 2100}
\uput[r](0,165){parties par million - CO$_2$}
\end{pspicture}

\vspace{0.6cm}

1 ppm de CO$_2$ = 1 partie par million de CO$_2$ = 1 milligramme de CO$_2$ par kilogramme d'air.
\end{center}
\smallskip

\begin{enumerate}
\item Déterminer graphiquement la concentration de CO$_2$ en ppm en 1995 puis en 2005.


\item On veut modéliser l'évolution de la concentration de CO$_2$ en fonction du temps à l'aide d'une
fonction $g$ où $g(x)$ est la concentration de CO$_2$ en ppm en fonction de l'année $x$.
	\begin{enumerate}
		\item Expliquer pourquoi une fonction affine semble appropriée pour modéliser la concentration
en CO$_2$ en fonction du temps entre 1995 et 2005.

		\item  Arnold et Billy proposent chacun une expression pour la fonction $g$ :
		
Arnold propose l'expression $g(x) = 2x - \np{3630}$ ;
		
Billy propose l'expression $g(x) = 2x - \np{2000}$.
		
Quelle expression modélise le mieux l'évolution de la concentration de CO$_2$ ? Justifier.

		\item  En utilisant la fonction que vous avez choisie à la question précédente, indiquer l'année pour laquelle la valeur de $450$~ppm est atteinte.
		
 	\end{enumerate}
\item  En France, les forêts, grâce à la photosynthèse, captent environ $70$~mégatonnes de CO$_2$ par an, ce qui représente $15$\,\% des émissions nationales de carbone (année 2016).
	
Calculer une valeur approchée à une mégatonne près de la masse M du CO$_2$ émis en France en 2016.

\end{enumerate}

\bigskip

