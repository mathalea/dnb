
\medskip

%Dans une classe de Terminale, huit élèves passent un concours d'entrée dans une école d'enseignement supérieur.
%
%Pour être admis, il faut obtenir une note supérieure ou égale à 10.
%
%Une note est attribuée avec une précision d'un demi-point (par exemple : 10 ; 10,5 ; 11 ; \dots)
%On dispose des informations suivantes :
%
%\smallskip
%\begin{tabularx}{\linewidth}{|>{\centering \arraybackslash}X|} \hline
%
%\textbf{Information 1}
%
%Notes attribuées aux 8 élèves de la classe qui ont passé le concours :
%		
%10; \quad 13; \quad 15; \quad 14,5; \quad 6; \quad 7,5; \quad \tikz[baseline={(0,-0.075)}]{\fill (0,-0.075)--(0.075,0)--(0,0.075)--(-0.075,0)--cycle;}; \quad \tikz[baseline={(0,-0.075)}]{\fill (0,0) circle (0.075);}\\ \hline
%\end{tabularx}
%
%\smallskip
%\begin{tabularx}{\linewidth}{|X|X|} \hline
%	
%\multicolumn{2}{|c|}{\textbf{Information 2}} \\
%	
%\parbox{6.5cm}{La série constituée des huit notes :
%		
%		\begin{itemize}
%			\item a pour étendue 9;
%			\item a pour moyenne 11,5;
%			\item a pour médiane 12.
%	\end{itemize}}
%	& \parbox{6.5cm}{75\,\% des élèves de la classe qui ont passé le concours ont été reçus.}\\ \hline
%\end{tabularx}

\begin{enumerate}
\item %Expliquer pourquoi il est impossible que l'une des deux notes désignées par \tikz[baseline={(0,-0.075)}]{\fill (0,-0.075)--(0.075,0)--(0,0.075)--(-0.075,0)--cycle;} ou \tikz[baseline={(0,-0.075)}]{\fill (0,0) circle (0.075);} soit 16.
Si l'une des notes inconnues était 16, l'étendue serait au moins égale à $16 - 6 = 10$ ; or celle-ci est égale à 9.
Il est donc impossible que l'une des deux notes inconnues soit égale à 16.	
\item %Est-il possible que les deux notes désignées par \tikz[baseline={(0,-0.075)}]{\fill (0,-0.075)--(0.075,0)--(0,0.075)--(-0.075,0)--cycle;} et \tikz[baseline={(0,-0.075)}]{\fill (0,0) circle (0.075);} soient 12,5 et 13,5 ?
Si les deux notes inconnues sont 12,5 et 13,5, alors 

-- l'étendue est égale à $15 - 6 = 9$ ;

-- la moyenne serait égale à $\dfrac{10 + 13 + 15 + 14,5 + 6 + 7,5 + 12,5 + 13,5}{8} = \dfrac{92}{8} = 11,5$ ; 

-- il y aurait 6 élèves sur 8 ayant une note supérieure ou égale à 10, donc une proportion de $\dfrac{6}{8} = \dfrac{3}{4} = \dfrac{3 \times 25}{4 \times 25} = \dfrac{75}{100} = 75$\,\% de candidat reçus ; 

-- La liste des notes serait donc :

6 ; 7,5 ; 10 ; 12,5 ; 13 ; 13,5 ; 14,5 ; 15 la médiane serait supérieure à 12,5 : ce n'est pas possible.
\end{enumerate}
