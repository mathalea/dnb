
\medskip

%On donne le programme ci-dessous où on considère $2$ lutins. Pour chaque lutin, on a écrit un
%script correspondant à un programme de calcul différent.
%
%\begin{center}
%\begin{tabularx}{\linewidth}{|X|c|}\hline
%Lutin 1 &Numéro d'instruction\\ \hline
%\begin{scratch}
%\blockinit{Quand \greenflag est cliqu\'e}\end{scratch}&1\\
%\begin{scratch}\blocksensing{demander \txtbox{Saisir un nombre} et attendre}\end{scratch}&2\\
%\begin{scratch}\blockvariable{mettre \ovalvariable{x} \`a {\ovaloperator{\ovalvariable{réponse} + \ovalnum{5}}}}\end{scratch}&3\\
%\begin{scratch}\blockvariable{mettre \ovalvariable{x} \`a {\ovaloperator{\ovalvariable{x} * \ovalnum{2}}}}\end{scratch}&4\\
%\begin{scratch} \blockvariable{mettre \ovalvariable{x} \`a\ovaloperator{\ovalvariable{x} - \ovalvariable{réponse}}} \end{scratch}&5\\
%\begin{scratch}\blocklook{dire \txtbox{regroupe} Le programme de calcul donne{\ovalvariable{x}}}\end{scratch}&6\\ \hline
%\end{tabularx}
%\end{center}
%
%\medskip
%
%\begin{center}
%\begin{tabularx}{\linewidth}{|X|c|}\hline
%Lutin 2 &Numéro d'instruction\\ \hline
%\begin{scratch}\blockinit{Quand je reçois \ovalvariable{nombre saisi}}\end{scratch}&1\\
%\begin{scratch}\blockvariable{mettre \ovalvariable{x} \`a {\ovaloperator{\ovalnum{7} * \ovalvariable{réponse}}}}\end{scratch}&3\\
%\begin{scratch}\blockvariable{mettre \ovalvariable{x} \`a {\ovalvariable{x} - \ovalnum{8}}}\end{scratch}&4\\
%\begin{scratch}\blocklook{dire \txtbox{regroupe} Le programme de calcul donne{\ovalvariable{x}}}\end{scratch}&5\\ \hline
%\end{tabularx}
%\end{center}

\begin{enumerate}
\item %Vérifier que si on saisit $7$ comme nombre, le lutin \no 1 affiche comme résultat $17$ et le lutin \no 2 affiche $41$.
Le premier programme donne : $7 \to 12 \to 24 \to 17$.

Le deuxième programme donne : $7  \to 49 \to 41$.
\item %Quel résultat affiche le lutin \no 2 si on saisit le nombre $- 4$ ?
On obtient successivement : $- 4 \to - 28 \to - 36$.
\item 
	\begin{enumerate}
		\item %Si on appelle $x$ le nombre saisi, écrire en fonction de $x$ les expressions qui traduisent le programme de calcul du lutin \no 1, à chaque étape (instructions 3 à 5).
		Le programme 1 donne : $x \to x + 5 \to 2(x + 5) \to 2(x + 5) - x$.
		\item %Montrer que cette expression peut s'écrire $x + 10$.
Le résultat final précédent d'écrit :

$2(x + 5) - x = 2x + 10 - x = x + 10$.
	\end{enumerate}
\item  %Célia affirme que plusieurs instructions dans le script du lutin \no 1 peuvent être supprimées et remplacées par celle ci-contre.\begin{scratch}\blockvariable{mettre \ovalvariable{x} \`a {\ovaloperator{\ovalvariable{r\' eponse} + \ovalnum{10}}}}\end{scratch}

%Indiquer, sur la copie, les numéros des instructions qui sont alors inutiles.
On peut supprimer les instructions 3, 4 et 5.
\item  %Paul a saisi un nombre pour lequel les lutins \no 1 et \no 2 affichent le même résultat. Quel
%est ce nombre ?
Le deuxième programme donne si on introduit le nombre $x$, \; $7x - 8$.

Donc les deux programmes donnent le même résultat si :

$x + 10 = 7x - 8$, soit $18 = 6x$ ou $6\times 3 = 6\times x$, d'où finalement $x = 3$.

Vérification : le lutin \no 1 donne $3 + 10 = 13$ et le lutin \no 2 donne $7\times 3 - 8 = 21 - 8 = 13$.
\end{enumerate}
