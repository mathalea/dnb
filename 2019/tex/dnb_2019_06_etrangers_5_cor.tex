
\medskip

\begin{enumerate}
\item On compare les longueurs des côtés des triangles OAB et ODC :

On a $\dfrac{\text{OA}}{\text{OD}} = \dfrac{36}{64} = \dfrac{4 \times 9}{4 \times 16} = \dfrac{9}{16}$ ;

$\dfrac{\text{OB}}{\text{OC}} = \dfrac{27}{48} = \dfrac{3 \times 9}{3 \times 16} = \dfrac{9}{16}$, donc 

$\dfrac{\text{OA}}{\text{OD}} = \dfrac{\text{OB}}{\text{OC}}$ : Comme les points O,A et D d'une part, et les points O, B et C d'autre part sont alignés dans le même ordre,
alors d'après la réciproque de la propriété de Thalès cette égalité montre que les droites(AB) et (CD) sont parallèles.
\item On sait que l'on a également $\dfrac{\text{OA}}{\text{OD}} =\dfrac{\text{AB}}{\text{CD}}$ ou encore en remplaçant par les valeurs connues :

$\dfrac{9}{16} = \dfrac{\text{AB}}{80}$, d'où en multipliant chaque membre par 80 : 

AB $ = 80 \times \dfrac{9}{16} = 16 \times 5 \times \dfrac{9}{16} = 5 \times 9 = 45$~(cm).
\item On sait que le triangle ACD est rectangle en C ; donc le théorème de Pythagore permet d'écrire :

$\text{AC}^2 + \text{CD}^2 = \text{AD}^2$. \quad (1)

Or CD $ = 80$ et AD = AO + OD $= 36 + 64 = 100$.

L'égalité (1) devient :

$\text{AC}^2 + 80^2 = 100^2$, d'où $\text{AC}^2 = 100^2 - 80^2 = \np{10000} - \np{6400} = \np{3600}$; d'où AC $ = \sqrt{3600} = 60$.

Chaque étagère a une hauteur de 60~cm avec un plateau de 2~cm soit une hauteur de 62~cm ; il y a 4 étagères, donc la hauteur totale du meuble est égale à : $4 \times 62 = 248$~(cm) plus le dernier plateau donc une hauteur totale de 250~cm.
\end{enumerate}

\vspace{0,5cm}

