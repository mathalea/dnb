
\medskip

Le graphique ci-dessous donne les hauteurs d'eau au port de La Rochelle le mercredi 15 août 2018.

\begin{center}
\psset{xunit=0.45cm,yunit=1cm}
%Origine : http://maree.info/127?d=20180815
\begin{pspicture}(-2,-0.75)(26,7.5)
\multido{\n=0+2}{14}{\psline[linewidth=0.2pt](\n,0)(\n,7)}
\multido{\n=0+1}{8}{\psline[linewidth=0.2pt](0,\n)(26,\n)}
\psaxes[linewidth=1.25pt,Dx=2](0,0)(0,0)(25.9,7)
\pscurve[linecolor=blue,linewidth=1.25pt](0,3)(0.5,2.25)(1,1.53)(1.5,0.95)(2,0.65)(3,1.05)(4,2.45)(5,3.93)(5.5,4.52)(6,4.99)(7,5.63)(8,5.88)(8.5,5.88)(9,5.81)(10,5.41)(11,4.62)(12,3.47)(13,2.12)(14,1.07)(14.17,0.98)(14.5,0.9)(15,1.07)(16,2.22)(17,3.74)(18,4.92)(19,5.61)(20,5.91)
(21,5.91)(22,5.66)(23,5.09)(23.917,4.22)(24,4.1)
\uput[r](0,7.2){Hauteur (m)}\uput[u](24,0){Heures}
\end{pspicture}
\end{center}

\begin{enumerate}
\item %Quel a été le plus haut niveau d'eau dans le port ?
Le niveau d'eau a frôlé les 6~m  vers 8 h et un peu après 20 h.
\item %À quelles heures approximativement la hauteur d'eau a-t-elle été de 5 m ?
Il y avait 5 m d'eau à 6~h, 10~h 30, 18~h et 23~h.
\item%En utilisant les données du tableau ci-contre, calculer:

%\medskip
%
%\parbox{0.43\linewidth}{
%\textbf{a.} le temps qui s'est écoulé entre la marée haute et la marée basse.
%
%\textbf{b.} la différence de hauteur d'eau entre la marée
%haute et la marée basse.} \hfill
%\parbox{0.55\linewidth}{\begin{tabularx}{\linewidth}{|c|*{2}{>{\centering \arraybackslash}X|}}\cline{2-3}
%\multicolumn{1}{c|}{~}	&Heure	&\small Hauteur (en m)\\ \hline
%Marée haute 			&8 h 16 &5,89\\ \hline
%Marée basse 			&14 h 30&0,90\\ \hline
%\end{tabularx}
%}
	\begin{enumerate}
		\item Entre la marée haute et la marée basse, il s'est écoulé 14 h 30 - 8 h 16 = 6 h 14.
		\item La hauteur de la marée (le marnage) a été $5,89 - 0,90 = 4,99$~m.
	\end{enumerate}

\item  %À l'aide des deux documents suivants, comment qualifier la marée du 15 août 2018 entre 8 h 16 et 14 h 30 à la Rochelle ?
On a vu que la marée était de 4,99~m, donc le coefficient de marée est égal à :

$C = \dfrac{4,99}{5,34}\times 100 \approx 93$ : c'était donc une marée de vives-eaux.
\end{enumerate}

%\parbox{0.48\linewidth}{
%\begin{tabularx}{\linewidth}{|X|}\hline
%\textbf{Document 1} :\\
%Le coefficient de marée peut être calculé de la façon suivante à La Rochelle :\\
%$C = \dfrac{H_{\text{h}} - H_{\text{b}}}{5,34} \times 100$\\
%avec :\\
%\hspace{1cm}$\bullet~~$ $H_{\text{h}}$ : hauteur d'eau à marée haute.\\
%\hspace{1cm}$\bullet~~$ $H_{\text{b}}$  : hauteur d'eau à marée basse.\\ \hline
%\end{tabularx}
%}\hfill
%\parbox{0.48\linewidth}{
%\begin{tabularx}{\linewidth}{|X|}\hline
%\textbf{Document 2 :}\\
%Le coefficient de marée prend une valeur comprise entre $20$ et $120$.\\
%\hspace{1cm}$\bullet~~$ Une marée de coefficient supérieur à 70 est qualifiée de marée de
%vives-eaux.\\
%\hspace{1cm}$\bullet~~$ Une marée de coefficient inférieur à
%70 est qualifiée de marée de mortes-eaux.\\ \hline
%\end{tabularx}
%}
%%%%%%%%
\psset{unit=.25}
%\def\Atom#1{%
%\begin{pspicture}[dimen=m](-12,-12)(12,12)
%    \pstVerb{/AA 1 5 atan def /RR 26 sqrt def}
%    \pscustom[fillstyle=eofill,fillcolor=red,linearc=#1]{%
%            \pscircle{3}
%            \moveto(5,-1)
%            \psLoop{12}{%
%                    \translate(5,0)
%                    \psline(4,-1)(4,1)(0,1)
%                    \translate(-5,0)
%                    \psarc(0,0){!RR}{!AA}{!30 AA sub}
%                    \rotate{30}
%            }%
%            \closepath
%    }%
\def\Atomi#1{%
\begin{pspicture}[dimen=m](-12,-12)(12,12)
    \pstVerb{/AA 1 5 atan def /RR 26 sqrt def}
    \pscustom[fillstyle=eofill,fillcolor=red,linearc=#1]{%
            \pscircle{3}
            \moveto(5,-1)
            \psLoop{18}{%
                    \translate(5,0)
                    \psline(4,-1)(4,1)(0,1)
                    \translate(-5,0)
                    \psarc(0,0){!RR}{!AA}{!20 AA sub}
                    \rotate{20}
            }%
            \closepath
    }%
\end{pspicture}}
%\Atom{1}
%\Atom{0}
\bigskip

\Atomi{0}