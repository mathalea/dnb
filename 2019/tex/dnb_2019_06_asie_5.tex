
\medskip

\parbox{0.65\linewidth}{La figure ci-contre est codée et réalisée à main levée.

Elle représente un quadrilatère ABCD dont les
diagonales se croisent en un point O.

On donne: OA $= 3,5$ cm et AB $= 5$ cm.}\hfill
\parbox{0.25\linewidth}{\psset{unit=0.75cm}
\begin{pspicture}(4,4)
%\psgrid
\pslineByHand(0.2,3.8)(3.6,0.5)(3.7,3.6)(0.5,0.5)(0.2,3.8)%ACBDA
\pslineByHand(0.5,0.5)(3.6,0.5)%DC
\pslineByHand(0.2,3.8)(3.7,3.6)%AB
\uput[ul](0.2,3.8){A} \uput[dr](3.6,0.5){C} \uput[ur](3.7,3.6){B} \uput[dl](0.5,0.5){D} \uput[u](2,2.1){O}
\psline(1.2,2.6)(1.4,2.8)\psline(2.6,1.2)(2.9,1.4)
\psline(2.7,2.8)(3,2.6)\psline(1.2,1.4)(1.4,1.2) 
\end{pspicture}}

\smallskip

On s'intéresse à la nature du quadrilatère ABCD qui a été représenté.

\medskip

\begin{enumerate}
\item Peut-on affirmer que ABCD est un rectangle ?
\item Peut-on affirmer que ABCD est un carré ?
\end{enumerate}

\bigskip

