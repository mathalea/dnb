
\medskip

\begin{enumerate}
\item Le grand cône est un agrandissement du petit cône de coefficient 

$k = \dfrac{\text{AB}}{\text{A}'\text{B}'} = \dfrac{60}{30} = 2$, donc SB = 2SB$'$ et SB$'$ = BB$'$ = 240~cm.

Par conséquent, SB $= 2 \times $SB$' = 2 \times 240 = 480$ cm.
\item  Le triangle SOB est rectangle en O, donc d'après le théorème de Pythagore, on a :

$\text{SB}^2 = \text{SO}^2 + \text{OB}^2$

$480^2 = \text{SO}^22 + 30^2$

$\np{230400} = \text{SO}^2 + 900$

$\text{SO}^2 = \np{230400} - 900$

$\text{SO}^2 = \np{229500}$

$\text{SO} > 0$, donc $\text{SO} = \sqrt{\np{229500}}$

SO $\approx 479$ cm.
\item  Je commence par exprimer le volume du grand cône :

$V_{\text{grand cône}} = \dfrac{30^2 \times  \pi \times \sqrt{\np{229500}}}{3} \approx  \np{451505}$~cm$^3$.

Le petit cône est une réduction du grand cône de coefficient $\dfrac{1}{2}$, son volume est donc :

$V_{\text{petit cône}} = \left(\dfrac{1}{2}\right)^3 \times  V_{\text{grand cône}} \approx \np{56438}$~cm$^3$.

On en déduit le volume du manche à air :

$V_{\text{manche à air}} = V_{\text{grand cône}} - V_{\text{petit cône}} \approx 
\np{451505} - \np{56438}$, soit $V_{\text{manche à air}} \approx \np{395067}$~cm$^3$.
\end{enumerate}

\vspace{0,5cm}

