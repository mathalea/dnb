
\medskip

Une association cycliste organise une journée de randonnée à  vélo.

Les participants ont le choix entre trois circuits de longueurs différentes: 42 km, 35 km et 27 km.

À l'arrivée, les organisateurs relèvent les temps de parcours des participants et calculent leurs vitesses moyennes. Ils regroupent les informations dans un tableau dont voici un extrait:

\begin{center}
\begin{tabularx}{\linewidth}{|l|*{5}{>{\centering \arraybackslash}X|}}\hline
Nom du sportif				& Alix 	&David 	&Gwenn 		&Yassin 	&Zoé\\ \hline
Distance parcourue (en km)	& 35 	&42 	&27 		&35 		&42\\ \hline
Durée de la randonnée 		&2 h 	&3 h 	&1 h 30 min &1 h 45 min &1 h 36 min\\ \hline
Vitesse moyenne (en km/h) 	&17,5	&		&			&			&\\ \hline
\end{tabularx}
\end{center}


\begin{enumerate}
\item Quelle distance David a-t-il parcourue ?
\item Calculer les vitesses moyennes de David et de Gwenn.
\item Afin d'automatiser les calculs, l'un des organisateurs décide d'utiliser la feuille de tableur ci-dessous :

\begin{center}
\begin{tabularx}{\linewidth}{|c|l|*{5}{>{\centering \arraybackslash}X|}}\hline
&A &B &C &D& E &F\\ \hline
1 &Nom du sportif 				&Alix 	&David 	&\footnotesize Gwenn 	&Yassin &Zoé\\ \hline
2 &Distance parcourue (en km)	& 35	& 42	&27 	&35 	&42\\ \hline
3 &Durée de la randonnée (en h)	& 2 	&3 		&1,5	&		&\\ \hline
4 &Vitesse moyenne (en km/h)	& 17,5	&		&		&		&\\ \hline
\end{tabularx}
\end{center}

	\begin{enumerate}
		\item Quel nombre doit-il saisir dans la cellule E3 pour renseigner le temps de Yassin ?
		\item Expliquer pourquoi il doit saisir 1,6 dans la cellule F3 pour renseigner le temps de Zoé.
		\item Quelle formule de tableur peut-il saisir dans la cellule B4 avant de l'étirer sur la ligne 4 ?
	\end{enumerate}
\item Les organisateurs ont oublié de noter la performance de Stefan.
	
Sa montre GPS indique qu'il a fait le circuit de 35 km à  la vitesse moyenne de 25 km/h.
	
Combien de temps a-t-il mis pour faire sa randonnée? On exprimera la durée de la randonnée en
heures et minutes.
\end{enumerate}

\bigskip

