
\medskip 

%Antoine crée des objets de décoration avec des vases, des billes et de l'eau colorée. 
%
%Pour sa nouvelle création, il décide d'utiliser le vase et les billes ayant les caractéristiques suivantes : 
%
%\begin{center}
%\begin{tabularx}{\linewidth}{|m{7cm}|X|}\hline
%\textbf{Caractéristiques du vase}&\textbf{Caractéristiques des billes}\\  
%\psset{unit=0.7cm}
%\begin{pspicture}(9.5,8.5)
%%\psgrid
%\psline(0.25,1.6)(1.85,0.25)(4.4,0.9)(4.4,6.8)(1.85,6.15)(0.25,7.5)(2.8,8.15)(4.4,6.8)
%\psline(1.85,0.25)(1.85,6.15)\psline(0.25,1.6)(0.25,7.5)
%\pspolygon[fillstyle=vlines](6.1,0.5)(9.1,0.5)(9.1,6.6)(9,6.6)(9,1.3)(6.2,1.3)(6.2,6.6)(6.1,6.6)
%\psline{<->}(5.9,0.5)(5.9,6.6)\psline{>-<}(6.2,6.8)(6.1,6.8)\psline{>-<}(9.1,6.8)(9,6.8)
%\psline{<->}(9.2,0.5)(9.2,1.3)\rput{90}(9.5,0.9){1,7 cm}
%\psline{<->}(6.1,0.3)(9.1,0.3)\uput[d](7.6,0.3){9 cm}
%\uput[u](9.05,6.8){0,2 cm} 
%\uput[u](6.15,6.8){0,2 cm} 
%\rput{90}(5.5,3.9){21,7 cm} 
%\end{pspicture}	&\psset{unit=1cm}
%\begin{pspicture}(5,3)
%%\psgrid
%\pscircle[gradangle=90,gradbegin = gray,gradmidpoint = 0.5,gradend = white,fillstyle=gradient](2,1){0.5}
%\psline{<->}(2.6,0.5)(2.6,1.5)\uput[r](2.6,1){1,8 cm}
%\end{pspicture}\\
%Matière: verre &Matière: verre \\
%Forme: pavé droit&Forme: boule \\ 
%Dimensions extérieures : 9 cm $\times$ 9 cm $\times$ 21,7 cm&Dimension : 1,8 cm de diamètre\\
%Épaisseur des bords : 0,2 cm &\\
%Épaisseur du fond : 1,7 cm&\\ \hline 
%\end{tabularx}
%\end{center}
%
%Il  met 150 billes dans le vase. Peut-il ajouter un litre d'eau colorée sans risquer le débordement ? 

\smallskip

\emph{On rappelle que le volume de la boule est donné par la formule : $\dfrac{4}{3}\times \pi \times \text{rayon}^3$.} 

Le pavé a pour base un carré de côtés $9 - 2\times 0,2 = 8,6$~cm et de hauteur $21,7 - 1,7 = 20$~cm.

Le volume du vase est donc égal à :

$8,6 \times 8,6 \times 20 = \np{1479,2}$~(cm$^3)$.

Une bille a un volume de : $\dfrac{4}{3}\times \pi \times 0,9^3 = 0,972\pi$, donc 150 billes occuperont un volume de $145,8\pi$.

Il restera  $\np{1479,2} - 145,8\pi \approx \np{1021,16}$~(cm$^3)$ soit plus de 1 dm$^3$ : Antoine pourra ajouter un litre d'eau colorée.
