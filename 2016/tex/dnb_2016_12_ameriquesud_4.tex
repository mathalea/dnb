
\medskip

Cristo Redentor, symbole brésilien, est une grande statue dominant la ville de Rio qui s'érige au
sommet du mont Corcovado.

Au pied du monument, Julien et Magali souhaitent mesurer la hauteur de la statue (socle
compris). Julien qui mesure 1,90~m, se place debout à quelques mètres devant la statue. Magali
place le regard au niveau du sol de telle manière qu'elle voit le sommet du Cristo (S) et celui de la tête de Julien (T) alignés; elle se situe alors à 10~m de la statue et à 50~cm de Julien.
La situation est modélisée ci-dessous par la figure qui n'est pas à l'échelle.

\begin{center}
\psset{unit=1cm}
\begin{pspicture}(-1,0)(7,6)
\psline[linewidth=1.25pt](1,1)(1,5.8)
\psline[linewidth=1.25pt](3.4,1)(3.4,2.3)
\psline[linestyle=dotted](1,1)(4.3,1)(1,5.8)
\uput[r](-0.8,4.8){Cristo}
\uput[r](-0.8,4.2){Redentor}
\psline{->}(0,4)(1,2.9)
\uput[l](2.4,1.7){Julien}
\psline{->}(2.2,1.5)(3.4,1.7)
\rput{-55}(3.9,2.){regard de Magali}
\psline[linestyle=dashed]{<->}(1,0.5)(4.3,0.5)
\uput[d](2.65,0.5){10~m}
\uput[u](1,5.8){S} \uput[d](1,1){C} \uput[l](3.4,2.4){T} \uput[d](3.4,1){J} \uput[dr](4.3,1){M} 
\end{pspicture}
\end{center}

Déterminer la hauteur SC de la statue en supposant que le monument et Julien sont
perpendiculaires au sol.

\vspace{0,5cm}

