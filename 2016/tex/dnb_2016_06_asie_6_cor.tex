
\medskip

%Dans chaque cas, dire si l'affirmation est vraie ou fausse. Justifier votre réponse.
%
%\medskip

\begin{enumerate}
\item Affirmation 1 :

%Deux nombres impairs sont toujours premiers entre eux.
Faux $3$ et $9$ impairs sont divisibles par 3 ; ils ne sont pas premiers entre eux.
\item Affirmation 2 :

%Pour tout nombre entier positif $a$ et $b$, $\sqrt{a} + \sqrt{b} = \sqrt{a + b}$.
Faux $\sqrt{1} + \sqrt{4} = 1 + 2 = 3$ et $\sqrt{1 + 4} = \sqrt{5} \ne 3$.
\item Affirmation 3 :

%Si on augmente le prix d'un article de 20\,\% puis de 30\,\% alors, au total, le prix a
augmenté de 56\,\%.
Augmenter de 20\,\% revient à multiplier par 1,20, puis augmenter de 30\,\% revient à multiplier par 1,30 ; donc les deux augmentations successives reviennent à multiplier par $1,20 \times 1,30 = 1,56$. L'affirmation est vraie.
\end{enumerate}

\bigskip

