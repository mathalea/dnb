
\medskip

%Lors des soldes, Rami, qui accompagne sa mère et s'ennuie un peu, compare trois étiquettes pour passer le temps :
%\begin{center}
%\psset{unit=1cm}
%\begin{pspicture}(12,6)
%\rput(2,5.5){1}\rput(6,5.5){2}\rput(10,5.5){3}
%\rput(1,4.5){VALEUR}
%\rput(1,3.5){\Large 120~\euro}
%\rput(2,2.5){SOLDÉ}
%\rput(2,1.5){\Large 105~\euro}
%\psframe[linewidth=1.5pt,framearc=0.3](0,0)(3.5,5)
%\psframe[linewidth=1.5pt,framearc=0.3](4,0)(7.5,5)
%\rput(5.75,4.5){Robe rouge}
%\rput(5.75,3.5){\Large 45~euros}
%\rput(5.75,1.5){\huge $- 30$\,\%}
%\psframe[linewidth=1.5pt,framearc=0.3](8,0)(11.5,5)
%\rput(9.25,4.5){\large SOLDES}
%\rput(9.25,3.5){\large \textbf{SOLDES}}
%\rput(9.25,2.5){\large \emph{SOLDES}}
%\rput(9.25,1.5){\large 25~\euro}
%\rput(9.25,0.5){\large $- 12,50$~\euro}
%\end{pspicture}
%\end{center}
%\medskip

\begin{enumerate}
\item %Quel est le plus fort pourcentage de remise ?
Premier solde : $\dfrac{15}{120} = \dfrac{5}{40} = \dfrac{1}{8} = \dfrac{25}{200} = 12,5\,\%$.

Deuxième solde : 30\,\%

Troisième solde : $\dfrac{12,50}{25} = 50$\,\% : c'est le plus fort pourcentage de remise.
\item %Est-ce que la plus forte remise en euros est la plus forte en pourcentage ?

Premier solde : moins 15~\euro ;

Deuxième solde : $0,30 \times 45 = 13,50$~\euro.

Troisième solde : 12,50~\euro

Donc la plus forte remise en euros (premier solde) n'est pas la plus forte en pourcentage.
\end{enumerate}

\vspace{0,5cm}

