
\medskip

\begin{enumerate}
\item Pour pouvoir simplifier le calcul \:$7^6 \times 7^6$, on utilise une des propriétés des puissances, à savoir: $a^p \times  a^n = a^{n+p}$.

Ainsi, on peut écrire : $7^6 \times 7^6 = 7^{6+6} = 7^{12}$. (Réponse B)
\item Appelons $S$ la superficie de la maison avant augmentation. Si après augmentation de 40\,\%, la superficie est égale à 210, on peut alors poser :

$210 = S + \dfrac{40}{100} \times  S = S + 0,4S  = S \times  (1 + 0,4)$ soit 

$210 = 1,4S$ ou $S = \dfrac{210}{1,4} = 150$. (Réponse C)

\emph{Remarque} : Pensez à poser une équation lorsque vous cherchez une inconnue
\item Commençons par lister les diviseurs de 6. Il y en a 4 qui sont: 1, 2, 3 et 6.

Nous savons que la probabilité d'un évènement $A$ quelconque peut s'écrire:

$P(A) = \dfrac{\text{Nombre de cas favorables}}{\text{Nombre de cas possibles}}$.

Or, étant donné qu'il existe 4 diviseurs, il y a donc 4 cas favorables. Le dé comportant 6 faces, on a donc 6 cas possibles.

Soit : $P(A) = \dfrac{4}{6} = \dfrac{2}{3}$.  (Réponse A)
\end{enumerate}

\vspace{0,25cm}

