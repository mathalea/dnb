
\medskip

Les continents occupent $\dfrac{5}{17}$ de la superficie totale de la Terre.


\medskip

\begin{enumerate}
\item %L'océan Pacifique recouvre la moitié de la superficie restante.
Les mers occupent $1 - \dfrac{5}{17} = \dfrac{17 - 5}{17} = \dfrac{12}{17}$ de la superficie de la Terre.

L'Océan Pacifique occupe donc à lui seul $\dfrac{6}{17}$ (un peu plus d'un tiers).
%Quelle fraction de la superficie totale de la Terre occupe-t-il ?
\item %Sachant que la superficie de l'océan Pacifique est de \np{180000000}~km$^2$, déterminer la superficie de la Terre.
Si $S$ est la superficie de la Terre, on a donc :

$\dfrac{6}{17}\times S = \np{180000000}$, d'où $6S = 17 \times \np{180000000}$ et 

$S = \dfrac{17 \times \np{180000000}}{6} = \np{510000000}$~km$^2$.

La Terre a une superficie d'environ \np{510000000}~km$^2$.
\end{enumerate}
