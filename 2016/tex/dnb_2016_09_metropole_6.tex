
\medskip

Mélanie construit une véranda contre l'un des murs de sa maison.

Pour couvrir le toit de la véranda, elle se rend chez un grossiste en matériaux qui lui fournit des renseignements concernant deux modèles de tuiles.

\medskip

\textbf{Document 1 : Informations sur la véranda}

\medskip

\begin{tabularx}{\linewidth}{|p{7cm} X|}\hline
\psset{unit=0.63cm}
\begin{pspicture}(10.5,7.5)
\pspolygon[fillstyle=solid,fillcolor=lightgray](3.2,2.3)(3.2,5.2)(4.8,7.1)(8.6,6)(10.1,4.8)(10.1,1.8)(7,0.7)
\psline[linecolor=white,linewidth=1.5pt](3.3,2.3)(3.3,5.2)(4.9,7.1)(8.7,6)(10.2,4.8)(10.2,1.8)(7.1,0.7)
\psline[linecolor=white,linewidth=1.5pt](8.6,6)(7.05,3.9)(3.2,5.2)
\multirput(3.25,5)(0.64,-0.215){7}{\psline(-1.9,-1.2)}
\psline(7.05,3.8)(7,0.7)
\psline(3.25,5)(7.05,3.7)
\psline(7.05,3)(5.25,2.42)%(1.4,3.8)(1.4,1.1)
\psline(5.2,2.42)(5.2,0.14)
\psline(1.4,3.8)(5.25,2.42)
\psline(1.4,3.8)(3.2,4.4)
\psline(1.4,3.8)(1.4,1.6)
\uput[l](5.2,0.14){A}\uput[dr](7.05,0.8){B}
\uput[r](7.05,3){C}\uput[r](7.05,3.8){D}
\uput[l](5.2,2.42){E}\uput[l](1.4,1.6){H}
\uput[l](3.2,5){G}\uput[l](1.4,3.8){F}
\uput[l](3.2,2.42){I}
\rput(1.4,6.5){\small Toit EDGH}
\rput(1.4,5.7){\small de la véranda}
\psframe(-0.1,6.8)(2.9,5.4)
\psline{->}(1.4,5.4)(2.2,4.4)
%\psgrid
\end{pspicture}&\vspace{-4.5cm}EC = 2,85 m

BC = 2,10 m

BD = 3,10 m

EF = 6,10 m

Le toit EDGF de la véranda est un rectangle.

\begin{pspicture}(4,4)
\psline(0.5,0.5)(0.5,2.6)(3.35,3.6)(3.35,0.5)
\psline(0.5,2.6)(3.35,2.6)
\psframe(0.5,2.6)(0.8,2.3)
\psframe(3.35,2.6)(3.05,2.3)
\uput[l](0.5,0.5){A}\uput[r](3.35,0.5){B}
\uput[r](3.35,2.6){C}\uput[l](0.5,2.6){E}
\uput[r](3.35,3.6){D}
\end{pspicture}

\qquad Croquis à l'échelle\\ \hline
\end{tabularx}

\medskip

\textbf{Document 2 : informations sur les tuiles}

\medskip

\begin{tabularx}{\linewidth}{|m{4cm}|*{2}{>{\centering \arraybackslash}X|}}\hline
Modèle &Tuile romane &Tuile régence\\ \hline
Coloris &\og littoral \fg &\og Brun vieilli \fg\\ \hline
Quantité au m$^2$ & 13 & 19\\ \hline
Poids au m$^2$ (en kg) & 44 & 44\\ \hline
Pente minimale pour permettre la pose & 15\degrees &18\degrees\\ \hline
Prix à l'unité & 1,79~\euro & 1,2~\euro\\ \hline
Prix au m$^2$&23,27~\euro&\begin{pspicture}(0,-0.3)(0.6,0.3)\pscurve*(0,0)(0.3,0.2)(0.5,0.3)(0.4,-0.1)(0.3,-0.2)(0.25,-0.2)(0.1,-0.25)(0,-0.3)\rput(0.7,0){\euro} \end{pspicture}\\ \hline
\end{tabularx}

\medskip

\begin{enumerate}
\item Une tache cache le prix au m$^2$ des \og tuiles régence \fg. Calculer ce prix.
\item La pente du toit de la véranda, c'est-à-dire l'angle $\widehat{\text{DEC}}$, permet-elle la pose de chaque modèle ?
\item Mélanie décide finalement de couvrir le toit de sa véranda avec des tuiles romanes. Ces tuiles sont vendues à l'unité.

Pour déterminer le nombre de tuiles à commander, le vendeur lui explique  :

\og Il faut d'abord calculer la surface à recouvrir. Il faut augmenter ensuite cette surface de 5\,\%.\fg
 
En tenant compte de ce conseil, combien de tuiles doit-elle prévoir d'acheter ?
\end{enumerate}
 
\vspace{0,5cm}

