
\medskip

Pour répondre à la question posée, il faut calculer SO.

Je commence par déterminer AO :

ABC est un triangle rectangle en B. D'après le théorème de Pythagore, on a :

$\text{AC}^2 = \text{AB}^2 + \text{BC}^2$

$\text{AC}^2 = 30^2 + 30^2$

$\text{AC}^2 = 900 + 900$

$\text{AC}^2 = \np{1800}$

$\text{AC} > 0$, donc $\text{AC} = \sqrt{1800} = \sqrt{900 \times 2} = 30\sqrt{2}$~(cm).

ABCD est un carré, donc ses diagonales se coupent en leur milieu et
$\text{AO} = \dfrac{30\sqrt{2}}{2} = 15\sqrt{2}$ cm.

Je calcule SO :

ASO est un triangle rectangle en O. D'après le théorème de Pythagore, on a :

$\text{AS}^2 = \text{AO}^2 + \text{SO}^2$

$552 = \left(15\sqrt{2}\right)^2 + \text{SO}^2$

$\np{3025} = 225 \times 2 + \text{SO}^2$

$3 025 = 450 + \text{SO}^2$

$\text{SO}^2 = \np{3025} - 450$

$\text{SO}^2 = \np{2575}$

$\text{SO} > 0$, donc $\text{SO} = \sqrt{\np{2575}}$.

$\text{SO} \approx  50,7> 50  $ (cm).

Le présentoir ne peut pas être placé dans la vitrine de hauteur 50~cm.

\vspace{0,5cm}

