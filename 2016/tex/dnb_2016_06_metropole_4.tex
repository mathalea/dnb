
\medskip 

Lors des soldes, un commerçant décide d'appliquer une réduction de 30\,\% sur l'ensemble des articles de son magasin. 

\medskip

\begin{enumerate}
\item L'un des articles coûte 54~\euro{} avant la réduction. Calculer son prix après la réduction. 
\item Le commerçant utilise la feuille de calcul ci-dessous pour calculer les prix des articles soldés . 

\begin{center}
\begin{tabularx}{\linewidth}{|c|l|*{5}{>{\centering \arraybackslash}X|}}\hline
&A  &   B   &C  &D&   E&   F\\ \hline     
1  &prix avant réduction&   12,00~\euro{}   &14,80~\euro{}   &33,00~\euro{}   &44,20~\euro{}&   85,50~\euro{}\\ \hline  
2 &  réduction de 30\,\%&   3,60~\euro{} &  4,44~\euro{}&   9,90~\euro{} & 13,26~\euro{} &  25,65~\euro{}\\ \hline   
3&   prix soldé &&&&&\\ \hline
\end{tabularx}
\end{center}

	\begin{enumerate}
		\item Pour calculer la réduction, quelle formule a-t-il pu saisir dans la cellule B2 avant de l'étirer sur la ligne 2 ? 
		\item Pour obtenir le prix soldé, quelle formule peut-il saisir dans la cellule B3 avant de l'étirer sur la ligne 3 ? 
	\end{enumerate}
\item Le prix soldé d'un article est 42,00~\euro. Quel était son prix initial ? 
\end{enumerate}

\bigskip

