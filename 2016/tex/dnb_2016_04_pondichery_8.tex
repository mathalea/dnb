
\medskip

\parbox{0.48\linewidth}{Afin de faciliter l'accès à sa piscine,
Monsieur Joseph décide de construire un escalier constitué de deux prismes superposés dont les bases sont des triangles rectangles.}\hfill\parbox{0.48\linewidth}{
\psset{unit=1cm}
\begin{pspicture}(5.5,4.2)
%\psgrid
\psframe(0,0)(5.5,1)
\psframe(1.75,2.2)(4,3.2)
\pspolygon(1.75,3.2)(2.92,4)(4,3.2)
\psline(0,1)(1.75,2.2)
\psline(5.5,1)(4,2.2)
\end{pspicture}}

Voici ses plans :

\begin{center}
\psset{unit=1.5cm}
\begin{pspicture}(5.5,4.2)
%\psgrid
\psframe(0,0)(5.5,1)
\psframe(1.75,2.2)(4,3.2)
\pspolygon(1.75,3.2)(2.92,4)(4,3.2)
\psline(0,1)(1.75,2.2)
\psline(5.5,1)(4,2.2)
\psline[linestyle=dotted](2.92,4)(2.92,2)
\psline[linestyle=dotted](1.75,2.2)(2.92,3)(5.5,1)
\psline[linestyle=dotted](0,0)(2.92,2)(5.5,0)
\rput{-38}(3.5,3.8){1,28 m}\rput{34}(2.4,3.8){1,36 m}
\rput(1.35,2.7){0,20 m}
\psline{<->}(0,0.2)(2.92,2.2)\rput{35}(1.46,1.4){3,40 m}
\psline{<->}(5.5,0.2)(2.92,2.2)\rput{-35}(4.2,1.4){3,20 m}
\psdots(1.75,2.7)(4,2.7)(0,0.5)(5.5,0.5)
\psline(2.8,3.9)(2.92,3.8)(3.06,3.89)
\psline[linestyle=dotted](2.8,2.9)(2.92,2.8)(3.06,2.89)
\end{pspicture}
\end{center}

\textbf{Information 1 :} Volume du prisme = aire de la base $\times$ hauteur ;\quad  1~L = 1~dm$^3$

\textbf{Information 2 :} Voici la reproduction d'une étiquette figurant au dos d'un sac de ciment
de 35~kg.

\begin{center}
\begin{tabularx}{\linewidth}{|m{2cm}|*{4}{>{\centering \arraybackslash}X|}}\hline
Dosage pour 1 sac de 35 kg	&Volume de béton obtenu	&Sable (seaux)	&Gravillons (seaux)	&Eau\\ \hline
Mortier courant 			&105 L					&10				&					&16 L\\ \hline
Ouvrages en béton courant	&100 L					&5				&8 					&17 L\\ \hline
Montage de murs 			&120 L 					&12				&					&18~L\\ \hline
\multicolumn{5}{m{11cm}}{\emph{Dosages donnés à titre indicatif et pouvant varier suivant les matériaux régionaux et le taux d'hygrométrie des granulats}} 
\end{tabularx}
\end{center}

\medskip

\begin{enumerate}
\item Démontrer que le volume de l'escalier est égal à \np{1,26208} m$^3$.
\item Sachant que l'escalier est un ouvrage en béton courant, déterminer le nombre de sacs
de ciment de 35 kg nécessaires à la réalisation de l'escalier.
\item Déterminer la quantité d'eau nécessaire à cet ouvrage.
\end{enumerate}

