
\medskip

\begin{enumerate}
\item 
	\begin{enumerate}
		\item On se place sur l'axe des abscisses au point d'abscisse  320 ;  le point de la courbe \og confort \fg{} ayant cette abscisse a une ordonnée  d'environ \np{2400}~(mètres).
		\item On se place sur  l'axe des ordonnées au point d'abscisse \np{1500} ; l'horizontale contenant ce point coupe la courbe \og rapide \fg{} au point d'abscisse d'environ $360$~(km/h).
	\end{enumerate}
\item 
	\begin{enumerate}
		\item Le point de la courbe \og confort \fg{} d'abscisse 260 a une ordonnée d'environ \np{1600} mètres. Or, les
deux premières sorties se trouvent à une distance inférieure à \np{1600} mètres. Il est donc évident que l'avion dépassera les sorties 1 et 2.
		\item Si le pilote doit s'arrêter obligatoirement à la sortie 1, il ne peut dépasser une distance de freinage de  $900$ mètres. S'il décide d'un freinage \og rapide \fg, graphiquement, on observe que cette distance est atteinte avec une vitesse
maximale de $280$km/h.
	\end{enumerate}
\end{enumerate}

\vspace{0,25cm}

