
\medskip

Le mont du Pain de Sucre est un pic situé à Rio à flanc de mer. Il culmine à $396$~mètres
d'altitude et est accessible par un téléphérique composé de deux tronçons.

\begin{center}
\psset{unit=0.8cm}
\begin{pspicture}(15.3,4.5)
%\psgrid
\pspolygon[fillstyle=solid,fillcolor=lightgray](0,0.4)(1,1.1)(2,2.2)(2.4,1.8)(2.6,2.2)(3.5,2.1)(5,1.7)(5.8,1.5)(7,1.8)(7.5,1.9)(8,1.8)(8.4,2.3)(9.1,3.6)(9.3,3.4)(9.5,3.5)(11,2.5)(13,1.4)(13.5,1.5)(15.3,0.4)
\psdots(2.6,2.2)(9.1,3.6)(9.1,2.2)
\psline(2.6,2.2)(9.1,3.6)
\psline[linestyle=dotted](2.6,2.2)(9.1,2.2)(9.1,3.6)
\pscircle(7.05,3.2){0.05}
\pscircle(6.95,3.18){0.05}
\psline(7,3.19)(7,2.9)
\psframe[fillstyle=solid,fillcolor=gray](6.83,2.8)(7.17,3)
\psframe[fillstyle=solid,fillcolor=cyan](6.86,2.9)(6.95,2.96)
\psframe[fillstyle=solid,fillcolor=cyan](6.97,2.9)(7.06,2.96)
\psframe[fillstyle=solid,fillcolor=cyan](7.08,2.9)(7.17,2.96)
\psline[linestyle=dotted]{->}(12,2.2)(9.2,2.2)\uput[r](12,2.2){altitude 220 m}
\psline[linestyle=dotted]{->}(12,3.6)(9.2,3.6)\uput[r](12,3.6){altitude 396 m}
\uput[u](2.6,2.2){U} \uput[u](9.1,3.6){S} \uput[dr](9.1,2.2){O} \uput[u](5.7,2.8){762 m}
\rput(3.6,4.2){\small 2\up{e} tronçon du téléphérique du Pain de Sucre}
\end{pspicture}

\medskip

Le dessin ci-dessus n'est pas à l'échelle.
\end{center}

On a représenté ci-dessus le deuxième tronçon du téléphérique qui mène du point U au
sommet S du pic.

\begin{tabularx}{\linewidth}{l*{2}{X}}
On donne : 	&Altitude du point S : 396 m&US = 762 m\\
			&Altitude du point U : 220 m&Le triangle UOS est rectangle en O.\\
\end{tabularx}

\medskip

\begin{enumerate}
\item Déterminer l'angle $\widehat{\text{OUS}}$ que forme le câble du téléphérique avec l'horizontale. On arrondira le résultat au degré.
\item Sachant que le temps de trajet entre les stations U et S est de 6~min~30~s, calculer la
vitesse moyenne du téléphérique entre ces deux stations en mètres par seconde.
On arrondira le résultat au mètre par seconde.
\item On a relevé la fréquentation du Pain de Sucre sur une journée et saisit ces informations
dans une feuille de calcul d'un tableur.

\begin{center}	
\begin{tabularx}{\linewidth}{|>{\footnotesize}c|>{\footnotesize}c|*{7}{>{\centering \arraybackslash}X|}}\hline
\multicolumn{9}{|l|}{H2\qquad  =SOMME(B2 : G2)}\\ \hline
&A &B &C& D&E &F &G &H\\ \hline
1&Horaires &\scriptsize 8 \negthickspace : 00- 10\negthickspace:00&\scriptsize 10\negthickspace:00 12\negthickspace:00&\scriptsize  12\negthickspace:00-14\negthickspace:00&\scriptsize  14\negthickspace:00-16\negthickspace:00&\scriptsize  16\negthickspace:00-18\negthickspace:00&\scriptsize  18\negthickspace:00-20\negthickspace:00&\\ \hline
2&Nombre de visiteurs& 122 &140 &\psellipse*(0.,0.1)(0.3,0.2)&63 &75 &118 &615\\ \hline
\end{tabularx}
\end{center}

On a saisi dans la cellule H2 la formule : =SOMME(B2\negthickspace:G2)
	\begin{enumerate}
		\item Interpréter le nombre calculé avec cette formule.
		\item Quel est le nombre de visiteurs entre 12~h 00 et 14~h 00 ?
	\end{enumerate}
\item Une formule doit être saisie pour calculer le nombre moyen de visiteurs par heure sur
cette journée. Parmi les propositions suivantes, recopier sans justification celle qui
convient:

\begin{center}	
\begin{tabularx}{\linewidth}{*{2}{X}}	
MOYENNE(B2\negthickspace:G2)&=MOYENNE(B2\negthickspace:G2)\\
MOYENNE(B2\negthickspace:G2)/2&=MOYENNE(B2\negthickspace:G2)/2\\
\end{tabularx}
\end{center}
\end{enumerate}
