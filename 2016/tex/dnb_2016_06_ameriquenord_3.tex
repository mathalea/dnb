
\medskip

Une station de ski a relevé le nombre de forfaits \og journée\fg{} vendus lors de la saison écoulée (de décembre à avril).
		
Les résultats sont donnés ci-dessous dans la feuille de calcul d'un tableur.

\begin{center}		
\begin{tabularx}{\linewidth}
				{|>{\columncolor[gray]{0.8}}>{\sf \arraybackslash}c|>{\small \arraybackslash}m{2cm} | *{6}{>{\centering\arraybackslash} X|}} \hline 
				
			\rowcolor[gray]{0.8} & \centering{\textsf{{\normalsize A}}} & \sf B & \sf C & \sf D & \sf E & \sf F & \sf G \\  \hline
			1 & mois & \small décembre &\small janvier &\small février &\small mars &\small avril &\small total \\  \hline
			2 & nombre de forfaits journées vendus & \np{60457} & \np{60457} & \np{148901} & \np{100058} & \np{10035} &  \\  \hline
			3 &  &  &  &  &  &  &  \\ \hline
\end{tabularx} 
\end{center}

		\begin{enumerate}
			\item \begin{enumerate}
				\item Quel est le mois durant lequel la station a vendu le plus de forfaits \og journée \fg{}?
				
				\item Ninon dit que la station vend plus du tiers des forfaits durant le mois de février.
				
A-t-elle raison ? Justifier.
			\end{enumerate}
			
			\item Quelle formule doit-on saisir dans la cellule \textsf{G2} pour obtenir le total des forfaits \og journée \fg{}   vendus durant la saison considérée ?
			
			\item Calculer le nombre moyen de forfaits \og journée \fg{} vendus par la station en un mois. On arrondira le résultat à l'unité.
		\end{enumerate}

\vspace{0,5cm}

