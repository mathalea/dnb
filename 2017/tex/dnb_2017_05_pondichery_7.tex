
\medskip

Alban souhaite proposer sa candidature pour un emploi dans une entreprise. Il doit envoyer dans une seule enveloppe: 2 copies de sa lettre de motivation et 2 copies de son Curriculum Vitæ (CV). Chaque copie est rédigée sur une feuille au format A4.

\medskip

\begin{enumerate}
\item Il souhaite faire partir son courrier en lettre prioritaire. Pour déterminer le prix du timbre, il obtient sur
internet la grille de tarif d'affranchissement suivante:

\begin{center}
\begin{tabularx}{0.5\linewidth}{|*{2}{>{\centering \arraybackslash}X|}}\hline
\multicolumn{2}{|c|}{Lettre prioritaire}\\ \hline
Masse jusqu'à& Tarifs nets\\ \hline
20~g &0,80~\euro\\ \hline
100~g &1,60~\euro\\ \hline
250~g &3,20~\euro\\ \hline
500~g &4,80~\euro\\ \hline
3~kg &6,40~\euro\\ \hline
\end{tabularx}
\end{center}

Le tarif d'affranchissement est-il proportionnel à la masse d'une lettre ?
\item Afin de choisir le bon tarif d'affranchissement, il réunit les informations suivantes:

\setlength\parindent{1cm}
\begin{itemize}
\item[$\bullet~~$] Masse de son paquet de $50$ enveloppes : $175$~g.
\item[$\bullet~~$] Dimensions d'une feuille A4 : $21$~cm de largeur et $29,7~$cm de longueur.
\item[$\bullet~~$] Grammage d'une feuille A4 : $80$~g/m$^2$ (le grammage est la masse par m$^2$ de feuille).
\end{itemize}

Quel tarif d'affranchissement doit-il choisir ?
\end{enumerate}
