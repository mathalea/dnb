
\medskip


Le 17 juillet 2016, une spectatrice regarde l'étape \og Bourg-en-Bresse
/ Culoz \fg{} du Tour de France.

Elle note, toutes les demi-heures, la distance parcourue par le
cycliste français Thomas Vœckler qui a mis 4~h 30~min pour
parcourir cette étape de 160~km ; elle oublie seulement de noter la
distance parcourue par celui-ci au bout de 1~h de course.

Elle obtient le tableau suivant : 

\begin{center}
\begin{tabularx}{\linewidth}{|c|*{10}{>{\centering \arraybackslash}X|}}\hline
Temps en heure &0 &0,5&1 		&1,5 	&2 	&2,5 	&3 	&3,5 &4 	&4,5\\ \hline
Distance en km &0 &15 &\ldots 	&55 	&70 &80 	&100&110 &135 	&160\\ \hline
\end{tabularx}
\end{center}

\medskip

\begin{enumerate}
\item Quelle distance a-t-il parcourue au bout de 2~h 30~min de course?
\item Montrer qu'il a parcouru 30 km lors de la troisième heure de course.
\item A-t-il été plus rapide lors de la troisième ou bien lors de la quatrième heure de
course ?
\item Répondre aux questions qui suivent sur la feuille ANNEXE , qui est
à rendre avec la copie.
	\begin{enumerate}
		\item Placer les 9 points du tableau dans le repère. On ne peut pas placer le
point d'abscisse 1 puisque l'on ne connaît pas son ordonnée.
		\item En utilisant votre règle, relier les points consécutifs entre eux.
 	\end{enumerate}
\item En considérant que la vitesse du cycliste est constante entre deux relevés,
déterminer, par lecture graphique, le temps qu'il a mis pour parcourir 75~km.
\item On considère que la vitesse du cycliste est constante entre le premier relevé
effectué au bout de 0,5~h de course et le relevé effectué au bout de 1,5~h de
course ; déterminer par lecture graphique la distance parcourue au bout de 1~h
de course.
\item Soit $f$ la fonction, qui au temps de parcours du cycliste Thomas Vœckler,
associe la distance parcourue. La fonction $f$ est-elle linéaire ?
\end{enumerate}

\begin{center}

\textbf{\large ANNEXE}

\bigskip

\textbf{À détacher du sujet et à joindre avec la copie}

\begin{flushleft}
\textbf{question 4}
\end{flushleft}

\medskip

\psset{xunit=1.7cm,yunit=0.08cm}
\begin{pspicture}(-0.5,-10)(5.5,200)
\uput[u](5,0){Temps en $h$}
\uput[r](0,192.50){Distance en km}
\multido{\n=0.0+0.5}{12}{\psline[linewidth=0.2pt](\n,0)(\n,190)}
\multido{\n=0+5}{39}{\psline[linewidth=0.2pt](0,\n)(5.5,\n)}
\psaxes[linewidth=1.25pt,Dy=20]{->}(0,0)(0,0)(5.5,190)
\psaxes[linewidth=1.25pt,Dy=20](0,0)(0,0)(5.5,190)
\end{pspicture}
\end{center}

\vspace{0,5cm}

