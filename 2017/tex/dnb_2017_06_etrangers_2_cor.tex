
\medskip

\textbf{Partie 1 }: 

%Pour réaliser une étude sur différents isolants, une société réalise 3 maquettes de maison strictement identiques à l'exception près des isolants qui diffèrent dans chaque maquette. On place ensuite ces 3 maquettes dans une chambre froide réglée à 6~\degres C. On réalise un relevé des températures ce qui permet de construire les 3 graphiques suivants: 
%
%\begin{center}
%\psset{xunit=.12,yunit=.3}
%\begin{pspicture}(-3,-.5)(100,24)
%\multido{\n=0+5}{21}{\psline[linewidth=0.2pt,linecolor=lightgray](\n,-0)(\n,22.5)}
%\multido{\n=0+2}{12}{\psline[linewidth=0.2pt,linecolor=lightgray](-.20,\n)(100,\n)}
%\psaxes[labelsep=.8mm,linewidth=.75pt,ticksize=-2pt 2pt,Dx=5,Dy=2,labelFontSize=\scriptstyle]{->}(0,0)(0,0)(100,23)
%\rput(10,21.5){{\scriptsize Température en $^\circ$C}}
%\psline[linewidth=1.25pt](0,20)(15,20)
%\pscurve[linewidth=1.25pt](15,20)(19,19.9)(76,6.1)(80,6)
%\psline[linewidth=1.25pt](80,6)(100,6)
%\rput(60,19.2){MAQUETTE A}
%\rput(87,1.5){{\scriptsize Durée en heures}}
%\end{pspicture}
%\end{center}
%
%
%
%\begin{center}
%\psset{xunit=.12,yunit=.3}
%\begin{pspicture}(-3,-.5)(100,24)
%\multido{\n=0+5}{21}{\psline[linewidth=0.2pt,linecolor=lightgray](\n,-0)(\n,22.5)}
%\multido{\n=0+2}{12}{\psline[linewidth=0.2pt,linecolor=lightgray](-.20,\n)(100,\n)}
%\psaxes[labelsep=.8mm,linewidth=.75pt,ticksize=-2pt 2pt,Dx=5,Dy=2,labelFontSize=\scriptstyle]{->}(0,0)(0,0)(100,23)
%\rput(10,21.5){{\scriptsize Température en $^\circ$C}}
%\psline[linewidth=1.25pt](0,20)(20,20)
%\pscurve[linewidth=1.25pt](20,20)(24,19.9)(66,6.1)(70,6)
%\psline[linewidth=1.25pt](70,6)(100,6)
%\rput(60,19.2){MAQUETTE B}
%\rput(87,1.5){{\scriptsize Durée en heures}}
%\end{pspicture}
%\end{center}
%
%\begin{center}
%\psset{xunit=.12,yunit=.3}
%\begin{pspicture}(-3,-.5)(100,24)
%\multido{\n=0+5}{21}{\psline[linewidth=0.2pt,linecolor=lightgray](\n,-0)(\n,22.5)}
%\multido{\n=0+2}{12}{\psline[linewidth=0.2pt,linecolor=lightgray](-.20,\n)(100,\n)}
%\psaxes[labelsep=.8mm,linewidth=.75pt,ticksize=-2pt 2pt,Dx=5,Dy=2,labelFontSize=\scriptstyle]{->}(0,0)(0,0)(100,23)
%\rput(10,21.5){{\scriptsize Température en $^\circ$C}}
%\psline[linewidth=1.25pt](0,20)(10,20)
%\pscurve[linewidth=1.25pt](10,20)(14,19.9)(51,6.1)(55,6)
%\psline[linewidth=1.25pt](55,6)(100,6)
%\rput(60,19.2){MAQUETTE C}
%\rput(87,1.5){{\scriptsize Durée en heures}}
%\end{pspicture}
%\end{center}

%\medskip

\begin{enumerate}
\item  %Quelle était la température des maquettes avant d'être mise dans la chambre froide ?
Les trois maquettes étaient à 20~\degres C. 

\item %Cette expérience a-t-elle duré plus de 2 jours? Justifier votre réponse. 
L'expérience a duré 100 heures soit $4 \times 24 + 4$ donc 4 jours et 4 heures.

\item %Quelle est la maquette qui contient l'isolant le plus performant? Justifier votre réponse.
La maquette la plus résistante au froid est la B car il lui faut  70~h pour descendre à 6~\degres C.
\end{enumerate}

\medskip

\textbf{Partie 2 }: 

%Pour respecter la norme RT2012 des maisons BBC (Bâtiments Basse Consommation), il faut que la résistance thermique des murs notée $R$ soit supérieure ou égale à 4. Pour calculer cette résistance thermique, on utilise la relation: 
%
%$$R=\dfrac{e}{c}$$ 
%
%où $e$ désigne l'épaisseur de l'isolant en mètre et $c$ désigne le coefficient de conductivité thermique de l'isolant. Ce coefficient permet de connaître la performance de l'isolant. 

\medskip

\begin{enumerate}
\item  %Noa a choisi comme isolant la laine de verre dont le coefficient de conductivité thermique est: $c = 0,035$. Il souhaite mettre 15~cm de laine de verre sur ses murs. 

%Sa maison respecte-t-elle la normé RT2012 des maisons BBC ? 
On a $R_{\text{Noa}} = \dfrac{0,15}{0,035} = \dfrac{150}{35} = \dfrac{30}{7} \approx 4,3$ donc supérieur à 4.
\item  %Camille souhaite obtenir une résistance thermique de 5 ($R = 5$). Elle a choisi comme isolant du liège dont le coefficient de conductivité thermique est: $c = 0,04$. 

%Quelle épaisseur d'isolant doit-elle mettre sur ses murs ? 
Il faut trouver $e$ tel que :

$R = \dfrac{e}{c}$, soit $5 = \dfrac{e}{0,04}$, donc $e = 5 \times 0,04 = 0,2$~(m) soit 20~cm.
\end{enumerate}

\bigskip

