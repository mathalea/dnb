
\medskip

Pour des raisons de santé, il est conseillé de limiter ses efforts durant des activités sportives, afin de ne pas dépasser un certain rythme cardiaque.

La fréquence cardiaque est donnée en pulsations/minute.

L'âge est donné en année.

\smallskip

Autrefois, la relation entre l'âge $x$ d'une personne et $f(x)$ la fréquence cardiaque maximale recommandée était décrite par la formule suivante :

\[f(x) = 220 - x.\]

Des recherches récentes ont montré que cette formule devait être légèrement modifiée.

La nouvelle formule est :
\[g(x) = 208 - 0,7x.\]

\begin{enumerate}
\item 
	\begin{enumerate}
		\item Avec la formule $f(x)$, quelle est la fréquence cardiaque maximale recommandée pour un enfant de $5$ ans ?
		\item Avec la formule $g(x)$, quelle est la fréquence cardiaque maximale recommandée pour un enfant de $5$ ans ?
	\end{enumerate}
\item  
	\begin{enumerate}
		\item Sur l'annexe 2, compléter le tableau de valeurs.
		\item Sur l'annexe 2, tracer la droite $d$ représentant la fonction $f$ dans le repère tracé.
		\item Sur le même repère, tracer la droite $d'$ représentant la fonction $g$.
	\end{enumerate}
\item  Un journal commente : \og Une des conséquences de l'utilisation de la nouvelle formule au lieu de l'ancienne est que la fréquence cardiaque maximale recommandée diminue
légèrement pour les jeunes et augmente légèrement pour les personnes âgées. \fg
	
Selon la nouvelle formule, à partir de quel âge la fréquence cardiaque maximale
recommandée est-elle supérieure ou égale à celle calculée avec l'ancienne formule ?
	
Justifier.
\item  Des recherches ont démontré que l'exercice physique est le plus efficace lorsque la
fréquence cardiaque atteint 80\,\% de la fréquence cardiaque maximale recommandée
donnée par la nouvelle formule.
	
Calculer pour une personne de $30$ ans la fréquence cardiaque, en pulsations/minute, pour
que l'exercice physique soit le plus efficace.
\end{enumerate}

\begin{center}
{\Large \textbf{Annexe 2}}

\bigskip

\begin{tabularx}{\linewidth}{|*{12}{>{\centering \arraybackslash}X|}}\hline
$x$		&5	&10	&20	&30	&40	&50	&60	&70	&80	&90	&100\\ \hline
$f(x)$	&	&	&	&	&	&	&	&	&	&	&\\ \hline
$g(x)$	&	&	&	&	&	&	&	&	&	&	&\\ \hline
\end{tabularx}

\bigskip

\psset{xunit =0.08cm,yunit=0.08cm}
\begin{pspicture}(-5,-5)(100,220)
\multido{\n=0+5}{21}{\psline[linecolor=cyan,linewidth=0.3pt](\n,0)(\n,220)}
\multido{\n=0+5}{45}{\psline[linecolor=cyan,linewidth=0.3pt](0,\n)(100,\n)}
\psaxes[linewidth=1.25pt,Dx=10,Dy=10,labelFontSize=\scriptstyle]{->}(0,0)(0,0)(100,220)
\psaxes[linewidth=1.25pt,Dx=10,Dy=10,labelFontSize=\scriptstyle](0,0)(0,0)(100,220)
\uput[r](0,217.5){fréquence cardiaque}
\uput[u](97,0){$x$}
\end{pspicture}
\end{center}
\vspace{0,5cm}

