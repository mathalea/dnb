
\vspace{0.3cm}
\textbf{Affirmation 1 :} \textbf{Les nombres 11 et 13 n'ont aucun multiple commun.}

$11\times13=143$

143 est un multiple de 11 car il s’écrit \textbf{$11 \times~$ entier},

et 143 est un multiple de 13 car il s'écrit \textbf{entier $~ \times13$},

donc 143 est un multiple commun aux nombres 11 et 13. Ainsi l'affirmation 1 est fausse.

\smallskip

\textbf{Affirmation 2 :} \textbf{Le nombre 231 est un nombre premier.}

Un nombre premier est un nombre entier admettant exactement deux diviseurs (et dans ce cas, ce sont nécessairement 1 et lui-même).

231 est divisible par 3 car $2+3+1=6$ et 6 est divisible par 3, donc d'après le critère de divisibilité par 3, 231 est divisible par 3.

231 admet plus de deux diviseurs : \quad 1 ; 231 et 3. Donc 231 n'est pas un nombre premier. Ainsi, l'affirmation 2 est fausse.

\smallskip

\textbf{Affirmation 3 :} \textbf{$\dfrac{2}{15}$ est le tiers de $\dfrac{6}{15}$.}

Prendre le tiers \textcolor{red}{\textbf{de}} $\dfrac{6}{15}$ s'est calculer $\dfrac{1}{3}\textcolor{red}{\textbf{$\times$}}\dfrac{6}{15}$
$\dfrac{1}{3}\times\dfrac{6}{15}=\dfrac{1\times6}{3\times15}$
$=\dfrac{1\times\cancel{3}\times2}{\cancel{3}\times15}$ 
$=\dfrac{2}{15}$. Ainsi, l'affirmation 3 est vraie. 

\textbf{Remarque} :  On aurait pu directement remarquer que $615=3\times 215$ et donc le tiers de $3\times 215$ est égal à $215$.

\smallskip

\textbf{Affirmation 4:} \textbf{$15 - 5 \times 7 + 3 = 73$. }

La multiplication est prioritaire sur la soustraction, donc : \quad $15 - \underbrace{5 \times 7}_{35} + 3 =15-35+3$
 $=-20+3$
$=-17$ \textcolor{red}{\textbf{$\ne73$}}. \quad Ainsi, l'affirmation 4 est fausse.

\smallskip

\textbf{Affirmation 5 :} \og Le triangle $ABC$ avec $AB = 4,5$ cm, $BC = 6$ cm et $AC = 7,5$ cm est rectangle en $B$\fg.

Dans le triangle $ABC$, $[AC]$ est le côté le plus long.

\begin{minipage}{6cm}
$AC^2 = 7,5^2 = 56,25$
\end{minipage}
\vline \hspace{1cm}\begin{minipage}{6cm}
$AB^2+BC^2 = 4,5^2+6^2$
$=24,75+36$
$=56,25$
\end{minipage}

\vspace{0.1cm}

On a bien \quad $AC^2=AB^2+BC^2$, donc d'après \textcolor{red}{\textbf{la réciproque}} du théorème de Pythagore, le triangle $ABC$ est rectangle en $B$.

Ainsi, l'affirmation 5 est vraie.

\vspace{0.5cm}

