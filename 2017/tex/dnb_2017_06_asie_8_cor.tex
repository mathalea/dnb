
\medskip

%\parbox{0.45\linewidth}{La figure ci-contre représente un solide constitué de l'assemblage de quatre cubes :
%\setlength\parindent{8mm}
%\begin{itemize}
%\item trois cubes d'arête 2~cm ;
%\item un cube d'arête 4~cm.
%\end{itemize}
%\setlength\parindent{0mm}}\hfill
%\parbox{0.5\linewidth}{\psset{unit=1cm}
%\def\unite{\pspolygon[fillstyle=solid,fillcolor=gray](0,0.3)(0.8,0)(0.8,1)(0,1.3)
%\pspolygon[fillstyle=solid,fillcolor=lightgray](0.8,0)(1.65,0.3)(1.65,1.3)(0.8,1)
%\pspolygon[fillstyle=solid,fillcolor=lightgray](0.8,1)(1.65,1.3)(0.85,1.6)(0,1.3)}
%\def\gros{\pspolygon[fillstyle=solid,fillcolor=gray](0,0.6)(1.6,0)(1.6,2)(0,2.6)
%\pspolygon[fillstyle=solid,fillcolor=lightgray](1.6,0)(3.3,0.6)(3.3,2.6)(1.6,2)
%\pspolygon[fillstyle=solid,fillcolor=lightgray](1.6,2)(3.3,2.6)(1.7,3.2)(0,2.6)}
%\begin{pspicture}(2.3,0)(8,6)
%\rput(2.9,0.1){\gros}
%\rput(2.1,2.7){\unite}\rput(5.35,0.1){\unite}
%\rput(2.85,0.8){\unite}
%\rput(1.5,2){Vue de }\rput(1.5,1.6){face}
%\rput(7.6,1){Vue de }\rput(7.6,0.6){droite}
%\rput(4.6,5){Vue de}
%\rput(4.6,4.6){dessus}
%\psline[linewidth=2pt]{->}(4.6,4.2)(4.6,3.3)
%\psline[linewidth=2pt]{->}(2.1,2.1)(2.7,2.4)
%\psline[linewidth=2pt]{->}(7.6,1.4)(6.4,2.)
%\end{pspicture}
%}
%
%\medskip

\begin{enumerate}
\item %Quel est le volume de ce solide ? 
$V = 3 \times 2^3 + 4^3 = 3 \times 8 + 64 = 24 + 64 = 88$~cm$^3$.
\item ~%On a dessiné deux vues de ce solide (elles ne sont pas en vraie grandeur).

%Dessiner la \textbf{vue de droite} de ce solide \textbf{en vraie grandeur}.
\begin{center}
\psset{unit=0.6cm}
\begin{pspicture}(-2,0)(4,6)
\psframe(4,4)
\psframe(2,0)(4,2)
\psframe(0,4)(2,6)
\psframe(0,2)(-2,4)
\end{pspicture}
\end{center}
\end{enumerate}

%\begin{center}
%\begin{tabularx}{\linewidth}{*{2}{>{\centering \arraybackslash}X}}
%\textbf{Vue de face}&\textbf{Vue de dessus}\\
%\psset{unit=0.6cm}
%\begin{pspicture}(8,6)
%\psframe(0,4)(2,6)\psframe(2,0)(6,4)\psframe(4,2)(6,4)\psframe(6,0)(8,2)
%\end{pspicture}&\psset{unit=0.6cm}
%\begin{pspicture}(8,6)
%\psframe(0,2)(2,4)\psframe(2,2)(6,6)\psframe(6,4)(8,6)\psframe(4,0)(6,2)
%\end{pspicture}
%\end{tabularx}
%\end{center}
