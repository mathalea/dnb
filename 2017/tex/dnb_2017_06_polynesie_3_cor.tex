
\medskip

\begin{enumerate}
\item 
	\begin{enumerate}
		\item %Tracer un triangle CDE rectangle en D tel que CD $= 6,8$~cm et DE $= 3,4$~cm.
Sur deux demi-droites perpendiculaires en D on place les points C et E tels que CD $= 6,8$~cm et DE $= 3,4$~cm.
		\item %Calculer CE au dixième de centimètre près.
D'après le théorème de Pythagore :

$\text{CE}^2 = \text{CD}^2 + \text{DE}^2 = 6,8^2 + 3,4^2 = 46,24 + 11,56 = 57,80$.

D'où $\text{CE} \approx 7,60$ soit 7,6~cm au dixième près.
 	\end{enumerate}
\item 
	\begin{enumerate}
		\item %Placer le point F sur [CD] tel que CF $= 2$~cm.
Voir la figure.
		\item %Placer le point G sur [CE] tel que FE $= 1$~cm.
Le point G est à l'intersection du segment [CE] et du cercle de centre F et de rayon 1~cm; il y a deux points G$_1$ et G$_2$ qui répondent à la question.
		\item %Les droites (FG) et (DE) sont-elles parallèles ?
%On a $\dfrac{\text{CF}}{\text{CD}} = \dfrac{2}{6,8} = \dfrac{1}{3,4}$ et $\dfrac{\text{GF}}{\text{ED}} = \dfrac{1}{3,4}$.
%
%Donc $\dfrac{\text{CF}}{\text{CD}} = \dfrac{\text{GF}}{\text{ED}}$ : d'après la réciproque de la propriété de Thalès les droites (FG) et (DE) sont parallèles.
Comme on peut construire deux points G répondant à la question \textbf{2. b.}, on ne peut pas dire si les droites (FG) et (DE) sont parallèles ou non.

	\end{enumerate}
\end{enumerate}
\begin{center}
\psset{unit=1cm}
\begin{pspicture}(9,4.1)
%\psgrid
\pspolygon(0.5,3.9)(0.5,0.5)(7.3,0.5)%EDC
\uput[l](0.5,3.9){E}\uput[dl](0.5,0.5){D}\uput[d](7.3,0.5){C}
\uput[d](5.3,0.5){F}\uput[70](5.3,1.5){G$_1$}\uput[30](6.1,1.1){G$_2$}
\psline(5.3,0.5)(5.3,1.5)
\psline(5.3,0.5)(6.1,1.1)
\pscircle(5.3,0.5){1}%F
\end{pspicture}
\end{center}
\bigskip

