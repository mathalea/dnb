
\medskip

\begin{minipage}{7cm}
Le tableau ci-contre indique l'apport énergétique en kilocalories par gramme (kcal/g) de quelques nutriments.
\end{minipage}
\hspace{0.5cm}\begin{minipage}{7cm}
\begin{tabularx}{\linewidth}{|X|X|}\hline
\multicolumn{2}{|c|}{\small Apport énergétique pour quelques nutriments}\\ \hline   
Lipides   	&9 kcal/g \\ \hline  
Protéines  	&4 kcal/g\\ \hline   
Glucides   	&4 kcal/g \\ \hline
\end{tabularx}  
\end{minipage}

\begin{enumerate}
\item Un \oe uf de 50 g est composé de:
\begin{itemize}
\item 5,3 g de lipides ; 
\item 6,4 g de protéines ; 
\item 0,6 g de glucides ; 
\item 37,7 g d'autres éléments non énergétiques. 
\end{itemize}
L'apport énergétique des lipides pour quelques nutriments est de 9~kcal pour 1~g. 

$5,3 \times 9=47,7$. L'apport énergétique des lipides pour un œuf de 50~g est de 47,7~kcal.

L'apport énergétique des protéines pour quelques nutriments est de 4~kcal pour 1~g.

$6,4 \times 4=25,6$. L'apport énergétique des protéines pour un œuf de 50~g est de 25,6~kcal.

L'apport énergétique des glucides pour quelques nutriments est de 4~kcal pour 1~g.

$0,6 \times 4 = 2,4$. L'apport énergétique des glucides pour un œuf de 50~g est de 2,4~kcal. 

$47,7+25,6+2,4=75,7$. La valeur énergétique totale d'un œuf de 50~g est de 75,7~kcal. 
%\end{enumerate}

\begin{minipage}{9cm}
%\begin{enumerate}[\textbf{2.}]
\item À partir de la partie de l'étiquette de la tablette de chocolat et du tableau de la question \textbf{1.}, on calcule l'apport énergétique des lipides et celui des protéines, pour 100~g de chocolat.

L'apport énergétique des lipides pour quelques nutriments est de 9~kcal pour 1~g.

$30 \times 9  = 270$. L'apport énergétique des lipides pour 100~g de chocolat est de 270~kcal.

L'apport énergétique des protéines pour quelques nutriments est de 4~kcal pour 1~g.

$4,5 \times 4 = 18$. L'apport énergétique des protéines pour 100~g de chocolat est de 18~kcal.

$270+18=288$. L'apport énergétique des lipides et des protéines pour 100~g de chocolat est de 288~kcal.
%\end{enumerate}
\end{minipage}
\begin{minipage}{7cm}
\psset{unit=1cm}
\begin{pspicture}(-1.5,-3)(6,2.6)
%\psgrid
\psclip{\pscurve(-1,-2.4)(2,-2.)(2.5,-1.4)(3,-0.8)(4,-1)(4.3,1)(4.8,1.4)(4.4,2.2)(3.5,2.3)(2,2.5)(0,2.4)(-0.5,2.5)(-1,0)(-1,-2.4)}
\rput(2.25,0){
\begin{tabularx}{\linewidth}{|m{3.cm}|X|}\hline
Valeurs nutritionnelles moyennes&Pour 100 g de chocolat\\ \hline 
Valeur énergétique				&520 kcal\\ \hline 
Lipides							&30 g \\ \hline 
Protéines						&4,5 g\\ \hline  
Glucides						& \\ \hline 
Autres éléments non énergétiques  	&\\ \hline
\end{tabularx}
}
\endpsclip
\end{pspicture}
\end{minipage}
%\begin{enumerate}[\phantom{\textbf{2.}}]
\item La valeur énergétique totale pour 100~g de chocolat est de 520~kcal.

$520-288=232$. L'apport énergétique des glucides pour 100~g de chocolat est de 232~kcal.

L'apport énergétique des glucides pour quelques nutriments est de 4~kcal pour 1~g.

$232 \div 4 = 58$. La masse de glucides pour 100~g de chocolat est de 58~g.

Dans 200~g de chocolat, la masse de glucides est deux fois plus grande.

$58 \times 2 = 116$. 

Dans cette tablette de 200 g de chocolat, la masse de glucides est égale à 116~g.
\end{enumerate}
