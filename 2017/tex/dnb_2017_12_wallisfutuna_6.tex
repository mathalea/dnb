
\medskip

Dans un laboratoire A, pour tester le vaccin contre la grippe de la saison hivernale prochaine, on a injecté la même souche de virus à 5 groupes comportant 29 souris chacun.

3 de ces groupes avaient été préalablement vaccinés contre ce virus.

Quelques jours plus tard, on remarque que :

\setlength\parindent{10mm}
\begin{itemize}
\item[$\bullet~~$] dans les $3$ groupes de souris vaccinées, aucune souris n'est malade ;
\item[$\bullet~~$] dans chacun des groupes de souris non vaccinées, $23$ souris ont développé la maladie.
\end{itemize}
\setlength\parindent{0mm} 

\medskip
 
\begin{enumerate}
\item 
	\begin{enumerate}
		\item En détaillant les calculs, montrer que la proportion de souris malades lors de ce test est $\dfrac{46}{145}$.
		\item Justifier sans utiliser la calculatrice pourquoi on ne peut pas simplifier cette fraction.
	\end{enumerate}	
\end{enumerate}
		
\textbf{Donnée utile} Le début de la liste ordonnée des nombres premiers est : 
		
		2,\: 3,\: 5,\: 7,\: 11,\: 13,\: 17,\: 19,\: 23,\:29.
		
Dans un laboratoire B on informe que $\dfrac{140}{870}$ des souris ont été malades.

\begin{enumerate}[resume]		
\item  
	\begin{enumerate}
		\item Décomposer $140$ et $870$ en produit de nombres premiers.
		\item En déduire la forme irréductible de la proportion de souris malades dans le laboratoire B.
	\end{enumerate}
\end{enumerate}

\vspace{0,5cm}

