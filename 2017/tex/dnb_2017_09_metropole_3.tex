
\medskip

Voici trois figures différentes, aucune n'est à l'échelle indiquée dans l'exercice :

\begin{center}
\begin{tabularx}{\linewidth}{*{3}{>{\centering \arraybackslash}X}}
\psset{unit=0.8cm}
\begin{pspicture}(4.9,4.3)
\psline[linecolor=blue,linewidth=1.5pt](0,0)(0,4.3)(3.8,4.3)(3.8,1)(1,1)(1,3.3)(2.7,3.3)(2.7,2.2)(2.2,2.2)
\end{pspicture}&
%\psset{unit=0.8cm}
%\begin{pspicture}(4.9,4.3)
%\psline[linecolor=blue](4.9,2.3)(3.6,0)(0.9,0)(0,2.2)(1,3.4)(2.4,3.4)(3,2.3)(2.7,1.7)(2,1.7)
%\end{pspicture}
\psset{unit=0.5cm}
\begin{pspicture}(-4.2,-3.5)(4.7,2.7)
\psStartPoint(0,0)
\psVector[linecolor=blue,
arrows=-,linewidth=1.5pt](1,0)(1;60)(2;120)(-3,0)(3;-120)(3;-60)(4,0)(4;60)
\end{pspicture}
&\psset{unit=0.9cm}
\begin{pspicture}(4.9,4.3)
\psline[linecolor=blue,linewidth=1.5pt ](0.5,0.5)(0.5,3.4)(3.4,3.4)(3.4,1.4)(1.2,1.4)(1.2,2.7)(2.7,2.7)(2.7,1.9)(1.9,1.9)
\end{pspicture}\\
figure 1 &figure 2 &figure 3\\
\end{tabularx}
\end{center}

Le programme ci-dessous contient une variable nommée \og \textbf{longueur} \fg.

\medskip

\begin{minipage}{.4\linewidth}
Script
\begin{scratch}
    \blockinit{Quand \greenflag est cliqué}
    \blocklook{cacher}
    \blockmove{aller à x: \ovalnum{0} y: \ovalnum{0}}
    \blockmove{s’orienter à \ovalnum{90\selectarrownum} degrés}
    \blockvariable{mettre \selectmenu{longueur} à   \ovalnum{30} }
    \blockpen{effacer tout}
    \blockpen{mettre la taille du stylo à \ovalnum{3}}
    \blockpen{stylo en position d'écriture}        
        \blockrepeat{répéter \ovalnum{2} fois}
    {
        \blockmoreblocks{un tour}
        \blockvariable{ajouter  à \selectmenu{longueur}    \ovalnum{30} }
    }
\end{scratch}
\end{minipage}
\begin{minipage}{.2\linewidth}
~
\end{minipage}
\begin{minipage}{.4\linewidth}
{Le~bloc~:~\textbf{un~tour}}

\begin{scratch}
	\initmoreblocks{Définir \namemoreblocks{un tour}}
		\blockrepeat{répéter \ovalnum{2} fois}
		{\blockmove{avancer de \ovalvariable{longueur} }
		\blockmove{tourner \turnleft{} de \ovalnum{90} degrés}
		}		
	\blockvariable{ajouter à  \selectmenu{longueur}  \ovalnum{30} }
		\blockrepeat{répéter \ovalnum{2} fois}
		{
		\blockmove{avancer de \ovalvariable{longueur} }
		\blockmove{tourner \turnleft{} de \ovalnum{90} degrés}
		}
\end{scratch}
\end{minipage}

\medskip

On rappelle que l'instruction \raisebox{-2.3ex}{\begin{scratch} \blockmove{s’orienter à \ovalnum{90\selectarrownum} degrés}  \end{scratch}} signifie que l'on s'oriente vers la droite avec le stylo.

\medskip

\begin{enumerate}
\item 
	\begin{enumerate}
		\item Dessiner la figure obtenue avec le bloc \og un tour\fg{} donné dans le cadre de droite ci-dessus, pour une longueur de départ égale à 30, étant orienté vers la droite avec le stylo, en début de tracé. On prendra 1~cm pour 30~unités de longueur, c'est-à-dire 30 pixels.
		\item Comment est-on orienté avec le stylo après ce tracé ? (aucune justification n'est demandée)
	\end{enumerate}
\item Laquelle des figures 1 ou 3 le programme ci-dessus permet-il d'obtenir ? Justifier votre réponse.
\item Quelle modification faut-il apporter au bloc \og \textbf{un tour}\fg{} pour obtenir la figure 2 ci-dessus ?
\end{enumerate}

\vspace{0,5cm}

