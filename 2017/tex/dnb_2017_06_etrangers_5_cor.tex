
\medskip

%Sarah vient de faire construire une piscine dont la forme est un pavé droit de 8 m de longueur, 4 m de largeur et 1,80 m de profondeur. Elle souhaite maintenant remplir sa piscine. Elle y installe donc son tuyau d'arrosage. 
%
%Sarah a remarqué qu'avec son tuyau d'arrosage, elle peut remplir un seau de 10 litres en 18 secondes.
%
%Pour remplir sa piscine, un espace de 20 cm doit être laissé entre la surface de l'eau et le haut de la piscine. 
%
%Faut-il plus ou moins d'une journée pour remplir la piscine ? Justifier votre réponse.
Le débit du tuyau est égal à $\dfrac{10}{18} = \dfrac{5}{9}$~l/s.

Le volume à remplir est celui d'un pavé de 8 m sur 4 m d'une hauteur de 1,6~m, donc égal à :

$8 \times 4 \times 1,6 = 51,2$~m$^3$ ou $\np{51200}$~dm$^3$ ou $\np{51200}$~l.

Le temps nécessaire est donc égal à :

$\dfrac{\np{51200}}{\frac{5}{9}} = \dfrac{\np{51200} \times 9}{5} = \np{92160}$~s soit $\dfrac{\np{92160}}{60} = \np{1536}$~min ou $\dfrac{\np{1536}}{60} = 25,6$~heures, donc plus d'une journée.
\bigskip

