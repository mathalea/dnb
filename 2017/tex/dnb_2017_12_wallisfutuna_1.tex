
\medskip

\emph{Cet exercice est un questionnaire à choix multiple (QCM). Pour chaque question, une seule des quatre réponses proposées est exacte. Sur la copie, indiquer le numéro de la question et la réponse choisie On ne demande pas de justifier. Aucun point ne sera enlevé en cas de mauvaise réponse.}

\emph{Indiquer sur la copie le numéro de la question et recopier la réponse exacte.}

\medskip

\begin{tabularx}{\linewidth}{|c|m{4cm}|*{4}{>{\centering \arraybackslash}X|}}\hline
\multicolumn{2}{|c|}{~}&A &B &C& D\\ \hline
1&Dans un club sportif, $\dfrac{1}{8}$ des adhérents ont plus de 42 ans et $\dfrac{1}{4}$
 ont moins de 25 ans. 

La proportion d'adhérents ayant un âge de 25 à 42 ans est \ldots&$\dfrac{1}{6}$&$\dfrac{3}{8}$&$\dfrac{5}{8}$&$\dfrac{1}{8}$\\ \hline
2&Une télé coûte \np{46000} F. Son prix est augmenté de 20\,\%. Je paierai donc \ldots&\np{36800} F &\np{55200} F &\np{46020} F &\np{48000} F\\ \hline
3 &On triple la longueur de l'arête d'un cube. Son volume est \ldots&inchangé &multiplié par 3 &multiplié par 9 &multiplié par 27\\ \hline
4 &Les nombres 23 et 37& sont premiers&sont divisibles par 3&n'ont aucun diviseur commun
&sont  pairs\\ \hline
5&L'image de 3 par la fonction $f$ définie par 

$f(x) = x^2 - 2x + 7$ est \ldots &10 &4 &22 &$- 8$\\ \hline
\end{tabularx}

\vspace{0,5cm}

