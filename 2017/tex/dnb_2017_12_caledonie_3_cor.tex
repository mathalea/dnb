
\medskip

%\emph{Magic The Gathering} est un jeu de cartes. Aurel voudrait participer à un tournoi le week-end prochain. Il décide de s'acheter de nouvelles cartes sur Internet.
%
%L'annexe 2  est une capture d'écran d'un tableau obtenu à l'aide d'un tableur. Il permet de calculer le coût des achats d'Aurel.
%
%\medskip

\begin{enumerate}
\item %Quelle formule peut-on saisir dans la cellule D2 avant de l'étirer sur la colonne D ?
$=\text{B}2 * \text{C}2$.
\item %Sur l'annexe 2, compléter chaque cellule de la colonne D par les prix obtenus.
Voir l'annexe.
\item %Aurel range ses cartes dans une boîte à chaussures. Il les place à plat au fond de la boîte comme indiqué sur la figure de façon à former des piles.

%On dispose des informations suivantes :
%
%\begin{center}
%\begin{tabularx}{\linewidth}{*{2}{>{\centering \arraybackslash}X}}
%Dimensions de la boîte &Dimensions de la carte\\
%\psset{unit=1cm}
%\begin{pspicture}(5,5)
%\psframe(0,0.3)(2.3,1.6)
%\psline(2.3,0.3)(4.6,2.6)(4.6,3.9)(2.3,1.6)
%\psline(4.6,3.9)(2.3,3.9)(0,1.6)
%\psline[linestyle=dotted](0,0.3)(2.3,2.6)(2.3,3.9)
%\psline[linestyle=dotted](2.3,2.6)(4.6,2.6)
%\pspolygon[fillstyle=solid,fillcolor=lightgray](2.3,0.3)(2.9,0.9)(2.4,0.9)(1.8,0.3)
%\pspolygon[fillstyle=solid,fillcolor=lightgray](2.9,0.9)(3.5,1.5)(3,1.5)(2.4,0.9)
%\pspolygon[fillstyle=solid,fillcolor=lightgray](1.8,0.3)(2.4,0.9)(1.9,0.9)(1.3,0.3)
%\psframe(0,0.3)(2.3,1.6)
%\psline{<->}(0,0.2)(2.3,0.2)\uput[d](1.15,0.2){24,5 cm}
%\psline{<->}(2.5,0.3)(4.8,2.7)\rput{46}(3.9,1.5){37,5 cm}
%\psline{<->}(4.8,2.6)(4.8,3.9)\rput{90}(5,3.25){17 cm}
%\end{pspicture}&
%\psset{unit=1cm}
%\begin{pspicture}(-2,0)(5,5)
%\psframe(0.3,0.3)(2.6,4.1)
%\rput(1.45,3.5){\textbf{\scriptsize Magic The Gathering}}
%\psline{<->}(0.3,0.1)(2.6,0.1)\uput[d](1.45,0.1){6,2 cm}
%\psline{<->}(2.8,0.3)(2.8,4.1)\rput{90}(3,2.2){8,7 cm}
%\end{pspicture}
%\\
%\end{tabularx}
%\end{center}
%
%\medskip

%Quel est alors le nombre maximum de piles que peut contenir cette boîte ? Justifier.
Dans la longueur il peut placer $\dfrac{37,5}{8,7} \approx 4,3$, donc 4 piles au plus et, 
dans la largeur $\dfrac{24,5}{6,2} \approx 3,95$ donc 3 piles au plus.

Il pourra donc faire au plus $4 \times 3 = 12$ piles de cartes.

On peut vérifier qu'en tournant les cartes il ferait $6 \times 2 = 12$ piles également.
\end{enumerate}
\begin{center}
\textbf{\large À RENDRE AVEC LA COPIE}

\bigskip

\textbf{\large ANNEXE 2}

\medskip

\begin{tabularx}{\linewidth}{|c|*{4}{>{\centering \arraybackslash}X|}}\hline
	&A 							&B 							&C		&D\\ \hline
1	&\textbf{Nouvelles cartes}	&\textbf{Quantité}&\textbf{Prix unitaire (en F)}& \textbf{Prix (en F)}\\ \hline
2	&\psset{unit=0.8cm}
\begin{pspicture}(2.2,3.2)
\psframe(2.2,3)\rput(1.1,2.6){\tiny{\textbf{Magic The Gathering 1}}}
\end{pspicture}					&2							&322	    &\red 644\\ \hline
3	&\psset{unit=0.8cm}
\begin{pspicture}(2.2,3.2)
\psframe(2.2,3)\rput(1.1,2.6){\tiny{\textbf{Magic The Gathering 2}}}
\end{pspicture}					&3							&112		&\red 336\\ \hline
4	&\psset{unit=0.8cm}
\begin{pspicture}(2.2,3.2)
\psframe(2.2,3)\rput(1.1,2.6){\tiny{\textbf{Magic The Gathering 3}}}
\end{pspicture}
								&4							&480		&\red \np{1920}\\ \hline
5	&\multicolumn{3}{r|}{Montant de la commande :}						&\np{2900}\\ \hline
6	&\multicolumn{3}{r|}{Frais de transport : + 10\,\% de la commande}	&\red 290\\ \hline
7	&\multicolumn{3}{r|}{Montant total :} 								&\red \np{3190}\\ \hline
\end{tabularx}
\end{center}
\vspace{0,5cm}

