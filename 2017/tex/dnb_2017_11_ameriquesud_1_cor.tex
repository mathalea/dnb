
\medskip

\begin{enumerate}
\item La probabilité qu’une boule porte le numéro $7$ est égale $\dfrac{4}{8}$ car l’urne contient $4$ boules portant le numéro $7$ sur un total de 8 boules.
\item Il y a 3 boules portant un numéro pair et 5 boules portant un numéro impair.

Wacim n'a pas plus de chance de tirer un numéro pair qu'un numéro impair car il y a moins de boules portant un numéro pair qu'un numéro impair. Il a donc tort.
\item Wacim a tiré la boule portant le numéro $5$ et la garde : il ne reste donc que $7$ boules dans l’urne;

La probabilité que la boule tirée par Baptiste porte le numéro $7$ est égale à $\dfrac{4}{7}$ car l’urne contient $4$ boules portant le numéro $7$ sur $7$ boules.
\end{enumerate}

\vspace{0.5cm}

