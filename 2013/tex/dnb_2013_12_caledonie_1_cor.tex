
\medskip
 
%Cet exercice est un questionnaire à choix multiples (QCM). Pour chaque question, une seule des trois réponses proposées est exacte. Sur la copie, indiquer le numéro de la question et recopier, sans justifier, la réponse choisie. Aucun point ne sera enlevé en cas de mauvaise réponse :
%
%\begin{center} 
%
%\begin{tabularx}{\linewidth}{|m{3.5cm}|*{3}{>{\centering \arraybackslash}X|}}\hline
%\multicolumn{1}{|c|}{}&\multicolumn{3}{c|}{Réponses proposées}\\ \cline{2-4}
%\multicolumn{1}{|c|}{Question posée}&A&B&C\\ \hline 
%\textbf{1.}\: Une fourmi se déplace à :&4 km/s &4 m/s &4 cm/s\\ \hline 
%\textbf{2.}\: La distance de la Terre à la Lune est :&$3,844 \times 10^5$ \small km &$3,844 \times 10^{-5}$ \small km & $3,844$ \small km \\ \hline
%\textbf{3.}\: Une écriture simplifiée de $\frac{125}{625}$	est :\rule[-3mm]{0mm}{9mm}&$\dfrac{1}{6}$ &$\dfrac{1}{5}$ &$125,625$\\ \hline
%\textbf{4.}\: $\sqrt{12}$ est égal à :\rule[-3mm]{0mm}{9mm}& 6 &$4\sqrt{3}$&$2\sqrt{3}$\\ \hline
%\end{tabularx}
%\end{center}
\begin{enumerate}
\item Les deux premières réponses sont invraisemblables : réponse C.
\item Les deux dernières réponses sont invraisemblables : réponse A.
\item $\dfrac{125}{625} = \dfrac{25 \times 5}{25 \times 25} = \dfrac{5}{25} = \dfrac{1}{5} = 0,2$. Réponse B.
\item $\sqrt{12} = \sqrt{4 \times 3} = \sqrt{4} \times \sqrt{3} = 2\sqrt{3}$. Réponse C.
\end{enumerate}

\bigskip

