
\medskip
 
%La 24\up{e} édition du Marathon International de Moorea a eu lieu le 18 février 2012.
% 
%Des coureurs de différentes origines ont participé à ce marathon :
%
%\setlength\parindent{6mm} 
%\begin{itemize}
%\item[$\bullet~~$] 90 coureurs provenaient de Polynésie Française dont 16 étaient des femmes 
%\item[$\bullet~~$] 7 coureurs provenaient de France Métropolitaine dont aucune femme, 
%\item[$\bullet~~$] 6 provenaient d'Autriche dont 3 femmes, 
%\item[$\bullet~~$] 2 provenaient du Japon dont aucune femme, 
%\item[$\bullet~~$] 11 provenaient d'Italie dont 3 femmes, 
%\item[$\bullet~~$] 2 provenaient des Etats-Unis dont aucune femme 
%\item[$\bullet~~$] Un coureur homme était Allemand.
%\end{itemize}
%\setlength\parindent{0mm}
%
%\medskip
 
\begin{enumerate}
\item Compléter le tableau ci-dessous à l'aide des données de l'énoncé. 

\medskip
\begin{tabularx}{\linewidth}{|*{8}{>{\centering \arraybackslash \footnotesize}X|}}\cline{2-8}
\multicolumn{1}{c|}{~}	&Polynésie	&Métropole	&Autriche	&Japon	&Italie	&États-Unis	&Allemagne\\ \hline
Femme					&16			&0			&3			&0		&3		&0			&0\\ \hline
\end{tabularx}
\medskip
 
\item %Combien de coureurs ont participé à ce marathon ?
$90 + 7 + 6 + 2 + 11 + 2 + 1 = 119$. 
\item %Parmi les participants à ce marathon, quel pourcentage les femmes polynésiennes représentent-elles ? Arrondir au dixième près.
Ob a $\dfrac{16}{119} \times 100 \approx 13,4\,\%$ au dixième près. 
\hspace*{-1cm}%À la fin du marathon, on interroge un coureur au hasard.
 
\item %Quelle est la probabilité que ce coureur soit une femme Autrichienne ?
$\dfrac{3}{119}$. 
\item %Quelle est la probabilité que ce coureur soit une femme ? 
$\dfrac{16 + 3 + 3}{119} = \dfrac{22}{119}$
\item %Quelle est la probabilité que ce coureur soit un homme Polynésien ?
$\dfrac{90 - 16}{119} = \dfrac{74}{119}$. 
\item %Quelle est la probabilité que ce coureur ne soit pas Japonais ? 
$\dfrac{119 - 2}{119} = \dfrac{117}{119}$
\item %Vaitea dit que la probabilité d'interroger un coureur homme Polynésien est exactement trois fois plus grande que celle d'interroger un coureur homme non Polynésien.
 
%A-t-il raison? Expliquer pourquoi.
Probabilité d'interroger un coureur homme Polynésien : $\dfrac{74}{119}$  (voir au-dessus) ;

Probabilité d'interroger un coureur homme non Polynésien : $\dfrac{7 + 3 + 2 + 8  + 2 + 1}{119} = \dfrac{23}{119}$.

Or $3 \times \dfrac{23}{119} = \dfrac{69}{119} \ne \dfrac{74}{119}$. Vaitea a tort.
\end{enumerate}

\bigskip

