
\medskip 

%Chaque année les professeurs de mathématiques de la Nouvelle-Calédonie organisent le Rallye maths des collégiens. Pour l'année 2013, l'équipe organisatrice est confrontée à un problème de répartition des cadeaux des trois premières classes figurant au classement final. 
%
%\medskip

\begin{enumerate}
\item %Avec 292 crayons, 219 règles et 73 calculatrices, combien de lots identiques peut-on constituer pour en avoir le plus possible et en utilisant tout le stock ? Justifier la réponse.
On voit que : $292 = 73 \times 4\:;\: 219 = 73 \times 3$
 et $73 = 73 \times 1$.
 
73 est donc le PGCD de 292, 219 et 73. On peut donc faire 73 lots identiques 
\item %Quelle serait alors la composition de chacun des lots ? Justifier la réponse.
Chaque lot se compose de  4 crayons, 3 règles et 1 calculatrice. 
\item %On suppose que le nombre de lots est de 73 lots. Sachant que l'effectif total de ces trois classes est de 80 élèves, quelle est la probabilité qu'un élève choisi au hasard ne reçoive aucun lot ? 
Il y aura $80 - 73 = 7$ élèves qui n'auront pas de lot.

La probabilité qu'un élève choisi au hasard ne reçoive aucun lot est donc égale à $\dfrac{7}{80} = \dfrac{1,75}{20} = \dfrac{8,75}{100} = 8,75\,\% = 0,0875$.
\end{enumerate}

\vspace{0,5cm}

