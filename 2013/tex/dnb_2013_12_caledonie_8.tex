
\bigskip
 
Dans un jeu vidéo on a le choix entre trois personnages : un guerrier, un mage et un chasseur. 

La force d'un personnage se mesure en points.
 
Tous les personnages commencent au niveau $0$ et le jeu s'arrête au niveau $25$. 

Cependant ils n'évoluent pas de la même façon :

\setlength\parindent{8mm} 
\begin{itemize}
\item[\decoone~~] Le guerrier commence avec 50 points et ne gagne pas d'autre point au cours du jeu. 
\item[\decoone~~] Le mage n'a aucun point au début mais gagne 3 points par niveau. 
\item[\decoone~~] Le chasseur commence à 40 points et gagne 1 point par niveau. 
\end{itemize}
\setlength\parindent{0mm}

\medskip

\begin{enumerate}
\item Au début du jeu, quel est le personnage le plus fort? Et quel est le moins fort ? 
\item Compléter le tableau de l'annexe 2 en page 5. 
\item À quel niveau le chasseur aura-t-il autant de points que le guerrier? 
\item Dans cette question, x désigne le niveau de jeu d'un personnage. 

Associer chacune des expressions suivantes à l'un des trois personnages: chasseur, mage ou guerrier :
 
\setlength\parindent{8mm} 
\begin{itemize}
\item[$\bullet~~$] $f(x) = 3x$ ; 
\item[$\bullet~~$] $g(x) = 50$ ; 
\item[$\bullet~~$] $h(x) = x + 40$.
\end{itemize}
\setlength\parindent{0mm}
 
\item Dans le repère de l'annexe 2, la fonction $g$ est représentée. 

Tracer les deux droites représentant les fonctions $f$ et $h$. 
\item Déterminer à l'aide du graphique, le niveau à partir duquel le mage devient le plus fort. 
\end{enumerate} 
\begin{center}

{\large \textbf{ANNEXE 2}}

\bigskip

\begin{tabularx}{\linewidth}{|*{7}{>{\centering \arraybackslash}X|}}\hline
Niveau						&0	&1	&5	&10	&15	&25\\ \hline
\small Points du Guerrier	&50	&50	&	&	&	&\\ \hline
\small Points du Mage		&0	&3	&	&	&	&\\ \hline
\small Points du Chasseur	&40	&41	&	&	&	&\\ \hline
\end{tabularx}

\vspace{1cm}

\psset{xunit=0.5cm,yunit=0.09cm}
\begin{pspicture}(-1,-5)(27,75)
\multido{\n=0+1}{28}{\psline[linestyle=dashed,linewidth=0.3pt](\n,0)(\n,75)}
\multido{\n=0+5}{16}{\psline[linestyle=dashed,linewidth=0.3pt](0,\n)(27,\n)}
\psline[linewidth=1.5pt](0,50)(27,50)
\psaxes[linewidth=1.25pt,Dy=5]{->}(0,0)(0,0)(27,75)
\psaxes[linewidth=1.25pt,Dy=5](0,0)(0,0)(27,75)
\uput[u](25.5,0){Niveau}
\uput[r](0,72.5){Points}
\uput[dl](0,0){O}
\end{pspicture}
\end{center}
