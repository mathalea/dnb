
\medskip 

On considère les deux programmes de calcul suivants : 

\medskip

\begin{tabularx}{\linewidth}{|m{6cm}|X|}\hline
\textbf{Programme A}&   \textbf{Programme B}\\   
$\bullet~~$ Choisir un nombre de départ &$\bullet~~$Choisir un nombre de départ \\  
$\bullet~~$  Soustraire 1 au nombre choisi&$\bullet~~$Calculer le carré du nombre choisi  \\ 
$\bullet~~$  Calculer le carré de la différence obtenue&$\bullet~~$Ajouter 1 au résultat   \\
$\bullet~~$  Ajouter le double du nombre de départ au résultat&$\bullet~~$Écrire le résultat obtenu\\   
$\bullet~~$  Écrire le résultat obtenu & \\ \hline
\end{tabularx}

\medskip

\begin{enumerate}
\item Montrer que, lorsque le nombre de départ est 3, le résultat obtenu avec le programme A est 10. 
\item Lorsque le nombre de départ est 3, quel résultat obtient-on avec le programme B ? 
\item Lorsque le nombre de départ est $- 2$, quel résultat obtient-on avec le programme A ? 
\item Quel(s) nombre(s) faut-il choisir au départ pour que le résultat obtenu avec le programme B soit 5 ? 
\item Henri prétend que les deux programmes de calcul fournissent toujours des résultats identiques. 
A-t-il raison ? Justifier la réponse. 
\end{enumerate}

\bigskip

