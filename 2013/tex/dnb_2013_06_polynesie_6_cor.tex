
\medskip
 
%Sur un parking, une commune veut regrouper 6 conteneurs à déchets du même modèle A ou B. Les deux modèles sont fabriqués dans le même matériau qui a partout la même épaisseur.
%
%\begin{center}
%\psset{unit=1cm}
%\begin{pspicture}(10,5.5) 
%\rput(1.5,5){le conteneur A} 
%\rput(6.5,5){le conteneur B}
%\pspolygon(0.3,0.4)(2,0.3)(2.9,0.6)(2.8,4.3)(1.9,4)(0.2,4.1)%ABCGFEA
%\psline(2.8,4.3)(1.1,4.4)(0.2,4.1)%GHE
%\psline(2,0.3)(1.9,4)%BF
%\psline[linestyle=dashed](0.3,0.4)(1.2,0.7)(1.1,4.4)%ADH
%\psline[linestyle=dashed](1.2,0.7)(2.9,0.6)
%\psline(5.5,1.2)(5.5,3.4)
%\psline(7.7,1.2)(7.7,3.4)
%\psarc(6.6,1.2){1.1}{-180}{0}
%\psarc(6.6,3.4){1.1}{0}{-180}
%\scalebox{.99}[0.3]{\psarc[linewidth=2pt](6.65,3.2){1.075}{180}{0}}%
%\scalebox{.99}[0.3]{\psarc[linewidth=2pt,linestyle=dashed](6.65,3.2){1.075}{0}{180}}%
%\scalebox{.99}[0.3]{\psarc[linewidth=2pt](6.65,12.){1.075}{180}{0}}%
%\scalebox{.99}[0.3]{\psarc[linewidth=2pt,linestyle=dashed](6.65,12.){1.075}{0}{180}}%
%\end{pspicture} 
%\end{center}
%
%\setlength\parindent{6mm}
%\begin{itemize}
%\item le conteneur A est un pavé droit à base carrée de côté 1~m, et de hauteur 2~m 
%\item le conteneur B est constitué de deux demi-sphères de rayon $0,58$~m et d'un cylindre de même rayon et de hauteur $1,15$~m
%\end{itemize}
%\setlength\parindent{0mm}
%
%\bigskip
 
\begin{enumerate}
\item 
	\begin{enumerate}
		\item %Vérifie que les 2 conteneurs ont pratiquement le même volume. 
Volume du conteneur A : $1 \times 1 \times 2 = 2$~m$^3$.
		
Volume du conteneur B : $\pi \times 0,58^2 \times 1,15 + \dfrac{4}{3}\pi \times 0,58^3 \approx 2,03$~m$^3$.
		\item %Quels peuvent être les avantages du conteneur A ?
A est plus facile à fabriquer, plus facile à nettoyer, plus stable que B.
	\end{enumerate} 
\item %On souhaite savoir quel est le conteneur le plus économique à fabriquer. 
	\begin{enumerate}
		\item %Calcule l'aire totale des 6 faces du conteneur A.
A a deux faces carrées de 1~m$^2$, et quatre  faces   de 2~m$^2$, soit une aire totale de 10~m$^2$.
		\item %Vérifie que, pour le conteneur B, l'aire totale, arrondie à 0,1~m$^2$ près, est 8,4 m$^2$. 
L’aire de la sphère (réunion des demi-sphères) est égale à $4 \pi \times 0,58^2 \approx 4,227 \approx 4,2$~m$^2$.

L’aire latérale du cylindre est égale à 

$2 \pi \times 0,58 \times 1,15 \approx 4,191$~m$^2$.

L’aire du conteneur B est donc à peu près 4,227 + 4,191 = 8,418 soit environ 8,4~m$^2$. 
		\item %Quel est le conteneur le plus économique à fabriquer ? Justifie ta réponse.
Les deux conteneurs sont faits avec le même matériau de même épaisseur. Il faut donc moins de matériau pour fabriquer le conteneur B.
	\end{enumerate}
\end{enumerate}

%\medskip
% 
%\textbf{Formulaire :}
%
%$b$ = base  ; $c$ = côté ;  $L$ = longueur ;  $l$ = largeur  ; $h$ = hauteur  ; $r$ = rayon 
%
%\medskip
%\begin{tabularx}{\linewidth}{|*{3}{>{\centering \arraybackslash}X|}}\hline
%Aire d'un rectangle 	&Aire d'un carré 	&Aire d'un triangle\\ \hline 
%$L \times l$			&$c \times c$		&$\dfrac{b x h}{2}$\\ \hline 
%Aire d'un disque 		&Aire latérale d'un cylindre &Aire d'une sphère\\ \hline 
%$\pi r^2$				& $2\pi r h$		& $4\pi r^2$\\ \hline 
%Volume d'un pavé droit 	&Volume d'un cylindre &Volume d'une sphère\\ \hline
%$L \times l \times h$	&$\pi r^2 \times h$	&$\dfrac{4}{3}\pi r^3$\\ \hline
%\end{tabularx}
%
%\medskip

\bigskip

