
\medskip 

\setlength\parindent{6mm}
\begin{itemize}
\item[$\bullet~~$] On donne 6 nombres répartis dans six bulles. 
\item[$\bullet~~$] Vous devez trouver des étapes de calcul permettant d'obtenir le résultat affiché au centre. 
\item[$\bullet~~$] Vous pouvez utiliser les 4 opérations autant de fois que vous le voulez. 
\item[$\bullet~~$] Vous ne pouvez pas utiliser deux fois le même nombre (ou la même expression). 
\item[$\bullet~~$] Vous n'êtes pas obligé d'utiliser tous les nombres (ou les expressions) affichés. 
\end{itemize}
\setlength\parindent{0mm}

Prenons pour exemple la liste des nombres donnée ci-dessous, en utilisant les 4 opérations, on doit trouver 155 : 

\parbox{0.4\linewidth}{\psset{unit=1cm}

\begin{pspicture}(-2.5,-2.5)(2.5,2.5)
\pscircle[linewidth=5pt,linecolor=gray](0,0){2cm}
\multido{\n=30+60}{6}{\rput(2;\n){\pscircle[fillstyle=solid,fillcolor=gray!20](0;0){0.5cm}}}
\psellipse[fillstyle=solid,fillcolor=gray!20](0,0)(1.4,1)
\rput(0,0.5){le nombre à}
\rput(0,0){trouver est}
\rput(0,-0.5){\textbf{155}}
\rput(2;30){3} \rput(2;90){25} \rput(2;150){10} 
\rput(2;210){50} \rput(2;270){2} \rput(2;330){8} 
\end{pspicture}}\hfill
\parbox{0.5\linewidth}{On peut, par exemple, proposer les étapes de calcul suivant :

\setlength\parindent{6mm}
\begin{itemize}
\item[$\bullet~~$] $50 \times 3 = 150$
\item[$\bullet~~$] $10 \div 2 = 5$
\item[$\bullet~~$] $150 + 5 = 155$ (qui est la solution à trouver).
\end{itemize}
\setlength\parindent{0mm}}

\medskip

\begin{enumerate}
\item Avec les données de l'exemple précédent, proposer des étapes de calcul pour obtenir 367. 
\item On donne maintenant la série de nombres suivante. 

Proposer des étapes de calcul permettant d'obtenir $\dfrac{15}{6}$.
 
\begin{center}
\psset{unit=1cm}
\begin{pspicture}(-2.5,-2.5)(2.5,2.5)
\pscircle[linewidth=5pt,linecolor=gray](0,0){2cm}
\multido{\n=30+60}{6}{\rput(2;\n){\pscircle[fillstyle=solid,fillcolor=gray!20](0;0){0.5cm}}}
\psellipse[fillstyle=solid,fillcolor=gray!20](0,0)(1.4,1)
\rput(0,0.5){le nombre à}
\rput(0,0){trouver est}
\rput(0,-0.5){\textbf{15/6}}
\rput(2;30){1/7} \rput(2;90){1/3} \rput(2;150){3} 
\rput(2;210){5/3} \rput(2;270){1/2} \rput(2;330){1/3} 
\end{pspicture} 
\end{center}

\item À partir des expressions réparties dans les six bulles ci-dessous, proposer des étapes de calcul permettant d'obtenir $4x^2 + 6x - 1$. 

\begin{center}
\psset{unit=1cm}
\begin{pspicture}(-2.5,-2.5)(2.5,2.5)
\pscircle[linewidth=5pt,linecolor=gray](0,0){2cm}
\multido{\n=30+60}{6}{\rput(2;\n){\pscircle[fillstyle=solid,fillcolor=gray!20](0;0){0.5cm}}}
\psellipse[fillstyle=solid,fillcolor=gray!20](0,0)(1.4,1)
\rput(0,0.5){l'expression}
\rput(0,0){à trouver est}
\rput(0,-0.5){\boldmath $4x^2+ 6x - 1$\unboldmath}
\rput(2;30){1} \rput(2;90){$x^2$} \rput(2;150){$- 5$} 
\rput(2;210){$- 2x$} \rput(2;270){$8x$} \rput(2;330){$5x^2$} 
\end{pspicture} 
\end{center} 

\end{enumerate}

\vspace{0,5cm}

