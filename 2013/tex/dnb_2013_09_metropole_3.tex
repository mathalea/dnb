
\medskip

ABCD est un rectangle tel que AB = 30 cm et BC = 24 cm.
 
On colorie aux quatre coins du rectangle quatre carrés identiques en gris. On délimite ainsi un rectangle central que l'on colorie en noir. 

\begin{center}
\psset{unit=1cm}
\begin{pspicture}(0,-0.2)(4.8,4)
\psframe(4.8,3.8)
\psframe[fillstyle=solid,fillcolor=lightgray](0,3.8)(1.1,2.7)
\psframe[fillstyle=solid,fillcolor=lightgray](0,1.1)(1.1,0)
\psframe[fillstyle=solid,fillcolor=lightgray](3.7,3.8)(4.8,2.7)
\psframe[fillstyle=solid,fillcolor=lightgray](3.7,1.1)(4.8,0)
\psframe*(1.1,2.7)(3.7,1.1)
\uput[ul](0,3.8){A}\uput[ur](4.8,3.8){B}
\uput[dr](4.8,0){C}\uput[dl](0,0){D}
\end{pspicture}
\end{center}
 
\begin{enumerate}
\item Dans cette question, les quatre carrés gris ont tous 7 cm de côté. Dans ce cas : 
	\begin{enumerate}
		\item quel est le périmètre d'un carré gris ? 
		\item quel est le périmètre du rectangle noir ? 
	\end{enumerate}
\item Dans cette question, la longueur du côté des quatre carrés gris peut varier. Par conséquent, les dimensions du rectangle noir varient aussi.
 
Est-il possible que le périmètre du rectangle noir soit égal à la somme des périmètres des quatre carrés gris ? 
\end{enumerate} 

\vspace{0,5cm} 

