
\bigskip
 
L'épreuve du concours australien de mathématiques est divisée en trois catégories :

\setlength\parindent{8mm} 
\begin{itemize}
\item[$\bullet~~$] \og Junior \fg{} qui regroupe les classes de 5\up{e} et 4\up{e} 
\item[$\bullet~~$] \og Intermédiaire \fg{} pour les classes de 3\up{e} et 2\up{nde} 
\item[$\bullet~~$] \og Senior \fg{} avec les classes de 1\up{re} et de terminale.
\end{itemize}
\setlength\parindent{0mm} 
 
Cette année 25 établissements se sont inscrits. Plus de \np{3000}~élèves, répartis comme l'indique le tableau de l'annexe 1, ont participé à ce concours.

\medskip
 
\begin{enumerate}
\item Compléter le tableau de l'annexe 1 en page 5. Les cases barrées ne sont pas à remplir. 
\item Quel est le niveau où il y a le plus d'inscrits ? 
\item Quelle est la catégorie ayant le moins d'inscrits ? 
\item En moyenne, combien d'élèves par établissement ont participé? Arrondir à l'unité. 
\item Le tableau de l'annexe est une copie d'écran d'un tableur. 

Quelle formule faut-il écrire dans la case G5 pour obtenir l'effectif total ?
\end{enumerate}
\begin{center}
{\large \textbf{ANNEXE 1}}

\bigskip

\begin{tabularx}{\linewidth}{|c|m{2cm}|*{6}{>{\centering \arraybackslash}X|}}\hline
&A&B&C&D&E&F&G\\ \hline
1&Catégorie				&\multicolumn{2}{|c|}{Junior}&\multicolumn{2}{|c|}{Intermédiaire}&\multicolumn{2}{|c|}{Senior}\\ \hline 
2&Effectif par catégorie&\multicolumn{1}{>{\columncolor{lightgray}}c|}{\quad}&\np{1958}&\multicolumn{1}{>{\columncolor{lightgray}}c|}{\quad}&\ldots&\multicolumn{1}{>{\columncolor{lightgray}}c|}{\quad}&308\\ \hline 
3&Niveau&5\up{e}		&4\up{e}&3\up{e}&2\up{nde}&1\up{re}&Term\\ \hline 
4&Effectif par niveau	& 989&969& 638&238& 172&\ldots\\ \hline 
5&\multicolumn{6}{|c|}{Effectif total}&\ldots  \\ \hline
\end{tabularx}
\end{center}

\bigskip 

