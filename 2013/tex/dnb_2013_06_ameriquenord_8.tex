
\medskip  

\parbox{0.5\linewidth}{Les longueurs sont données en centimètres. 

ABCD est un trapèze.}\hfill
\parbox{0.48\linewidth}{\psset{unit=0.8cm}
\begin{pspicture}(7,3)
\pspolygon(0,0)(1,3)(4,3)(7,0)
\psline[linestyle=dashed](0,0)(0,3)(1,3)
\psline[linestyle=dashed](4,3)(7,3)(7,0)
\psframe(0,3)(0.2,2.8)
\psframe(7,3)(6.8,2.8)
\rput(2.5,3){o}\rput(5.5,3){o}\rput(7,1.5){o}
\uput[r](7,1.5){3}\uput[u](0.5,3){1}\uput[d](3.5,0){7}
\uput[ur](1,3){A} \uput[ur](4,3){B} \uput[dr](7,0){C} \uput[dl](0,0){D} 
\end{pspicture}} 

\medskip

\begin{enumerate}
\item 
	\begin{enumerate}
		\item Donner une méthode permettant de calculer l'aire du trapèze ABCD. 
		\item Calculer l'aire de ABCD.
	\end{enumerate} 
\item \textbf{Dans cette question, si le travail n'est pas terminé, laisser tout de même une trace de la recherche. Elle sera prise en compte dans l'évaluation.}
 
L'aire d'un trapèze $A$ est donnée par l'une des formules suivantes. Retrouver la formule juste en expliquant votre choix.
 
\begin{center}
\psset{unit=0.8cm}
\begin{pspicture}(0,-0.2)(7,3.2)
\pspolygon(0,0)(1,3)(4,3)(7,0)
\psline[linestyle=dashed](4,3)(7,3)(7,0)
\psframe(7,3)(6.8,2.8)
\uput[u](2.5,3){$b$}
\uput[d](3.5,0){$B$}
\uput[r](7,1.5){$h$} 
\end{pspicture}
\end{center}
\medskip

\begin{tabularx}{\linewidth}{*{3}{X}}
$A = \dfrac{(b . B)h}{2}$& 
$A = \dfrac{(b + B)h}{2}$& 
$A = 2(b + B)h$
\end{tabularx} 
\end{enumerate}
