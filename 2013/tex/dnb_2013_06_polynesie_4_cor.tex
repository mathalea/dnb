
\medskip

\begin{enumerate}
\item %Construis un triangle ABC rectangle en C tel que AB = 10 cm et AC = 8 cm. 
Ce triangle rectangle en C est inscrit dans un demi-cercle dont [ÆB] est un diamètre. On trace donc le milieu de ce diamètre (tracé de la médiatrice), puis un demi-cercle de diamètre [AB] ; le cercle de centre A et de rayon 8 coupe le demi-cercle au point C.
\begin{center}
\psset{unit=0.75cm}
\begin{pspicture}(-5,-0.5)(5,6)
%\psgrid
\psarc(0,0){5}{0}{180}
\psarc(-5,0){8}{30}{50}
\pspolygon(-5,0)(5,0)(1.4,4.8)
\uput[l](-5,0){A} \uput[r](5,0){B}
\uput[ur](1.4,4.8){C}
\psdots[dotstyle=+,dotangle=45,dotscale=1.6](-3,0)(-3.7,1)(2.1,3.85)
\uput[d](-3,0){M}\uput[ul](-3.7,1){E}\uput[u](2.1,3.85){F}
\psline(-3,0)(-3.7,1)
\psline(-3,0)(3.4,4.8)
\end{pspicture}
\end{center}
\item %Calcule la longueur BC (en justifiant précisément).
Le théorème de Pythagore permet d’écrire :

$\text{AB}^2 =  \text{AC}^2 + \text{CB}^2$, d’où $\text{CB}^2 =  \text{AB}^2 - \text{AC}^2 = 10^2 - 8^2 = 100 - 64 = 36$, d’où CB $ = \sqrt{36} = 6$~cm.
\item 
	\begin{enumerate}
		\item %Place le point M de l'hypoténuse [AB] tel que AM = $2$~cm. 
Voir sur la figure.
		\item %Trace la perpendiculaire à [AC] passant par M. Elle coupe [AC] en E. 
		\item %Trace la perpendiculaire à [BC] passant par M. Elle coupe [BC] en F. 
		\item %À l'aide des données de l'exercice, \textbf{recopie sur ta copie} la proposition que l'on peut directement utiliser pour prouver que le quadrilatère MFCE est un rectangle.
Le quadrilatère MFCE a trois (et donc quatre) angles droits : c’est un rectangle. (proposition 3)
	\end{enumerate}

\medskip
	 
%\textbf{Proposition 1 :} Si un quadrilatère a 4 angles droits alors c'est un rectangle. 
%
%\textbf{Proposition 2 :} Si un quadrilatère est un rectangle alors ses diagonales ont la même longueur. 
%
%\textbf{Proposition 3 :} Si un quadrilatère a 3 angles droits alors c'est un rectangle. 

\end{enumerate}

\bigskip

