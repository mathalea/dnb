
\medskip  

Un éleveur a acheté $40$ m de grillage; il veut adosser un enclos rectangulaire à sa grange, contre un mur de $28$~m de long. 
 
\begin{center}
\psset{unit=1cm}
\begin{pspicture}(8,5.5)
\psframe[fillstyle=vlines](0,4.5)(8,5) \uput[u](4,5){mur}
\psline(0.5,4.5)(0.5,0.5)(7.5,0.5)(7.5,4.5)
\rput{90}(0.1,2.5){grillage}\rput{90}(7.7,2.5){grillage}
\rput(4,0.3){grillage}
\uput[d](4,0.1){$x$}\uput[r](7.9,2.5){$y$}
\psline{<->}(0.5,0.1)(7.5,0.1)
\psline{<->}(7.9,0.5)(7.9,4.5)
\end{pspicture}
\end{center}

Il souhaite offrir ainsi le maximum de place à ses brebis en utilisant le grillage. 

\begin{enumerate}
\item 
	\begin{enumerate}
		\item Pour $x = 4$~m , calculer la longueur $y$, puis l'aire $A$ de l'enclos en m$^2$. 
		\item Recopier et compléter le tableau ci-dessous : 

\begin{center}
\begin{tabularx}{0.6\linewidth}{|c|*{4}{>{\centering \arraybackslash}X|}}\hline
$x$ (en m)						&4	&10	&20	&28\\ \hline   
$y$ (en m)						&	&	&	&\\ \hline             
$A \left(\text{en m}^2\right)$	&	&	&	&\\ \hline
\end{tabularx}
\end{center}

	\end{enumerate}   
\item Déterminer $y$ en fonction de $x$.

 En déduire que $A = 20x - 0,5x^2$. 
\item Voici la plage de cellules réalisées dans un tableur-grapheur qui permettra de calculer la valeur de $A$. 

\begin{center}    
\begin{tabularx}{0.4\linewidth}{|c|*{2}{>{\centering \arraybackslash}X|}}\hline    
				&Valeur de $x$	&   Valeur de $A$\\ \hline     
2				&4				&\\ \hline
3				&6				&\\ \hline
4				&8				&\\ \hline
5				&10				&\\ \hline
6				&12				&\\ \hline
7				&14				&\\ \hline
8				&16				&\\ \hline
9				&18				&\\ \hline            
11				&22				&\\ \hline         
12				&24				&\\ \hline         
13				&26				&\\ \hline      
14				&28				&\\ \hline       
\end{tabularx}
\end{center}

Quelle formule doit-il saisir dans la cellule B2 et qui pourra être étendue sur toute la colonne B ? 
\item Le graphique ci-dessous représente l'aire $A$ en fonction de la longueur $x$ compris entre 4~m et 28m. 

\begin{center}
\psset{xunit=0.25cm,yunit=0.025cm}
\begin{pspicture}(-2,-15)(30,250)
\multido{\n=0+5}{7}{\psline[linewidth=0.2pt](\n,0)(\n,250)}
\multido{\n=0+10}{26}{\psline[linewidth=0.3pt](0,\n)(30,\n)}
\multido{\n=0+50}{6}{\psline[linewidth=0.8pt](0,\n)(30,\n)}
\psaxes[linewidth=1.5pt,Dx=5,Dy=50](0,0)(30,250)
\psplot[plotpoints=5000,linewidth=1.25pt,linecolor=blue]{4}{28}{20 x mul x dup mul 0.5 mul sub}
\end{pspicture}
\end{center}

À l'aide de ce graphique répondre aux questions suivantes en donnant des valeurs approchées : 

	\begin{enumerate}
		\item Quelle est l'aire de cet enclos pour $x = 14$~m ? 
		\item Pour quelle(s) valeur(s) de $x$ l'aire de l'enclos est égale à $192$~m$^2$ ? 
		\item Pour quelle(s) valeur(s) de $x$ l'aire de l'enclos est maximale ? 

En déduire les dimensions de l'enclos pour que les brebis aient le maximum de place.
	\end{enumerate} 
\end{enumerate}

\bigskip

