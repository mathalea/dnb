
\medskip
 
On considère le programme de calcul suivant : 

\begin{center}
\begin{tabular}{|l|}\hline
$\bullet~~$ Choisir un nombre\\
$\bullet~~$ Ajouter $5$\\
$\bullet~~$ Prendre le carré de cette somme\\ \hline
\end{tabular}
\end{center}

\begin{enumerate}
\item Quel résultat obtient-on lorsqu'on choisit le nombre $3$ ? le nombre $- 7$ ? 
\item 
	\begin{enumerate}
		\item Quel nombre peut-on choisir pour obtenir $25$ ? 
		\item Peut-on obtenir $- 25$ ? Justifier la réponse.
	\end{enumerate} 
\item On appelle $f$ la fonction qui, au nombre choisi, associe le résultat du programme de calcul. 
	\begin{enumerate}
		\item Parmi les fonctions suivantes, quelle est la fonction $f$ ? 

\[\begin{array}{l l}
	x \longmapsto x^2 + 25 &	x \longmapsto (x + 5)^2\\ 
	x \longmapsto x^2 + 5& 	x \longmapsto 2(x + 5)
	\end{array}\]
	 
		\item Est-il vrai que $- 2$ est un antécédent de $9$ ? 
	\end{enumerate}		
\item
	\begin{enumerate}
		\item Résoudre l'équation $(x + 5)^2 = 25$. 
		\item En déduire tous les nombres que l'on peut choisir pour obtenir $2$5 à ce programme de calcul. 
	\end{enumerate}
\end{enumerate}

\bigskip

