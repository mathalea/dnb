
\medskip 

Un professeur de SVT demande aux 29 élèves d'une classe de sixième de faire germer des graines de blé chez eux.
 
Le professeur donne un protocole expérimental à suivre: 

\setlength\parindent{10mm}
\begin{itemize}
\item mettre en culture sur du coton dans une boîte placée dans une pièce éclairée, de température entre 20~\degres{} et 25~\degres C ; 
\item arroser une fois par jour ; 
\item il est possible de couvrir les graines avec un film transparent pour éviter l'évaporation de l'eau.
\end{itemize}
\setlength\parindent{0mm}
 
Le tableau ci-dessous donne les tailles des plantules (petites plantes) des 29 élèves à 10 jours après la mise en germination. 

\medskip

\renewcommand\arraystretch{1.4}
\begin{tabularx}{\linewidth}{|m{2cm}|*{11}{>{\centering \arraybackslash}X|}}\hline
Taille en cm&0 &8 &12 &14 &16 &17 &18 &19 &20 &21 &22\\ \hline 
Effectif &1 &2 &2 &4 &2 &2 &3 &3 &4 &4 &2\\ \hline
\end{tabularx}
\renewcommand\arraystretch{1}

\medskip
 
\begin{enumerate}
\item Combien de plantules ont une taille qui mesure au plus 12~cm ? 
\item Donner l'étendue de cette série. 
\item Calculer la moyenne de cette série. Arrondir au dixième près. 
\item Déterminer la médiane de cette série et interpréter le résultat. 
\item On considère qu'un élève a bien respecté le protocole si la taille de la plantule à 10 jours est supérieure ou égale à 14~cm.
 
Quel pourcentage des élèves de la classe a bien respecté le protocole ? 
\item Le professeur a fait lui-même la même expérience en suivant le même protocole. Il a relevé la taille obtenue à 10 jours de germination.
 
Prouver que, si on ajoute la donnée du professeur à cette série, la médiane ne changera pas. 
\end{enumerate}

\bigskip

