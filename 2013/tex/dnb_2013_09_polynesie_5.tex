
\medskip

Teiki se promène en montagne et aimerait connaître la hauteur d'un Pinus (ou Pin des Caraibes) situé devant lui. Pour cela, il utilise un bâton et prend quelques mesures au sol. Il procède de la façon suivante :

\medskip

\setlength\parindent{8mm}
\begin{itemize}
\item Il pique le bâton en terre, verticalement, à $12$ mètres du Pinus. 
\item La partie visible (hors du sol) du bâton mesure $2$~m. 
\item Teiki se place derrière le bâton, de façon à ce que son œil, situé à $1,60$~m au dessus du sol, voie en alignement le sommet de l'arbre et l'extrémité du bâton. 
\item Teiki marque sa position au sol, puis mesure la distance entre sa position et le bâton. Il trouve alors $1,2$~m.
\end{itemize}
\setlength\parindent{0mm}

\medskip
 
On peut représenter cette situation à l'aide du schéma ci-dessous :

\begin{center} 
\psset{unit=0.85cm}
\begin{pspicture}(13.6,7)
\psline(0,0.5)(13.6,0.5) \rput(6.8,0.25){Sol}
\psline[linewidth=1.5pt](1.6,0.5)(1.6,2.3) \pspolygon[linewidth=1.5pt](1.6,6.5)(0,2.3)(3.2,2.3)
\rput(1.6,4){Pinus}
\psline[linestyle=dashed](1.6,2.1)(12.8,2.1)(12.8,0.5)
\psline[linewidth=1.5pt](11.5,0.5)(11.5,2.65)
\psline(1.6,6.5)(12.8,2.1)
\psline[linewidth=0.5pt]{<->}(1.6,0.7)(11.5,0.7)\uput[u](6.55,0.7){12 m}
\psline[linewidth=0.5pt]{<->}(11.4,0.5)(11.4,2.6)\uput[l](11.4,1.55){2 m}
\psline[linewidth=0.5pt]{<->}(11.6,0.4)(12.8,0.4)\uput[d](12.2,0.4){1,2 m}
\psline[linewidth=0.5pt]{<->}(13.2,0.5)(13.2,2.1)\uput[l](13.2,1.25){1,60 m}
\end{pspicture}
\end{center}

Quelle est la hauteur du Pinus au-dessus du sol ? 

\bigskip

