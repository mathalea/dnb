
\medskip 

M. Cotharbet décide de monter au Pic Pointu en prenant le funiculaire\up{1}  entre la gare inférieure et la gare supérieure, la suite du trajet s'effectuant à pied. 

{\footnotesize(1) Un funiculaire est une remontée mécanique équipée de véhicules circulant sur des rails en pente.}

\begin{center} 
\psset{unit=0.9cm}
\begin{pspicture}(13.5,7.5)
\def\fenetre{\psframe(0.29,0.29)\psframe(0.31,0)(0.6,0.29)\psline(0,0)(0,0.31)(0.61,0.32)(0.61,0)}
\pspolygon[fillstyle=solid,fillcolor=lightgray](0.3,1.3)(9.8,1.3)(0.3,5.1)
\pspolygon[fillstyle=solid,fillcolor=white,linecolor=white](1.4,3.75)(1.4,1.3)(2.3,1.3)(2.3,3.75)
\pswedge[linecolor=white,fillstyle=solid,fillcolor=white](1.85,3.75){0.45}{0}{180}
\pspolygon[fillstyle=solid,fillcolor=white,linecolor=white](3.6,2.95)(3.6,1.3)(4.5,1.3)(4.5,2.95)
\pswedge[linecolor=white,fillstyle=solid,fillcolor=white](4.05,2.95){0.45}{0}{180}
\pspolygon[fillstyle=solid,fillcolor=white,linecolor=white](5.7,2.05)(5.7,1.3)(6.6,1.3)(6.6,2.05)
\pswedge[linecolor=white,fillstyle=solid,fillcolor=white](6.15,2.05){0.45}{0}{180}
\pswedge[linecolor=white,fillstyle=solid,fillcolor=white](8.35,1.3){0.45}{0}{180}
\pspolygon[fillstyle=solid,fillcolor=gray](0.3,1.3)(1.1,1.3)(1.1,4.8)(0.3,5.1)
\psline[linewidth=0.8pt](0.3,1.3)(9.8,1.3)
\rput(3.5,4.3){\fenetre}\psframe(3.5,4)(3.8,4.29)\psframe(3.8,4)(4.1,4.29)
\rput(4.11,3.99){\fenetre}\psframe(4.11,3.71)(4.41,4)\psframe(4.4,3.71)(4.7,4.)
\rput(4.72,3.68){\fenetre}\psframe(4.72,3.41)(5.02,3.7)\psframe(5.02,3.41)(5.32,3.7)
\rput(5.33,3.37){\fenetre}\psframe(5.33,3.11)(5.63,3.4)\psframe(5.63,3.11)(5.93,3.4)
\psframe(5.93,3.11)(6.23,3.4)\psframe(6.23,3.11)(6.53,3.4)
\psline(6.53,3.11)(6,2.9)(3.5,3.95)(3.5,4.95)(5.93,3.7)
\rput(2.2,6.8){Pic Pointu (altitude \np{1165} m)} 
\rput(4.3,5.8){Gare supérieure (altitude \np{1075} m)} 
\rput(7.5,4.5){Funiculaire} 
\rput(11,3){Gare inférieure (altitude 415 m)} 
\rput(10.5,6){Sur le dessin ci-contre, les points}   
\rput(9.5,5.6){I, L et K sont alignés,}
\rput(9.2,5.2){ainsi que I, S et J.} 
\uput[ul](0.3,5.1){J} \uput[u](1.1,4.82){S} \uput[r](9.8,1.3){I} 
\uput[d](1.1,1.3){L} \uput[d](0.3,1.3){K}
\psline{<->}(1.1,0.7)(9.8,0.7)
\uput[d](5.45,0.7){880 m}
\psline{->}(1.8,6.6)(0.3,5.1) 
\psline{->}(3.8,5.6)(1.1,4.8)
\psline{->}(7.4,4.3)(5.3,4)
\psline{->}(10.7,2.8)(9.8,1.3)
\end{pspicture} 
\end{center}
 
\begin{enumerate}
\item À l'aide des altitudes fournies, déterminer les longueurs SL et JK. 
\item  
	\begin{enumerate}
		\item Montrer que la longueur du trajet SI entre les deux gares est \np{1100}~m. 
		\item Calculer une valeur approchée de l'angle $\widehat{\text{SIL}}$. On arrondira à un degré près.
 	\end{enumerate} 
\item Le funiculaire se déplace à la vitesse moyenne constante de 10 km.h$^{-1}$, aussi bien à la montée qu'à la descente. 

Calculer la durée du trajet aller entre les deux gares. On donnera le résultat en min et s. 
\item Entre la gare supérieure et le sommet, M. Cotharbet effectue le trajet en marchant. 

Quelle distance aura-t-il parcourue à pied ? 
\end{enumerate}

\bigskip

