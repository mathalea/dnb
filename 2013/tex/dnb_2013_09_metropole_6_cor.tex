
\medskip

%On souhaite construire une structure pour un skatepark, constituée d'un escalier de six marches identiques permettant d'accéder à un plan incliné dont la hauteur est égale à 96 cm. 
%Le projet de cette structure est présenté ci-dessous. 
%Schéma
%
%\begin{center}
%\psset{unit=0.75cm,arrowsize=4pt 2}
%\begin{pspicture}(0,-0.5)(16,3.5)
%\pspolygon(0,0)(0,1.8)(2.2,3.1)(3.3,3.1)(7.1,1.4)(4.9,0)
%\psline(0,1.8)(1.1,1.8)(3.3,3.1)
%\psline(2.2,2.4)(5.95,0.65)
%\psline(3.4,0)(3.4,0.3)(2.9,0.3)(2.9,0.6)(2.4,0.6)(2.4,0.9)(2,0.9)(2,1.2)(1.5,1.2)(1.5,1.5)(1.1,1.5)(1.1,1.8)
%\psline(3.4,0)(4.5,0.7)(4.5,1)(3.4,0.3)
%\psline(2.9,0.3)(4,1)(4,1.3)(2.9,0.6)
%\psline(2.4,0.6)(3.5,1.3)(3.5,1.6)(2.4,0.9)
%\psline(2,0.9)(3.1,1.6)(3.1,1.9)(2,1.2)
%\psline(1.5,1.2)(2.6,1.9)(2.6,2.2)(1.5,1.5)
%\psline(4.5,1)(4,1)(4,1.3)(3.5,1.3)(3.5,1.6)(3.1,1.6)(3.1,1.9)(2.6,1.9)(2.6,2.2)(2.2,2.2)(2.2,2.4)
%\psline(1.1,1.5)(2.2,2.2)
%\psline(4.5,0.7)(6,0.7)
%\pspolygon(7.3,0.1)(7.3,2.8)(9,2.8)(14.6,0.1)
%\psline[linestyle=dashed](9,2.8)(9,0.1)
%\psline[linewidth=0.4pt]{<->}(8.8,2.8)(8.8,0.1)
%\rput{90}(8.5,1.45){96 cm}
%\psline[linewidth=0.4pt]{<->}(9,-0.4)(12.3,-0.4)
%\uput[d](10.65,-0.4){55 cm}
%\psline[linewidth=0.4pt]{<->}(12.3,-0.4)(14.6,-0.4)
%\uput[d](13.45,-0.4){150 cm}
%\psline(9,2.4)(9.68,2.4)(9.68,1.96)(10.36,1.96)(10.36,1.52)(11.04,1.52)(11.04,1.08)(11.72,1.08)(11.72,0.64)(12.4,0.64)(12.4,0.1)
%\psline[linestyle=dotted](11.04,1.52)(11.04,2.6)
%\psline[linestyle=dotted](11.72,1.08)(11.72,2.6)
%\psline[linestyle=dotted](11.04,1.52)(14.2,1.52)
%\psline[linestyle=dotted](11.72,1.08)(14.2,1.08)
%\psline[linewidth=0.4pt]{<->}(11.04,2.6)(11.72,2.6)
%\psline[linewidth=0.4pt]{>-<}(14.2,1.78)(14.2,0.82)
%\uput[u](11.38,2.8){\footnotesize profondeur}
%\uput[u](11.38,2.6){\footnotesize d'une marche}
%\rput(15.2,1.6){\footnotesize hauteur}
%\rput(15.2,1.1){\footnotesize d'une marche}
%\uput[ur](9,2.8){A}\uput[dl](9,0.1){B}
%\uput[d](12.4,0.1){C}\uput[dr](14.6,0.1){D}
%\rput(8,3.4){Schéma}
%\end{pspicture}
%\end{center}
%
%\begin{tabularx}{\linewidth}{|X|}\hline
%Normes de construction de l'escalier :\\ 
%$60 \leqslant 2 h + p \leqslant 65$ où $h$ est la hauteur d'une marche et $p$ la profondeur d'une marche, en cm.\\
% ~\\
%Demandes des habitués du skate park :\\ 
%Longueur du plan incliné (c'est-à-dire la longueur AD) comprise entre $2,20$~m et $2,50$~m. \\
% Angle formé par le plan incliné avec le sol (ici l'angle $\widehat{\text{BDA}}$) compris entre 20\degres{} et 30\degres.\\ \hline
%\end{tabularx}
%
%\medskip
 
\begin{enumerate}
\item %Les normes de construction de l'escalier sont-elles respectées ?
On a $6h = 96$, soit $h = 16$~cm.

D’autre part $5p = 150$, soit $p = 30$~cm.

On a donc $2h + p =  32 + 30 = 62$, donc les normes de construction de l'escalier sont respectées.


\item %Les demandes des habitués du skatepark pour le plan incliné sont-elles satisfaites ?
Dans le triangle ABD rectangle en B, le théorème de Pythagore permet d’écrire 

AD$^2 = \text{AB}^2 + \text{BD}^2 = 96^2 + 205^2 = \np{9216} + \np{42025} = \np{51241}$.

Donc AD $ = \sqrt{\np{51241}} \approx 226,4$~cm soit environ 2,26~m. La première demande est respectée.

Enfin on a $\tan \widehat{\text{BDA}} = \dfrac{\text{AB}}{\text{BD}} = \dfrac{96}{205}$ ; la calculatrice donne $\widehat{\text{BDA}}  \approx 25,1$\degres.

La deuxième demande n’est pas respectée (de peu !)
\end{enumerate}
 
\bigskip  

