
\medskip 

%Cet exercice est un questionnaire à choix multiples (QCM). Aucune justification n'est demandée. 
%
%Pour chacune des questions, trois réponses sont proposées ; une seule est exacte. Toute réponse inexacte ou toute absence de réponse n'enlève pas de point. 
%
%\medskip
%
%\textbf{On indiquera sur la copie le numéro de chacune des cinq questions et on recopiera la réponse exacte.}
%
%\begin{center} 
%\begin{tabularx}{\linewidth}{|c|m{4cm}|*{3}{>{\centering \arraybackslash}X|}}\hline
% &Énoncé&   Réponse A&   Réponse B &  Réponse C \\ \hline  
%1&$\dfrac{5}{3} -  \dfrac{2}{3} : \dfrac{5}{3} + \dfrac{2}{3}$   est égal à \ldots \rule[-3mm]{0mm}{9mm}&$\dfrac{3}{3} : \dfrac{7}{3}$&$\dfrac{5}{3} - \dfrac{2}{5} + \dfrac{2}{3}$&$\dfrac{3}{3} \times \dfrac{3}{5} + \dfrac{2}{3}$\\ \hline      
%2&   Pour $x = 2\sqrt{5}$ ,  l'expression  $x^2 + 2x + 1$ vaut ...& $25\sqrt{5}$& $24\sqrt{5} + 1$&   $21 + 4\sqrt{5}$\\ \hline   
%3&   L'écriture scientifique de \np{0,00723} est \ldots&   $723 \times 10^{-5}$&  $7,23 \times  10^{- 3}$&   $7,23 \times  10^3$\\ \hline           
%4&   Soit la fonction $f$ définie par :  $f(x) = x^2 - x$&  L'image de $- 1$  est $- 2$&  L'image de  $- 1$   est $0$& $0$ a pour     antécédents $0$ et $1$\\ \hline          
%5&   Un élève a eu les notes suivantes :                 
% 6 ; 6 ; 9 ; 11 ; 12 ; 12 ; 14.   La médiane de ses notes est \ldots&   $10$&     $11$&       $12$\\ \hline
%\end{tabularx}
%\end{center}                
\begin{enumerate}
\item $\dfrac{2}{3} : \dfrac{5}{3} = \dfrac{2}{3} \times \dfrac{3}{5} = \dfrac{2}{5}$. Donc  $\dfrac{5}{3} -  \dfrac{2}{3} : \dfrac{5}{3} + \dfrac{2}{3} = \dfrac{5}{3} - \dfrac{2}{5} + \dfrac{2}{3}$. Réponse B.
\item On a $\left(2\sqrt{5} \right)^2 + 2 \times 2\sqrt{5} + 1 = 20 + 1 + 4\sqrt{5} = 21 + 4\sqrt{5}$. Réponse C.
\item Réponse B.
\item On a $x^2 - x = 0$ si $x(x - 1) = 0$ soit $x = 0$ ou $x - 1 = 0$. Réponse C.
\item 11 partage la série en deux séries de même taille. 11 est la médiane. Réponse B.
\end{enumerate}

\bigskip

