
\medskip

\parbox{0.4\linewidth}{\psset{unit=0.75cm}
\begin{pspicture}(-3.8,-4.5)(3.8,4.5)
\pscircle(0,0){3.5}\psline[linestyle=dashed](0.8,3.4)(-0.8,-3.4)
\pspolygon(1.8,-3)(-3.1,-1.6)(0.8,3.4)%CBA
\uput[ur](0.8,3.4){A} \uput[dl](-3.1,-1.6){B} \uput[dr](1.8,-3){C} \uput[ur](0,0){O}\uput[d](-0.8,-3.4){M}
\psdots(0,0)
\rput(-1.6,1){5} 
\end{pspicture}}\hfill
\parbox{0.55\linewidth}{On considère un triangle ABC isocèle en A tel que l'angle $\widehat{\text{BAC}}$ mesure 50\degres{} et AB est égal à 5~cm.
 
On note O le centre du cercle circonscrit au triangle ABC. La droite (OA) coupe ce cercle, noté ($C$), en un autre point M.

\medskip

\begin{enumerate}
\item Quelle est la mesure de l'angle $\widehat{\text{BAM}}$ ? Aucune justification n'est demandée. 
\item Quelle est la nature du triangle BAM ? Justifier. 
\item Calculer la longueur AM et en donner un arrondi au dixième de centimètre près. 
\item La droite (BO) coupe le cercle ($C$) en un autre point K. Quelle est la mesure de l'angle $\widehat{\text{BKC}}$ ?

Justifier. 
\end{enumerate}}

\bigskip

