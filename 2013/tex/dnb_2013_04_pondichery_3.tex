
\medskip
 
Le poids d'un corps sur un astre dépend de la masse et de l'accélération de la pesanteur.
 
On peut montrer que la relation est $P = mg$,
 
$P$ est le poids (en Newton) d'un corps sur un astre (c'est-à-dire la force que l'astre exerce sur le corps),
 
$m$ la masse (en kg) de ce corps,
 
$g$ l'accélération de la pesanteur de cet astre.

\medskip
 
\begin{enumerate}
\item Sur la terre, l'accélération de la pesanteur de la Terre $g_{T}$ est environ de $9,8$. Calculer le poids (en Newton) sur Terre d'un homme ayant une masse de $70$~kg. 
\item Sur la lune, la relation $P = mg$ est toujours valable.
 
On donne le tableau ci-dessous de correspondance poids-masse sur la Lune : 

\medskip

\begin{tabularx}{\linewidth}{|l|*{5}{>{\centering \arraybackslash}X|}}\hline
Masse (kg)	&3	&10	&25		&40	&55 \\ \hline
Poids (N)	&5,1&17 &42,5	&68	&93,5\\ \hline
\end{tabularx}

\medskip
 
	\begin{enumerate}
		\item Est-ce que le tableau ci-dessus est un tableau de proportionnalité ? 
		\item Calculer l'accélération de la pesanteur sur la lune noté $g_{L}$ 
		\item Est-il vrai que l'on pèse environ 6 fois moins lourd sur la lune que sur la Terre ?
	\end{enumerate} 
\item Le dessin ci-dessous représente un cratère de la lune. BCD est un triangle rectangle en D. 

\begin{center}
\psset{unit=0.9cm}
\begin{pspicture}(13,8)
\pspolygon[fillstyle=solid,fillcolor=lightgray](0,2.3)(7.9,2.3)(7.9,4.7)(6.8,4.7)(5.4,3.6)(2.5,3.6)(1,4.7)(0,4.7)
\psline(4,3.6)(11.8,6.6)
\psline{->}(11.5,6.7)(9.5,5.9)
\psline{->}(11.8,6.2)(9.7,5.4)
\psline[linestyle=dashed](4,5.7)(4,0.6)(6.8,0.6)(6.8,5.7)
\psline{<->}(4,0.6)(6.8,0.6)\uput[d](5.4,0.6){29 km}
\rput{20}(10,6.8){rayons solaires}
\psline(4,3.6)(12,3.6)
\psarc(4,3.6){4.9cm}{0}{21}
\uput[u](1,4.7){A}\uput[ur](6.8,4.7){B}
\uput[dl](4,3.6){C}\uput[dr](6.8,3.6){D}
\rput(9.2,4.6){4,3\degres}
\end{pspicture}
\end{center}

	\begin{enumerate}
		\item Calculer la profondeur BD du cratère. Arrondir au dixième de km près. 
		\item On considère que la longueur CD représente 20\,\% du diamètre du cratère. Calculer la longueur AB du diamètre du cratère.
	\end{enumerate} 
\end{enumerate} 


