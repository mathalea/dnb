
\medskip

Flora fait des bracelets avec de la pâte à modeler. Ils sont tous constitués de 8 perles rondes et de 4 perles longues.

\medskip

\parbox{0.55\linewidth}{ Cette pâte à modeler s'achète par blocs qui ont tous la forme d'un pavé droit dont les dimensions sont précisées ci-contre.
 
La pâte peut se pétrir à volonté et durcit ensuite à la cuisson.}\hfill
\parbox{0.4\linewidth}{\begin{center}\psset{unit=1cm}\begin{pspicture}(5,2.2)
\psframe(1.3,0.4)(3.9,1.2)
\psline(3.9,0.4)(4.6,1.4)(4.6,2.1)(3.9,1.2)
\psline(4.6,2.1)(2,2.1)(1.3,1.2)
\psline[linestyle=dotted](1.3,0.4)(2,1.4)(2,2.1)
\psline[linestyle=dotted](2,1.4)(4.6,1.4)
\psline[linewidth=0.5pt]{<->}(1.1,0.4)(1.1,1.2)
\psline[linewidth=0.5pt]{<->}(1.3,0.3)(3.9,0.3)
\psline[linewidth=0.5pt]{<->}(4.1,0.4)(4.8,1.4)
\rput{90}(0.7,0.8){\footnotesize 2 cm}\uput[d](2.6,0.4){\footnotesize 6 cm}
\rput{60}(4.6,0.8){\footnotesize 6 cm}
\end{pspicture}\end{center}}

\textbf{Information sur les perles :}
 
\medskip
\begin{center}
\begin{tabularx}{0.75\linewidth}{|*{2}{>{\centering \arraybackslash}X|}}\hline 
Une perle ronde &Une perle longue\\
\psset{unit=1cm}\begin{pspicture}(-0.5,-1)(0.5,1)
\pscircle(0,0){0.3cm}
\scalebox{1}[0.3]{\psarc(0,0){0.3}{180}{0}}%
\scalebox{1}[0.3]{\psarc[linestyle=dashed](0,0){0.32}{0}{180}}%
\end{pspicture}&\begin{pspicture}(-0.5,-0.5)(0.5,0.5)
\scalebox{1}[0.3]{\psarc[linewidth=2pt](0,-0.7){0.3}{180}{0}}%
\scalebox{1}[0.3]{\psarc[linestyle=dashed,linewidth=2pt](0,-0.7){0.32}{0}{180}}%
\psellipse(0,0.7)(0.3,0.12)
\psline(-0.3,-0.26)(-0.3,0.7)
\psline(0.3,-0.26)(0.3,0.7)
\end{pspicture}\\ 
 Boule de diamètre 8mm& Cylindre de hauteur 16 mm et de diamètre 8 mm\\ \hline
 \end{tabularx}
 \end{center}
  
Flora achète deux blocs de pâte à modeler: un bloc de pâte à modeler bleue pour faire les perles rondes et un bloc de pâte à modeler blanche pour faire les perles longues.
 
Combien de bracelets peut-elle ainsi espérer réaliser ?
 
\begin{center}
\begin{tabularx}{0.75\linewidth}{|X|}\hline
On rappelle les formules suivantes :\\ 
Volume d'un cylindre : $V = \pi \times \text{rayon}^2 \times \text{hauteur}$\\ 
Volume d'une sphère : $V = \dfrac{4}{3}\times  \pi \times \text{rayon}^3$\\ \hline
\end{tabularx}
\end{center}  

