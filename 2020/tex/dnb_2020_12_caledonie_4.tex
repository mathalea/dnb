\textbf{EXERCICE 4 : La régate \hfill 16 points}

\medskip

%

AB $= 400$, AC $= 300$, BC $= 500$ et CD $= 700$.

\medskip
\begin{center}
\begin{tabularx}{\linewidth}{lX} 
\psset{unit=1cm}
\begin{pspicture}(7,5.2)
%\psgrid
\pspolygon(1.5,3.2)(0.5,0.5)(6.2,5.14)(5,1.9)%ABDE
\uput[u](1.5,3.2){A} \uput[l](0.5,0.5){B} \uput[u](3,2.6){C} \uput[r](6.2,5.14){D} \uput[r](5,1.9){E} 
\end{pspicture}&\vspace{-2.5cm}\begin{tabular}{|l|} \hline

Les droites (AE) et (BD) se coupent en C \\
~\\
Les droites (AB) et (DE)sont parallèles\\ \hline
\end{tabular}\\
\end{tabularx}

\end{center}

\begin{enumerate}
\item Calculer la longueur DE.
\item Montrer que le triangle ABC est rectangle,
\item Calculer la mesure de l'angle $\widehat{\text{ABC}}$. Arrondir au degré.
\end{enumerate}

Lors d'une course les concurrents doivent effectuer plusieurs tours du parcours représenté ci-dessus. Ils partent du point A, puis passent par les points B, C, D et E dans cet ordre puis de nouveau par le point C pour ensuite revenir au point A.

\smallskip

Maltéo, le vainqueur, a mis 1~h 48~min pour effectuer les $5$ tours du parcours. La distance parcourue pour faire un tour est \np{2880}~m.

\begin{enumerate}[resume]
\item Calculer la distance totale parcourue pour effectuer les $5$~tours du parcours. 
\item Calculer la vitesse moyenne de Maltéo. Arrondir à l'unité.
\end{enumerate}

\vspace{0,5cm}

