\textbf{EXERCICE 8 : Scratch \hfill 11 points}

\medskip


Le script suivant permet de tracer le carré de côté $50$ unités .


\begin{center}
\begin{scratch}
\blockinit{quand \greenflag est cliqué}
\blockmove{s’orienter à \ovaloperator{\selectmenu{90}}}
\blockpen{stylo en position d’écriture}
\blockinfloop{répéter \ovalnum{4} fois}
{
\blockmove{avancer de \ovalnum{50}}
\blockmove{tourner \turnleft{} de \ovalnum{90} degrés}
}
\end{scratch}
\end{center}

\medskip

\begin{enumerate}
\item Sur l'annexe page \pageref{annexe2}, compléter le script pour obtenir un triangle équilatéral de coté $80$ unités.

\medskip

On a lancé le script suivant :

\begin{center}
\begin{scratch}
\blockinit{quand \greenflag est cliqué}
\blockmove{s’orienter à \ovaloperator{\selectmenu{90}}}
\blockmove{mettre \selectmenu{longueur} à \ovalnum{40}}
\blockpen{stylo en position d’écriture}
\blockinfloop{répéter \ovalnum{12} fois}
{
\blockmove{avancer de \selectmenu{longueur}}
\blockmove{tourner \turnleft{} de \ovalnum{90} degrés}
\blockmove{ajouter  à \selectmenu{longueur}\ovalnum{10}}
}
\end{scratch}
\end{center}
\item Entourer sur l'annexe page \pageref{annexe2}, la figure obtenue avec ce script.
\end{enumerate}



\begin{center}
\label{annexe2}
\textbf{\Large ANNEXE 2}

\vspace{2cm}

\textbf{question 1}

\bigskip

\textbf{Script à compléter}

\bigskip

\begin{scratch}
\blockinit{quand \greenflag est cliqué}
\blockmove{s’orienter à \ovalnum{\selectmenu{90}}}
\blockpen{stylo en position d’écriture}
\blockinfloop{répéter \ovalnum{\ldots} fois}
{
\blockmove{avancer de \ovalnum{\ldots}}
\blockmove{tourner \turnleft{} de \ovalnum{\ldots} degrés}
}
\end{scratch}


\vspace{1.5cm}

\textbf{question 2}

\bigskip

\begin{tabularx}{\linewidth}{*{3}{>{\centering \arraybackslash}X}}
Figure 1&Figure 2& Figure 3\\
\psset{unit=1mm,linecolor=red}
\begin{pspicture}(33,33)
%\psgrid
\psline(0,0)(0,34)(32,34)(32,4)(4,4)(4,30)(28,30)(28,8)(8,8)(8,26)(24,26)
(24,12)(12,12)(12,22)(20,22)(20,16)(16,16)
\end{pspicture}&
\psset{unit=1mm,linecolor=red}
\begin{pspicture}(34,34)
%\psgrid
\psline(0,0)(0,34)(32,34)(32,4)(4,4)(4,30)(28,30)(28,8)(8,8)(8,26)(24,26)
(24,12)(12,12)
\end{pspicture}&
\psset{unit=1mm,linecolor=red}
\begin{pspicture}(-20,-12)(20,20)
%\psgrid
\psline(20;210)(18;90)(16.2;-30)(14.58;210)(13.122;90)(11.8098;-30)(10.629;210)(9.56;90)(8.61;-30)(7.75;210)(6.974;90)(6.276;-30)(5.649;210)
\end{pspicture}
\end{tabularx}
\end{center}
