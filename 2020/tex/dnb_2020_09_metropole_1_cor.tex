
\medskip

%Cet exercice est un questionnaire à choix multiples (QCM).
%
%Pour chaque question, une seule des trois réponses proposées est exacte.
%
%Sur la copie, indiquer le numéro de la question et recopier, sans justifier, la réponse choisie.
%
%\medskip

\begin{center}
\begin{tabularx}{\linewidth}{|m{6.5cm}|*{3}{>{\centering \arraybackslash}X|}}\hline
\textbf{Questions}									&Réponse A	&Réponse B	&Réponse C\\ \hline
\textbf{1.~} On donne la série de nombres suivante : 

10 ; 6 ; 2 ; 14 ; 25 ; 12 ; 22.

La médiane est :							&12			&13			&14\\ \hline
\textbf{2.~} Un sac opaque contient 50 billes bleues, 45 rouges, 45 vertes et 60 jaunes.

Les billes sont indiscernables au toucher.

On tire une bille au hasard dans ce sac.

La probabilité que cette bille 
soit jaune est :							&60			&0,3					&$\dfrac{1}{60}$\\ \hline
\textbf{3.~} La décomposition en facteurs
 premiers de \np{2020} est :				&$2 \times 10 \times 101$	&$5 \times 5 \times 101$&$2 \times 2 \times 5 \times 101$\\ \hline
\textbf{4.~} La formule qui permet de calculer
 le volume d'une boule de rayon $R$ est :	&$2\pi R$&$\pi R^2$&$\dfrac{4}{3}\pi R^3$\\ \hline
\textbf{5.~} Une homothétie de centre A et de 
rapport $-2$ est une transformation qui :
								&agrandit les longueurs&réduit les longueurs&conserve les longueurs\\ \hline
\end{tabularx}
\end{center}

\bigskip

\begin{enumerate}
\item En ordonnant la série des 7 valeurs : 2 ; 6 ; 10  ; 12 ; 14 ; 22 ; 25, on voit que la 4\up{e}, 12 est la médiane.
\item La probabilité de tirer une bille jaune est $\dfrac{60}{50 + 45 + + 45 + 60} = \dfrac{60}{200} = \dfrac{30}{100} = 30\,\% = 0,3$.
\item $\np{2020} = 202 \times 10 = 2 \times 101 \times 2 \times 5 = 2 \times 2\times 5\times 101$.
\item $V = \dfrac{4}{3}\pi R^3$.
\item Elle agrandit les longueurs.
\end{enumerate}

\bigskip

