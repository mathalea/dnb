\textbf{\large Exercice 3 \hfill 23 points}

\medskip

Un club de handball souhaite commander des maillots avec le nom du club inscrit dessus. À l'issue de sa commande, le club veut recevoir exactement $350$~maillots.

Après quelques recherches, deux sites internet ont été sélectionnés:

\setlength\parindent{1cm}
\begin{itemize}
\item[$\bullet~~$] sur le site A : les maillots sont vendus à 12~\euro{} l'unité;
\item[$\bullet~~$] sur le site B : les maillots sont vendus à 13~\euro{} l'unité, avec la promotion :

\begin{center}\og 10 maillots offerts pour 100 achetés \fg.\end{center}
\end{itemize}
\setlength\parindent{0cm}

\medskip

\begin{enumerate}
\item Déterminer le montant, exprimé en euro, de la commande du club envisagée sur le site A.
\item Un tableur ci-dessous présente des exemples de dépenses en fonction du nombre de maillots payés sur le site B. Voici une copie d'écran de ce tableur.

\begin{center}
\begin{tabularx}{\linewidth}{|c|>{\small}p{4cm}|*{8}{>{\centering \arraybackslash}X|}}\hline
 &A								&B	&C	&D	&E		& F	& G	& H	&I\\ \hline 
1&Nombre de maillots payés		&50 &100&150&200 &250	&300&350&400\\ \hline
2&Nombre de maillots offerts	&0	&10	&10	&20	&20	&30	&30	&40\\ \hline
3&Nombre total de maillots reçus&50 &110 &160	&220	&270&330&380 &440\\ \hline
4&Coût total (en \euro)			&650&\np{1300}&\np{1950}&\np{2600}&\np{3250}&\np{3900}&\np{4550}& \np{5200}\\ \hline
\end{tabularx}
\end{center}

	\begin{enumerate}
		\item À la lecture de ce tableur, le trésorier du club affirme que le montant de la commande
sera compris entre \np{3900}~\euro{} et \np{4550}~\euro. Son affirmation est-elle vraie ?
		\item Sachant que les lignes 1 et 2 du tableur ont été complétées auparavant, quelle formule a-t-on pu saisir ensuite dans la cellule B3 avant de l'étirer jusqu'à la cellule I3, pour remplir la ligne 3 du tableur ?
		\item Le coût total exprimé en euro est-il proportionnel au nombre de maillots reçus ?
	\end{enumerate}
\item Sur quel site le club doit-il passer sa commande pour recevoir exactement $350$~maillots, tout en payant le moins cher ?
\item Le club souhaite que ces $350$~maillots soient répartis entre des maillots noirs et des maillots rouges dans le ratio 5 : 2.

Combien faut-il commander de maillots noirs et de maillots rouges ?
\item Le club a aussi commandé des gourdes. Les cartons reçus sont indiscernables tant par leurs dimensions que par leur forme. 

Il y a 4 cartons de gourdes blanches et 3 cartons de gourdes bleues.

On ouvre un carton au hasard. Quelle est la probabilité qu'il contienne des gourdes bleues ?
\end{enumerate}

\bigskip

