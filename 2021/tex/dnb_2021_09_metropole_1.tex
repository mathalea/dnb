
\medskip

Cet exercice est un Q.C.M. (Questionnaire à Choix Multiples).

Chaque question n'a qu'une seule bonne réponse.

Pour chaque question, précisez sur la copie le numéro de la question et la réponse choisie. 

Aucune justification n'est demandée pour cet exercice.

\begin{center}\begin{tabularx}{\linewidth}{|c|m{4cm}|*{3}{>{\centering \arraybackslash}X|}}\hline
\multicolumn{2}{|c|}{\textbf{Question}}& \textbf{Réponse A} &\textbf{Réponse B} &\textbf{Réponse C}\\ \hline
\textbf{1.}&\rule[-4mm]{0mm}{9mm}$\dfrac{4}{7} + \dfrac{5}{21} = \ldots$& $\dfrac{9}{21}$&$\dfrac{9}{28}$&$\dfrac{17}{21}$\rule[-4mm]{0mm}{9mm}\\ \hline
\textbf{2.}& Une urne contient 3 boules jaunes, 2 boules bleues et 4 boules vertes, indiscernables
au toucher.

On tire une boule au hasard. Quelle est la probabilité d'obtenir une boule verte ?&$\dfrac{4}{5}$&$\dfrac{4}{9}$&$\dfrac{5}{9}$\\ \hline
\textbf{3.}&Sur quelle figure a-t-on représenté une flèche et son image par une rotation de centre O et d'angle 90\degres ?&
\psset{unit=0.75cm}\def\fleche{\pspolygon(0,0)(1,0)(1,-0.2)(1.3,0.25)(1,0.7)(1,0.5)(0,0.5)}
\begin{pspicture}(-1.7,-1)(1.7,1)
\psdots[dotstyle=+](0,0)\rput(0.3,0.3){\fleche}\rput{180}(-0.3,-0.3){\fleche}
\uput[dr](0,0){O}
\end{pspicture}&\psset{unit=0.75cm}\def\fleche{\pspolygon(0,0)(1,0)(1,-0.2)(1.3,0.25)(1,0.7)(1,0.5)(0,0.5)}
\begin{pspicture}(-1.7,0)(1.7,2)
\psdots[dotstyle=+](0,0)\rput(0.3,0.3){\fleche}\rput{90}(-0.3,0.3){\fleche}
\uput[dr](0,0){O}
\end{pspicture} &\psset{unit=0.75cm}\def\fleche{\pspolygon(0,0)(1,0)(1,-0.2)(1.3,0.25)(1,0.7)(1,0.5)(0,0.5)}
\begin{pspicture}(-1.7,0)(1.7,2)
\psdots[dotstyle=+](0,0)\rput(0.3,0.3){\fleche}\rput{45}(-0.3,0.3){\fleche}
\uput[dr](0,0){O}
\end{pspicture}\\ \hline
\textbf{4.}& La décomposition en produit de facteurs premiers de $117$ est :&$3 \times 3 \times 13$&$9 \times 13$ &$3 \times 7 \times 7$\\ \hline
\textbf{5.}&\rule[-4mm]{0mm}{9mm}$\dfrac{1}{(-2) \times (-2) \times (-2)} = \ldots$&$(-2)^{-3}$&$(-2)^3$& $2^{-3}$\\ \hline
\end{tabularx}
\end{center}

\medskip

