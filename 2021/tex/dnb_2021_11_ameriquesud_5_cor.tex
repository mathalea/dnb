\textbf{\large Exercice 5 \hfill 20 points}

\medskip

Une usine de fabrication de bougies reçoit des cubes de cire d'abeille d'arête 6 cm.
Ils sont disposés dans des cartons remplis (sans espace vide).

\begin{center}
\begin{tabular}{l p{5mm} l}
\begin{tabular}{|p{5cm}|}\hline
\textbf{Informations sur les cartons:}\\
\\
Forme: pavé droit\\ 
\\
Dimensions : \\
%\begin{itemize}
~~-- largeur : 60 cm\\
~~-- hauteur : 36 cm\\
~~-- profondeur : 36 cm\\
\\
%\end{itemize}\\
(On ne tient pas compte de l'épaisseur des cartons)\\ \hline
\end{tabular} &~&
\begin{tabular}{|p{5cm}|}\hline
\textbf{Information sur la cire d'abeille:}\\
\\
Masse volumique: 0,95 g/cm$^3$\\ \hline
\end{tabular}
\end{tabular}
\end{center}

\medskip

\begin{enumerate}
\item 
	\begin{enumerate}
		\item %Montrer que chaque carton contient 360 cubes de cire d'abeille.
La largeur du carton est de 60~cm, et chaque cube de cire a une arête de 6~cm; on met donc $\dfrac{60}{6}=10$ cubes dans la largeur.

La profondeur du carton est de 36~cm, et chaque cube de cire a une arête de 6~cm; on met donc $\dfrac{36}{6}=6$ cubes dans la profondeur.
		
La hauteur du carton est de 36~cm, et chaque cube de cire a une arête de 6~cm; on met donc $\dfrac{36}{6}=6$ cubes dans la hauteur.

On met donc $10\times 6 \times 6=360$ cubes de cire dans un carton.		
				
		\item% Quelle est la masse de cire d'abeille contenue dans un carton rempli de cubes? On donnera la réponse en kg, arrondie à l'unité près, en ne tenant pas compte de la masse du carton.
Le volume de cire contenu dans un carton est en cm$^3$: $60\times 36\times 36	=\np{77760}$.

La masse volumique de la cire est de $0,95$~g/cm$^3$ donc la masse de $\np{77760}$~cm$^3$ est en gramme de $\np{77760}\times 0,95 = \np{73872}$, c'est-à-dire en arrondissant à l'unité, 74~kg. 
		\end{enumerate}
		
\item À l'usine, on découpe les cubes de cire d'abeille afin d'obtenir des cylindres de hauteur 6 cm et de diamètre 6 cm avec lesquels on fera des bougies en installant une mèche.

\begin{center}
\begin{tabular}{|*{5}{c}|}\hline
\psset{unit=1cm,arrowsize=15pt,arrowinset=0.8}
\begin{pspicture}(3,3)
%\psgrid
\pspolygon(0,0.6)(1,0)(2.7,0.3)(2.7,2.1)(1.7,2.8)(0,2.5)
\psline(2.7,2.1)(1,1.8)(0,2.5)\psline(1,0)(1,1.8)
\psline[linestyle=dotted](0,0.6)(1.7,0.9)(2.7,0.3)
\psline[linestyle=dotted](1.7,0.9)(1.7,2.8)
\end{pspicture}&\begin{pspicture}(1.5,3)\psline[linewidth=5pt,arrowsize=15pt,arrowlength=0.5,arrowinset=0.1]{->}(0,1.5)(1.5,1.5)\end{pspicture}&\begin{pspicture}(3,3)
\pspolygon(0,0.6)(1,0)(2.7,0.3)(2.7,2.1)(1.7,2.8)(0,2.5)
\psline(2.7,2.1)(1,1.8)(0,2.5)\psline(1,0)(1,1.8)
\psline[linestyle=dotted](0,0.6)(1.7,0.9)(2.7,0.3)
\psline[linestyle=dotted](1.7,0.9)(1.7,2.8)
\def\bougie{
\psset{unit=1cm}
\begin{pspicture}(-1,-0.5)(1,2.3)
\psline(-0.95,0)(-0.95,1.8)\psline(0.95,0)(0.95,1.8)
\psellipse(0,1.8)(0.95,0.4)
\psellipticarc[linewidth=1.25pt,linestyle=dotted](0,0)(0.95,0.4){0}{180}
\psellipticarc[linewidth=1.25pt](0,0)(0.95,0.4){180}{360}
\end{pspicture}}
\rput(1.28,1.43){\bougie}
\end{pspicture}&\begin{pspicture}(1.5,3)\psline[linewidth=5pt,arrowsize=15pt,arrowlength=0.5,arrowinset=0.1]{->}(0,1.5)(1.5,1.5)\end{pspicture}&
\begin{pspicture}(-1,-0.5)(1,2.3)
%\psgrid
\psline(-1,0)(-1,2)\psline(1,0)(1,2)
\psellipse(0,2)(1,0.37)
\psellipticarc[linewidth=1.25pt,linestyle=dotted](0,0)(1,0.37){0}{180}
\psellipticarc[linewidth=1.25pt](0,0)(1,0.37){180}{360}
\end{pspicture}\\
Cube de cire	&&&&Bougie cylindrique\\
d'abeille		&&&&(sans sa mèche)\\
Arête: 6 cm		&&&& Hauteur : 6 cm\\
				&&&& Diamètre: 6 cm\\ \hline
\end{tabular}
\end{center}

\emph{On ne tiendra pas compte de la masse, du volume et du prix de Ja mèche dans la suite de l'exercice.}

\medskip

	\begin{enumerate}
		\item 
Le volume d'un cylindre de rayon $r$ et de hauteur $h$ est donné par la formule : 
$V = \pi \times  r^2 \times h$,
donc le volume de la bougie est en cm$^3$:
$V=\pi \times 3^3\times 6 \approx 169,65$,
soit environ 170~cm$^3$.
		\item En découpant les cubes de cire d'abeille d'arête 6~cm pour former des bougies cylindriques, la cire perdue est réutilisée pour former à nouveau d'autres cubes de cire d'abeille d'arête 6~cm. 

Le cube de cire a pour volume, en cm$^3$, $6\times 6\times 6=216$, et la bougie a pour volume 170~cm$^3$; à chaque découpe de cube, on récupère $216-170=46$~cm$^3$ de cire.

$\dfrac{216}{46}\approx 4,7$ donc il faut découper 5 cubes pour pouvoir reconstituer un cube de cire d'abeille d'arête 6~cm, avec la cire perdue.
		
%Combien de cubes au départ doit-on découper pour pouvoir reconstituer un cube de cire d'abeille d'arête 6~cm, avec la cire perdue ?
	\end{enumerate}
	
\item Un commerçant vend les bougies de cette usine au prix de $9,60$~\euro{} l'unité. Il les vend 20\,\% plus chères qu'il ne les achète à l'usine.

Ajouter 20\,\%, c'est multiplier par $1+\dfrac{20}{100}=1,20$. 

On cherche le prix auquel il faut rajouter 20\,\% pour obtenir $9,60$: c'est $\dfrac{9,60}{1,20}=8$.

Le commerçant paie à l'usine 8~\euro{} pour l'achat d'une bougie.
\end{enumerate}
