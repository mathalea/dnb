\textbf{\large Exercice 2 \hfill 19 points}

\medskip

Une mère et sa fille rentrent chez elles à pied en empruntant le même trajet de
10 kilomètres. La mère décide de s'y rendre en marchant et sa fille en courant.

Le graphique ci-dessous modélise les parcours de la mère et de la fille depuis leur départ.

\medskip

\begin{center}
\psset{xunit=2.8cm,yunit=0.7cm,arrowsize=2pt 3}
\begin{pspicture}(-0.25,-1)(2.25,10)
\psgrid[unit=0.7cm,gridcolor=lightgray,subgriddiv=1,gridlabels=0](0,0)(9,10)
\psaxes[linewidth=1pt]{->}(0,0)(0,0)(2.25,10)
\psline[linecolor=blue,linewidth=1.5pt](0,0)(0.25,3)(0.75,3)(1.833,10)
\psline[linecolor=red,linestyle=dotted,linewidth=1.5pt](0,0)(2,10)
\uput[r](0,10.2){Distance parcourue (en km)}
\rput(0.5,5.5){\blue Parcours de la fille}\psline[linecolor=blue]{->}(0.5,5.2)(0.3,3)
\rput(0.75,1.2){\red Parcours de la mère}\psline[linecolor=red]{->}(0.6,1.3)(0.45,2.2)
\uput[u](1.85,0){Temps de parcours (en h)}
\end{pspicture}
\end{center}

%\medskip

\begin{enumerate}
\item 
	\begin{enumerate}
		\item %Indiquer le temps mis par la mère pour rentrer chez elle, avec la précision que permet la lecture du graphique.
La droite représentant le trajet de la mère passe par le point de coordonnées $(2~;~10)$ donc  le temps mis par la mère pour faire les 10~km pour rentrer chez elle est de 2 heures.
		
		\item %Déterminer la vitesse moyenne en km/h de la mère sur l'ensemble de son parcours.
La mère fait 10~km en 2~h donc sa vitesse moyenne est de $\dfrac{10}{2}$ soit 5~km/h.

		\item% La distance parcourue par la mère est-elle proportionnelle au temps ?
Le trajet de la mère est une droite passant par l'origine, donc la distance  parcourue par la mère est proportionnelle au temps.
		
	\end{enumerate}
\item La fille est partie à $16$~h et est arrivée chez elle à $17$~h $50$. Elle a fait une pause durant sa course. 
	\begin{enumerate}
		\item La durée de la pause de la fille est de 30 minutes.
		 
		\item %Quand a-t-elle couru le plus vite: avant ou après sa pause ?

\begin{list}{\textbullet}{Le trajet de la fille peut être décomposé en 3 parties.}
\item Les 15 premières minutes, elle parcourt 3~km, ce qui fait une vitesse de 12~km/h.
\item Les 30 minutes suivantes, elle fait une pause.
\item Elle parcourt 7~km dans la 3\ieme{} partie. Son trajet total se déroule de 16~h à 17~h~50, donc dure 1~h~50. La 3\ieme{} partie dure donc 1~h~05.

En parcourant 7~km en 1~h~05, la vitesse est inférieure à 7~km/h donc inférieure à 12~km/h.
\end{list}		

La fille a donc couru plus vite avant sa pause qu'après.
		\end{enumerate}
		
\item %Combien de fois la mère et la fille se sont retrouvées au même endroit et au même moment, au cours de leur trajet ?
Les graphiques représentant les deux trajets de la mère et de la fille se coupent en deux points, donc  la mère et la fille se sont retrouvées deux fois au même endroit et au même moment, au cours de leur trajet.

\item Dans cette question, on note $f$ la fonction qui, au temps de parcours $x$ (exprimé en heure) de la mère depuis le départ, associe la distance parcourue (exprimée en kilomètre) par la mère depuis le départ.
Parmi les propositions suivantes, l'expression de $f(x)$ est:

\hfill$f(x) = \dfrac{1}{5}x$ \quad{} ;\quad  \fbox{$f(x) = 5x$} \quad{} ;\quad  $f(x) = x +5$\hfill{}

\end{enumerate}


