
\medskip

%Voici trois programmes réalisés avec l'application Scratch.
%
%\small{\begin{center}
%\begin{tabularx}{\linewidth}{|*{3}{>{\centering \arraybackslash}X|}}\hline
%Programme 1 &Programme 2 &Programme  3\\
%\begin{scratch}
%\setscratch{num blocks=true}
%\blockinit{quand \greenflag est cliqué}
%\blockpen{stylo en position d'écriture}
%\blockrepeat{répéter \ovalnum{3} fois}
%{
%\blockmove{avancer de \ovalnum{100} pas}
%\blockmove{tourner \turnleft{} de \ovalnum{120} degrés}
%}
%\blockmove{avancer de \ovalnum{50} pas}
%\blockrepeat{répéter \ovalnum{4} fois}
%{
%\blockmove{avancer de \ovalnum{ ? } pas}
%\blockmove{tourner \turnleft{} de \ovalnum{90} degrés}
%}
%\end{scratch}&\begin{scratch}
%\setscratch{num blocks=true}
%\blockinit{quand \greenflag est cliqué}
%\blockpen{stylo en position d'écriture}
%\blockrepeat{répéter \ovalnum{3} fois}
%{
%\blockmove{avancer de \ovalnum{100} pas}
%\blockmove{tourner \turnleft{} de \ovalnum{120} degrés}
%}
%\blockmove{avancer de \ovalnum{100} pas}
%\blockrepeat{répéter \ovalnum{4} fois}
%{
%\blockmove{avancer de \ovalnum{ ? } pas}
%\blockmove{tourner \turnleft{} de \ovalnum{90} degrés}
%}
%\end{scratch}&\begin{scratch}
%\setscratch{num blocks=true}
%\blockinit{quand \greenflag est cliqué}
%\blockpen{stylo en position d'écriture}
%\blockrepeat{répéter \ovalnum{3} fois}
%{
%\blockmove{avancer de \ovalnum{100} pas}
%\blockmove{tourner \turnleft{} de \ovalnum{120} degrés}
%}
%\blockmove{tourner \turnleft{} de \ovalnum{60} degrés}
%\blockrepeat{répéter \ovalnum{4} fois}
%{
%\blockmove{avancer de \ovalnum{ ? } pas}
%\blockmove{tourner \turnleft{} de \ovalnum{90} degrés}
%}
%\end{scratch}\\ \hline
%\end{tabularx}
%\end{center}}

\begin{enumerate}
\item %Ils donnent les trois figures suivantes constituées de triangles et de quadrilatères \textbf{identiques}.

%\begin{center}
%\begin{tabularx}{\linewidth}{|*{3}{>{\centering \arraybackslash}X|}}\hline
%Figure A &Figure B& Figure C\\
%\psset{unit=1cm}
%\begin{pspicture}(-3.2,-1)(1.3,2.5)
%%\psgrid
%\pspolygon(1.1;-30)(1.1;90)(1.1;210)
%\rput{60}(-0.98,-0.58){\psframe(1.95,1.95)}
%\end{pspicture}&\psset{unit=1cm}
%\begin{pspicture}(-1.3,-1)(3.2,2.5)
%%\psgrid
%\pspolygon(1.1;-30)(1.1;90)(1.1;210)
%\psframe(0,-0.55)(1.95,1.4)
%\end{pspicture}&\psset{unit=1cm}
%\begin{pspicture}(-1.3,-1)(3.2,2.5)
%%\psgrid
%\pspolygon(1.1;-30)(1.1;90)(1.1;210)
%\psframe(0.96,-0.55)(2.91,1.4)
%\end{pspicture}\\ \hline
%\end{tabularx}\end{center}

	\begin{enumerate}
		\item %Quelle est la nature du triangle et du quadrilatère sur chaque figure ? Aucune justification n'est attendue.
Dans chaque cas le triangle est équilatéral et le quadrilatère est un carré.
		\item %Quelle est la valeur manquante à la ligne 8 dans ces 3 programmes?
Avancer de 100 pas.
		\item %Indiquer sur la copie, pour chaque figure, le numéro du programme qui permet de l'obtenir.
Programme 1 : figure B ; Programme 2 : figure C ; Programme 3 : figure A.
	\end{enumerate}
\item  
	\begin{enumerate}
		\item %Maintenant nous allons modifier les programmes précédents pour construire d'autres figures pour lesquelles le périmètre du quadrilatère est égal au périmètre du triangle. Quelle valeur du pas doit-on alors choisir à la ligne 8 de chaque programme ?
Si $c$ est la longueur du côté du carré et $t$ la longueur du côté du triangle, on doit avoir $4c = 3t$.
		
Donc si $t = 100$, alors $4c = 300$, soit $c =  75$.

Il faut donc écrire à la ligne 8  : avancer de 75 pas.
		\item ~%Représenter la figure A obtenue avec cette nouvelle valeur, en prenant 1 cm pour 25 pas.
\begin{center}
\psset{unit=1cm}
\begin{pspicture}(5,3)
\pspolygon(1.8,0)(4.8,0)(3.3,2.6)
\rput{60}(1.8,0){\psframe(2.25,2.25)}
\end{pspicture}
\end{center}
	\end{enumerate}
\end{enumerate}

\medskip

