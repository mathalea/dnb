
\medskip

%Le quadrilatère ABCD est un carré de côté de longueur 1 cm. Il est noté carré \ding{172}
%.
%Les points A, B, E et H sont alignés, ainsi que les points A, D, G et J.
%
%On construit ainsi une suite de carrés (carré \ding{172}  carré \ding{173}, carré \ding{174}, \ldots) en doublant la longueur du côté du carré, comme illustré ci-dessous pour les trois premiers carrés.
%
%\emph{La figure n'est pas en vraie grandeur}
%
%\begin{center}
%\psset{unit=1.25cm}
%\begin{pspicture}(6,4)
%\psframe[linewidth=1.25pt](4,4)
%\def\dbv{\psline(0,0.1)(0,-0.1)\psline(0.1,0.1)(0.1,-0.1) }
%\def\dbh{\psline(-0.1,0)(0.1,0)\psline(-0.1,-0.1)(0.1,-0.1) }
%\multido{\n=0.5+1.0}{4}{\rput(\n,4){\dbv}}
%\multido{\n=0.5+1.0}{4}{\rput(0,\n){\dbh}}
%\psline[linestyle=dashed](0,4)(4,0)
%\psline[linestyle=dashed](0,3)(1,3)(1,4)
%\psline[linestyle=dashed](0,2)(2,2)(2,4)
%\uput[ul](0,4){A} \uput[u](1,4){B} \uput[dl](1,3){C} \uput[l](0,3){D} 
%\uput[u](2,4){E} \uput[d](2,2){F} \uput[l](0,2){G} \uput[u](4,4){H} 
%\uput[dr](4,0){I} \uput[l](0,0){J} 
%\rput(5.5,2.5){Carré \ding{172} : ABCD}
%\rput(5.5,2){Carré \ding{173} : AEFG}
%\rput(5.5,1.5){Carré \ding{174} : AHIJ}
%\psdots[dotstyle=+,dotangle=45,dotscale=1.5](1,4)(2,4)(3,4)(0,1)(0,2)(0,3)
%\end{pspicture}
%\end{center}

\begin{enumerate}
\item %Calculer la longueur AC.
La diagonale [AC] partage le carré ABCD en deux triangles rectangles isocèles.

Dans le triangle ABC rectangle et isocèle en B, le théorème de Pythagore s'écrit :

$\text{AB}^2 + \text{BC}^2 = \text{AC}^2$, soit $1^2 + 1^2 = \text{AC}^2$.

Donc $\text{AC}^2 = 2$ et $\text{AC} = \sqrt{2} \approx 1,414$~(cm).
\item On choisit un carré de cette suite de carrés. 

\emph{Aucune justification n'est demandée pour les questions} 2. a. et 2. b.
	\begin{enumerate}
		\item %Quel coefficient d'agrandissement des longueurs permet de passer de ce carré au carré suivant ?
La suite des carrés est obtenue en doublant les longueurs  : le coefficient d'agrandissement des longueurs qui permet de passer de ce carré au carré suivant est donc 2.
		\item %Quel type de transformation permet de passer de ce carré au carré suivant ?
Tous ces carrés ont A pour l'un de leurs sommets : la transformation permettant de passer d'un carré au suivant est donc l'homothétie de centre A et de rapport 2.
		
%\begin{center}
%\fbox{symétrie axiale} \quad \fbox{homothétie} \quad  \fbox{rotation \phantom{é}} \quad  \fbox{symétrie centrale}  \quad \fbox{translation} 
%\end{center}

\item 
%L’affirmation \og la longueur de la diagonale du carré \ding{174} est trois fois plus grande que la longueur de la diagonale du carré \ding{172} \fg{} est-elle correcte ?
On a par doublement des longueurs :

AF $=  2 \times \text{AC} = 2\sqrt{2}$ ;

AI $= 2 \times \text{AF}  = 2 \times 2\sqrt{2} = 4\sqrt{2}$, donc AI $ = 4\text{AC}$ et non pas $3\text{AC}$ : l'affirmation est fausse.
	\end{enumerate}
\item %Déterminer, à l'aide de la calculatrice, une valeur approchée de la mesure de l'angle 
$\widehat{\text{AJB}}$ au degré près.
Le triangle AJB est rectangle en A. Pour l'angle $\widehat{\text{AJB}}$ on connait les longueurs du côté opposé et du côté adjacent ; on peut calculer sa tangente :

$\tan \widehat{\text{AJB}} = \dfrac{\text{AB}}{\text{AJ}} = \dfrac{1}{4} = 0,25$.

La calculatrice donne avec la fonction inverse de la tangente $\widehat{\text{AJB}} \approx 14,04$ soit $14\degres$ au degré près. $\widehat{\text{AJB}} \approx 14(\degres)$.
\end{enumerate}
\bigskip

