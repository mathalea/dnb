
\medskip

%Voici un algorithme:
%
%\begin{center}
%\psset{unit=1cm,arrowsize=2pt 3}
%\begin{pspicture}(-6,0)(6,6.5)
%%\psgrid
%\rput(0,6){Choisir un nombre $N$}
%\psframe(-2,5.6)(2,6.3)
%\psline{->}(0,5.8)(0,5)
%\rput(0,4){A-t-on : $N > 15$?}
%\pspolygon(-2.5,4)(0,5)(2.5,4)(0,3.)
%\psline{->}(-2.5,4)(-4,4)(-4,2)\psline{->}(2.5,4)(4,4)(4,2)
%\rput(-4,1.5){Calculer $100 - N \times 4$}\rput(4,1.5){Calculer $2 \times (N + 10)$}
%\psframe(-5.6,1)(-2.3,2)\psframe(2.4,1)(5.6,2)
%\end{pspicture}
%\end{center}

\begin{enumerate}
\item %Justifier que si on choisit le nombre $N$ de départ égal à 18, le résultat final de cet algorithme est $28$.
Comme $18 > 15$, l'algorithme calcule $100 - 4 \times 18 = 100 - 72 = 28$.
\item %Quel résultat final obtient-on si on choisit $14$ comme nombre $N$ de départ ?
Comme $14 > 15$ est faux l'algorithme calcule $2 \times (14 + 10) = 2 \times 24 = 48$.
\item %En appliquant cet algorithme, deux nombres de départ différents permettent d'obtenir $32$ comme résultat final. Quels sont ces deux nombres ?
\starredbullet~~ Si $N > 15$ on a donc $100 - 4N = 32$ ou $100 - 32 = 4N$ soit $68 = 4N$ ou $4 \times 17 = 4 \times N$, donc en simplifiant par 4 : $N = 17$ (qui est bien supérieur à 15).

\starredbullet~~ Si $N <  15$ on a donc $2(N + 10) = 32$ ou $2(N + 10) = 2\times 16$ et en simplifiant par 2 : $N + 10 = 16$ et enfin $N = 6$ (qui est bien inférieur à 15).

Les deux nombres introduits dans l'algorithme et rendant le nombre 32 sont 6 et 17.
\item %On programme l'algorithme précédent:

%\begin{center}
%\begin{tabular}{r l}
%\multicolumn{1}{l}{\textbf{Numéros}}&\\
%\multicolumn{1}{l}{\textbf{de ligne}}&\\
%\psline{->}(-1,0.2)(0.5,0.2)(0.5,0)&\\
%&\begin{scratch}
%\setscratch{num blocks=true}
%\blockinit{quand \greenflag est cliqué}
%\blocksensing{demander \txtbox{Choisir un nombre} et attendre}
%\blockifelse{si \booloperator{réponse > \txtbox{\phantom{de}} alors}}
%{\blocksound{dire \ovalnum{100} - réponse * \ovalnum{4} pendant \ovalnum{2} secondes}}
%{\blocksound{dire \ovalnum{} * \ovalnum{} + \ovalnum{} pendant \ovalnum{2} secondes}}
%\end{scratch}
%\\
%\end{tabular}
%\end{center}

	\begin{enumerate}
		\item %Recopier la ligne 3 en complétant les pointillés: 
		
		ligne 3 : si réponse $> 15$  alors
		\item %Recopier la ligne 6 en complétant les pointillés:
		
		ligne 6 : dire 2 $*(\text{réponse}  + 10)$ pendant 2 secondes
	\end{enumerate}
\item %On choisit au hasard un nombre premier entre 10 et 25 comme nombre $N$ de départ.

%Quelle est la probabilité que l'algorithme renvoie un multiple de 4 comme résultat final ?

\starredbullet~~ 11 donne $2 \times (11+ 10) = 2\times 21 = 42$ qui n'est pas multiple de 4.

\starredbullet~~ 13 donne $2 \times (13+ 10) = 2\times 23 = 46$ qui n'est pas multiple de 4.

\starredbullet~~ 17 donne $100 - 4 \times 17 = 100 - 68 = 32$ qui est  multiple de 4.

\starredbullet~~ 19 donne $100 - 4 \times 19 = 100 - 76 = 24$ qui est  multiple de 4.

\starredbullet~~ 23 donne $100 - 4 \times 23 = 100 - 92 = 8$ qui est  multiple de 4.

Il y a donc 3 nombres premiers sur 5 qui donnent un résultat multiple de 4 : la probabilité demandée est donc : $\dfrac{3}{5} = \dfrac{6}{10} = 0,6 = \dfrac{60}{100} = 60\,\%$.
\end{enumerate}

\bigskip

