
\medskip

\parbox{0.48\linewidth}{On considère le programme de calcul ci-contre.}\hfill
\parbox{0.48\linewidth}{\begin{tabularx}{\linewidth}{|X|}\hline
$\bullet~~$Choisir un nombre.\\
$\bullet~~$Ajouter 2 à ce nombre.\\
$\bullet~~$Prendre le carré du résultat précédent.\\
$\bullet~~$Soustraire le carré du nombre de départ au résultat précédent.\\ \hline
\end{tabularx}}

\medskip

On a utilisé la feuille de calcul ci-dessous pour appliquer ce programme de calcul au nombre 5 ; le résultat obtenu est 24.

\begin{center}
\begin{tabularx}{\linewidth}{|c|l|X|}\hline
&A&B\\ \hline
1& Programme&Résultat\\ \hline
2& Choisir un nombre &5\\ \hline
3& Ajouter 2 à ce nombre &7\\ \hline
4& Prendre le carré du résultat précédent &49\\ \hline
5& Soustraire le carré du nombre de départ au résultat précédent &24\\ \hline
\end{tabularx}
\end{center}

\smallskip

\begin{enumerate}
\item Pour les questions suivantes, faire apparaître les calculs sur la copie.
	\begin{enumerate}
		\item Si on choisit 2 comme nombre de départ, vérifier qu'on obtient 12 comme résultat.
		\item Si on choisit $- 8$ comme nombre de départ, quel résultat obtient-on ?
	\end{enumerate}
\item Parmi les trois propositions suivantes, recopier sur votre copie la formule qui a été saisie dans la cellule B5.

\begin{center}
\begin{tabularx}{\linewidth}{|*{3}{>{\centering \arraybackslash}X|}}\hline
=B4 $-$ B2 * B2& =B2 + 2 &= B3 * B3\\ \hline
\end{tabularx}
\end{center}

\item 
	\begin{enumerate}
		\item Si l'on choisit $x$ comme nombre de départ, exprimer en fonction de $x$, le résultat final de ce programme de calcul.
		\item Montrer que $(x + 2)^2 - x^2 = 4x + 4$.
	\end{enumerate}
\item Si on choisit un nombre entier au départ, est-il exact que le résultat du programme est toujours un multiple de 4 ? Justifier.
\end{enumerate}
% Résultat

\medskip

