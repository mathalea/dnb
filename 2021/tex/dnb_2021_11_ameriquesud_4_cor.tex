\textbf{\large Exercice 4 \hfill 14 points}

\medskip

%\emph{Dans tout cet exercice, aucune justification n'est demandée}
%
%\smallskip

On donne le programme suivant : 

\begin{center}
\begin{tabular}{l @{\hspace{2cm}} l}
Script principal&le bloc Carré\\
\begin{scratch}
\blockinit{quand \greenflag est cliqué}
\blocklook{effacer tout}
\blockmove{aller à x: \ovalnum{0} y : \ovalnum{0}}
\blockmove{s'orienter à \ovaloperator{\selectmenu{90}}}
\blockrepeat{répéter \ovalnum{4} fois}
{\blockmoreblocks{Carré}
\blockmove{avancer de \ovalnum{50}}
}
\end{scratch}&\begin{scratch}
\initmoreblocks{définir \namemoreblocks{Carré}}
\blockpen{stylo en position d'écriture}
\blockrepeat{répéter \ovalnum{4} fois}
{\blockmove{avancer de \ovalnum{50}}
\blockmove{tourner \turnleft{} de \ovalnum{90} degrés}
}
\blockpen{relever le stylo}
\end{scratch}\\
\end{tabular}
\end{center}

%On rappelle que l'instruction \begin{scratch}\blockmove{s'orienter à \ovalnum{90}}\end{scratch}
%signifie que l'on s'oriente vers la droite.

\begin{enumerate}
\item On lance le programme et on construit  la figure obtenue (1 cm pour 25 unités de longueur).

\begin{center}
\psset{unit=1cm}
\begin{pspicture}(0,0)(8,2)
%\psgrid[gridcolor=lightgray,subgriddiv=1,gridlabels=0]
\psframe(0,0)(8,2)
\psline(2,0)(2,2) \psline(4,0)(4,2) \psline(6,0)(6,2) 
\end{pspicture}
\end{center}

\end{enumerate}

\smallskip

On modifie le Script principal et on obtient deux scripts ci-dessous:

\begin{center}
\begin{tabular}{l @{\hspace{2cm}} l}
Script principal A&Script principal B\\
\begin{scratch}
\blockinit{quand \greenflag est cliqué}
\blocklook{effacer tout}
\blockmove{aller à x: \ovalnum{0} y : \ovalnum{0}}
\blockmove{s'orienter à \ovaloperator{\selectmenu{90}}}
\blockrepeat{répéter \ovalnum{3} fois}
{\blockmoreblocks{Carré}
\blockmove{avancer de \ovalnum{25}}
}
\end{scratch}&\begin{scratch}
\blockinit{quand \greenflag est cliqué}
\blocklook{effacer tout}
\blockmove{aller à x: \ovalnum{0} y : \ovalnum{0}}
\blockmove{s'orienter à \ovaloperator{\selectmenu{90}}}
\blockrepeat{répéter \ovalnum{4} fois}
{\blockmoreblocks{Carré}
\blockmove{tourner \turnleft{} de \ovalnum{90} degrés}
}
\end{scratch}\\
\end{tabular}
\end{center}

\begin{enumerate}
\setcounter{enumi}{1}
\item Parmi les trois figures ci-dessous, associer sur votre copie chacun des deux scripts principaux A et B à la figure qu'il permet de réaliser:

\begin{center}
\begin{tabularx}{\linewidth}{*{3}{>{\centering \arraybackslash}X}}
Figure 1 &Figure 2 &Figure 3\\
\psset{unit=1.2cm}
\begin{pspicture}(0,0)(2,2.2)
\psgrid[gridlabels=0pt,subgriddiv=1,gridwidth=1.2pt]
\end{pspicture}
&
\psset{xunit=0.6cm,yunit=1.2cm}
\begin{pspicture}(4,1)
\psgrid[gridlabels=0pt,subgriddiv=1,gridwidth=1.2pt]
\end{pspicture}
&
\psset{unit=1.2cm}
\begin{pspicture}(3,2)
\psgrid[gridlabels=0pt,subgriddiv=1,gridwidth=1.2pt]
\end{pspicture}\\
{\blue Script principal B} & {\blue Script principal A} & \\ 
\end{tabularx}

\end{center}

\end{enumerate}

\psset{unit=1cm}
\begin{pspicture}(-8,-1.5)(1.5,1.5)
\uput[r](-8.1,0){On souhaite réaliser la figure suivante :}
\psframe(-1,-1)(1,1)\psline(-1,0)(1,0)\psline(0,-1)(0,1)
\rput{45}(0,0){\psframe(-1,-1)(1,1)\psline(-1,0)(1,0)\psline(0,-1)(0,1)}
\end{pspicture}

\smallskip

Le point de départ se situe au centre de la figure.

\begin{enumerate}
\setcounter{enumi}{2}
\item On donne le nouveau script principal ci-dessous.% en recopiant sur la copie uniquement les lignes 5 et 7. Pour mémoire, l'énoncé rappelle ci-dessous à droite le descriptif du bloc Carré.

\begin{center}
\begin{tabular}{r l @{\hspace{2cm}} l}
Numéros de ligne \psset{arrowsize=2pt 3}
\psline{->}(0,0)(0.55,-0.8)&Script principal&le bloc Carré\\
&\begin{scratch}[num blocks]
\blockinit{quand \greenflag est cliqué}
\blocklook{effacer tout}
\blockmove{aller à x: \ovalnum{0} y : \ovalnum{0}}
\blockmove{s'orienter à \ovaloperator{\selectmenu{90}}}
\blockrepeat{répéter \ovalnum{\ldots\ldots} fois}
{\blockmoreblocks{Carré}
\blockmove{\ldots\ldots\ldots}
}
\end{scratch}&\begin{scratch}
\initmoreblocks{définir \namemoreblocks{Carré}}
\blockpen{stylo en position d'écriture}
\blockrepeat{répéter \ovalnum{4} fois}
{\blockmove{avancer de \ovalnum{50}}
\blockmove{tourner \turnleft{} de \ovalnum{90} degrés}
}
\blockpen{relever le stylo}
\end{scratch}\\
\end{tabular}
\end{center}

\end{enumerate}

\medskip

\begin{list}{\textbullet}{On complète les lignes 5 et 7.}
\item Ligne 5: \og répéter 8 fois \fg{} (il y a 8 carrés tracés)
\item Ligne 7: \og tourner \quad \rput{-60}(0,0.1){$\curvearrowleft$} \quad de 45 degrés \fg{} (on passe d'un carré au suivant en effectuant une rotation d'angle 45\degre{} dont le centre est le centre de la figure)
\end{list}


