
\medskip

Le quadrilatère ABCD est un carré de côté de longueur 1 cm. Il est noté carré \ding{172}.

Les points A, B, E et H sont alignés, ainsi que les points A, D, G et J.

On construit ainsi une suite de carrés (carré \ding{172}  carré \ding{173}, carré \ding{174}, \ldots) en doublant la longueur du côté du carré, comme illustré ci-dessous pour les trois premiers carrés.

\emph{La figure n'est pas en vraie grandeur}

\begin{center}
\psset{unit=1.25cm}
\begin{pspicture}(6,4)
\psframe[linewidth=1.25pt](4,4)
\def\dbv{\psline(0,0.1)(0,-0.1)\psline(0.1,0.1)(0.1,-0.1) }
\def\dbh{\psline(-0.1,0)(0.1,0)\psline(-0.1,-0.1)(0.1,-0.1) }
\multido{\n=0.5+1.0}{4}{\rput(\n,4){\dbv}}
\multido{\n=0.5+1.0}{4}{\rput(0,\n){\dbh}}
\psline[linestyle=dashed](0,4)(4,0)
\psline[linestyle=dashed](0,3)(1,3)(1,4)
\psline[linestyle=dashed](0,2)(2,2)(2,4)
\uput[ul](0,4){A} \uput[u](1,4){B} \uput[dl](1,3){C} \uput[l](0,3){D} 
\uput[u](2,4){E} \uput[d](2,2){F} \uput[l](0,2){G} \uput[u](4,4){H} 
\uput[dr](4,0){I} \uput[l](0,0){J} 
\rput(5.5,2.5){Carré \ding{172} : ABCD}
\rput(5.5,2){Carré \ding{173} : AEFG}
\rput(5.5,1.5){Carré \ding{174} : AHIJ}
\psdots[dotstyle=+,dotangle=45,dotscale=1.5](1,4)(2,4)(3,4)(0,1)(0,2)(0,3)
\end{pspicture}
\end{center}

\begin{enumerate}
\item Calculer la longueur AC.
\item On choisit un carré de cette suite de carrés. 

\emph{Aucune justification n'est demandée pour les questions} 2. a. et 2. b.
	\begin{enumerate}
		\item Quel coefficient d'agrandissement des longueurs permet de passer de ce carré au carré suivant ?
		\item Quel type de transformation permet de passer de ce carré au carré suivant ?
		
\begin{center}
\fbox{symétrie axiale} \quad \fbox{homothétie} \quad  \fbox{rotation \phantom{é}} \quad  \fbox{symétrie centrale}  \quad \fbox{translation} 
\end{center}

\item 
L’affirmation \og la longueur de la diagonale du carré \ding{174} est trois fois plus grande que la longueur de la diagonale du carré \ding{172} \fg{} est-elle correcte ?
	\end{enumerate}
\item Déterminer, à l'aide de la calculatrice, une valeur approchée de la mesure de l'angle $\widehat{\text{AJB}}$
au degré près.
\end{enumerate}
\bigskip

