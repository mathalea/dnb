
\medskip

%Pour chacun des six énoncés suivants, écrire sur la copie le numéro de la question et la réponse choisie. 
%
%Il ya une seule réponse correcte par énoncé. 
%
%\emph{On rappelle que toutes les réponses doivent être justifiées.}
%
%\begin{center}
%\begin{tabularx}{\linewidth}{|c|m{5cm}|*{3}{>{\centering \arraybackslash}X|}}\hline
%&&\textbf{Réponse A}&\textbf{Réponse B}&\textbf{Réponse C}\\ \hline
%
%1&Le nombre 126 a pour diviseur&252&20&6\\ \hline
%2&On considère la fonction $f$ définie par :
%
%\[f(x) =x^2 - 2.\]&L’image de 2 par $f$ est  $- 2$&$f(- 2) = 0$&$f(0) = - 2$\\ \hline
%3&Dans la cellule A2 du tableur ci-dessous, on a saisi la formule
%
%\[= - 5*\text{A}1 * \text{A}1 + 2 * \text{A}1  - 14\]
%
%puis on l'a étirée vers la droite.
%
% Quel nombre obtient-on dans la cellule B2 ?
%
%\begin{center}
%$\begin{array}{c|c|c}\hline
%&\text{A}&\text{B}\\ \hline
%1&	-4&- 3\\ \hline
%2&- 102&\\ \hline
%\end{array}$\end{center}&$- 65$&205&25\\ \hline
%4&Les solutions de l'équation $x^2 = 16$ sont …&$- 8$ et 8&$- 4$ et 4&$- 32$ et 32\\ \hline
%5& $2 \times  2^{400}$ est égal à …&$2^{401}$&$4^{400}$&$2^{800}$\rule[-3mm]{0mm}{9mm}\\ \hline
%6&La largeur et la hauteur d'une télévision suivent le
%ratio 16 ~:~ 9. Sachant que la hauteur de cette télévision 
%est de 54 cm, combien mesure sa largeur ?&
%94 cm&  96 cm
%& 30,375 cm\\ \hline
%\end{tabularx}
%\end{center}
\begin{enumerate}
\item $126 = 120 + 6 = 6 \times 20 + 6 \times 1 = 6 \times (20 + 1) = 6 \times 21$ : 126 est donc un multiple de $6$ ou 6 divise 126. \textbf{Réponse C}
\item On a :

\begin{itemize}
\renewcommand{\labelitemi}{\decosix~~}
\item $f(2) = 2^2 - 2 = 2$ ;
\item $f(-2) = (- 2)^2 - 2 = 4 - = 2$ ;
\item $f(0) = 0^2 - 2 = - 2$. \textbf{Réponse C}.
\end{itemize}

\item le tableur a calculé : $- 5 \times (- 4)^2 + 2 \times (- 4) - 14 = - 80 - 8 - 14 = - 102$, donc 

$- 5 \times (- 3)^2 + 2 \times (- 3) - 14 = - 45 - 6 - 14 = - 65$. \textbf{Réponse A}
\item $x^2 = 16$ ou $x^2 - 16 = 0$ ou $x^2 - 4^2 = 0$ ou $(x + 4)(x - 4) = 0$. Ce produit est nul si l'un des facteurs est nul, soit $\left\{\begin{array}{lc l}
x + 4&=&0\\
&\text{ou}&\\
x - 4&=&0
\end{array}\right.$ Les deux solutions sont donc $- 4$ et 4. \textbf{Réponse B}
\item $2 \times 2^{400} = 2^1 \times 2^{400} = 2^{1 + 400} = 2^{401}$.\textbf{Réponse A}
\item Si le poste a une longueur $L$ et une hauteur $h$, un ration de 16~:~9 signifie que $\dfrac{L}{h} = \dfrac{16}{9}$.

Si le poste a une hauteur de 54~(cm), on a donc $\dfrac{L}{54} = \dfrac{16}{9}$ d'où en multipliant par 54 :

$L = \dfrac{16 \times 54}{9} = \dfrac{16 \times 9 \times 6}{9} = 16 \times 6 = 96~$(cm). \textbf{Réponse B}
\end{enumerate}

\bigskip

