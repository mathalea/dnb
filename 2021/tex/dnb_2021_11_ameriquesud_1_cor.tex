\textbf{\large Exercice 1 \hfill 24 points}

\medskip

Pour chacune des six affirmations suivantes, indiquer si elle est vraie ou fausse.

On rappelle que toutes les réponses doivent être justifiées.

\begin{center}
\begin{tabularx}{\linewidth}{|X|}\hline
\textbf{Affirmation 1}: 72 est un multiple commun des nombres 12 et 18.\\ \hline
\textbf{Affirmation 2} : pour tout nombre $n$, on a l'égalité suivante: $(n- 5)^2 = n^2 - 5^2$.\rule{0pt}{12pt}\\ \hline
On considère la fonction $f$ définie par $f(x) = 2x +5$.\\
\textbf{Affirmation 3}: l'antécédent de 6 par la fonction $f$ est égal à $\dfrac{1}{2}$.\rule[-10pt]{0pt}{0pt}\\ \hline
Voici les températures relevées en degré Celsius (noté \degres C) pendant six jours dans une même ville: 5~\degres C, 7~\degres C, 11~\degres C, 8~\degres C, 5~\degres C et 6~\degres C.\\
\textbf{Affirmation 4} : la moyenne de ces six températures est égale à 6,5~\degres C.\\ \hline
\parbox{0.48\linewidth}{Les points B, D et A sont alignés.

Les points B, E et C sont alignés.

Le triangle ABC est rectangle en B.

BA $= 12$ cm ; BC $= 9$ cm ;

BD $= 8$ cm et BE $= 6$ cm.

\emph{La figure ci-contre n'est pas à l'échelle.}}
\hfill \parbox{0.43\linewidth}{\psset{unit=0.6cm}
\begin{pspicture}(9.5,6.4)
\pspolygon(2.9,0)(9.5,0)(9.5,4)%DBE
\pspolygon(0,0)(9.5,0)(9.5,6)%ABC
\psframe(9.5,0)(9.2,0.3)
\uput[d](0,0){A}\uput[d](9.5,0){B}\uput[r](9.5,6){C}
\uput[d](2.9,0){D}\uput[r](9.5,4){E}
\end{pspicture}}\\
\textbf{Affirmation 5}: la longueur AC est égale à $15$~cm.\\
\textbf{Affirmation 6}: les droites (AC) et (DE) sont parallèles.\\ \hline
\end{tabularx}
\end{center}

\begin{list}{\textbullet}{}
\item $72=12\times 6 = 18 \times 4$ donc 72 est un multiple commun à 12 et à 18.

\hfill\textbf{Affirmation 1 vraie}
\item Pour $n=10$, on a: $(n-5)^2=(10-5)^2=25$ et $n^2-5^2=10^2-5^2=75$; donc $(n-5)^2 \neq n^2-5^2$.

\hfill\textbf{Affirmation 2 fausse}

\item Pour déterminer les antécédents de 6 par $f$, on résout l'équation $f(x)=6$:

$f(x)=6$ équivaut à $2x+5=6$ équivaut à $2x=1$ équivaut à $x=\dfrac{1}{2}$. 

Donc $\dfrac{1}{2}$ est l'antécédent de 6 par $f$.

\hfill\textbf{Affirmation 3 vraie}

\item $\dfrac{5+7+11+8+4+6}{6}=\dfrac{42}{6}=7 \neq 6,5$

\hfill\textbf{Affirmation 4 fausse}

\item Le triangle ABC est rectangle en B donc, d'après le théorème de Pythagore:

$\text{AC}^2 = \text{BA}^2 + \text{BC}^2 = 12^2+9^2=144+81=225$ donc $\text{AC}=\sqrt{225}=15$

\hfill\textbf{Affirmation 5 vraie}

\item $\dfrac{\text{BD}}{\text{BA}}=\dfrac{8}{12}=\dfrac{2}{3}$ et $\dfrac{\text{BE}}{\text{BC}}=\dfrac{6}{9}=\dfrac{2}{3}$
donc $\dfrac{\text{BD}}{\text{BA}}=\dfrac{\text{BE}}{\text{BC}}$ 

\smallskip

D'après la réciproque du théorème de Thalès, les droites (AC) et (DE) sont parallèles.

\hfill\textbf{Affirmation 6 vraie}
\end{list}


