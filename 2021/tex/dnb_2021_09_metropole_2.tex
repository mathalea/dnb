
\medskip

Sur l'île de Madagascar, un scientifique mène une étude sur les tortues vertes.

La tortue verte a pour nom scientifique: « Chelonia Mydas ».

La carapace mesure en moyenne $115$~cm et l'animal pèse entre $80$ et $130$~kg.

Elle est classée comme espèce « En Danger».

\medskip

Afin de surveiller la bonne santé des tortues, elles sont régulièrement pesées. Voici les données relevées par ce scientifique en mai 2021.

\begin{center}
\begin{tabularx}{\linewidth}{|m{3cm}|*{7}{>{\centering \arraybackslash}X|}}\hline
Lettres de marquage&A-001&A-002&A-003&A-004&A-005&A-006&A-007\\ \hline
Sexe de la tortue&Mâle&Femelle&Femelle&Femelle&Mâle& Femelle&Femelle\\ \hline 
Masse (en kg)&113&96&125&87&117&104&101\\ \hline
\end{tabularx}
\end{center}


\begin{enumerate}
\item Calculer l'étendue de cette série statistique.
\item Calculer la masse moyenne de ces 7 tortues. Arrondir le résultat à l'unité.
\item Déterminer la médiane de cette série statistique. Interpréter le résultat.
\item Est-il vrai que les mâles représentent moins de 20\,\% de cet échantillon?
\item L'île de Madagascar a pour coordonnées géographiques (20 Sud ; 45 Est).

Placer une croix sur le planisphère fourni afin de marquer la position de l'île de Madagascar.
\smallskip

\begin{center}
	\textbf{\Large À rendre avec la copie}
	
	\vspace{1cm}
	
	\textbf{Question 5}
	
	\vspace{3cm}
	
	\begin{pspicture}(15,10)
	%\psgrid
	\rput(7.5,5){\includegraphics[width=14.cm]{Mappemonde_Metropole_sept_2021}}
	\psdots[linecolor=yellow,dotscale=1.](2.8,6.7)(7.22,7.18)(8.9,4)(12.2,6.675)
	\end{pspicture}
	\end{center}
\end{enumerate}

\medskip

