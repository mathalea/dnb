\textbf{\large Exercice 5 \hfill 20 points}

\medskip

Une usine de fabrication de bougies reçoit des cubes de cire d'abeille d'arête 6 cm.

Ils sont disposés dans des cartons remplis (sans espace vide).

\begin{center}
\begin{tabular}{l p{5mm} l}
\begin{tabular}{|p{5cm}|}\hline
\textbf{Informations sur les cartons:}\\
Forme: pavé droit\\ 
Dimensions : \\
\begin{itemize}
\item largeur : 60 cm
\item hauteur : 36 cm
\item profondeur : 36 cm
\end{itemize}\\
(On ne tient pas compte de l'épaisseur des cartons)\\ \hline
\end{tabular} &~&
\begin{tabular}{|p{5cm}|}\hline
\textbf{Information sur la cire d'abeille:}\\
Masse volumique: 0,95 g/cm$^3$\\ \hline
\end{tabular}
\end{tabular}
\end{center}

\medskip

\begin{enumerate}
\item 
	\begin{enumerate}
		\item Montrer que chaque carton contient 360 cubes de cire d'abeille.
		\item Quelle est la masse de cire d'abeille contenue dans un carton rempli de cubes? On donnera la réponse en kg, arrondie à l'unité près, en ne tenant pas compte de la masse du carton.
	\end{enumerate}
\item À l'usine, on découpe les cubes de cire d'abeille afin d'obtenir des cylindres de hauteur 6 cm et de diamètre 6 cm avec lesquels on fera des bougies en installant une mèche.

\begin{center}
\begin{tabular}{|*{5}{c}|}\hline
\psset{unit=1cm,arrowsize=15pt,arrowinset=0.8}
\begin{pspicture}(3,3)
%\psgrid
\pspolygon(0,0.6)(1,0)(2.7,0.3)(2.7,2.1)(1.7,2.8)(0,2.5)
\psline(2.7,2.1)(1,1.8)(0,2.5)\psline(1,0)(1,1.8)
\psline[linestyle=dotted](0,0.6)(1.7,0.9)(2.7,0.3)
\psline[linestyle=dotted](1.7,0.9)(1.7,2.8)
\end{pspicture}&\begin{pspicture}(1.5,3)\psline[linewidth=5pt,arrowsize=15pt,arrowlength=0.5,arrowinset=0.1]{->}(0,1.5)(1.5,1.5)\end{pspicture}&\begin{pspicture}(3,3)
\pspolygon(0,0.6)(1,0)(2.7,0.3)(2.7,2.1)(1.7,2.8)(0,2.5)
\psline(2.7,2.1)(1,1.8)(0,2.5)\psline(1,0)(1,1.8)
\psline[linestyle=dotted](0,0.6)(1.7,0.9)(2.7,0.3)
\psline[linestyle=dotted](1.7,0.9)(1.7,2.8)
\def\bougie{
\psset{unit=1cm}
\begin{pspicture}(-1,-0.5)(1,2.3)
\psline(-0.95,0)(-0.95,1.8)\psline(0.95,0)(0.95,1.8)
\psellipse(0,1.8)(0.95,0.4)
\psellipticarc[linewidth=1.25pt,linestyle=dotted](0,0)(0.95,0.4){0}{180}
\psellipticarc[linewidth=1.25pt](0,0)(0.95,0.4){180}{360}
\end{pspicture}}
\rput(1.28,1.43){\bougie}
\end{pspicture}&\begin{pspicture}(1.5,3)\psline[linewidth=5pt,arrowsize=15pt,arrowlength=0.5,arrowinset=0.1]{->}(0,1.5)(1.5,1.5)\end{pspicture}&
\begin{pspicture}(-1,-0.5)(1,2.3)
%\psgrid
\psline(-1,0)(-1,2)\psline(1,0)(1,2)
\psellipse(0,2)(1,0.37)
\psellipticarc[linewidth=1.25pt,linestyle=dotted](0,0)(1,0.37){0}{180}
\psellipticarc[linewidth=1.25pt](0,0)(1,0.37){180}{360}
\end{pspicture}\\
Cube de cire	&&&&Bougie cylindrique\\
d'abeille		&&&&(sans sa mèche)\\
Arête: 6 cm		&&&& Hauteur : 6 cm\\
				&&&& Diamètre: 6 cm\\ \hline
\end{tabular}
\end{center}

\emph{On ne tiendra pas compte de la masse, du volume et du prix de Ja mèche dans la suite de l'exercice.}

\medskip

	\begin{enumerate}
		\item Montrer que le volume-d'une bougie est d'environ $170$~cm$^3$.
		
		\smallskip
		
On rappelle que le volume d'un cylindre de rayon $r$ et de hauteur $h$ est donné par la formule : 

\[V = \pi \times  r^2 \times h.\]

		\item En découpant les cubes de cire d'abeille d'arête 6~cm pour former des bougies cylindriques, la cire perdue est réutilisée pour former à nouveau d'autres cubes de cire d'abeille d'arête 6~cm. 
		
Combien de cubes au départ doit-on découper pour pouvoir reconstituer un cube de cire d'abeille d'arête 6~cm, avec la cire perdue ?
	\end{enumerate}
\item Un commerçant vend les bougies de cette usine au prix de $9,60$~\euro{} l'unité. Il les vend 20\,\% plus chères qu'il ne les achète à l'usine.

Combien paie-t-il à l'usine pour l'achat d'une bougie ?
\end{enumerate}
