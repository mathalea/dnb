
\medskip

%Une famille a acheté une étagère qu'elle souhaite placer le long d'un mur.

\medskip

\begin{enumerate}
\item %L'étagère était affichée au prix de 139,90 €. La famille a obtenu une réduction de 10 %.
%Quel a été le montant de cette réduction?
10\,\% de 139,90 est égal à $139,9 \times 0,1 = 13,99$~(\euro) de réduction.
\item %Voici l'image de profil qu'on peut voir sur le guide de montage de l'étagère; ce dessin n'est
pas à l'échelle.

%\begin{center}
%\psset{unit=0.8cm,arrowsize=2pt 3}
%\begin{pspicture}(17.2,7.6)
%\def\fleche{\psline[linewidth=1pt]{->}(-4.5,0.6)(-4.5,1.6)}
%%\psgrid
%\psline(0,7.6)(17.2,7.6)
%\psline(0,7)(17.2,7)\rput(8.6,7.3){Plafond}
%\psline(0,0.6)(17.2,0.6)
%\rput(2.2,3.7){AB = AB$' = 0,80$~m}
%\rput(2.2,3){BC = B$'$C$' = 2,25$~m}
%\pspolygon[fillstyle=solid,fillcolor=gray!20](4.5,0.6)(10.4,0.6)(9.4,3.1)
%%\pscircle(7,0.6){2.5}\pscircle(9.5,0.6){2.8}
%\psarc[linestyle=dotted,linewidth=1.25pt](10.4,0.6){5.9}{61}{180}
%\psline(13.1,0.6)(13.1,7)
%\psline(13.8,0.6)(13.8,7)
%\pspolygon[fillstyle=penrose,psscale=0.3](13.1,0.6)(13.1,7)(13.8,7)(13.8,0.6)
%\uput[r](13.8,3.8){Mur}
%\psline{<->}(15.3,0.6)(15.3,7)\uput[r](15.3,3.8){2,40 m}
%\rput(7.6,0.9){\small Position 1}
%\rput(8,0.3){Sol de la pièce}
%\pspolygon[linestyle=dashed,fillstyle=solid,fillcolor=gray!30](10.4,0.6)(11,3.5)(6.2,4.8)
%\pspolygon[linestyle=dashed,fillstyle=solid,fillcolor=gray!40](10.4,0.6)(12.8,2)(10.4,6.5)
%\pspolygon[linestyle=dashed,fillstyle=solid,fillcolor=gray!50](10.4,0.6)(13.1,0.6)(13.1,5.7)
%\pspolygon(4.5,0.6)(10.4,0.6)(9.4,3.1)
%\pspolygon[linestyle=dashed](10.4,0.6)(11,3.5)(6.2,4.8)
%\pspolygon[linestyle=dashed](10.4,0.6)(12.8,2)(10.4,6.5)
%\pspolygon[linestyle=dashed](10.4,0.6)(13.1,0.6)(13.1,5.7)
%\rput(12.1,0.9){\small Position 2}
%%\rput{-20}(10.4,0.6){\fleche}
%%\rput{-55}(10.4,0.6){\fleche}
%%\rput{-90}(10.4,0.6){\fleche}
%\psarc{<-}(10.4,0.6){4.5}{140}{160}
%\psarc{<-}(10.4,0.6){4.5}{95}{115}
%\psarc{<-}(10.4,0.6){4.5}{65}{75}
%\rput{-154}(9.4,3.1){\psframe(0.35,0.35)}
%\uput[d](4.5,0.6){C}\uput[d](10.4,0.6){A}\uput[u](9.4,3.1){B}\uput[l](6.2,4.8){D}\uput[u](10.4,6.5){E}
%\uput[ul](13.1,5.7){C$'$}\uput[d](13.1,0.6){B$'$}
%\rput{27}(7.1,2.2){2,25~m}\rput{-69}(10.1,2.1){0,80~m}
%\end{pspicture}
%\end{center}


L'étagère a été montée à plat sur le sol de la pièce ; elle est donc en position 1.

On veut s'assurer qu'elle ne touchera pas le plafond au moment de la relever pour atteindre la position 2. 

On ne dispose d'aucun instrument de mesure.

Avec les données du schéma précédent, vérifier que l'étagère ne touchera pas le plafond.
Le triangle ABC est rectangle en B, donc d'après le théorème de Pythagore :

$\text{AC}^2 = \text{AB}^2 + \text{BC}^2 = 0,8^2 + 2,25^2 = 0,64 + \np{5,0625} = \np{5.7025}$.
On en déduit que AC $ = \sqrt{\np{5.7025}} \approx 2,388 < 2,40$.

On a donc AE $ < 2,40$ : l'étagère passe (juste !)
\item %Dans cette question, on supposera que le meuble a pu être disposé contre le mur.

%On installe maintenant quatre tablettes horizontales régulièrement espacées et représentées ici par les segments [DE], [FG], [HI] et [JK].

%\begin{center}
%\psset{unit=1cm}
%\begin{pspicture}(5,6.8)
%%\psgrid
%\psline(0,0.8)(5,0.8)%AB'
%\psline(4,0.8)(4,6.8)(0.6,0.8)
%\psframe(4,0.8)(3.7,1.1)
%\psline(4,2)(1.26,2)%KJ
%\psline(4,3.2)(1.94,3.2)%IH
%\psline(4,4.4)(2.62,4.4)%GF
%\psline(4,5.6)(3.3,5.6)%ED
%\rput(2.5,0.1){Sol de la pièce}
%\uput[d](0,0.8){A} \uput[d](5,0.8){B$'$} \uput[u](4,6.8){C$'$} \uput[l](3.3,5.6){D} 
%\uput[r](4,5.6){E} \uput[l](2.62,4.4){F} \uput[r](4,4.4){G} \uput[l](1.94,3.2){H} 
%\uput[r](4,3.2){I} \uput[l](1.26,2){J} \uput[r](4,2){K}
%\def\doublebarre{\psline(-0.1,-0.05)(0.1,-0.05)\psline(-0.1,0.05)(0.1,0.05)}
%\multirput(4,1.4)(0,1.2){5}{\doublebarre}
%\end{pspicture}
%\end{center}

	\begin{enumerate}
		\item %Calculer la longueur C$'$E.
On a $\text{C}'\text{E} = \dfrac{\text{B}'\text{C}'}{5} \dfrac{2,25}{5} = \dfrac{4,5}{10} = 0,45$~(m).
		\item %Calculer la longueur de la tablette [DE].
Le droites horizontales sont parallèles : on peut donc appliquer le théorème de Thalès :
		
$\dfrac{\text{DE}}{\text{AB}'} = \dfrac{\text{C}'\text{E}}{\text{C}'\text{B}'}$, soit :

$\dfrac{\text{DE}}{0,8} = \dfrac{0,45}{2,25} = 0,2$, d'on on a DE $ = 0,8 \times 0,2 = 0,16$~(m).
		\item %Calculer la longueur de la tablette [HI].
On a de la même façon :

$\dfrac{\text{HI}}{\text{AB}'} = \dfrac{\text{C}'\text{I}}{\text{C}'\text{B}'}$, soit $\dfrac{\text{HI}}{0,8} = \dfrac{3 \times 0,45}{2,25} = 0,6$, d'où on a HI $ = 0,8 \times 0,6 = 0,48$~(m).
	\end{enumerate}
	
%\begin{center}
%	\begin{tabular}{|l|}\hline
%Rappels des données :\\
%B$'$C$' = 2,25$ m\\
%AB$' = 0,80$ m\\ \hline
%\end{tabular}
%\end{center}
\end{enumerate}

