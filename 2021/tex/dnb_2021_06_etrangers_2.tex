
	\textbf{Partie 1}
	
	Dans cette première partie, on lance un dé bien équilibré à six faces numérotées de 1 à 6, puis on note le numéro de la face du dessus.
	
	\begin{enumerate}
		\item Donner sans justification les issues possibles.
		
		\item Quelle est la probabilité de l'évènement A : « On obtient 2 » ?
		
		\item Quelle est la probabilité de l'évènement B : « On obtient un nombre impair » ?
	\end{enumerate}

	\textbf{Partie 2}
	
	Dans cette deuxième partie, on lance simultanément deux dés bien équilibrés à six faces, un rouge et un vert. On appelle « score » la somme des numéros obtenus sur chaque dé.
	
	\begin{enumerate}
		\item Quelle est la probabilité de l'évènement C : « le score est 13 » ? Comment appelle-t-on un tel évènement ?
		
		\item Dans le tableau à double entrée donné en ANNEXE, on remplit chaque case avec la somme des numéros obtenus sur chaque dé.
		
		\begin{enumerate}
			\item Compléter, sans justifier, le tableau donné en ANNEXE à rendre avec la copie.
			
			\item Donner la liste des scores possibles.
		\end{enumerate}
	
		\item \begin{enumerate}
			\item Déterminer la probabilité de l'évènement D : « le score est 10 ».
			
			\item Déterminer la probabilité de l'évènement E : « le score est un multiple de 4 ».
			
			\item Démontrer que le score obtenu a autant de chance d'être un nombre premier qu'un nombre strictement plus grand que 7.
		\end{enumerate}
	\end{enumerate}

\vspace{0,5cm}

\textbf{Annexe à rendre avec la copie - Partie 2, question 2. a.}

\begin{center}
	\begin{tabularx}{0.8\linewidth}{|>{\centering \arraybackslash}p{3cm}|*{6}{>{\centering \arraybackslash \rule[-3mm]{0mm}{10mm}}X|}}  \hline
	\backslashbox{\rule{0mm}{6mm}Dé rouge}{Dé vert}&1&2&3&4&5&6\\ \hline
	1& & & & & & \\ \hline
	2& & & & & & \\ \hline
	3& & & &7& & \\ \hline
	4& &6& & & & \\ \hline
	5& & & & & & \\ \hline
	6& & & & & & \\ \hline
\end{tabularx}
\end{center}

