\textbf{\large Exercice 4 \hfill 18 points}

\medskip

\begin{enumerate}
\item 
	\begin{enumerate}
		\item $({\red 7} + 5) \times ({\red 7} - 5) + 25 = 12 \times 2 + 25 = 24 + 25  = 49$.

Avec 5 au départ on obtient bien 49 en sortie.
		\item $({\red - 4} + 5)({\red - 4} - 5) + 25 = 1 \times (- 9) +25 = - 9 + 25
= 16$.

Avec $- 4$ au départ on obtient $16$ en sortie.
	\end{enumerate}
\item
	\begin{enumerate}
		\item $(x + 5)(x - 5) + 25$ 
		\item On développe $(x + 5)(x - 5) = x^2 - 5^2
= x^2 - 25$.
		\item D'après le calcul précédent : $(x + 5)(x - 5) + 25 = x^2 - 25 + 25 = x^2$.
Sarah a raison.
	\end{enumerate}
\end{enumerate}

\bigskip

