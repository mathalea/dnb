\textbf{{\large \textsc{Exercice 3}} \hfill 20 points}

\bigskip

La figure ci-dessous est un pavage constitué de cerfs-volants.

Les triangles SLP et PLA ainsi formés sont des triangles équilatéraux.

\begin{center}
\psset{unit=1cm}
\begin{pspicture}(7.8,4.4)
\pspolygon(0.2,0.2)(4.6,0.2)(6.8,4.01)(2.4,4.01)%SLAP
\psline(2.4,4.01)(4.6,0.2)%PL
\pspolygon[fillstyle=solid,fillcolor=lightgray](2.4,4.01)(1.3,2.105)(2.4,1.4)(3.5,2.105)
\pspolygon[fillstyle=solid,fillcolor=lightgray](3.5,2.105)(4.6,0.2)(5.7,2.105)(4.6,2.7)
\psline(2.4,0.2)(2.4,1.4)
\psline(4.6,4.01)(4.6,2.7)
\uput[dl](0.2,0.2){S} \uput[dr](4.6,0.2){L} \uput[ur](6.8,4.01){A} \uput[ul](2.4,4.01){P} \uput[u](3.5,2.105){J}
\rput(2.4,2.5){1} \rput(1.5,0.9){2} \rput(3.4,0.9){3} \rput(3.7,3.3){4} \rput(5.4,3.3){5} \rput(4.6,1.7){6}
\end{pspicture}
\end{center}

\textbf{PARTIE A :}

\medskip

\begin{enumerate}
\item Déterminer la mesure de l'angle $\widehat{\text{PSL}}$.
\item Quelle est l'image du cerf-volant 2 par la symétrie d'axe (PL)? On ne demande pas de
justification.
\item Déterminer par quelle transformation du plan le cerf-volant 1 devient le cerf-volant 6 ? 

On ne demande pas de justification.
\end{enumerate}

\bigskip

\textbf{PARTIE B :}

\medskip

Dans cette partie, on se propose de construire le cerf-volant ci-dessous.

Essya, Nicolas et Tiago souhaitent construire cette figure à l'aide d'un logiciel de programmation.

\begin{center}
\psset{unit=1cm}
\begin{pspicture}(3.6,3.6)
%\psgrid
\pspolygon(0.2,0.2)(3.6,0.2)(3.6,1.9)(2.17,2.95)
\rput{-120}(2.17,2.95){\psframe(0.2,0.2)}
\rput{90}(3.6,0.2){\psframe(0.2,0.2)}
\psline(2.75,2.4)(2.95,2.6)
\psline(3.5,1.)(3.7,1.1)
\psline(2,0.05)(2,0.35)\psline(2.1,0.05)(2.1,0.35)
\psline(1.15,1.8)(1.35,1.6)\psline(1.2,1.9)(1.4,1.7)
\end{pspicture}
\end{center}

Ils écrivent tous un programme \og  Cerf-volant \fg{} différent.

\begin{center}
\begin{tabularx}{\linewidth}{|*{3}{>{\centering \arraybackslash}X|}}\hline
Programme de Essya&Programme de Nicolas&Programme de Tyago\\ \hline
\begin{scratch}
\initmoreblocks{définir \namemoreblocks{Cerf-volant}}
\blockmove{avancer de \ovalnum{300} pas}
\blockmove{tourner \turnleft{} de \ovalnum{90} degrés}
\blockmove{avancer de \ovalnum{173} pas}
\blockmove{tourner \turnleft{} de \ovalnum{60} degrés}
\blockmove{avancer de \ovalnum{173} pas}
\blockmove{tourner \turnleft{} de \ovalnum{90} degrés}
\blockmove{avancer de \ovalnum{300} pas}
\end{scratch}&\begin{scratch}
\initmoreblocks{définir \namemoreblocks{Cerf-volant}}
\blockmove{avancer de \ovalnum{300} pas}
\blockmove{tourner \turnleft{} de \ovalnum{120} degrés}
\blockmove{avancer de \ovalnum{300} pas}
\blockmove{tourner \turnleft{} de \ovalnum{120} degrés}
\blockmove{avancer de \ovalnum{300} pas}
\end{scratch}&\begin{scratch}
\initmoreblocks{définir \namemoreblocks{Cerf-volant}}
\blockmove{avancer de \ovalnum{173} pas}
\blockmove{tourner \turnleft{} de \ovalnum{60} degrés}
\blockmove{avancer de \ovalnum{300} pas}
\blockmove{tourner \turnleft{} de \ovalnum{90} degrés}
\blockmove{avancer de \ovalnum{173} pas}
\blockmove{tourner \turnleft{} de \ovalnum{120} degrés}
\blockmove{avancer de \ovalnum{300} pas}
\end{scratch}\\ \hline
\end{tabularx}
\end{center}

\begin{enumerate}
\item Tracer le programme \og Cerf-Volant \fg{} de Nicolas, en prenant $1$~cm pour $100$ pas.
\item Un élève a écrit le script correct. Donner le nom de cet élève en justifiant la réponse.
\end{enumerate}

\bigskip

