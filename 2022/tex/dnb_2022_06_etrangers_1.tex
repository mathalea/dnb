
\medskip

Les deux parties de cet exercice sont indépendantes. 

\medskip

\textbf{Partie A :}

\medskip

Cette partie est un questionnaire à choix multiples (QCM). 

Pour chaque question, trois réponses sont proposées, une seule est exacte. Recopier le numéro de la question et indiquer, sans justifier dans cette partie seulement, la réponse choisie.

Dans toute cette partie, on considère la fonction définie par: 

\[f(x) = 2x + 3.\]

\begin{center}
\begin{tabularx}{\linewidth}{|p{5cm}|*{3}{>{\centering \arraybackslash}X|}}\hline
&Réponse A&Réponse B&Réponse C\\ \hline
\vspace*{-3cm}\textbf{1.~}La représentation graphique de cette fonction est:&
\psset{unit=0.475cm}
\begin{pspicture*}(-2.5,-2)(3,4.5)
\psaxes[linewidth=1.25pt,labelFontSize=\scriptstyle]{->}(0,0)(-2,-1.95)(3,4.5)
\psplot[plotpoints=500,linewidth=1.2pt,linecolor=blue]{-2}{3}{2 x mul 3 add}
\end{pspicture*}&\psset{unit=0.475cm}
\begin{pspicture*}(-2.5,-2)(3,4.5)
\psaxes[linewidth=1.25pt,labelFontSize=\scriptstyle]{->}(0,0)(-2,-1.95)(3,4.5)
\psplot[plotpoints=500,linewidth=1.2pt,linecolor=blue]{-2}{3}{3 }
\end{pspicture*}&\psset{unit=0.475cm}
\begin{pspicture*}(-2.5,-2)(3,4.5)
\psaxes[linewidth=1.25pt,labelFontSize=\scriptstyle]{->}(0,0)(-2,-1.95)(3,4.5)
\psplot[plotpoints=500,linewidth=1.2pt,linecolor=blue]{-2}{3}{2 x mul}
\end{pspicture*}\\ \hline
\textbf{2.~}L'image de $- 2$ par la fonction $f$ est \ldots&$-7$&$- 1$&3\\ \hline
\begin{tabular}{|*{4}{c|}}\hline
	&A		&B		&C\\ \hline
1	&$x$	&$-2$	&$-1$\\ \hline
2	&$f(x)$	&		&\\ \hline
\end{tabular}

\textbf{3.~}Dans cette feuille de calcul extraite d'un tableur, la formule à saisir dans la cellule B2 avant de l'étirer vers la droite est :&=2*A1 +3&=2*B1 +3&=2*$(-2)$ +3\\ \hline
\end{tabularx}
\end{center}

\medskip

\textbf{Partie B :}

\medskip

\begin{enumerate}
\item Montrer que: $(2x - 1)(3x + 4) - 2x = 6x^2 + 3x - 4$.
\item On considère le triangle CDE tel que : CD $= 3,6$~cm ; CE $= 4,2$~cm et DE $= 5,5$~cm.

Le triangle CDE est-il rectangle ?
\end{enumerate}

\medskip

