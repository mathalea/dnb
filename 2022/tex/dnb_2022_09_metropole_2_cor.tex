\textbf{{\large \textsc{Exercice 2}} \hfill 20 points}

\bigskip

%Yanis vit en France métropolitaine. Il part cet été en Guadeloupe en vacances.
%
%Il se renseigne quant aux locations de véhicules.
%
%Une société de location de voitures à Pointe-à-Pitre propose les tarifs suivants pour un véhicule  5 places de taille moyenne,  assurances non comprises :

\begin{itemize}
\item[$\bullet~~$]Tarif \og Affaire \fg{} : 0,50~\euro{} par kilomètre parcouru.
\item[$\bullet~~$]Tarif \og Voyage court \fg{} : un forfait de 120~\euro{} puis 20 centimes par 
kilomètre parcouru
\item[$\bullet~~$]Tarif \og Voyage long \fg : un forfait de 230~\euro, quel que soit le nombre de kilomètres effectués.
\end{itemize}

\medskip

\begin{enumerate}
\item %Yanis a préparé son plan de route et il fera $280$~km. Il choisit le tarif


%Combien va-t-il payer ?
Avec le tarif \og Affaire \fg{}  Yanis va payer $280 \times 0,50 = 140$~\euro.
\item %S'il parcourt $450$~km, quelle offre est la plus avantageuse financièrement ?
$\bullet~~$Avec le tarif \og Affaire \fg{} il va payer $450 \times 0,5 = 225$~(\euro)  ;

$\bullet~~$Tarif \og Voyage court \fg{} il va payer $120 + 450 \times 0,20 = 120 + 90 = 210$~(\euro)

$\bullet~~$Avec le Tarif \og Voyage long \fg{} il va payer 230~\euro.

Le tarif le plus intéressant est le \og Voyage court \fg.
\item Dans la suite, $x$ désigne le nombre de kilomètres parcourus en voiture.

On considère les trois fonctions  $l,\:m,\:n$ suivantes :

\[l(x) = 230 \qquad m(x) = 0,5x \qquad n(x) = 0,2x + 120\]

	\begin{enumerate}
		\item %Associer, sans justifier, chacune de ces fonctions au tarif correspondant.
		
$l$ correspond au tarif \og Voyage long{} ;

$m$ correspond au tarif \og Affaire \fg{} ;

$n$ correspond au tarif \og Voyage court.
		\item %Déterminer le nombre de kilomètres à parcourir pour que le tarif \og Voyage court \fg{} soit égal au tarif \og Affaire \fg.
		Il faut trouver $x$ tel que $l(x) = n(x)$, soit $ 230 = 0,2x + 120$, d'où $110 = 0,2x$ et en multipliant chaque membre par 5 : $550 = x$.
		
		Pour un voyage de 550 km on paiera le même prix avec le tarif \og Affaire \fg{} ou le tarif \og Voyage court \fg.
	\end{enumerate}
\item 
	\begin{enumerate}
		\item Sur le document joint, tracer les courbes représentatives des fonctions $l,\:m$ et $n$.	
		\item Déterminez graphiquement le nombre de kilomètres que devra atteindre Yanis  pour que le tarif \og Voyage long \fg{} soit le plus avantageux.
		\emph{On laissera les traits de constructions apparents sur le graphique.}

		On constate que pour une distance supérieure à 550 km, le tarif \og Voyage long \fg{} est le plus avantageux.

	\psset{xunit=0.0325cm,yunit=0.05cm,arrowsize=2pt 3}
	\begin{pspicture}(600,260)
	\multido{\n=0+20}{31}{\psline[linewidth=0.15pt](\n,0)(\n,260)}
	\multido{\n=0+20}{14}{\psline[linewidth=0.15pt](0,\n)(600,\n)}
	\psaxes[linewidth=1.25pt,Dx=40,Dy=40]{->}(0,0)(0,0)(600,260)
	\uput[u](528,0){distance parcourue en km}
	\uput[r](0,255){coût en euro}
	\psline[linewidth=1.25pt](0,230)(600,230)
	\psplot[plotpoints=500,linewidth=1.25pt,linecolor=blue]{0}{500}{0.5 x mul}
	\psplot[plotpoints=500,linewidth=1.25pt,linecolor=red]{0}{600}{0.2 x mul 120 add}
	\uput[u](40,230){$l(x) = 230$}
	\rput{36}(40,25){\blue $m(x) = 0,5x$}
	\rput{16}(40,135){$\red n(x) = 0,2x + 120$}
	\psline[linewidth=1.25pt,linestyle=dashed]{->}(550,230)(550,0)
	\uput[d](550,2.5){550}
	\end{pspicture}
	\end{enumerate}
\end{enumerate}

\bigskip

