
\medskip

Une collectionneuse compte ses cartes Pokémon afin de les revendre.

Elle possède $252$ cartes de type \og{}\textit{feu}\fg{} et $156$ cartes de type \og{}\textit{terre}\fg{}.

\begin{enumerate}[itemsep=1em]
\item \begin{enumerate}
  \item Parmi les trois propositions suivantes, laquelle correspond à la décomposition en produit de facteurs premiers du nombre $252$~?

    \begin{center}
      \renewcommand{\arraystretch}{1.5}
      \begin{tabular}{|*{3}{c|}}
      \hline
      Proposition 1 & Proposition 2 & Proposition 3 \\
      $2^2\times9\times7$ & $2\times2\times3\times21$ & $2^2\times3^2\times7$\\
      \hline
    \end{tabular}
  \end{center}
  \item Donner la décomposition en produit de facteurs premiers du nombre $156$.
  \end{enumerate}
\item Elle veut réaliser des paquets identiques, c'est-à-dire contenant chacun le même nombre de cartes \og{}\textit{terre}\fg{}  et le même nombre de cartes \og{}\textit{feu}\fg{}  en utlisant toutes les cartes.
  \begin{enumerate}
  \item Peut-elle faire $36$ paquets~?
  \item Quel est le nombre maximum de paquets qu'elle peut réaliser~?
  \item Combien de cartes de chaque type contient alors chaque paquet~?
  \end{enumerate}
\item Elle choisit une carte au hasard parmi toutes ses cartes. On suppose les cartes indiscernables au toucher.\medskip

  Calculer la probabilité que ce soit une carte de type \og{}\textit{terre}\fg{}.
\end{enumerate}
\medskip

