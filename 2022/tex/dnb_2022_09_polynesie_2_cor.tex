\textbf{{\large \textsc{Exercice 2}} \hfill 22 points}

\medskip

\begin{minipage}{0.53\linewidth}
Un poteau électrique vertical [BC] de $5,2$ m de haut est retenu par un câble métallique [AC] comme montré sur le schéma 1 qui n'est pas en vraie grandeur.
\end{minipage}\hfill
\begin{minipage}{0.43\linewidth}
\psset{unit=0.9cm}
\begin{pspicture}(5.5,5)
\psline[linewidth=1.5pt](3.7,0.3)(3.7,4.7)%BC
\psline(3.7,4.7)(0.5,0.3)(3.7,0.3)%CAB
\psframe(3.7,0.3)(3.5,0.5)
\uput[r](3.7,2.5){Poteau : 5,2 m}
\uput[d](2.1,0.3){Sol : 3,9 m}
\uput[ul](2.1,2.5){Câble}
\rput(1.7,4){\textbf{Schéma 1}}
\uput[dl](0.5,0.3){A} \uput[dr](3.7,0.3){B} \uput[ur](3.7,4.7){C} 
\end{pspicture}
\end{minipage}

\medskip

\begin{enumerate}
\item %Montrer que la longueur du câble [AC] est égale à $6,5$~m.
le théorème de Pythagore appliqué au triangle ABC rectangle en B, s'écrit :

$\text{AB}^2 + \text{BC}^2 = \text{AC}^2$, soit $3,9^2 + 5,2^2 = \text{AC}^2$, ou encore $15,21 + 27,04 = \text{AC}^2$, soit $\text{AC}^2 = 42,25$.

On a donc AC $= \sqrt{42,25} = 6,5$~(m).
\item %Calculer la mesure de l'angle $\widehat{\text{ACB}}$ au degré près.
On a par exemple $\cos \widehat{\text{ACB}} = \dfrac{\text{BC}}{\text{AC}} = \dfrac{5,2}{6,5} = \dfrac{52}{65} = \dfrac{4 \times 13}{5 \times 13} = \dfrac{4}{5} = \dfrac{8}{10} = 0,8$.

La calculatrice donne $\widehat{\text{ACB}} \approx 36,9$.

La mesure de l'angle $\widehat{\text{ACB}}$ est $37\degres$ au degré près.
\end{enumerate}

Deux araignées se trouvant au sommet du poteau (point C) décident de rejoindre le bas du câble (point A) par deux chemins différents.

\begin{enumerate}[resume]
\item %La première araignée se déplace le long du câble [AC] à une vitesse de $0,2$~m/s. 

%Vérifier qu'il lui faut 32,5 secondes pour atteindre le bas du câble.
On a $v = \dfrac dt$, avec $v = 0,2$ et $d = \text{CA} = 6,5$.

Donc $0,2 = \dfrac{6,5}{t}$, d'où $0,2t = 6,5$ et $t = \dfrac{6,5}{0,2} = 6,5 \times 5 = 32,5$~(s).
\item La deuxième araignée décide de parcourir le chemin CFHA indiqué en pointillés sur le schéma 2 (qui n'est pas en vraie grandeur) : elle suit le morceau de câble [CF] en marchant, puis descend verticalement le long de [FH] grâce à son fil et enfin marche sur le sol le long de [HA].

%Calculer les longueurs FH et HA.
Les droites (FH) et (CB) étant toutes les deux perpendiculaires à la droite (AB) sont parallèles.

$\bullet~~$D'après le théorème de Thalès : $\dfrac{\text{FH}}{\text{BC}} = \dfrac{\text{AF}}{\text{AC}}$, soit $\dfrac{\text{FH}}{5,2} = \dfrac{4}{6,5}$, d'où en multipliant chaque membre par 6,5 : $\text{FH} = \dfrac{4 \times 5,2}{6,5} = \dfrac{4 \times 52}{65} = \dfrac{4 \times 4 \times 13}{5 \times 13} = \dfrac{16}{5} = \dfrac{32}{10} = 3,2$~(m).

$\bullet~~$On a de même toujours d'après Thalès : $\dfrac{\text{AH}}{\text{AB}} = \dfrac{\text{AF}}{\text{AC}}$, soit $\dfrac{\text{AB}}{3,9} = \dfrac{4}{6,5}$, d'où en multipliant chaque membre par 3,9 : $\text{AF} = \dfrac{3,9 \times 4}{6,5} = \dfrac{39 \times 4}{65} = \dfrac{3 \times 13 \times 4}{5 \times 13} = \dfrac{12}{5} = \dfrac{24}{10} = 2,4$~(m).
\begin{center}
\psset{unit=0.9cm,arrowsize=2pt 3}
\begin{pspicture}(7,6.8)
%\psgrid
\psline[linewidth=1.5pt](5.4,0.8)(5.4,6.1)%BC
\psline(3.9,4.1)(1.5,0.8)%FA
\psline[linestyle=dashed,ArrowInside=->](5.4,6.1)(3.9,4.1)(3.9,0.8)(1.5,0.8)%CFHA
\psline(3.9,0.8)(5.4,0.8)%HB
\uput[dl](1.5,0.8){A} \uput[dr](5.4,0.8){B} \uput[ur](5.4,6.1){C} \uput[r](3.9,4.1){F} \uput[ur](3.9,0.8){H} 
\rput(1.4,6){\textbf{Schéma 2}}
\psline[linewidth=0.6pt]{<->}(5.8,0.8)(5.8,6.1)\rput{90}(6.05,3.45){5,2~m}
\psline[linewidth=0.6pt]{<->}(1.5,0.4)(5.4,0.4)\uput[d](3.45,0.4){3,9 m}
\psline[linewidth=0.6pt]{<->}(1.3,1)(3.7,4.3)\rput{57.03}(2.4,2.7){4 m}
\psline[linewidth=0.6pt]{<->}(1,1.4)(4.9,6.7)\rput{57.03}(2.8,4.2){6,5 m}
\psframe(3.9,0.8)(3.7,1)\psframe(5.4,0.8)(5.2,1)
\end{pspicture}
\end{center}

\item %La deuxième araignée marche à une vitesse de 0,2 m/s le long des segments [CF] et [HA] et descend le long du segment [FH] à une vitesse de 0,8 m/s.

%Laquelle des deux araignées met le moins de temps à arriver en A ?
De $v = \dfrac{d}{t}$, on tire $d = v \times t$ et $t = \dfrac{d}{v}$.

La deuxième araignée parcourt CF + HA $= (6,5 - 4)  + 2,4 = 4,9$~(m) a la vitesse de 0,2~(m/s).

Elle met donc $t_1 = \dfrac{4,9}{0,2} = 4,9 \times 5 = 24,5$~(s) pour parcourir ces deux segments.

Pour parcourir le segment [FH], elle met $t_2 = \dfrac{3,2}{0,8} = \dfrac{32}{8} = 4$~(s).

Elle met donc au total : $24,5 + 4 = 28,5$~(s) : c'est elle qui met le moins de temps pour arriver en A.
\end{enumerate}

\bigskip

