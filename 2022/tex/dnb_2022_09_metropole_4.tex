\textbf{{\large \textsc{Exercice 4}} \hfill 20 points}

\bigskip

Voici le nombre de passages de véhicules au péage du pont de l'île de Ré au cours de l'année 2020, reporté dans une feuille de calcul :

\begin{center}
\begin{tabularx}{0.5\linewidth}{|c|*{2}{>{\centering \arraybackslash}X|}}\hline
	&A		&B\\ \hline
1 	&Mois 	&Nombre de passages\\ \hline
2 	&Janvier&\np{210320}\\ \hline
3 	&Février&\np{218464}\\ \hline
4 	&Mars 	&\np{138395}\\ \hline
5 	&Avril 	&\np{62930}\\ \hline
6 	&Mai 	&\np{179699}\\ \hline
7 	&Juin 	&\np{295333}\\ \hline
8 	&Juillet&\np{389250}\\ \hline
9 	&Août	&\np{376551}\\ \hline
10 	&Septembre&\np{313552}\\ \hline
11 	&Octobre &\np{267864}\\ \hline
12 	&Novembre&\np{142152}\\ \hline
13 	&Décembre&\np{206662}\\ \hline
14 	&Total&\np{2801172}\\ \hline
\end{tabularx}
\end{center}

\begin{enumerate}
\item Quelle formule a-t-on saisi dans la cellule B14 pour obtenir le nombre total de passages en 2020?
\item Calculer le nombre moyen de passages par mois.
\item Donner l'étendue de la série.
\item Afin d'étudier les effets du confinement de 2020, on souhaite comparer le nombre de
passages de véhicules sur le pont de l'île de Ré du mois de mai 2020 avec celui du mois de mai 2021.

En mai 2021, \np{305214} véhicules ont passé le péage du pont.

Calculer le pourcentage d'augmentation du nombre de passages de véhicules entre mai 2020 et mai 2021. Arrondir à l'unité.
\item Sachant que le pont a une longueur de \np{3000} mètres, quelle est la vitesse moyenne, exprimée en km/h, d'un cycliste qui le traverse en $10$~minutes ?
\end{enumerate}

\bigskip

