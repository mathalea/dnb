\textbf{\large Exercice 2 \hfill 20 points}

\medskip

Paris-Nice est une course cycliste qui se déroule chaque année et qui mène les coureurs de la région parisienne à la région niçoise. L'édition 2021 s'est déroulée en 7 étapes décrites ci-dessous:

\begin{center}
\begin{tabularx}{\linewidth}{|c|c|c|>{\small}X|c|}\hline
\textbf{Étape} &\textbf{Date}&\textbf{Profil}&\textbf{Parcours}&\textbf{Distance}
\\ \hline
1&Dimanche 7 mars&Accidenté&\footnotesize Saint-Cyr-l'École$\to$Saint-Cyr-l'École&166 km\\ \hline
2&Lundi 8 mars&Plat&Oinville-sur-Montcient$\to$Amilly&188 km\\ \hline
3&Mercredi 10 mars&Accidenté& Chalon-sur-Saône$\to$Chiroubles&187,5 km\\ \hline
4&Jeudi 11 mars &Plat&Vienne$\to$Bollène&200 km\\ \hline
5&Vendredi 12 mars&Accidenté&Brignoles$\to$Biot&202,5 km\\ \hline
6&Samedi 13 mars&Montagneux&Le Broc$\to$Valdeblore La Colmiane&119,5 km\\ \hline
7&Dimanche 14 mars&Accidenté& Le Plan-du-Var$\to$Levens&93 km\\ \hline
\end{tabularx}
\end{center}

\begin{enumerate}
\item On étudie la série des distances parcourues par étape.
	\begin{enumerate}
		\item La distance moyenne parcourue par étape est en km:%, arrondie au dixième de km. 
		
$\dfrac{166+188+187,5+200+202,5+119,5+93}{7}=\dfrac{\np{1156,5}}{7} \approx 165,2$.
		
		\item Pour calculer la médiane des distances parcourues par étape, on commence par ranger les distances en ordre croissant:
		
93 -- $119,5$ -- 166 -- $187,5$ -- 188 -- 200 -- $202,5$

Il y a un nombre impair de distances donc la médiane est la distance située \og au milieu \fg{} donc la 4\ieme{}, c'est-à-dire $187,5$~km.
		
		\item L'étendue de la série formée par les distances parcourues par étape est égale à $202,5-93$ soit $109,5$~km.
	\end{enumerate}	
	
\item Un journaliste affirme: \og Environ 57\,\% du nombre total d'étapes de cette édition se sont déroulées sur un parcours accidenté. \fg{} 

%A-t-il raison ? Expliquer votre réponse.
Il y a en tout 4 étapes sur 7 en profil accidenté soit un pourcentage de 

$\dfrac{4}{7} \times 100 \approx 0,571 \times 100$, soit environ $57\,\%$ : le journaliste a raison.
%Le nombre total d'étapes sur parcours accidenté correspond à une distance de
%$166+187,5+202,5+93$ soit $649$~km.
%
%Le nombre total de kilomètres parcourus est de $\np{1156,5}$.
%
%$\dfrac{640}{\np{1156,5}}\times 100 \approx 56,1$; donc le journaliste s'est trompé de 1\,\%.

\item L'allemand Maximilian SCHACHMANN a remporté la course en 28~h 50~min.

Le dernier au classement général a effectué l'ensemble du parcours en 30~h 12~min.

De 28~h 50~min à 29~h, il y a 10~min, et de 29~h à 30~h~12~min, il y a 1~h~12~min; donc de 28~h 50~min à 30~h~12~min, il y a 1~h~22~min.

Le dernier au classement a donc  accumulé 1 heure et 22 minutes de retard par rapport au vainqueur.

\item L'Irlandais Sam BENNETI a remporté la première étape en 3~h 51~min, soit $3\times 60+51=231$~min.
Sa vitesse moyenne en km/h répond à la question: il a parcouru 166 kilomètres en 231 minutes, combien de kilomètres  a-t-il parcourus en 60 minutes?

$\dfrac{166}{231}\times 60 \approx 43$ donc la vitesse moyenne du vainqueur est de 43~km/h.

%Déterminer sa vitesse moyenne en km/h, arrondie à l'unité, lors de cette étape.
\end{enumerate}

\bigskip

