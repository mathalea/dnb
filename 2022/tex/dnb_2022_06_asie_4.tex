
\medskip

Une boutique en ligne vend des photos et affiche les tarifs suivants :

%\begin{tblr}{colspec={Q[c,m] X[c,m]}, hlines, vlines}
\begin{tabularx}{\linewidth}{|m{6cm}|>{\centering \arraybackslash}X|}\hline
%\SetRow{gray8} Nombre de photos commandées & Prix à payer\\
Nombre de photos commandées & Prix à payer\\ \hline
De 1 à 100 photos & 0,17 \euro{} par photo\\
Plus de 100 photos & 17~\euro{} pour l’ensemble des 100 premières photos
et 0,13~\euro{} par photo supplémentaire\\ \hline
\end{tabularx}
%\end{tblr}


\begin{enumerate}
	\item \begin{enumerate}
		\item Quel est le prix à payer pour 35 photos ?
		\item Vérifier que le prix à payer pour 150 photos est 23,50 \euro{}.
		\item On dispose d’un budget de 10 \euro{}. Combien de photos peut-on commander au maximum ?
	\end{enumerate}
\end{enumerate}

On a commencé à construire un programme qui doit permettre de calculer le prix à payer en fonction du nombre de photos commandées :

\smallskip

%\begin{tblr}{|l|X[l]|}\hline
\begin{tabularx}{\linewidth}{|X|p{5cm}|}\hline
	\begin{scratch}[num blocks,scale=0.9]
		\blockinit{quand \greenflag est cliqué}
		\blocksensing{demander \ovalnum{Nombre de photos à commander ?} et attendre}
		\blockvariable{mettre \selectmenu{Nb photos} à \ovalsensing{réponse}}
		\blockifelse{si \booloperator{\ovalvariable{Nb photos} < \ovalnum{}}}
		{\blockvariable{mettre \selectmenu{Prix} à \ovaloperator{\ovalvariable{Nb photos} * \ovalnum{}}}}
		{\blockvariable{mettre \selectmenu{Nb photos supplémentaires} à \ovaloperator{\ovalvariable{Nb photos} - \ovalnum{100}}}
		\blockvariable{mettre \selectmenu{Prix} à \ovaloperator{\ovalnum{} + \ovalvariable{Nb photos supplémentaires} * \ovalnum{0.13}}}}
		\blocklook{dire \ovaloperator{regrouper \ovalnum{Prix à payer en euros} et \selectmenu{Prix}}}
	\end{scratch} &\textbf{Informations :}

Le programme comporte trois variables :

%		\bigskip
%
%		\begin{minipage}{\linewidth}
%		\begin{itemize}[leftmargin=*]
%			\item \ovalvariable{Nb photos}
%
%	 		Nombre de photos commandées
%
%			\item \ovalvariable{Nb photos supplémentaires}
%
%			Nombre de photos commandées au-delà des 100 premières photos commandées.
%
%			\item \ovalvariable{Prix}
%		\end{itemize}
%		\end{minipage}

\bigskip

$\bullet~~$\ovalvariable{Nb photos}

Nombre de photos commandées

\bigskip

$\bullet~~$\ovalvariable{Nb photos supplémentaires}

Nombre de photos commandées au-delà des 100 premières photos commandées.

\bigskip

$\bullet~~$\ovalvariable{Prix}\\ \hline
	\end{tabularx}
%\end{tblr}

\smallskip

\begin{enumerate}[resume]
	\item \emph{Dans cette question, aucune justification n'est attendue.}

	Par quelles valeurs peut-on compléter les instructions des lignes 4, 5 et 8 pour que le programme permette de calculer le prix à payer en fonction du nombre de photos commandées ?

	\textbf{\emph{Sur la copie, écrire le numéro de chaque ligne à compléter et la valeur correspondante.}}

	\item En période des soldes, le site offre une réduction de 30\,\% sur le prix à payer, pour toute commande supérieure à 20 \euro{}.
	\begin{enumerate}
		\item Calculer le prix a payer pour $150$ photos en période des soldes.

		\item \emph{Dans cette question, aucune justification n’est attendue.}

\begin{minipage}[t]{8 cm}
On modifie le programme pour qu’il donne le prix à payer en période des soldes en insérant le bloc ci-contre entre les lignes 8 et  9.

Dans la liste suivante, indiquer une proposition qui convient pour compléter la case vide :
		\end{minipage} \hfill
		\begin{scratch}
			\blockif{si \booloperator{\ovalvariable{Prix} > \ovalnum{20} } alors}
			{\blockvariable{mettre \selectmenu{Prix} à \ovalnum{}}}
		\end{scratch}
%		\begin{tblr}{@{} l @{~}X[l] l@{~}l}
\begin{tabular}{l l l l}\hline
			Proposition 1 :&\ovaloperator{\ovalvariable{Prix} - \ovalnum{30}}&
			Proposition 2 :&\ovaloperator{\ovalvariable{Prix} - \ovaloperator{\ovalvariable{Prix} * \ovalnum{0.3}}}\\
			Proposition 3 :&\ovaloperator{\ovalvariable{Prix} * \ovaloperator{\ovalnum{30} / \ovalnum{100}}}&
			Proposition 4 :&\ovaloperator{\ovalvariable{Prix} * \ovalnum{0.7}}\\
\end{tabular}			
%		\end{tblr}
	\end{enumerate}
\end{enumerate}

\medskip

