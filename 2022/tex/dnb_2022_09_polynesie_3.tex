\textbf{{\large \textsc{Exercice 3}} \hfill 17 points}

\medskip

On utilise un logiciel de programmation.

On rappelle que \og s'orienter à $0\degres$ \fg{} signifie qu'on oriente le stylo vers le haut.

On considère les deux scripts suivants:

\begin{center}
\begin{tabularx}{\linewidth}{X X}
\multicolumn{1}{c}{Script 1}&\multicolumn{1}{c}{Script 2}\\
\begin{scratch}
\blockinit{Quand \greenflag est cliqué}
\blockpen{effacer tout}
\blockpen{stylo en position d'écriture}
\blockmove{s’orienter à \ovalnum{0} }
\blockrepeat{répéter \ovalnum{2} fois}
{\blockmove{avancer de \ovalnum{20}}
\blockmove{tourner \turnright{} de \ovalnum{90} degrés}
\blockmove{avancer de \ovalnum{40}}
\blockmove{tourner \turnleft{} de \ovalnum{90} degrés}
}
\end{scratch}&
\begin{scratch}
\blockinit{Quand \greenflag est cliqué}
\blockpen{effacer tout}
\blockpen{stylo en position d'écriture}
\blockmove{s’orienter à \ovalnum{0} }
\blockvariable{mettre \selectmenu{longueur} à \ovalnum{20}}
\blockrepeat{répéter \ovalnum{2} fois}
{\blockmove{avancer de \selectmenu{longueur}}
\blockmove{tourner \turnright{} de \ovalnum{90} degrés}
\blockmove{avancer de \selectmenu{longueur}}
\blockmove{tourner \turnleft{} de \ovalnum{90} degrés}
\blockvariable{ajouter à \selectmenu{longueur} \ovalnum{20}}
}
\end{scratch}\\
\end{tabularx}
\end{center}

\medskip

\begin{enumerate}
\item On exécute le script 1 ci-dessus.

Représenter le chemin parcouru par le stylo sur le document à rendre avec la copie.

\medskip
\hrulefill

\medskip
\begin{minipage}{0.48\linewidth}
\psset{unit=5mm,arrowsize=2pt 3}
\begin{pspicture}(10,12)
\multido{\n=0+1}{10}{\psline[linewidth=0.2pt](\n,3)(\n,12)}
\multido{\n=3+1}{10}{\psline[linewidth=0.2pt](0,\n)(9,\n)}
\multido{\n=0+1}{10}{\multido{\na=3+1}{10}{\psdots(\n,\na)}}
\psframe(9,2)
\rput(4.5,1){Position de départ}
%\psline[linewidth=1.5pt](1,4)(1,5)(2,5)(2,6)(3,6)
\psline{->}(4.5,2)(1,4)
\end{pspicture}
\end{minipage}\hfill
\begin{minipage}{0.48\linewidth}
Chaque côté de carreau mesure 20 pixels.

La position de départ du stylo est indiquée sur la figure ci-contre.
\end{minipage}

\item Quel dessin parmi les trois ci-dessous correspond au script 2 ? 

On expliquera pourquoi les deux autres dessins ne correspondent pas au script 2.

Chaque côté de carreau mesure 20 pixels.

\begin{center}
\begin{tabularx}{\linewidth}{*{3}{>{\centering \arraybackslash}X}}
\textbf{Dessin 1}&\textbf{Dessin 2} &\textbf{Dessin 3}\\
\psset{unit=4.5mm,arrowsize=2pt 3}
\begin{pspicture}(9,12)
\multido{\n=0+1}{10}{\psline[linewidth=0.2pt](\n,3)(\n,12)}
\multido{\n=3+1}{10}{\psline[linewidth=0.2pt](0,\n)(9,\n)}
\psframe(9,2)
\rput(4.5,1){Position de départ}
\psline[linewidth=1.5pt](1,4)(1,5)(2,5)(2,6)(3,6)
\psline{->}(4.5,2)(1,4)
\end{pspicture}&
\psset{unit=4.5mm,arrowsize=2pt 3}
\begin{pspicture}(9,12)
\multido{\n=0+1}{10}{\psline[linewidth=0.2pt](\n,3)(\n,12)}
\multido{\n=3+1}{10}{\psline[linewidth=0.2pt](0,\n)(9,\n)}
\psframe(9,2)
\rput(4.5,1){Position de départ}
\psline[linewidth=1.5pt](1,4)(1,5)(2,5)(2,7)(4,7)
\psline{->}(4.5,2)(1,4)
\end{pspicture}&
\psset{unit=4.5mm,arrowsize=2pt 3}
\begin{pspicture}(9,12)
\multido{\n=0+1}{10}{\psline[linewidth=0.2pt](\n,3)(\n,12)}
\multido{\n=3+1}{10}{\psline[linewidth=0.2pt](0,\n)(9,\n)}
\psframe(9,2)
\rput(4.5,1){Position de départ}
\psline[linewidth=1.5pt](1,4)(2,4)(2,5)(4,5)(4,7)
\psline{->}(4.5,2)(1,4)
\end{pspicture}
\end{tabularx}
\end{center}

\item \phantom{rrr}

\begin{minipage}{0.5\linewidth}
On souhaite maintenant obtenir le motif représenté sur le dessin 4 : 

\begin{center}
\psset{unit=4.5mm,arrowsize=2pt 3}
\begin{pspicture}(9,13)
\multido{\n=0+1}{10}{\psline[linewidth=0.2pt](\n,3)(\n,12)}
\multido{\n=3+1}{10}{\psline[linewidth=0.2pt](0,\n)(9,\n)}
\psframe(9,2)
\uput[u](4.5,12){\textbf{Dessin 4}}
\rput(4.5,1){Position de départ}
\pspolygon[linewidth=1.5pt](1,4)(1,5)(3,5)(3,9)(5,9)(5,10)(6,10)(6,4)
\psline{->}(4.5,2)(1,4)
\end{pspicture}
\end{center}

Compléter sans justifier les trois cases du script 3 donné en document à rendre avec la copie, permettant d'obtenir le dessin 4.
\end{minipage}
\hfill
\begin{minipage}{0.4\linewidth}
	\psset{unit=1cm,arrowsize=2pt 3}
\begin{pspicture}(12,12)
%\psgrid
\rput(1,11.5){Script 3}\psframe(0,11)(2,12)
\psline{->}(3.5,6)(3,6)
\psline{->}(3.5,1.75)(3,1.75)
\psline{->}(3.5,-1)(3,-1)
\rput(4,2.25){\psscaleboxto(1,9){\}}}
\rput(2,4){\begin{scratch}[scale=0.8]
\blockinit{Quand \greenflag est cliqué}
\blockpen{effacer tout}
\blockpen{stylo en position d'écriture}
\blockmove{s’orienter à \ovalnum{0}}
\blockmove{avancer de \ovalnum{20}}
\blockmove{tourner \turnright{} de \ovalnum{90} degrés}
\blockmove{avancer de \ovalnum{\dots\dots}}
\blockmove{tourner \turnleft{} de \ovalnum{90} degrés}
\blockmove{avancer de \ovalnum{80}}
\blockmove{tourner \turnright{} de \ovalnum{90} degrés}
\blockmove{avancer de \ovalnum{40}}
\blockmove{tourner \turnleft{} de \ovalnum{90} degrés}
\blockmove{avancer de \ovalnum{\dots\dots}}
\blockmove{tourner \turnright{} de \ovalnum{90} degrés}
\blockmove{avancer de \ovalnum{20}}
\blockmove{tourner \turnright{} de \ovalnum{90} degrés}
\blockmove{avancer de \ovalnum{\dots\dots}}
\blockmove{tourner \turnright{} de \ovalnum{90} degrés}
\blockmove{avancer de \ovalnum{100}}
\end{scratch}}
\rput(6.5,2.2){Trois cases à compléter}
\psframe(4.5,1.7)(8.5,2.7)
\end{pspicture}
\end{minipage}

\vspace*{30mm}
\item À partir du motif représenté sur le dessin 4, on peut obtenir le pavage ci-dessous :

\begin{center}
\psset{unit=4.5mm}
\begin{pspicture}(24,18)
\psline[linewidth=1.5pt](0,0)(1,0)(1,1)(3,1)(3,5)(5,5)(5,7)(3,7)(3,11)(1,11)(1,13)(3,13)(3,17)(5,17)(5,18)(7,18)(7,17)(9,17)(9,13)(11,13)(11,11)(9,11)(9,7)(7,7)(7,5)(9,5)(9,1)(11,1)(11,0)(13,0)
\psline[linewidth=1.5pt](13,0)(13,1)(15,1)(15,5)(17,5)(17,7)(15,7)(15,11)(13,11)(13,13)(15,13)(15,17)(17,17)(17,18)(19,18)(19,17)(21,17)(21,13)(23,13)(23,11)(21,11)(21,7)(19,7)(19,5)(21,5)(21,1)(23,1)(23,0)(25,0)
\psframe[linewidth=1.5pt](18,12)\psframe[linewidth=1.5pt](6,0)(12,18)
\psframe[linewidth=1.5pt](18,6)
\psline[linewidth=1.5pt](6,12)(6,18)\psline[linewidth=1.5pt](12,12)(12,18)
\psline[linewidth=1.5pt](18,12)(18,18)
\psline[linewidth=1.5pt](12,18)(19,18)\psline[linewidth=1.5pt](18,12)(23,12)
\psline[linewidth=1.5pt](18,6)(19,6)\psline[linewidth=1.5pt](18,0)(23,0)
\uput[ul](1,12){A} \uput[ul](6,12){B} \uput[dl](12,18){C} \uput[ur](13,12){D} 
\uput[ur](18,12){E} \uput[ur](0,6){F} \uput[ur](12,6){G}
\multido{\n=4.5+3.0,\na=1+1}{6}{\rput(\n,14){\na}}
\multido{\n=2.+3.0,\na=7+1}{7}{\rput(\n,9){\na}}
\multido{\n=2.+3.0,\na=14+1}{7}{\rput(\n,3){\na}}
\end{pspicture}
\end{center}

Répondre aux questions suivantes sur votre copie en indiquant le numéro du motif qui convient (on ne demande pas de justifier la réponse) :

	\begin{enumerate}
		\item Quelle est l'image du motif 1 par la translation qui transforme le point B en E ?
		\item Quelle est l'image du motif 1 par la symétrie de centre B ?
		\item Quelle est l'image du motif 16 par la symétrie de centre G ?
		\item Quelle est l'image du motif 2 par la symétrie d'axe (CG) ?
	\end{enumerate}	
\end{enumerate}

\bigskip

