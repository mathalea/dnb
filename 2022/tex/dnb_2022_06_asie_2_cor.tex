\textbf{\large Exercice 2 \hfill 20 points}

\medskip

\begin{enumerate}
\item Les droites (AB) et (CD) sont parallèles et les droites (AD) et (BC) sont sécantes en E donc d'après le théorème de Thalès on a :

$\dfrac{\text{AB}}{\text{DC}} = \dfrac{\text{EB}}{\text{EC}} = \dfrac{\text{EA}}{\text{ED}}$, soit $\dfrac{9}{6}  = \dfrac{7,2}{\text{EC}}$.

On en déduit que $\text{EC}\times 9 = 6 \times 7,2$, puis $\text{EC} = \dfrac{6 \times 7,2}{9} = 6 \times 0,8 = 4,8$~cm.
\item $\text{DC}^2 = 6^2= 36$ et $\text{ED}^2 + \text{EC}^2 = 3,6^2+ 4,8^2 = 12,96 + 23,04 = 36$.

Donc DC$^2 = \text{ED}^2 +\text{EC}^2$ : par conséquent d'après la réciproque du théorème de Pythagore le triangle EDC est rectangle en E.
\item Le triangle ABE est l'image du triangle EDC par l'homothétie de centre E et de
rapport $-\dfrac{9}{6} = - \dfrac32 = - 1,5$.
\item D'après la question 3 nous savons que l'aire du triangle ABE est $1,5^2$ fois plus grande que l'aire du triangle EDC.

L'affirmation est fausse, le coefficient d'agrandissement doit être mis au carré pour l'image d'une aire.
\end{enumerate}

\bigskip

