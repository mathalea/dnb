
\medskip

\begin{enumerate}
\item Dans le triangle HMS, rectangle en H, on connaît MH = 5 cm et MS = 13 cm.

D'après le théorème de Pythagore, on sait que : \qquad $\mathrm{MS}^2 = \mathrm{MH}^2 + \mathrm{HS}^2$

En remplaçant les longueurs connues : \qquad $13^2 = 5^2 + \mathrm{HS}^2$

Donc : \qquad $\mathrm{HS}^2 = 13^2 - 5^2 =169 - 25 =144 $

Comme HS est une longueur, elle est donc positive, et donc on en déduit :

$\mathrm{HS} =\sqrt{144} = 12$ cm.

\item On sait que :
\begin{itemize}[label = $ \bullet $]
	\item Les points H, M et T sont alignés, dans cet ordre;
	\item Les points S, M et A sont alignés dans le même ordre;
	\item Les droites (HS) et (MT) sont parallèles entre elles, car elles sont perpendiculaires à la même troisième droite (HT).
\end{itemize}

D'après le théorème de Thalès appliqué dans cette configuration, on en déduit :

$ \dfrac{\mathrm{MH}}{\mathrm{MT}} = \dfrac{\mathrm{MS}}{\mathrm{MA}} = \dfrac{\mathrm{HS}}{\mathrm{AT}}$.
Notamment : \qquad $ \dfrac{\mathrm{MH}}{\mathrm{MT}} = \dfrac{\mathrm{HS}}{\mathrm{AT}} $

Soit, en remplaçant par les valeurs connues : \qquad $ \dfrac{5}{7} = \dfrac{12}{\mathrm{AT}} $

À l'aide d'un produit en croix, on a donc : $\mathrm{AT} = \dfrac{12\times 7}{5} = 16,8$ cm

\emph{Remarque :} On aurait aussi pu utiliser la notion de triangle semblable.

\item Dans le triangle HMS, rectangle en H, on peut utiliser la trigonométrie. Ici, comme toutes les longueurs du triangle sont connues, on peut utiliser le sinus, le cosinus ou la tangente.

Notamment :
\qquad
$\cos \left(\widehat{\text{HMS}}\right) = \dfrac{\mathrm{HM}}{\mathrm{MS}} = \dfrac{5}{13}$.

On en déduit :\qquad $ \widehat{\text{HMS}} = \arccos\left(\dfrac{5}{13}\right) \approx 67$\textdegree, arrondi au degré près.

\item Les triangles MAT et MHS sont semblables, mais ils n'ont pas les mêmes dimensions, donc les symétries (axiales et centrales), les rotations et les translations conservant les longueurs, ce n'est pas possible.

Par contre, une homothétie est possible. Ici, si on veut préciser, c'est une homothétie, de centre M et de rapport $ -\dfrac{7}{5} $.

\emph{Remarque :} Ici, aucune justification ni précision n'était attendue.

\item L'affirmation est fausse : on sait que le triangle MAT est un agrandissement de MSH de rapport $k =1,4$, donc les longueurs seront bien multipliées par 1,4, mais les surfaces seront multipliées par $k^2 = 1,4^2 = 1,96$

\emph{Remarque :} une autre justification possible est de calculer les aires des triangles rectangles. Puisque la base \textbf{et} la hauteur sont multipliées par 1,4, l'aire est bien multipliée par $1,4^2$.
\end{enumerate}

\bigskip

