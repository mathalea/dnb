\textbf{{\large \textsc{Exercice 4}} \hfill 20 points}

\bigskip

Voici le nombre de passages de véhicules au péage du pont de l'île de Ré au cours de l'année 2020, reporté dans une feuille de calcul :

\begin{center}
\begin{tabularx}{0.55\linewidth}{|c|*{1}{>{\centering \arraybackslash}X|}c|}\hline
	&A		&B\\ \hline
1 	&Mois 	&Nombre de passages\\ \hline
2 	&Janvier&\np{210320}\\ \hline
3 	&Février&\np{218464}\\ \hline
4 	&Mars 	&\np{138395}\\ \hline
5 	&Avril 	&\np{62930}\\ \hline
6 	&Mai 	&\np{179699}\\ \hline
7 	&Juin 	&\np{295333}\\ \hline
8 	&Juillet&\np{389250}\\ \hline
9 	&Août	&\np{376551}\\ \hline
10 	&Septembre&\np{313552}\\ \hline
11 	&Octobre &\np{267864}\\ \hline
12 	&Novembre&\np{142152}\\ \hline
13 	&Décembre&\np{206662}\\ \hline
14 	&Total	&\np{2801172}\\ \hline
\end{tabularx}
\end{center}

\begin{enumerate}
\item %Quelle formule a-t-on saisi dans la cellule B14 pour obtenir le nombre total de passages en 2020 ?
On saisit en B14 \fbox{=somme(B2:B13)}.
\item %Calculer le nombre moyen de passages par mois.
Il y a eu en moyenne $\dfrac{\np{2801172}}{12} = \np{233431}$ passages par mois en 2020.
\item %Donner l'étendue de la série.
L'étendue de la série est $\np{389250} - \np{62930} = \np{326320}$.
\item %Afin d'étudier les effets du confinement de 2020, on souhaite comparer le nombre de
%passages de véhicules sur le pont de l'île de Ré du mois de mai 2020 avec celui du mois de mai 2021.

%En mai 2021, \np{305214} véhicules ont passé le péage du pont.

%Calculer le pourcentage d'augmentation du nombre de passages de véhicules entre mai 2020 et mai 2021. Arrondir à l'unité.
L'augmentation du nombre de passages de véhicules entre mai 2020 et mai 2021 est en pourcentage :

$\dfrac{\np{305214} - \np{179699}}{\np{179699}} \times 100 = \dfrac{\np{125515}}{\np{179699}} \times 100 \approx 69,8$, soit 70\,\% à 1\,\% près.
\item %Sachant que le pont a une longueur de \np{3000} mètres, quelle est la vitesse moyenne, exprimée en km/h, d'un cycliste qui le traverse en $10$~minutes ?
Le cycliste parcourt \np{3000}~m en 10~min soit $6 \times \np{3000} = \np{18000}$~m ou 18 km en $6 \times 10 = 60$ soit une heure : sa vitesse moyenne est donc égale à 18~km/h.
\end{enumerate}

\bigskip

