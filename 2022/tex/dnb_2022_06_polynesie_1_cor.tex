\textbf{\large Exercice 1 \hfill 20 points}

\medskip

\begin{enumerate}
\item On a $- \dfrac75 +  \dfrac65 \times \dfrac47  = - \dfrac75 \times \dfrac77 + \dfrac65 \times \dfrac47 = \dfrac{- 49 + 24}{5\times 7} = \dfrac{- 25}{5\times 7} = \dfrac{- 5}{7} = - \dfrac{5}{7}$.

L'affirmation 1 est fausse.
\item Les points G, A et R sont alignés dans cet ordre et les points E, A et M sont alignés dans ce même ordre.

On a d'une part : $\dfrac{\text{AM}}{\text{AE}} = \dfrac{3}{4,2} = \dfrac{30}{42} = \dfrac{1 \times 3 \times 5}{2 \times 3 \times 7} = \dfrac57$ et d'autre part :

$\dfrac{\text{AR}}{\text{AG}} = \dfrac{7}{9,8} = \dfrac{70}{98} = \dfrac{7 \times 2 \times 5}{2 \times 7 \times 7} = \dfrac57$.

Par conséquent $\dfrac{\text{AM}}{\text{AE}} = \dfrac{\text{AR}}{\text{AG}}$ :  d'après la réciproque du théorème de Thalès les droites (MR) et (GE) sont parallèles. L'affirmation 2 est donc vraie.
\item ~

\begin{minipage}{0.3\linewidth}
\begin{tabular}{r|r}
126 &2\\
63 &3\\
21& 3\\
7&7\\
1&\\
\end{tabular}
\end{minipage}\hfill
\begin{minipage}{0.65\linewidth}
Donc $126 = 2 \times 3^2 \times 7$.
\end{minipage}

L'affirmation 3 est fausse car 9 n'est pas un nombre premier.
\item  Il y a en tout $1+3+7 =11$ portions pour un volume total de 330 mL.

Le volume de la portion est donc : $\dfrac{330}{11} = 30$~(mL).

Le volume d'huile utilisé pour $330$ mL de sauce salade est donc égal à $7 \times  30 =210$ (mL). L'affirmation 4 est donc vraie.

\end{enumerate}

\bigskip

