\textbf{{\large \textsc{Exercice 1}} \hfill 20 points}

\medskip

%Cet exercice est un QCM (Questionnaire à Choix Multiples).
%
%Chaque question n'a qu'une seule bonne réponse.
%
%Pour chaque question, précisez \textbf{sur la copie} le numéro de la question et la réponse choisie.
%
%Aucune justification n'est demandée pour cet exercice.
%
%Aucun point ne sera retiré en cas de mauvaise réponse.

%\begin{center}
%\renewcommand\arraystretch{2}
%\begin{tabularx}{\linewidth}{|m{5cm}|*{3}{>{\centering \arraybackslash}X|}}\cline{2-4}
%\multicolumn{1}{c|}{~}&\textbf{Réponse A}&\textbf{Réponse B}&\textbf{Réponse C}\\ \hline
%\textbf{1.~} $\dfrac{5^7 \times 5^3}{5^2}$&$5^{13}$&$5^5$&$5^8$\\ \hline
%\textbf{2.~} La fraction irréductible égale à $\dfrac{630}{882}$ est :&$\dfrac57$&$\dfrac{35}{49}$&$\dfrac{315}{441}$\\ \hline
%\textbf{3.~} Une expression développée de 
%$A = (x - 2)(3x + 7)$ est:&$3x^2 +13x +14$&$3x^2 +x +5$&$3x^2 +x -14$\\ \hline
%\textbf{4.~} 
%Les solutions de l'équation $(2x + 1)(- x + 3) = 0$ sont:&2 et $- 3$&$- \dfrac12$ et 3&$- 1$ et $- 3$\\ \hline
%\textbf{5.~} 
%Une urne contient 9 boules indiscernables au toucher :
%\begin{itemize}
%\item[$\bullet~~$]3 boules noires,
%\item[$\bullet~~$]4 boules blanches,
%\item[$\bullet~~$]2 boules rouges.
%\end{itemize}
%
%Quelle est la probabilité de ne pas tirer de boule noire ?&$\dfrac29$&$\dfrac13$&$\dfrac69$\\ \hline
%\end{tabularx}
%\renewcommand\arraystretch{1}
%\end{center}
\begin{enumerate}
\item $\dfrac{5^7 \times 5^3}{5^2} = \dfrac{5^{10}}{5^2} = 5^8$.
\item $\dfrac{630}{882} = \dfrac{9 \times 70}{2 \times 441} = \dfrac{9 \times 7 \times 2 \times 5}{2 \times 21^2}= \dfrac{2 \times 3^2 \times 7 \times 5}{2 \times 3^2 \times 7^2} = \dfrac57$.
\item $A = (x - 2)(3x + 7) = 3x^2 + 7x - 6x - 14 = 3x^2 + x - 14$.
\item $(2x + 1)(- x + 3) = 0$ ou $\left\{\begin{array}{l c l}
2x + 1&=&\\- x + 3&=&0\\
\end{array}\right.$ ou $\left\{\begin{array}{l c l}
2x &=&- 1\\3&=&x\\
\end{array}\right.$ et enfin $\left\{\begin{array}{l c l}
x &=&- \dfrac12\\3&=&x\\
\end{array}\right.$
\item La probabilité de tirer une boule noire est $p(N) = \dfrac39 = \dfrac13$, donc la probabilité de ne pas tirer de boule noire est égale à $1 - \dfrac13 = \dfrac23$.
\end{enumerate}

\bigskip

