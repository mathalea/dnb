
\medskip

Le graphique ci-dessous représente les deux tarifs pratiqués dans une salle de sport, selon le nombre d'heures effectuées :

\setlength\parindent{9mm}
\begin{itemize}
\item[$\bullet~~$] la droite $\left(d_1\right)$ est la représentation graphique du tarif \og liberté \fg
\item[$\bullet~~$]la droite $\left(d_2\right)$ est la représentation graphique du tarif \og abonné \fg
\end{itemize}
\setlength\parindent{0mm}

%%%%%%
\begin{center}
\psset{xunit=1cm,yunit=0.2cm}
\begin{pspicture}(-1,-4)(11,35)
\multido{\n=0+1}{12}{\psline[linewidth=0.4pt](\n,0)(\n,35)}
\multido{\n=0+5}{8}{\psline[linewidth=0.4pt](0,\n)(11,\n)}
\psaxes[linewidth=1.25pt,Dy=5,labelFontSize=\scriptstyle]{->}(0,0)(0,0)(11,35)
\psplot[plotpoints=400,linewidth=1.25pt,linecolor=red]{0}{11}{2.5 x mul 7.5 add}
\psplot[plotpoints=400,linewidth=1.25pt,linecolor=blue]{0}{7}{5 x mul}
\uput[u](6.5,33){\blue $\left(d_1\right)$}
\uput[u](10.3,33){\red $\left(d_2\right)$}
\rput{90}(-1,26){Prix payé en euro}
\uput[d](9,-2){Nombres d'heures effectuées}
\end{pspicture}
\end{center}

\begin{enumerate}
\item Le prix payé avec le tarif \og liberté \fg{} est-il proportionnel au nombre d'heures effectuées dans la salle de sport ? Expliquer la réponse.
\item On appelle:

\setlength\parindent{9mm}
\begin{itemize}
\item[$\bullet~~$]$f$ la fonction qui, au nombre d'heures effectuées, associe le prix payé en euro avec le tarif \og liberté \fg
\item[$\bullet~~$]$g$ la fonction qui, au nombre d'heures effectuées, associe le prix payé en euro avec le tarif \og abonné \fg
\end{itemize}
\setlength\parindent{0mm}

Répondre aux questions suivantes par lecture graphique :
	\begin{enumerate}
		\item Quelle est l'image de 5 par la fonction $f$ ?
		\item Quel est l'antécédent de 10 par la fonction $g$ ?
	\end{enumerate}
\item À l'aide du graphique, indiquer le tarif parmi les deux proposés qui est le plus avantageux pour une personne selon le nombre d'heures qu'elle souhaite effectuer dans la salle de sport.
\item Déterminer le prix payé avec le tarif \og liberté \fg{} pour $15$ heures effectuées. 

Expliquer la démarche, même si elle n'est pas aboutie.
\end{enumerate}

\bigskip

