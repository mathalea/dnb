
\medskip

On considère le programme de calcul suivant :

\begin{center}
\psset{unit=0.75cm,arrowsize=2pt 3}
\begin{pspicture}(-5,0)(5,11)
%\psgrid
\rput(0,10.8){Choisir un nombre}
\psline{->}(0,10.7)(0,10)
\psframe(-1.5,9)(1.5,10)
\rput(-3,8.5){Ajouter 5}
\psline{->}(0,9)(-3,7.6)\psline{->}(0,9)(3,7.6)
\psframe(-4,6.6)(-2,7.6)\psframe(4,6.6)(2,7.6)
\rput(3,8.5){Soustraire 5}
\psline{->}(-3,6.6)(0,4)(0,3)\psline(3,6.6)(0,4)
\rput(0,5.8){Multiplier les deux}
\rput(0,5.4){résultats}
\psframe(-1,3)(1,2)
\psline{->}(0,2)(0,1)
\rput(1.3,1.6){Ajouter 25}
\psframe(-1,1)(1,0)
\end{pspicture}
\end{center}

%\medskip

\begin{enumerate}
\item 
	\begin{enumerate}
		\item Si on choisit le nombre $7$, vérifier qu'on obtient 49 à la fin du programme
		\item Si on choisit le nombre $- 4$, quel résultat obtient-on à la fin du programme ?
	\end{enumerate}	
\item On note $x$ le nombre choisi au départ
	\begin{enumerate}
		\item Exprimer en fonction de $x$ le résultat obtenu. 
		\item Développer et réduire $(x + 5)(x - 5)$.
		\item Sarah dit : \og Avec ce programme de calcul, quel que soit le nombre choisi au départ, le résultat obtenu est toujours le carré du nombre de départ \fg.
		
Qu'en pensez-vous ?
	\end{enumerate}
\end{enumerate}

\bigskip

