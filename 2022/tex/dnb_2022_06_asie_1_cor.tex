\textbf{\large Exercice 1 \hfill 19 points}

\medskip

$\bullet~~$\textbf{Situation 1}

\begin{enumerate}
\item $(10- 7)\times 5- 2 \times10 = 3 \times 5 - 20 = 15 - 20 = -5$.
\item L'expression $C$ correspond au résultat du programme : $5(x - 7) - 2x$
\end{enumerate}

\medskip

$\bullet~~$\textbf{Situation 2}

\begin{minipage}{0.48\linewidth}
On lit sur le graphique 
\end{minipage}
\begin{minipage}{0.48\linewidth}\psset{unit=0.3cm}
\begin{pspicture}(-4,-6)(5,8)
\psgrid[gridlabels=0pt,subgriddiv=1,gridwidth=0.2pt,griddots=6]
\psaxes[linewidth=1.25pt,Dx=2,Dy=2,labelFontSize=\scriptstyle]{->}(0,0)(-4,-6)(5,8)
\psline(-3,-6)(4,8)\psdots[dotstyle=+,dotangle=45](3,6)\uput[d](3,3){A(3~;~6)}
\uput[u](4.9,0){$x$} \uput[r](0,7.8){$y$}
\psline[linewidth=1.25pt,linestyle=dashed,linecolor=blue,ArrowInside=->](-2,0)(-2,-4)(0,-4)
\end{pspicture}
\end{minipage}
%A(3;6)-:

\begin{enumerate}
\item $f(- 2) = - 4$
\item $f$ est une fonction linéaire donc de la forme $f(x) = a x$ avec $a$ un nombre réel.

La droite représentative de la fonction $f$ passe par les points de coordonnées (0~;~0)
et (3~;~6) donc $f(3) = 6$, soit $3 \times a = 6$, d'où $a = \dfrac63 = 2$.
Par conséquent, [(x = 2x
\end{enumerate}

$\bullet~~$\textbf{Situation 3}

\medskip

$V_{\text{pyramide}} = \dfrac{A_{\text{base}}  \times h}{3} = 
\dfrac{30 \times 40 \times 55 }{3} =  10  \times  40  \times  55
= \np{22000}$ cm$^3$
= 22 L car 1 L = 1 dm$^3 = \np{1000}$ cm/$^3$.

Par conséquent le volume de cette pyramide est supérieur à $20$ L.

\bigskip

