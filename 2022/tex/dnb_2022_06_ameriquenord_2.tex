
\medskip

\emph{Dans cet exercice, aucune justification n'est attendue.}

\medskip

Cet exercice est un questionnaire à choix multiple. Pour chaque question, une seule des quatre réponses est exacte.

\medskip

\textbf{Sur la copie}, écrire le numéro de la question et la réponse choisie.

\begin{center}
\begin{tabularx}{\linewidth}{|c|m{4.5cm}|*{4}{>{\centering \arraybackslash}X|}}\hline
&&\textbf{Réponse A} &\textbf{Réponse B} &\textbf{Réponse C} &\textbf{Réponse D}\\ \hline
1&On lance un dé équilibré à 20 faces numérotées de 1 à 20. La probabilité pour que le numéro tiré soit inférieur ou égal à 5 est \ldots&$\dfrac{1}{20}$&$\dfrac{1}{4}$&$\dfrac{1}{5}$&$\dfrac{5}{6}$\\ \hline
2&Une boisson est composée de sirop et d'eau dans la proportion d'un volume de sirop pour sept volumes d'eau (c'est-à-dire dans le ratio 1~:~7).

La quantité d'eau nécessaire pour préparer $560$ mL de cette boisson est \ldots&70 mL
&80 mL&400 mL&490 mL\\ \hline
3&La fonction linéaire $f$ telle
que $f\left(\dfrac45\right) = 1$ est \ldots&$f(x) = x + \dfrac15$&$f(x) = \dfrac45 x$&$f(x) = \dfrac54 x$&$f(x) = x  - \dfrac15$\\ \hline
4&La décomposition en produit de facteurs premiers de $195$ est \ldots&$5\times 39$&$3 \times 5 \times 13$&\footnotesize $1 \times 100 + 9~\times~10 + 5$&$3 \times 65$\\ \hline
5&\psset{unit=1cm}
\begin{pspicture}(4,3)
%\psgrid
\pspolygon(0.6,0.6)(2.2,0.6)(0.6,1.6)
\psline(2.2,0.6)(3.9,1.9)(2.2,2.9)(0.6,1.6)
\psline[linestyle=dotted,linewidth=1.6pt](0.6,0.6)(2.2,1.9)(3.9,1.9)
\psline[linestyle=dotted,linewidth=1.6pt](2.2,1.9)(2.2,2.9)
\psframe(2.2,1.9)(2.35,2.05)\psframe(0.6,0.6)(0.75,0.75)
\psline[linewidth=0.6pt]{<->}(0.6,0.4)(2.2,0.4)\uput[d](1.4,0.4){\small 5 cm}
\psline[linewidth=0.6pt]{<->}(2.3,0.5)(4,1.8)\uput[dr](3.15,1.15){\small 8 cm}
\psline[linewidth=0.6pt]{<->}(0.4,0.6)(0.4,1.6)\rput{90}(0.2,1.1){\small 3 cm}
\end{pspicture}

Le volume de ce prisme droit est \ldots&40 cm$^3$&60 cm$^3$&64 cm$^3$&120 cm$^3$\\ \hline
\end{tabularx}
\end{center}

\bigskip

