\textbf{\large Exercice 3 \hfill 23 points}

\medskip

\textbf{Partie A}

\medskip

\begin{enumerate}
\item Il y a $24$ données  ; la médiane est donc la moyenne entre la 12\up{e} et 13\up{e} données de la série rangées dans l'ordre croissant. D'après l'effectif cumulé croissant du tableau donné la 12\up{e} et la 13\up{e} donnée sont $350$, donc la médiane est de $350$ mL. Ce qui signifie que 50\,\% des données sont des volumes inférieurs à $350$~mL et que 50\,\% sont des volumes supérieurs à $350$~mL.
\item L'étendue est égale à : $357 - 344 = 13$ mL.
\item On a trouvé 2 briques de $350$~mL.La probabilité d'obtenir une brique contenant exactement $350$ mL est donc égale à $\dfrac{2}{24} = \dfrac{1}{12}$ soit environ 8,3\,\%.
\item 
Nombre de briques ayant un volume compris entre $345$mL et $355$ mL :
$24 - (1 + 1 + 1) = 24 - 3 = 21$.

Le pourcentage de briques pouvant être vendues est donc : $\dfrac{21}{24} \times 100 = \dfrac78 \times 100$ soit  87,5\,\%.
\end{enumerate}

\textbf{Partie B}

\medskip

\begin{enumerate}
\item Aire base  $6,4 \times 5 = 32$ cm?
\item $V_{\text{pavé}} = L \times l \times h$.

Soit $400 = 6,4 \times 5 \times  h$ ou $400 = 32h$, d'où $h = \dfrac{400}{32} = 12,5$ cm.

Il faut une hauteur de 12,5~cm pour obtenir une brique de $400$~cm$^3$.
\end{enumerate}

\bigskip

