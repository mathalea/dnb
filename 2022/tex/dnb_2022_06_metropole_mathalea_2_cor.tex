
\medskip
\begin{description}[itemsep=1em]
\item[Question 1] Réponse A
\begin{quote}
    \textit{ Remarque~: l'homothétie ne conserve pas les longueurs et la symétrie axiale ne conserve pas l'orientation} 
  \end{quote}
\item[Question 2] Réponse B (le point d'ordonnée $2$ a $1$ comme abscisse).\\
\begin{quote}
    \textit{Remarque~: Si vous avez répondu 4 c'est que vous avez confondu l'antécédent \textbf{de} 2 et le nombre qui a 2 \textbf{pour} antécédent.
}
\end{quote}
\item[Question 3] Réponse B
  \begin{align*}
    f(3) &= 3\times3^2-7\\
    f(3) &= 3\times9-7\\
    f(3) &= 27-7\\
    f(3) &= 20\\
  \end{align*}
  \begin{quote}
      \textit{Remarque~: si vous avez répondu "29 est l'image de 2 par la fonction f", vous avez sans doute effectué la substitution $f(2)=3\times 2^2-7 $ puis effectué la multiplication $3\times2=6$ \textbf{avant} de faire le carré...\\
  $f$ n'est pas une fonction affine car sont expression développée et réduite comporte un terme en $x^2$ : c'est une fonction du second degré.}
  \end{quote}
  
\item[Question 4] Réponse B

  La série ordonnée est~:
  
    \begin{tabular}{*{13}{>{$}c<{$}}}
       3,41 & 3,7 & 4,01 & 4,28 & 4,3 & 4,62 & 4,91 & 5,15 & 5,25 & 5,42 & 5,82 & 6,07 & 6,11
    \end{tabular}
    \medskip

    Il y a $13$ valeurs et $13\div2=6,5$ donc la médiane est la 7\ieme{} valeur de cette série ordonnée, soit $4,91$.
    
    7 n'est pas la médiane, c'est le rang de la médiane dans la série ordonnée.
    5,15 est la $7^e$ valeur dans la série \textbf{non ordonnée}.
    
    \begin{quote}
    \textit{Remarque~: en répondant $7$ vous donnez le rang de la médiane et non sa valeur et en  répondant $5,15$ vous avez oublié d'ordonner la valeur} 
  \end{quote}
  \item[Question 5] Réponse C
  
    Le facteur d'agrandissement des longueurs est donné par $\dfrac{BU}{LA}=\dfrac{6,3}{2,1}$ donc il vaut $3$.

    Si les longueurs sont multipliées par $3$ alors les aires sont multipliées par $3^2=9$.
    
    \begin{quote}
    \textit{Remarque~: En répondant $3$ vous oubliez que le facteur n'est pas le même pour les longueurs et pour les aires. En répondant $6$ vous avez sans doute pensé que $3^2$ valait la même chose que $3\times2$} 
  \end{quote}
\end{description}
\medskip

\clearpage
