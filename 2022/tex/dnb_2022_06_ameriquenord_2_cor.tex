
\medskip

\emph{Dans cet exercice, aucune justification n'est attendue. On en donne quand même dans ce corrigé.}


\begin{enumerate}
	\item \textbf{Réponse B}

	Puisque le dé est équilibré et à 20 faces, on a un univers avec 20 issues possibles, et on est en situation d'équiprobabilité. Il y a 5 issues favorables à l'événement décrit: 1; 2; 3; 4 et 5.

	La probabilité de l'événement est donc : $ \dfrac{5}{20} = \dfrac{1}{4} $

	\item \textbf{Réponse D}

	La boisson est composée de huit volumes : un volume de sirop et sept volumes d'eau. Pour arriver à \np[mL]{560}, un volume doit donc être de $560 \div 8 = \np[mL]{70}$.

	Les sept volumes d'eau totalisent donc un volume de $7 \times 70 = \np[mL]{490}$.

	\item \textbf{Réponse C}

	Si $f$ est linéaire, alors il existe un nombre $a$ tel que l'expression de $f$ est $f(x) = ax$, cela élimine les propositions \textbf{A} et \textbf{D}.

	Pour avoir $f\left(\dfrac{4}{5}\right)  = 1$, il faut donc : $a\times \dfrac{4}{5} = 1 \iff a = 1 \times \dfrac{5}{4} = \dfrac{5}{4}$.

	Remarque : on peut aussi calculer l'image de $ \dfrac{4}{5} $ par la proposition \textbf{B}, trouver $ \dfrac{16}{25} $ et donc éliminer aussi cette proposition.

	\item \textbf{Réponse B}

	On peut vérifier que la proposition \textbf{B} est correcte : $3 \times 5 \times 13 = 195$, et les trois facteurs : 3, 5 et 13 sont bien des nombres premiers.

	On peut aussi procéder par élimination : $5 \times 39 = 195$, mais 39 est un nombre composé : $39 = 3\times 13$. $1 \times 100 + 9\times 10 + 5$ est bien une décomposition de 195, mais pas sous la forme d'un produit, c'est une somme (dont les termes ne sont pas tous premiers, qui plus est).
	Enfin pour $3\times 65 = 195$, le problème est encore que 65 est un nombre composé : $65 = 5 \times 13$.

	\item \textbf{Réponse B}

	Le volume d'un prisme est le produit de l'aire de la base par la hauteur. Ici, il s'agit d'un prisme à base triangulaire.

	L'aire de la base est donc : $\mathcal{A}_\mathrm{base} = \dfrac{5 \times 3}{2} = \np[cm^2]{7,5}$

	La hauteur du prisme est $h =\np[cm]{8}$, donc le volume du prisme est :

	 $\mathcal{V}_\mathrm{prisme} = \mathcal{A}_\mathrm{base} \times h = 7,5 \times 8 = \np[cm^3]{60}$.
\end{enumerate}

\bigskip

