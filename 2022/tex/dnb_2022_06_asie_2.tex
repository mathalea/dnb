
\medskip

\begin{minipage}[t]{10cm}
\emph{La figure ci-contre est réalisée à main levée.}

Les droites (AB) et (CD) sont parallèles.

Les droites (AD) et (BC) sont sécantes en E.

On a : \begin{tabular}[t]{ll}
	ED = 3,6 cm & CD = 6 cm \\
	EB = 7,2 cm & AB = 9 cm
\end{tabular}

\begin{enumerate}
	\item Démontrer que le segment [EC] mesure 4,8 cm.

	\item Le triangle ECD est-il rectangle ?
\end{enumerate}
\end{minipage}
\hfill
\begin{tikzpicture}[pencildraw/.style={	smooth,black, decorate, decoration={random steps,segment length=10pt,amplitude=1.2pt}},baseline=(B.base)]
	\draw[pencildraw] (0,0) node [below left] {A}
			--(2.6,1.2) node [above right=0.2] {E}
			--(4.5,1.8) node [above right] {D}
			--(3.7,-1) node [below] {C}
			--(2.6,1.2)
			--(1.3,4.7)node[above] (B) {B}--cycle;
\end{tikzpicture}

\begin{enumerate}[resume]
	\item Parmi les transformations ci-dessous, quelle est celle qui permet d’obtenir le triangle ABE à partir du triangle ECD ?

	\emph{Recopier la réponse sur la copie. Aucune justification n’est attendue.}

%	\begin{tblr}{*{4}{|X[c,m]|p{1mm}}|X[c,m]|} \cline{1,3,5,7,9}
%		Symétrie axiale && Homothétie && Rotation && Symétrie centrale && Translation \\ \cline{1,3,5,7,9}
%	\end{tblr}
\begin{center}\fbox{Symétrie axiale}\quad  \fbox{Homothétie}\quad  \fbox{Rotation}\quad  \fbox{Symétrie centrale}\quad  \fbox{Translation}\end{center}
	\item On sait que la longueur BE est 1,5 fois plus grande que la longueur EC.

	L’affirmation suivante est-elle vraie ? \emph{On rappelle que la réponse doit être justifiée.}

	\textbf{Affirmation :} \og L’aire du triangle ABE est 1,5 fois plus grande que l’aire du triangle ECD. \fg
\end{enumerate}


