
\medskip

\emph{Les deux parties de cet exercice sont indépendantes}

\medskip

Une entreprise produit et vend des jus de fruit contenus dans des briques en carton qui ont la forme d'un pavé droit.

\bigskip

\textbf{PARTIE A : Briques de jus de pomme}

\medskip

Ces briques sont fabriquées pour contenir 350 mL de jus de pomme.

\smallskip

Lors d'un contrôle, 24 briques sont prélevées au hasard et analysées.

Le tableau ci-dessous donne le volume de jus de pomme (en mL) contenu dans ces briques :

\begin{center}
\begin{tabularx}{\linewidth}{|m{2.5cm}|*{11}{>{\centering \arraybackslash}X|}}\hline
Volume en mL&344 &347 &348 &349 &350 &351& 352 &353 &354 &356 &357\\ \hline
Effectif &1&2 &4 &4 &2 &3 &1&2 &3 &1 &1\\ \hline
\end{tabularx}
\end{center}

\medskip

\begin{enumerate}
\item Déterminer la médiane des volumes de cette série. Interpréter ce résultat
\item Calculer l'étendue de cette série
\item On prélève au hasard une brique parmi celles contrôlées, quelle est la probabilité qu'elle contienne exactement $350$~mL de jus de pomme ?
\item Lorsque le volume de jus de pomme contenu dans une brique est compris entre $345$~mL et $355$~mL, cette brique peut être vendue. 

Quel est le pourcentage de briques que l'entreprise peut vendre parmi les briques contrôlées ?
\end{enumerate}

\bigskip

\textbf{PARTIE B : Briques de jus de raisin}

\medskip

\begin{minipage}{0.67\linewidth}
L'entreprise souhaite commercialiser une nouvelle brique en forme de pavé droit pour le jus de raisin. Sa base est un rectangle de longueur $6,4$ cm et de largeur $5$ cm.

\medskip

\begin{enumerate}
\item Calculer l'aire de la base de cette brique
\item Quelle doit être la hauteur de cette brique pour que son volume soit de $400$~cm$^3$ ?
\end{enumerate}
\end{minipage}\hfill
\begin{minipage}{0.28\linewidth}
\psset{unit=1cm}
\begin{pspicture}(3.4,4.5)
%\psgrid
\pspolygon(0,3.8)(0,0.8)(2,0.1)(2,3.1)
\psline(2,0.1)(3.4,0.7)(3.4,3.7)(2,3.1)
\psline(3.4,3.7)(1.4,4.4)(0,3.8)
\psline[linestyle=dashed](0,0.8)(1.4,1.4)(3.4,0.7)
\psline[linestyle=dashed](1.4,1.4)(1.4,4.4)
\uput[d](1,0.42){6,4 cm}\uput[d](2.7,0.35){5 cm}
\end{pspicture}
\end{minipage}

\bigskip

