
\medskip

Dans cet exercice, toutes les longueurs sont exprimées en pixel.

\medskip

\textbf{Partie A :}

\medskip

Un professeur donne à ses élèves un motif en forme de parallélogramme et le script, en partie rédigé, qui permet de tracer ce motif. 

On précise que le lutin est au point de départ, comme indiqué sur la figure ci- dessous, et qu'il est orienté vers la droite:

\begin{tabularx}{\linewidth}{m{8cm} X}
\textbf{Parallélogramme obtenu :}&\textbf{Script du motif}\\
\psset{unit=1cm}
\begin{pspicture}(8,2.4)
\rput(1,0.2){Départ $\to$}
\pspolygon(1.9,0.2)(5.6,0.2)(7.7,2)(4,2)
\psarc(1.9,0.2){0.4}{0}{40}\psarc(5.6,0.2){0.4}{40}{180}
\uput[u](5.5,2){40 pixels}\uput[r](6.5,0.9){30 pixels}
\rput(2.55,0.4){$40\degres$}\rput(5.3,0.75){$140\degres$}
\end{pspicture}&\begin{scratch}
\initmoreblocks{définir \namemoreblocks{motif}}
{
\blockrepeat{répéter \ovalnum{2} fois }
{
\blockmove{avancer de \ovalnum{40}}
\blockspace[1]
}
}
\end{scratch}\\
\end{tabularx}

Recopier dans le bon ordre, sur votre copie, les instructions suivantes à insérer dans le script du motif permettant de tracer le parallélogramme ci-dessus:

\begin{scratch}
\blockmove{avancer de \ovalnum{30}}\end{scratch}\quad
\begin{scratch}\blockmove{tourner \turnleft{} de \ovalnum{40} degrés}\end{scratch}\quad
\begin{scratch}\blockmove{tourner \turnleft{} de \ovalnum{140} degrés}
\end{scratch}


\medskip

\textbf{Partie B :}

\medskip

Le professeur demande ensuite à ses élèves d'intégrer ce script dans un programme de leur choix permettant de tracer des figures composées de plusieurs de ces motifs.

Voici les programmes écrits par deux élèves.

\begin{center}
\begin{tabularx}{\linewidth}{*{2}{X}}
\textbf{Programme de l'élève A}& \textbf{Programme de l'élève B}\\
\begin{scratch}[num blocks]
\blockinit{Quand \selectmenu{flèche droite} est cliqué}
\blockpen{effacer tout}
\blockmove{aller à x: \ovalnum{-230} y: \ovalnum{-170}}
\blockmove{s'orienter à \ovalnum{\selectmenu{90}} degrés}
\blockrepeat{répéter \ovalnum{9} fois}
{\blockpen{stylo en position d'écriture}
\blocklook{Motif}
\blockpen{relever le stylo}
\blockmove{avancer de \ovalnum{50}}}
\end{scratch}&
\begin{scratch}[num blocks]
\blockinit{Quand \selectmenu{espace} est cliqué}
\blockpen{effacer tout}
\blockmove{aller à x: \ovalnum{0} y: \ovalnum{0}}
\blockpen{stylo en position d'écriture}
 \blockrepeat{répéter \ovalnum{9} fois}
{\blocklook{Motif}
\blockmove{tourner \turnleft{} de \ovalnum{40} degrés}}
\blockpen{relever le stylo}
\end{scratch}\\
\end{tabularx}
\end{center}

On rappelle que \og s'orienter à 90 \fg{} signifie que l'on est orienté vers la droite.

\medskip

\begin{enumerate}
\item Quelle action au clavier permet de lancer le programme de l'élève B ?
\item Parmi les figures suivantes, indiquer, ici \textbf{sans justifier }:
	\begin{enumerate}
		\item laquelle est obtenue avec le programme de l'élève A ? 
		\item laquelle est obtenue avec le programme de l'élève B ?
	\end{enumerate}
\end{enumerate}

\begin{center}
\begin{tabularx}{\linewidth}{|c|>{\centering \arraybackslash}X|}\hline
\textbf{Figure 1}&\textbf{Figure 2}\\
\psset{unit=0.8cm}
\def\losange{\pspolygon(0,0)(1.058,0)(1.666,0.51)(0.608,0.51)}
\begin{pspicture}(0,-1.9)(12,1.9)
\multido{\n=0.00+1.258}{9}{\rput(\n,0){\losange}}
\end{pspicture}&
\psset{unit=1cm}
\def\losange{\pspolygon(0,0)(1.058,0)(1.666,0.51)(0.608,0.51)}
\begin{pspicture}(-1.9,-1.9)(1.9,1.9)
\pspolygon(0.48;-70)(0.48;-30)(0.48;10)(0.48;50)(0.48;90)(0.48;130)(0.48;170)(0.48;210)(0.48;250)
\multido{\n=-110+40,\na=-40+40}{9}{\rput{\na}(0.5;\n){\losange}}
\end{pspicture}\\ \hline
\textbf{Figure 3}&\textbf{Figure 4}\\
\psset{unit=0.8cm}
\def\losange{\pspolygon(0,0)(1.058,0)(1.666,0.51)(0.608,0.51)}
\begin{pspicture}(0,-1.9)(12,1.9)
\multido{\n=0.00+1.058}{9}{\rput(\n,0){\losange}}
\end{pspicture}&\psset{unit=1cm}
\def\losange{\pspolygon(0,0)(1.058,0)(1.666,0.51)(0.608,0.51)}
\begin{pspicture}(-1.9,-1.9)(1.9,1.9)
\multido{\n=0+40}{9}{\rput{\n}(0;\n){\losange}}
\end{pspicture}\\ \hline
\end{tabularx}
\end{center}

