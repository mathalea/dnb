
\medskip

Pour fêter les 25 ans de sa boutique, un chocolatier souhaite offrir aux premiers clients de la journée une boîte contenant des truffes au chocolat.

\medskip

\begin{enumerate}
\item Il a confectionné $300$ truffes: $125$ truffes parfumées au café et $175$ truffes enrobées de noix de coco. Il souhaite fabriquer ces boîtes de sorte que:

\setlength\parindent{1cm}
\begin{itemize}
\item[$\bullet~~$] Le nombre de truffes parfumées au café soit le même dans chaque boîte;
\item[$\bullet~~$] Le nombre de truffes enrobées de noix de coco soit le même dans chaque boîte;
\item[$\bullet~~$] Toutes les truffes soient utilisées.
\end{itemize}
\setlength\parindent{0cm}

	\begin{enumerate}
		\item Décomposer $125$ et $175$ en produit de facteurs premiers.
		\item En déduire la liste des diviseurs communs à $125$ et $175$.
		\item Quel nombre maximal de boîtes pourra-t-il réaliser ?
		\item Dans ce cas, combien y aura-t-il de truffes de chaque sorte dans chaque boîte?
	\end{enumerate}	
\item Le chocolatier souhaite fabriquer des boîtes contenant $12$ truffes. Pour cela, il a le choix entre deux types de boites qui peuvent contenir les $12$ truffes, et dont les caractéristiques sont données ci-dessous:

\begin{center}
\begin{tabularx}{0.75\linewidth}{|*{2}{>{\centering \arraybackslash}X|}}\hline
\textbf{Type A}&\textbf{Type B}\\
\psset{unit=1cm}
\begin{pspicture}(4,3.4)
\pspolygon(0,0)(2.9,0)(3.7,1)(1.85,3.2)(2.9,0)
\psline(0,0)(1.85,3.2)
\psline[linestyle=dashed](0,0)(0.8,1)(3.7,1)
\psline[linestyle=dashed](0.8,1)(1.85,3.2)
\end{pspicture}&
\psset{unit=1cm}
\begin{pspicture}(4,3.4)
\psframe(0.4,0.2)(3.1,2.1)
\psline(3.1,0.2)(3.7,1.1)(3.7,3)(3.1,2.1)
\psline(3.7,3)(1,3)(0.4,2.1)
\psline[linestyle=dashed](0.4,0.2)(1,1.1)(3.7,1.1)
\psline[linestyle=dashed](1,1.1)(1,3)
\end{pspicture}\\ \hline
Pyramide à base carrée&Pavé droit\\
de côté 4,8 cm&de longueur 5 cm,\\
et de hauteur 5 cm& de largeur 3,5 cm\\
&et de hauteur 3,5 cm\\ \hline
\end{tabularx}
\end{center}

Dans cette question, chacune des $12$ truffes est assimilée à une boule de diamètre $1,5$~cm.

À l'intérieur d'une boîte, pour que les truffes ne s'abîment pas pendant le transport, le volume occupé par les truffes doit être supérieur au volume non occupé par les truffes.

Quel(s) type(s) de boîte le chocolatier doit-il choisir pour que cette condition soit respectée?
\end{enumerate}

\medskip

\begin{center}
\renewcommand\arraystretch{2}
\begin{tabularx}{\linewidth}{|X|}\hline
\textbf{Rappels :}\\
Le volume d'une boule de rayon $r$ est: $\dfrac43 \times \pi \times r^3$\\ 
Le volume d'une pyramide est : $\dfrac{\text{aire de la base} \times \text{hauteur}}{3}$\\
Le volume d'un pavé droit est : longueur $\times$ largeur $\times$ hauteur\\ \hline
\end{tabularx}
\end{center}
