
\medskip


\begin{enumerate}[itemsep=1em]
\item Les droites $(AC)$ et $(BD)$ sont perpendiculaires à la même droite $(AB)$.\\
Or, si deux droites sont perpendiculaires à une même troisième droite alors elles sont parallèles entre elles.\\
Donc les droites $(AC)$ et $(BD)$ sont parallèles.
\item Les droites $(AC)$ et $(BD)$ sont parallèles, de plus les droites $(CD)$ et $(AE)$ sont sécantes en $E$ donc d'après le théorème de Thalès : 

\medskip
$\dfrac{AE}{EB}=\dfrac{CE}{ED}=\dfrac{AC}{BD}$ soit $\dfrac{20}{5}=\dfrac{CE}{ED}=\dfrac{AC}{1}$

\medskip
Donc : $AC=\dfrac{20\times1}{5}=4$.

\medskip
Finalement, la largeur $AC$ de la rivière est de 4 pas.

\medskip
\textit{Remarque : on pouvait aussi montrer que $ACE$ est un agrandissement de $BDE$ de coefficient 4.}

\item Le triangle $ACE$ est rectangle en $A$ donc d'après le théorème de Pythagore :

\medskip
$CE^2=AC^2+AE^2$ ce qui équivaut avec des longueurs en pas à  : 

$CE^2=4^2+20^2=416$ 

\medskip
Soit $CE=\sqrt{416}~\text{pas}=\sqrt{416}\times65~\text{cm}\approx{1~330~\text{cm}}$ soit environ 13,3~m.

\bigskip
\textbf{Autre méthode :}
\textit{On pouvait aussi convertir toutes les longueurs en mètres avant d'utiliser le théorème de Pythagore}

$AC = 4~\text{pas} = 4 \times 65~\text{cm}=260~\text{cm}$\\
$AE = 20~\text{pas} = 20 \times 65~\text{cm}=1~300~\text{cm}$

Avec les longueurs en centimètres, on a donc : 
$CE^2=260^2+1~300^2=1~757~600$ donc $CE=\sqrt{1~757~600}\approx{1~330}$.

La longueur $CE$ est d'environ 1~330~cm soit 13,3~m.
\item 
    \begin{enumerate}
        \item Le bâton parcourt environ 13,3~m en 5~secondes, sa vitesse est donc : 
        
        \medskip
        $\dfrac{13,3~\text{m}}{5~\text{s}}=2,66~\text{m/s}$.
        
        \item \begin{tabular}{|l|c|c|}
        \hline
            Distance (en m) & 2,66 & $d$ \\
        \hline
            Temps (en s) & 1 & 3~600 \\
        \hline
        \end{tabular}
    \end{enumerate}
    
    \medskip
    C'est une situation de proportionnalité, donc : $d=\dfrac{2,66\times3~600}{1}=9~576$.
    
    \medskip
    Parcourir 2,66 mètres en une seconde est équivalent à 9~576 m en 3~600~s soit 9,576 kilomètres en une heure.
    
    \medskip
    C'est donc vrai : la vitesse moyenne est légèrement inférieure à 10~km/h.

\end{enumerate}
\medskip


