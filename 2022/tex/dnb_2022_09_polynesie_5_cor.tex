\textbf{{\large \textsc{Exercice 5}} \hfill 19 points}

\medskip

Une urne contient 20 boules rouges, 10 boules vertes, 5 boules bleues et 1 boule noire.

Un jeu consiste à tirer une boule au hasard dans l'urne.

Lorsqu'un joueur tire une boule noire, il gagne 10 points.

Lorsqu'il tire une boule bleue, il gagne 5 points.

Lorsqu'il tire une boule verte, il gagne 2 points.

Lorsqu'il tire une boule rouge, il gagne 1 point.

\medskip

\begin{enumerate}
\item Un joueur tire au hasard une boule dans l'urne.
	\begin{enumerate}
		\item %Quelle est la probabilité qu'il gagne $10$ points?
Il gagne 10 points s'il tire une boule noire ; il y a 1 boule noire sur un total de $20 + 10 + + 5 + 1 = 36$ : la probabilité est égale à $\dfrac{1}{36}$.
		\item %Quelle est la probabilité qu'il gagne plus de $3$ points?
		Il gagnera plus de 3 points s'il tire une boule noire (1 seule) ou une boule bleue (5 boules bleues) : la probabilité est égale à $\dfrac{6}{36} = \dfrac{6 \times 1}{6 \times 6} = \dfrac16$.
		\item %A-t-il plus de chance de gagner $2$ points ou de gagner $5$ points ?
Il y a plus de boules vertes que de boules bleues : Il a plus de chance de gagner $2$ points que de gagner $5$ points.
	\end{enumerate}	

\item~

%\begin{minipage}{0.48\linewidth}	
%Le tableau ci-contre récapitule les scores obtenus par 15 joueurs:
	\begin{enumerate}
		\item %Quelle est la moyenne des scores obtenus par ces joueurs ?
		La moyenne des scores est : $\dfrac{2 + 1 + 1 + \ldots + 2}{15} = \dfrac{50}{15} = \dfrac{5 \times 10}{5 \times 3} = \dfrac{10}{3} = 3,333$~(points).
		\item %Quelle est la médiane des scores ?
Les scores sont dans l'ordre croissant :
		
1 1 1 1 1 2 2 2 2 2 \ldots : la médiane est entre la 7\up{e} et la 8\up{e} valeur soit 2.
		\item %Déterminer la fréquence du score \og 10 points \fg.
		La fréquence du score 10 est $\dfrac{2}{15}$.
		\end{enumerate}
%\end{minipage}	\hfill
%\begin{minipage}{0.48\linewidth}
%\begin{tabularx}{\linewidth}{|*{2}{>{\centering \arraybackslash}X|}}\hline
%\textbf{JOUEUR}&\textbf{SCORE}\\ \hline
%JOUEUR A&2 points\\ \hline
%JOUEUR B& 1 point\\ \hline
%JOUEUR C&1 point\\ \hline
%JOUEUR D&5 points \\ \hline
%JOUEUR E&10 points\\ \hline
%JOUEUR F&2 points\\ \hline
%JOUEUR G&2 points\\ \hline
%JOUEUR H&5 points\\ \hline
%JOUEUR I&1 point\\ \hline
%JOUEUR J&2 points\\ \hline
%JOUEUR K&5 points\\ \hline
%JOUEUR L&10 points\\ \hline
%JOUEUR M&1 point\\ \hline
%JOUEUR N&1 point\\ \hline
%JOUEUR O&2 points\\ \hline
%\end{tabularx}
%\end{minipage}

\item Mille joueurs ont participé au jeu. Peut-on estimer le nombre de joueurs ayant obtenu le score de $10$ points ?

La réponse, affirmative ou négative, devra être argumentée.

On a vu à la question précédente que la fréquence du score 10 points est égale à $\dfrac{1}{36}$.

Donc pour \np{1000} joueurs il y aura à peu près :

$\np{1000} \times \dfrac{1}{36} = \dfrac{\np{1000}}{36} = \dfrac{250}{9} \approx 27,7$

Environ 28 joueurs auront un score de 10 points.
\end{enumerate}

