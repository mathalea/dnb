\textbf{\large Exercice 3 \hfill 21 points}

\medskip

On considère la figure suivante, où toutes les longueurs sont données en centimètre. Les points C, A et E sont alignés et les points B, A et D sont alignés. 

\emph{La figure n'est pas représentée en vraie grandeur.}

\begin{center}
\psset{unit=0.9cm}
\begin{pspicture}(7.5,4.4)
%\psgrid
\psline(0,0.2)(7.4,0.2)\psline(0,4)(7.4,4)
\psline(0,0)(7.3,4.3)\psline(2.2,0)(2.2,4.2)
\uput[l](2.1,2.7){9,6}\uput[dr](4.5,2.7){19,2}
\uput[ul](1.2,0.8){8}\rput(1.8,0.7){$60\degres$}
\psframe(2.2,0.2)(2,0.4)
\uput[ur](2.2,0.9){A} \uput[ur](2.2,0.2){B} \uput[ul](0.2,0.2){C} \uput[u](2.2,4){D} \uput[ul](7,4){E}\psarc(2.2,1.3){0.4}{210}{270}
\end{pspicture}
\end{center}

\begin{enumerate}
\item %Prouver que le segment [AB] mesure 4 cm.
Le triangle ABC est rectangle en B donc
$\cos \left ( \widehat{\text{BAC}} \right )=\dfrac{\text{AB}}{\text{AC}}$.

Or $\widehat{\text{BAC}}=60\degres$ et $\cos \left (60\degres\right )=\dfrac{1}{2}$.
De plus $\text{AC}=8$.

Donc
$\dfrac{1}{2}= \dfrac{\text{AB}}{8}$
donc $\text{AB}=4$.
Le segment [AB] mesure 4 cm.

\item% En utilisant la question précédente, démontrer que les droites (BC) et (DE) sont parallèles.
Les points B, A, D d'une part, et C, A, E d'autre part sont alignés dans cet ordre.

$\dfrac{\text{AC}}{\text{AB}} = \dfrac{8}{4}=2$
et
$\dfrac{\text{AE}}{\text{AD}} = \dfrac{19,2}{9,6}=2$
donc
$\dfrac{\text{AC}}{\text{AB}} = \dfrac{\text{AE}}{\text{AD}}$

D'après la réciproque du théorème de Thalès, on peut en conclure que les droites (BC) et (DE) sont parallèles.

\item% En déduire que la droite (DB) est perpendiculaire à la droite (DE).
La droite (DB) est perpendiculaire à la droite (BC), et les droites (BC) et (DE) sont parallèles.
Or, quand deux droites sont parallèles, toute droite perpendiculaire à l'une est perpendiculaire à l'autre.

On en déduit que  la droite (DB) est perpendiculaire à la droite (DE).

\item L'aire du triangle ADE, rectangle en D, est:
$\dfrac{\text{DE}\times \text{AD}}{2}$; on calcule DE.

Le triangle ADE est rectangle en D donc, d'après le théorème de Pythagore, on a:

$\text{AE}^2 = \text{AD}^2 + \text{DE}^2$
donc
$19,2^2 = 9,6^2 + \text{DE}^2$ 
donc
$\text{DE}^2 = 276,48$ donc $\text{DE}= \ds\sqrt{276,48}$

$\dfrac{\text{DE}\times \text{AD}}{2}=\dfrac{\ds\sqrt{276,48}\times 9,6}{2}
\approx 80$

L'aire du triangle ADE vaut environ 80~cm$^2$.

\end{enumerate}

%\bigskip

