
\medskip

\begin{center}
\textbf{PARTIE A}
\end{center}

\begin{enumerate}
\item Si le nombre de départ est 15, le programme de calcul donne :
\begin{itemize}[label=$\bullet$~]
	\item $15^2 = 225$
	\item $225 + 15 = 240$
\end{itemize}
Le nombre obtenu à l'arrivée est bien 240.

\item Dans la cellule B2, on veut afficher le nombre obtenu à l'arrivée quand le nombre choisi au départ est celui qui se lit dans la cellule A2.

On doit saisir la formule : \og =A2\^{}2 + A2\fg{}.

\item Si le nombre de départ est $x$, le programme de calcul donne :
\begin{itemize}[label=$\bullet$~]
	\item $x^2$
	\item $x^2 + x$
\end{itemize}
L'expression, en fonction de $x$, du nombre obtenu à l'arrivée est $x^2 + x$.

\end{enumerate}

\begin{center}
\textbf{PARTIE B}
\end{center}

On considère l'affirmation suivante:

\og Pour obtenir le résultat du programme de calcul, il suffit de multiplier le nombre de départ par le nombre entier qui suit. \fg

\begin{enumerate}[resume]
\item Quand le nombre entier choisi au départ est 9, le nombre que l'on obtient est 90 (voir cellule B11 du tableau).

Si on multiplie 9 par le nombre entier qui suit (10), on obtient bien aussi : $9\times 10 = 90$.

L'affirmation est correcte pour un nombre choisi égal à 9.

\item En général, si $x$ est le nombre entier choisi au départ, on a établi à la question \textbf{A. 3} que le nombre obtenu à l'arrivée est $x^2 + x$.

En, factorisant $x$ dans cette expression, il vient : $x^2 + x = x (x + 1)$.

Le nombre obtenu est donc bien, dans tous les cas le produit du nombre choisi au départ ($x$) par le nombre entier suivant ($x+1$). L'affirmation est donc bien vraie quel que soit le nombre entier choisi au départ.

\item Rappelons que le produit d'un nombre entier pair par un nombre entier quelconque est toujours pair.

Il y a deux cas de figure possibles :
\begin{itemize}[label=$\bullet$~]
	\item si le nombre de départ $x$ est pair, alors le nombre d'arrivée est pair, car c'est le produit d'un nombre pair ($x$) par un nombre entier ($x+1$).
	\item si le nombre de départ $x$ est un entier impair, alors l'entier suivant ($x+1$) est pair et donc le nombre d'arrivée est encore pair.
\end{itemize}

Dans tous les cas de figure, le nombre d'arrivée est bien un nombre pair, quel que soit le nombre entier choisi au départ.
\end{enumerate}
