\textbf{{\large \textsc{Exercice 5}} \hfill 19 points}

\medskip

Une urne contient 20 boules rouges, 10 boules vertes, 5 boules bleues et 1 boule noire.

Un jeu consiste à tirer une boule au hasard dans l'urne.

Lorsqu'un joueur tire une boule noire, il gagne 10 points.

Lorsqu'il tire une boule bleue, il gagne 5 points.

Lorsqu'il tire une boule verte, il gagne 2 points.

Lorsqu'il tire une boule rouge, il gagne 1 point.

\medskip

\begin{enumerate}
\item Un joueur tire au hasard une boule dans l'urne.
	\begin{enumerate}
		\item Quelle est la probabilité qu'il gagne $10$ points?
		\item Quelle est la probabilité qu'il gagne plus de $3$ points?
		\item A-t-il plus de chance de gagner $2$ points ou de gagner $5$ points ?
	\end{enumerate}	

\item~

\begin{minipage}{0.48\linewidth}	
Le tableau ci-contre récapitule les scores obtenus par 15 joueurs:
	\begin{enumerate}
		\item Quelle est la moyenne des scores obtenus par ces joueurs ?
		\item Quelle est la médiane des scores ?
		\item Déterminer la fréquence du score \og 10 points \fg.
		\end{enumerate}
\end{minipage}	\hfill
\begin{minipage}{0.48\linewidth}
\begin{tabularx}{\linewidth}{|*{2}{>{\centering \arraybackslash}X|}}\hline
\textbf{JOUEUR}&\textbf{SCORE}\\ \hline
JOUEUR A&2 points\\ \hline
JOUEUR B& 1 point\\ \hline
JOUEUR C&1 point\\ \hline
JOUEUR D&5 points \\ \hline
JOUEUR E&10 points\\ \hline
JOUEUR F&2 points\\ \hline
JOUEUR G&2 points\\ \hline
JOUEUR H&5 points\\ \hline
JOUEUR I&1 point\\ \hline
JOUEUR J&2 points\\ \hline
JOUEUR K&5 points\\ \hline
JOUEUR L&10 points\\ \hline
JOUEUR M&1 point\\ \hline
JOUEUR N&1 point\\ \hline
JOUEUR O&2 points\\ \hline
\end{tabularx}
\end{minipage}

\item Mille joueurs ont participé au jeu. Peut-on estimer le nombre de joueurs ayant obtenu le score de $10$ points ?

La réponse, affirmative ou négative, devra être argumentée.
\end{enumerate}

