\textbf{{\large \textsc{Exercice 5}} \hfill 20 points}

\bigskip

%Lya passe la journée dans un parc aquatique. 
%
%Elle y trouve une cabane dans un chêne d'où part une tyrolienne qui mène au-dessus d'une piscine.
%
%Le câble de la tyrolienne relie la cabane et le pied du peuplier situé juste derrière la piscine.
%
%\medskip

\begin{center}
\psset{unit=1cm,arrowsize=2pt 3}
\begin{pspicture}(0,-1)(12,5)
%\psgrid
\psframe[fillstyle=solid,fillcolor=blue!30](5,-0.8)(8.5,0.8)
\rput(6,4.7){\textbf{Document 1} : schéma de la situation}
\pspolygon(1,0)(1,2.8)(11.6,0)%BAC
\psframe(1,0)(1.2,0.2)\psline(7,0)(7,1.2)
\psframe(7,0)(7.2,0.2)\uput[dl](7,0){D}\uput[ul](7,1.2){E}
\rput{90}(6.8,0.6){1,50 m}
\rput(2,3.6){Départ}\rput(8,2.4){Arrivée}
\psline{->}(2,3.4)(1,2.8)\psline{->}(8,2.2)(7,1.2)
\uput[ul](1,0){B} \uput[ul](1,2.8){A} \uput[dr](11.6,0){C} 
\rput{90}(11.6,01){peuplier}\rput{90}(0.8,3.8){chêne}
\rput{90}(0.8,1.4){5,50 m}\rput(7,-0.6){Piscine}
\psline[linewidth=0.6pt]{<->}(1,-0.6)(5,-0.6)\uput[d](3,-0.6){12,20 m}
\psline[linewidth=0.6pt]{<->}(8.5,-0.6)(11.6,-0.6)\uput[d](10.05,-0.6){1,80 m}
\end{pspicture}
\end{center}

\textbf{Document 2 :}  La réglementation exige que l'angle formé par le câble de la tyrolienne et l'horizontale ait une mesure inférieure à $30\degres$.

\textbf{Document 3 :} La piscine a la forme d'un parallélépipède rectangle de longueur
$6$~m, largeur $6$~m et profondeur $1,60$~m.

\textbf{Document 4 :}  Lorsque Lya est suspendue à la tyrolienne, corps et bras tendus, elle mesure exactement $1,50$~m.

\medskip

\begin{enumerate}
\item %Vérifier par un calcul que BC $= 20$~m.
La piscine a une longueur de 6~m, donc BC $= 12,20 + 6 + 1,80 = 20$~(m).
\item %Le positionnement de la tyrolienne est-il conforme à la réglementation en vigueur ?
Le triangle ABC est rectangle en B, donc $\tan \widehat{\text{BCA}} = \dfrac{\text{AB}}{\text{BC}} = \dfrac{5,5}{20} = 0,275$.

La calculatrice donne $\widehat{\text{BCA}} \approx 15,37\degres$. C'est une mesure inférieure à $30\degres$ : la tyrolienne est réglementaire.
\item %Déterminer la longueur AC, en mètres, de câble nécessaire. Arrondir à l'unité.
Dans le triane ABC rectangle en B, le théorème de Pythagore s'écrit :

$\text{AB}^2 + \text{BC}^2 = \text{AC}^2 = 2,5^2 + 20^2 = 30,25 + 400 = 430,25$.

Donc AC $= \sqrt{430,25} \approx 20,7$ soit 21~m à l'unité près.
\item %Lya est suspendue à la tyrolienne verticalement. À quelle distance DC du peuplier,
%en mètres, les pieds de Lya toucheront-ils l'eau de la piscine ? Arrondir au centième.
Les droites (AB) et (ED) sont toutes deux perpendiculaires à la droite (BC) : elles sont donc parallèles  ; le théorème de Thalès donne en particulier :

$\dfrac{\text{ED}}{\text{AB}} = \dfrac{\text{DC}}{\text{BC}}$ soit $\dfrac{1,5}{5,5} = \dfrac{\text{DC}}{20}$; d'où en multipliant par 20 chaque membre :

$\text{DC} = \dfrac{15}{55} \times 20  = \dfrac{3}{11} \times 20 = \dfrac{60}{11} \approx 5,454$, soit 5,45~(m) au centième près. 
\item  %Calculer le volume de la piscine, en m$^3$ ?

%\emph{Rappel: Le volume d'un parallélépipède rectangle est } \:$V = \text{Longueur} \times \text{largeur} \times  \text{hauteur}$.
$V_{\text{piscine}} = 6 \times 6 \times 1,6 = 36 \times 1,6 = 57,6~\left(\text{m}^3\right)$.
\end{enumerate}

