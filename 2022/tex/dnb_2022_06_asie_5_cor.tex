
\medskip

\begin{enumerate}
\item Miami: 80\degres 0 ; 25\degres N \quad Canberra: 150°E;35°S
\item Avec un rayon de $\np{6371} + 380$ l'orbite a une longueur de :

$p_{\text{orbite ISS}} =  2 \times  (\np{6371} + 380) \times  \pi = 2 \times  \np{6751}\pi = \np{13502}\pi \approx \np{42418}$~(km).

La longueur de l'orbite de l'ISS est environ \np{42400} km arrondie à la centaine près.
\item 
	\begin{enumerate}
		\item On dresse un tableau de proportionnalité :
		
		\begin{center}
		\begin{tabularx}{0.5\linewidth}{|l|*{2}{>{\centering \arraybackslash}X|}}\hline
Distance &\np{27600}~km&\np{42400}~km\\ \hline
Temps &60~min &$x$ \\ \hline
\end{tabularx}
\end{center}

ON a $x \times \np{27600} = 60 \times \np{42400}$, d'où $x = \dfrac{60 \times \np{42400}}{\np{27600}} \approx 92,17$~(min), soit 1~h 32~min et $0,17 \times 60 \approx 10$~(s).

Il faut donc environ 1~h~32~min à l'ISS pour effectuer un tour complet de la Terre.
		\item Durée de sortie de Thomas Pesquet :
		
21 h 45 $-$ 14 h 30 = 7 h 15 min, soit  $7 \times 60 + 15 = 420 + 15 = 435$~(min).

L'ISS met environ 92 minutes pour faire un tour complet de la Terre. 

Or (division euclidienne de 435 par 92)  : \quad $435 = 92 \times 4 + 67$.

Thomas Pesquet a donc fait 4 tours complets de la terre durant sa sortie extravéhiculaire en restant attaché à l'ISS.
	\end{enumerate}
\end{enumerate}	
