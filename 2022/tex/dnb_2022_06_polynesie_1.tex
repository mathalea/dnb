
\medskip

Pour chacune des quatre affirmations suivantes, dire si elle vraie ou fausse en expliquant soigneusement la réponse.

\medskip

\begin{enumerate}
\item Adriana doit effectuer le calcul suivant :

\[- \dfrac75 + \dfrac65 \times \dfrac47\]


\textbf{Affirmation 1 :} Le résultat qu'elle obtient sous forme de fraction irréductible est $- \dfrac{4}{35}$.
\item Sur la figure ci-dessous, qui n'est pas à l'échelle, les points G, A et R sont alignés et les points E, A et M sont alignés.

\begin{center}
\psset{unit=1cm}
\begin{pspicture}(11.5,5)
%\psgrid
\psline(0,5)(5.5,0)%GE
\psline(6,5)(11.5,0)%MR
\psline(0.5,4.58)(10.7,0.7)%GR
\psline(3.5,1.8)(8.58,2.7)%EM
\uput[l](0.4,4.6){G} \uput[u](6.43,2.37){A} \uput[ur](8.58,2.7){M}
\uput[dl](3.5,1.8){E} \uput[ur](10.7,0.7){R}
\uput[u](3.8,3.35){9,8 cm} \uput[u](7.4,2.6){3 cm}
\uput[d](5.2,2.1){4,2 cm}\uput[d](8.5,1.5){7 cm}
\end{pspicture}
\end{center}

\textbf{Affirmation 2 :} Les droites (GE) et (MR) sont parallèles.
\item \textbf{Affirmation 3 :} La décomposition en produit de facteurs premiers de $126$ est $2 \times 7 \times 9$.

\item Dans la recette de sauce de salade de Thomas, les volumes de moutarde, de vinaigre et d'huile sont dans le ratio de 1 : 3 : 7.

\smallskip

\textbf{Affirmation 4 :} Pour obtenir $330$ mL de sauce de salade, il faut utiliser $210$ mL d'huile
\end{enumerate}

\bigskip

