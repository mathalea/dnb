
\medskip

\emph{Dans cet exercice, aucune justification n'est attendue.}

\smallskip

On a créé un jeu de hasard à l'aide d'un logiciel de programmation.

Lorsqu'on appuie sur le drapeau, le lutin dessine trois motifs côte à côte.

Chaque motif est dessiné aléatoirement: soit c'est une croix, soit c'est un rectangle.

Le joueur gagne si l'affichage obtenu comporte trois motifs identiques.

Au lancement du programme, le lutin est orienté horizontalement vers la droite:

\begin{center}
\begin{tabular}{|p{9cm}|p{5cm}|}\hline
\textbf{Programme principal}

\begin{scratch}[num blocks]
\blockinit{Quand \greenflag est cliqué}
\blockpen{effacer tout}
\blockmove{aller à x: \ovalnum{-110} y: \ovalnum0}
\blockrepeat{répéter \ovalnum{3} fois}
{\blockifelse{si \booloperator{\ovaloperator{nombre aléatoire entre \ovalnum{1} et \ovalnum{2}} = \ovalnum{1}} alors}
{\blockmove{croix}} %sinon
{\blockmove{rectangle}}
\blockmove{avancer de \ovalnum{100} pas}
}
\end{scratch}
&\textbf{Bloc \og~rectangle~\fg}

\begin{scratch}
\initmoreblocks{définir \namemoreblocks{rectangle}}
\blockpen{stylo en position d'écriture}
\blockrepeat{répéter \ovalnum{2} fois}
{\blockmove{avancer de \ovalnum{60} pas}
\blockmove{tourner \turnleft{} de \ovalnum{90} degrés}
\blockmove{avancer de \ovalnum{80} pas}
\blockmove{tourner \turnleft{} de \ovalnum{90} degrés}
}
\blockpen{relever le stylo}
\end{scratch}

\textbf{Bloc \og croix \fg}

Le script n'est pas donné.\\ \hline
\multicolumn{2}{|m{14cm}|}{Explication de l'instruction \og  nombre aléatoire entre ... \fg{} sur un exemple:}\\
\multicolumn{2}{|m{14cm}|}{\ovaloperator{nombre aléatoire entre \ovalnum{1} et \ovalnum{4}}~ renvoie un nombre au hasard parmi 1, 2, 3 et 4.}\\ \hline
\end{tabular}
\end{center}

\smallskip

\begin{enumerate}
\item En prenant pour échelle 1 cm pour 20 pas, représenter le motif obtenu par le bloc \og rectangle \fg.
\item ~

\begin{minipage}{0.65\linewidth}
Voici un exemple d'affichage obtenu en exécutant le programme principal :

Quelle est la distance $d$ entre les deux rectangles sur l'affichage, exprimée en pas?
\end{minipage}\hfill
\begin{minipage}{0.34\linewidth}
\psset{unit=1cm}
\begin{pspicture}(5.2,2)
\psline(0,0.5)(1.2,2)\psline(1.2,0.5)(0,2)
\psframe(1.8,0.5)(2.9,2)\psframe(3.6,0.5)(4.7,2)
\psline{<->}(2.9,0.5)(3.6,0.5)\uput[d](3.25,0.5){$d$}
\end{pspicture}
\end{minipage}
\item Quelle est la probabilité que le premier motif dessiné par le lutin soit une croix ?
\item Dessiner à main levée les 8 affichages différents que l'on pourrait obtenir avec le programme principal.
\item On admettra que les 8 affichages ont la même probabilité d'apparaître. Quelle est la probabilité que le joueur gagne ?
\item On souhaite désormais que, pour chaque motif, il y ait deux fois plus de chances d'obtenir un rectangle qu'une croix. Pour cela, il faut modifier l'instruction dans la ligne 5.

\textbf{Sur la copie}, recopier l'instruction suivante en complétant les cases:

\begin{center}\booloperator{\ovaloperator{nombre aléatoire entre \ovalnum{~~} et \ovalnum{~~}} = \ovalnum{~~}}
\end{center}
\end{enumerate}

\bigskip

