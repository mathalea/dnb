\textbf{{\large \textsc{Exercice 1}} \hfill 22 points}

\medskip

%Cet exercice est constitué de six questions indépendantes.
%
%\medskip

\begin{enumerate}
\item %Calculer $\dfrac56 + \dfrac78$ et donner le résultat sous la forme d'une fraction irréductible.

%On détaillera les calculs.
$\dfrac56 + \dfrac78 = \dfrac{5 \times 4}{6 \times 4} + \dfrac{7 \times 3}{8\times 3} = \dfrac{20 + 21}{24} = \dfrac{41}{24}$.
\item 
	\begin{enumerate}
		\item %Donner, sans justifier, la décomposition en facteurs premiers de $198$ et de $84$.
		
$\bullet~~$$198 = 9 \times 22 = 3 \times 3 \times 2 \times 11 = 2 \times 3^2 \times 11$ ;
		
$\bullet~~$$84 = 4 \times 21 = 2 \times 2 \times 3 \times 7 = 2^2 \times 3 \times 7$.
		\item %En déduire la forme irréductible de la fraction $\dfrac{198}{84}$.
$\dfrac{198}{84} = \dfrac{2 \times 3^2 \times 11}{2^2 \times 3 \times 7} = \dfrac{3 \times 11}{2 \times 7} = \dfrac{33}{14}$.
	\end{enumerate}	
\item  %On donne l'expression littérale suivante : $E = 5(3x - 4) - (2x - 7)$.

%Développer et réduire $E$.
$E = 5(3x - 4) - (2x - 7) = 15x - 20 - 2x + 7 = 13x - 13$.
\item  On désigne par $b$ un nombre positif.

\begin{minipage}{0.58\linewidth}
%Déterminer la valeur de $b$ telle que le périmètre du rectangle ci-contre soit égal à 25.
Le rectangle a une largeur de 4,5 et une longueur de 3b + 2,9.

Son périmètre est égal à : $2(4,5 + 3b + 2,9) =$

$ 2(7,4 + 3b) = 14,8 + 6b$.

Il faut que $14,8 + 6b = 25$, soit $6b = 25 - 14,8$ ou 

$6b = 10,2$, soit $b = \dfrac{10,2}{6} = \dfrac{3 \times 3,4}{3 \times 2} = 1,7$.
\end{minipage}\hfill
\begin{minipage}{0.38\linewidth}
\psset{unit=1cm,arrowsize=2pt 3}
\begin{pspicture}(5.8,3.5)
\psframe(1,0)(5.8,2.8)
\psline(1.9,2.7)(1.9,2.9)\psline(2.8,2.7)(2.8,2.9)\psline(3.7,2.7)(3.7,2.9)
\psline(1.4,2.7)(1.5,2.9)\psline(2.3,2.7)(2.4,2.9)\psline(3.2,2.7)(3.3,2.9)
\psline[linewidth=0.6pt]{<->}(1,3)(1.9,3)\uput[u](1.45,3){$b$}
\psline[linewidth=0.6pt]{<->}(3.7,3)(5.8,3)\uput[u](4.75,3){$2,9$}
\psline[linewidth=0.6pt]{<->}(0.8,0)(0.8,2.8)\uput[l](0.8,1.4){4,5}
\end{pspicture}
\end{minipage}

\item ~

\begin{minipage}{0.58\linewidth}
%Calculer le volume de la pyramide à base rectangulaire de hauteur SH $= 6$ ci-contre.
On sait que $V = \dfrac13 \times B \times h$, avec $B = 3 \times 4 = 12$ et $h = 6$, d'où :

$V = \dfrac13 \times 12 \times 6 = 24$.
\end{minipage}\hfill
\begin{minipage}{0.38\linewidth}
\psset{unit=1cm,arrowsize=2pt 3}
\begin{pspicture}(-2.5,-1)(2.6,4.6)
\psline(0,4.2)(-2.5,-0.5)(0.9,-0.5)(0,4.2)(2.5,0.5)(0.9,-0.5)
\pspolygon[linestyle=dashed](-2.5,-0.5)(2.5,0.5)(-0.9,0.5)
\psline[linestyle=dashed](0,0)(0,4.2)(-0.9,0.5)(0.9,-0.5)
\uput[l](0,2.1){6}\uput[d](-0.8,-0.5){4}\uput[dr](1.7,0){3}
\uput[u](0,4.2){S}\uput[d](0,0){H}
\end{pspicture}
\end{minipage}
\item %Le nombre d'habitants d'une ville a augmenté de 12\,\% entre 2019 et 2020.

%Cette ville compte \np{20692}~habitants en 2020.

%Quel était le nombre d'habitants de cette ville en 2019 ?
Augmenter de 12\,\%, c'est multiplier par $1 + \dfrac{12}{100} = 1 + 0,12 = 1,12$.

Si $x$ est le nombre d'habitants en 2019, alors :

$x \times 1,12 = \np{20692}$, d'où en multipliant chaque membre par $\dfrac{1}{1,12}$, \quad $x = \dfrac{\np{20692}}{1,12} = \np{18475}$.

Il y avait en 2019, \np{18475} habitants.
\end{enumerate}

\bigskip

