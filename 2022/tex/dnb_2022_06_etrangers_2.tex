
\medskip

Paris-Nice est une course cycliste qui se déroule chaque année et qui mène les coureurs de la région parisienne à la région niçoise. L'édition 2021 s'est déroulée en 7 étapes décrites ci-dessous:

\begin{center}
\begin{tabularx}{\linewidth}{|c|c|c|>{\small}X|c|}\hline
\textbf{Étape} &\textbf{Date}&\textbf{Profil}&\textbf{Parcours}&\textbf{Distance}
\\ \hline
1&Dimanche 7 mars&Accidenté&\footnotesize Saint-Cyr-l'École$\to$Saint-Cyr-l'École&166 km\\ \hline
2&Lundi 8 mars&Plat&Oinville-sur-Montcient$\to$Amilly&188 km\\ \hline
3&Mercredi 10 mars&Accidenté& Chalon-sur-Saône$\to$Chiroubles&187,5 km\\ \hline
4&Jeudi 11 mars &Plat&Vienne$\to$Bollène&200 km\\ \hline
5&Vendredi 12 mars&Accidenté&Brignoles$\to$Biot&202,5 km\\ \hline
6&Samedi 13 mars&Montagneux&Le Broc$\to$Valdeblore La Colmiane&119,5 km\\ \hline
7&Dimanche 14 mars&Accidenté& Le Plan-du-Var$\to$Levens&93 km\\ \hline
\end{tabularx}
\end{center}

\begin{enumerate}
\item On étudie la série des distances parcourues par étape.
	\begin{enumerate}
		\item Calculer la distance moyenne parcourue par étape, arrondie au dixième de km. 
		\item Calculer la médiane des distances parcourues par étape.
		\item Calculer l'étendue de la série formée par les distances parcourues par étape.
	\end{enumerate}	
\item Un journaliste affirme: \og Environ 57\,\% du nombre total d'étapes de cette édition se sont déroulées sur un parcours accidenté. \fg{} 

A-t-il raison ? Expliquer votre réponse.
\item L'Allemand Maximilian SCHACHMANN a remporté la course en 28~h 50~min.

Le dernier au classement général a effectué l'ensemble du parcours en 30~h 12~min.

Combien de retard le dernier au classement a-t-il accumulé par rapport au vainqueur ?
\item L'Irlandais Sam BENNETI a remporté la première étape en 3~h 51~min. 

Déterminer sa vitesse moyenne en km/h, arrondie à l'unité, lors de cette étape.
\end{enumerate}

\medskip

