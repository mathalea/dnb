\textbf{{\large \textsc{Exercice 4}} \hfill 20 points}

\medskip

\begin{enumerate}
\item Voici un tableau de valeurs d'une fonction $f$ :

\begin{center}
\begin{tabularx}{\linewidth}{|*{8}{>{\centering \arraybackslash}X|}}\hline
$x$		&$-2$	&$-1$	&0	&1		& 3		& 4		& 5\\ \hline
$f(x)$	&5 		&3 		&1 	&$-1$ 	&$-5$	& $-7$	& $-9$\\ \hline
\end{tabularx}
\end{center}

	\begin{enumerate}
		\item Quelle est l'image de $3$ par la fonction $f$ ?
		\item Donner un nombre qui a pour image $5$ par la fonction $f$.
		\item Donner un antécédent de $1$ par la fonction $f$.
	\end{enumerate}
\item On considère le programme de calcul suivant:
\begin{center}
\begin{tabular}{|l|}\hline
Choisir un nombre\\
Ajouter 1\\
Calculer le carré du résultat\\ \hline
\end{tabular}
\end{center}
	\begin{enumerate}
		\item Quel résultat obtient-on en choisissant $1$ comme nombre de départ? Et en choisissant $- 2$ comme nombre de départ?
		\item On note $x$ le nombre choisi au départ et on appelle $g$ la fonction qui à $x$ fait correspondre le résultat obtenu avec le programme de calcul.
		
Exprimer $g(x)$ en fonction de $x$.
	\end{enumerate}
\item La fonction $h$ est définie par $h(x) = 2x^2 - 3$.
	\begin{enumerate}
		\item Quelle est l'image de $3$ par la fonction $h$ ?
		\item Quelle est l'image de $-4$ par la fonction $h$ ?
		\item Donner un antécédent de $5$ par la fonction $h$. En existe-t-il un autre ?
	\end{enumerate}	
\item On donne les trois représentations graphiques suivantes qui correspondent chacune à une des fonctions $f$, $g$ et $h$ citées dans les questions précédentes.

Associer à chaque courbe la fonction qui lui correspond, en expliquant la réponse.

\begin{center}
\begin{tabularx}{\linewidth}{*{3}{>{\centering \arraybackslash}X}}
\psset{unit=4.5mm,arrowsize=2pt 3}
\begin{pspicture*}(-4,-5)(4,6)
\rput(0,5.5){Représentation \no 1}
\psaxes[linewidth=1.25pt,labelFontSize=\scriptstyle]{->}(0,0)(-4,-4.9)(4,5)
\psplot[plotpoints=1000,linewidth=1.25pt]{-2}{5}{1 2 x mul sub}
\end{pspicture*} &
\psset{xunit=4.4mm,yunit=2.2mm,arrowsize=2pt 3}
\begin{pspicture*}(-4,-4.5)(4,18.2)
\rput(0,17){Représentation \no 2}
\psaxes[linewidth=1.25pt,Dy=2,labelFontSize=\scriptstyle]{->}(0,0)(-4,-4)(4,16)
\psplot[plotpoints=1000,linewidth=1.25pt]{-3.1}{3.1}{x dup mul 2 mul 3 sub}
\end{pspicture*}&
\psset{xunit=4.5mm,,yunit=4mm,arrowsize=2pt 3}
\begin{pspicture*}(-5,-1.5)(3,11)
\rput(-1,10.5){Représentation \no 3}
\psaxes[linewidth=1.25pt,labelFontSize=\scriptstyle]{->}(0,0)(-5,-1.5)(3,10)
\psplot[plotpoints=1000,linewidth=1.25pt]{-4.1}{2.1}{1  x add  dup mul}
\end{pspicture*}\\
\end{tabularx}
\end{center}
\end{enumerate}

\bigskip

