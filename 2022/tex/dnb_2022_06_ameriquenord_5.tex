
\medskip

\begin{minipage}{0.48\linewidth}
On considère le programme de calcul suivant, appliqué à des nombres entiers:
\end{minipage}\hfill
\begin{minipage}{0.48\linewidth}
\begin{center}
\psset{unit=1cm}
\begin{pspicture}(-3.7,0)(3.7,-6.5)
%\psgrid
\rput(0,-0.4){Nombre choisi}\rput(0,-0.8){ au départ}
\psframe(-1.4,-0.2)(1.4,-1)
\psline[linewidth=4.5pt]{->}(0,-1)(0,-1.8)
\rput(0,-2.){Programme de calcul}\rput(0,-2.4){$\bullet~~$Calculer le carré du nombre de départ}
\rput(0,-2.8){$\bullet~~$Ajouter le nombre de départ}
\psline[linewidth=4.5pt]{->}(0,-3)(0,-3.9)
\psframe(-3.4,-1.8)(3.4,-3)\rput(0,-4.1){Nombre obtenu à}
\rput(0,-4.5){l'arrivée}
\psframe(-1.6,-3.9)(1.6,-4.8)
\end{pspicture}
\end{center}
\end{minipage}

\medskip

\begin{center}
\textbf{PARTIE A}
\end{center}

\begin{enumerate}
\item Vérifier que si le nombre de départ est 15, alors le nombre obtenu à l'arrivée est 240.
\end{enumerate}

\begin{minipage}{0.48\linewidth}
\begin{enumerate}[resume]
\item Voici un tableau de valeurs réalisé à l'aide d'un tableur:

Il donne les résultats obtenus par le programme de calcul en fonction de quelques valeurs du nombre choisi au départ.

Quelle formule a pu être saisie dans la cellule B2 avant d'être étirée vers le bas ?

\emph{Aucune justification n'est attendue}.
\item On note $x$ le nombre de départ.

Écrire, en fonction de $x$, une expression du résultat obtenu avec ce programme de calcul.
\end{enumerate}
\end{minipage}\hfill
\begin{minipage}{0.48\linewidth}
\begin{tabularx}{\linewidth}{|c|*{2}{>{\centering \arraybackslash}X|}}\hline
&A&B\\ \hline
1&Nombre choisi au départ&Nombre obtenu à l'arrivée\\ \hline
2&0&0 \\ \hline
3&1&2\\ \hline
4&2&6\\ \hline
5& 3& 12\\ \hline
6& 4& 20\\ \hline
7& 5& 30\\ \hline
8& 6& 42\\ \hline
9& 7& 56\\ \hline
10& 8& 72\\ \hline
11&9& 90\\ \hline
12&10& 110\\ \hline
\end{tabularx}
\end{minipage}

\begin{center}
\textbf{PARTIE B}
\end{center}

On considère l'affirmation suivante:

\og Pour obtenir le résultat du programme de calcul, il suffit de multiplier le nombre de départ par le nombre entier qui suit. \fg

\begin{enumerate}[resume]
\item Vérifier que cette affirmation est vraie lorsque le nombre entier choisi au départ est 9.
\item Démontrer que cette affirmation est vraie quel que soit le nombre entier choisi au départ.
\item Démontrer que le nombre obtenu à l'arrivée par le programme de calcul est un nombre pair quel que soit le nombre entier choisi au départ.
\end{enumerate}
