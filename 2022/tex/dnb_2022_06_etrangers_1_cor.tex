\textbf{\large Exercice 1 \hfill 19 points}

\medskip

%Les deux parties de cet exercice sont indépendantes. 
%
%\medskip

\textbf{Partie A}

\medskip

%Cette partie est un questionnaire à choix multiples (QCM). 
%
%Pour chaque question, trois réponses sont proposées, une seule est exacte. Recopier le numéro de la question et indiquer, sans justifier dans cette partie seulement, la réponse choisie.

Dans toute cette partie, on considère la fonction définie par: 
$f(x) = 2x + 3.$

\begin{center}
\begin{tabularx}{\linewidth}{|m{5cm}|*{3}{>{\centering \arraybackslash}X|}}\hline
&Réponse A&Réponse B&Réponse C\\ 
\hline
\vspace*{-3cm}\textbf{1.~}La représentation graphique de cette fonction est:\newline\newline \newline 
&
\psset{unit=0.5cm}
\begin{pspicture*}(-2,-2)(3,4.5)
\psaxes[linewidth=1.25pt,labelFontSize=\scriptstyle]{->}(0,0)(-2,-1.95)(3,4.5)
\psplot[plotpoints=500,linewidth=1.2pt,linecolor=blue]{-2}{3}{2 x mul 3 add}
\end{pspicture*}
&
\psset{unit=0.5cm}
\begin{pspicture*}(-2,-2)(3,4.5)
\psaxes[linewidth=1.25pt,labelFontSize=\scriptstyle]{->}(0,0)(-2,-1.95)(3,4.5)
\psplot[plotpoints=500,linewidth=1.2pt,linecolor=blue]{-2}{3}{3 }
\end{pspicture*}
&
\psset{unit=0.5cm}
\begin{pspicture*}(-2,-2)(3,4.5)
\psaxes[linewidth=1.25pt,labelFontSize=\scriptstyle]{->}(0,0)(-2,-1.95)(3,4.5)
\psplot[plotpoints=500,linewidth=1.2pt,linecolor=blue]{-2}{3}{2 x mul}
\end{pspicture*}\\ 
\hline
\textbf{2.~}L'image de $- 2$ par la fonction $f$ est \ldots&$-7$&$- 1$&3\\ 
\hline
~\newline
\hspace*{1cm}\begin{tabular}{|*{4}{c|}}
\hline
	&A		&B		&C\\ 
\hline
1	&$x$	&$-2$	&$-1$\\ 
\hline
2	&$f(x)$	&		&\\ 
\hline
\end{tabular}
\newline\newline
\textbf{3.~}Dans cette feuille de calcul extraite d'un tableur, la formule à saisir dans la cellule B2 avant de l'étirer vers la droite est :&=2*A1 +3&=2*B1 +3&=2*$(-2)$ +3\\ 
\hline
\end{tabularx}
\end{center}

\begin{enumerate}
\item $f(0)=3$ et la droite représentant $f$ ne peut pas être horizontale.

\hfill\textbf{Réponse A}
\item $f(-2)=2\times(-2)+3=-4+3=-1$

\hfill\textbf{Réponse B}
\item L'image de A1 se trouve en A2, et l'image de B1 se trouve en B2; dans la cellule B2 on cherche donc l'image de ce qui se trouve en B1 en entrant \fbox{= 2*B1 + 3}

\hfill\textbf{Réponse B}
\end{enumerate}
\medskip

\textbf{Partie B}

\medskip

\begin{enumerate}
\item $(2x - 1)(3x + 4) - 2x 
=(2x)\times(3x)+(2x)\times 4 - 1\times (3x) -1 \times 4 - 2x
= 6x^2 +8x -3x -4 - 2x\\
\phantom{(2x - 1)(3x + 4) - 2x }
= 6x^2 + 3x - 4$.

\item On considère le triangle CDE tel que : CD $= 3,6$~cm ; CE $= 4,2$~cm et DE $= 5,5$~cm.

Seul [DE] le côté le plus long pourrait être l'hypoténuse ; on vérifie si oui ou non :

$\text{DE}^2 = \text{CD}^2 + \text{DE}^2$. Or 

$\text{CD}^2=3,6^2=12,96$; $\text{CE}^2=4,2^2=17,64$ et $\text{DE}^2=5,5^2=30,25$

$12,96 + 17,64 = 30,6 \neq 30,25$ donc, le théorème de Pythagore ne s'applique pas  à ce triangle, on peut donc dire que le triangle CDE n'est pas rectangle.
%Le triangle CDE est-il rectangle ?
\end{enumerate}

\bigskip

