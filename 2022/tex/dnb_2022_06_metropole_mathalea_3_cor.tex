
\medskip
\begin{enumerate}[itemsep=1em]
\item \begin{enumerate}[itemsep=1em]
  \item Voici la décomposition en produit de facteurs premiers de $252$~:
    \begin{align*}
      252 &= 2\times 126\\
      252 &= 2\times 2\times 63\\
      252 &= 2\times 2\times 3 \times 21\\
      252 &= 2\times 2\times 3 \times 3\times 7\\
      252 &= 2^2\times3^2\times 7
    \end{align*}
    La proposition correcte est donc la \no{3}.
    
    \medskip
    \textit{Remarque : dans la proposition 1, 9 n'est pas un nombre premier ; dans la proposition 2, 21 n'est pas un nombre premier et dans la proposition 3, les 3 facteurs sont premiers et le produit est bien égal à 252. }
  \item Voici la décomposition en produit de facteurs premiers de $156$~:
    \begin{align*}
      156 &= 2 \times 78\\
      156 &= 2 \times 2 \times 39\\
      156 &= 2 \times 2 \times  \times 13\\
      156 &= 2^2 \times 3 \times 13
    \end{align*}
  \end{enumerate}
\item \begin{enumerate}[itemsep=1em]
  \item On a $252=36 \times 7$ et $156=36\times4+12$. Donc $36$ n'est pas un diviseur commun à $252$ et $156$.

    En conséquence, elle ne pourra pas faire $36$ paquets.
  \item Cherchons $N$ le plus grand commun diviseur de $252$ et $156$.

    Dans les deux décompositions en produit de facteurs premiers de ces deux nombres, on choisit les facteurs qui sont communs aux deux produits. Il vient $N=2^2\times3$ soit $N=12$.

    La collectionneuse pourra faire au maximum $12$ paquets.
  \item Elle fait $12$ paquets. On a~:
    \begin{align*}
      252 &= 12\times 21 & 156 &= 12\times13
    \end{align*}
    Il y aura donc $21$ cartes \textit{feu} et $13$ cartes \textit{terre}.
  \end{enumerate}
\item Soit $E$ l'événement \og \textit{La carte tirée est de type Terre} \fg.

  Il y a équiprobabilité donc la probabilité $p(E)$ de l'événement $E$ correspond à la proportion de cartes \textit{feu} parmi toutes les cartes. Donc~:
  \begin{align*}
    p(E) &= \frac{156}{252+156}\\[1em]
    p(E) &= \frac{156}{408}\\[1em]
    p(E) &= \frac{13}{34}\\[1em]
    p(E) &\approx 0,4\\
  \end{align*}
\end{enumerate}
\medskip

