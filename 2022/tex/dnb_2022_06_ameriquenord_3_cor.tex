
\medskip

\begin{enumerate}
\item D'après le communiqué de presse, 81\,\% des $1,6$ million d'adolescents de 11 à 17 ans interrogés ne respectent pas cette recommandation.

Cela représente : $0,81\times 1,6 \times 10^{6} = \np{1296000}$ personnes, soit 1,296 million d'adolescents.

\item
	\begin{enumerate}
		\item La valeur maximale de la série est celle du jour 4, pour 1 h 40 min, et la valeur minimale est celle du jour 14 pour 0 min.

		L'étendue des 14 durées quotidiennes notées dans le calendrier est donc la différence entre les deux, soit 1 h 40 min.

		\item Pour donner une médiane de ces 14 durées quotidiennes, il nous faut commencer par ranger les valeurs dans l'ordre croissant :

		0 min; 15 min; 15 min; 30 min; 30 min; 40 min; \textbf{50 min}; \textbf{1 h}; 1 h; 1 h; 1 h; \linebreak 1~h~30~min; 1 h 30 min ; 1 h 40 min.

		Il y a 14 valeurs en tout, donc la médiane est la moyenne des deux valeurs centrales, (écrites en gras, ci- dessus). La médiane est donc de 55 min.
	\end{enumerate}
\item
	\begin{enumerate}
		\item Calculons la durée moyenne de pratique physique pour cet  adolescent. Pour simplifier les calculs, convertissons toutes les durées en minutes, et établissons un tableau d'effectif :

	\begin{center}
		\begin{tabularx}{\linewidth}{|l|*{8}{>{\centering \arraybackslash}X|}} \hline
			Durée (min) &0&15&30&40&50&60&90&100\\ \hline
			effectif&1&2&2&1&1&4&2&1\\ \hline
		\end{tabularx}
	\end{center}

		La durée moyenne est donc de :

		$\dfrac{0\times 1 + 15 \times 2 + 30 \times 2 + 40 \times 1 + 50 \times 1 + 60 \times 4 + 90\times 2+100\times1}{14} = \dfrac{700}{14} = 50$.

		En moyenne, l'adolescent a eu une pratique physique de 50 minutes par jour, donc l'objectif n'est pas atteint.


		\item Pour que la moyenne soit exactement d'une heure sur les 21 jours, il faut que pendant ces 21 jours, il ait eu $21 \times 60 = \np[min]{1260}$ de pratique physique.

		Comme il en a déjà effectué 700 pendant les 14 premiers jours, cela lui laisse 560 minutes à effectuer pendant les 7 jours suivants (donc $560\div 7 = 80$~min par jour, en moyenne.)
	\end{enumerate}
\end{enumerate}

\bigskip

