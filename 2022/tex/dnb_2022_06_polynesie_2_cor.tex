\textbf{\large Exercice 2 \hfill 16 points}

\medskip

\begin{enumerate}
\item Le prix payé avec le tarif \og liberté \fg{} est représenté par la droite 
$\left(d_1\right)$ qui passe par l'origine donc ce prix est bien proportionnel au nombre d'heures effectuées dans la salle de sport. (la fonction associée, voir plus bas est une fonction linéaire)
\item 
	\begin{enumerate}
		\item $f(5) = 25$. L'image de 5 par $f$ est 25.
		\item L'antécédent de 10 par la fonction g est 1
	\end{enumerate}
Voir les lectures sur la figure ci-dessous.
	\begin{center}

\psset{xunit=1cm,yunit=0.2cm,arrowsize=2pt 4}
\begin{pspicture}(-1,-4)(11,35)
\multido{\n=0+1}{12}{\psline[linewidth=0.4pt](\n,0)(\n,35)}
\multido{\n=0+5}{8}{\psline[linewidth=0.4pt](0,\n)(11,\n)}
\psaxes[linewidth=1.25pt,Dy=5,labelFontSize=\scriptstyle]{->}(0,0)(0,0)(11,35)
\psplot[plotpoints=400,linewidth=1.25pt,linecolor=red]{0}{11}{2.5 x mul 7.5 add}
\psplot[plotpoints=400,linewidth=1.25pt,linecolor=blue]{0}{7}{5 x mul}
\uput[u](6.5,33){\blue $\left(d_1\right)$}
\uput[u](10.3,33){\red $\left(d_2\right)$}
\rput{90}(-1,26){Prix payé en euro}
\uput[d](9,-2){Nombres d'heures effectuées}
\psline[linewidth=1.25pt,linestyle=dashed,ArrowInside=->](5,0)(5,25)(0,25)
\psline[linewidth=1.25pt,linestyle=dashed,ArrowInside=->](0,10)(1,10)(1,0)
\end{pspicture}
\end{center}

\item ~
\begin{itemize}
\item[$\bullet~~$] Si la personne effectue moins de 3h dans la salle de sport il est plus avantageux qu'elle choisisse le tarif \og liberté\fg (car la droite 
$\left(d_1\right)$ est en dessous de la droite $\left(d_2\right)$)
\item[$\bullet~~$] Si la personne effectue 3~h il est équivalent qu'elle choisisse l'un ou l'autre des deux tarifs.
\item[$\bullet~~$] Si la personne effectue plus de 3~h il est plus avantageux qu'elle choisisse le tarif \og abonné \fg (car la droite $\left(d_2\right)$ est en dessous de la droite $\left(d_1\right)$).
\end{itemize}

\item  Comme la droite $\left(d_1\right)$ passe par l'origine elle représente une fonction linéaire, donc $f(x) = ax$ avec $a$ un réel.

$\left(d_1\right)$ passe par les points de coordonnées (0~;~0) et (3~;~15) donc
$a = \dfrac{15-0}{3 - 0} =  \dfrac{15}{3} = 5$.

Par conséquent $f(x) = 5x$. Et $f(15) = 5 \times 15 = 75$.

Le prix payé avec le tarif \og liberté\fg{} pour 15 heures effectuées est de $75$~\euro.
\end{enumerate}

\bigskip

