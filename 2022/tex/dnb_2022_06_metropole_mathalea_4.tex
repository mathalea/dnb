
\medskip

Dans cet exercice, $x$ est un nombre strictement supérieur à 3.

On s'intéresse aux deux figures géométriques dessinées ci-dessous :

\begin{itemize}[label=$\bullet$]
\item un rectangle dont les côtés ont pour longueurs $x-3$ et $x+7$ ;
\item un carré de côté $x$.
\end{itemize}

\par\vspace{1cm}
\begin{tikzpicture}[x=1cm, y=1cm]
\draw[thick] (-5,-1)-- (-2,-1);
\draw[thick] (-2,-1)-- (-2,-3);
\draw[thick] (-2,-3)-- (-5,-3);
\draw[thick] (-5,-3)-- (-5,-1);
\draw (-6.3,-1.8) node[anchor=north west] {$x-3$};
\draw (-4,-3) node[anchor=north west] {$x+7$};
\draw[thick] (1,-1)-- (3,-1);
\draw[thick] (3,-1)-- (3,-3);
\draw[thick] (3,-3)-- (1,-3);
\draw[thick] (1,-3)-- (1,-1);
\draw (0.5,-1.8) node[anchor=north west] {$x$};
\draw (1.8,-3) node[anchor=north west] {$x$};
\end{tikzpicture}

\begin{enumerate}[itemsep=1em]
\item Quatre propositions sont écrites ci-dessous.\\
Recopier sur la copie celle qui correspond à l'aire du carré. On ne demande pas de justifier.\\
\renewcommand{\arraystretch}{1.5}
\begin{tabular}{|>{\centering\arraybackslash}p{3cm}|>{\centering\arraybackslash}p{3cm}|>{\centering\arraybackslash}p{3cm}|>{\centering\arraybackslash}p{3cm}|}
\hline
    $4x$ &  $4+x$ & $x^2$ & $2x$ \\
 \hline
\end{tabular}
\renewcommand{\arraystretch}{1}

\item Montrer que l'aire du rectangle est égale à : $x^2+4x-21$
\item On a écrit le script ci-dessous dans Scratch.\\
On veut que ce programme renvoie l'aire du rectangle lorsque l'utilisateur a rentré une valeur de $x$ (strictement supérieure à 3).\\
Écrire sur la copie les contenus des trois cases vides des lignes 5, 6 et 7, en précisant les numéros de lignes qui correspondent à vos réponses.\\
     \begin{scratch}[scale=0.75,print]
		\blockinit{quand la touche  \selectmenu{espace} est pressée}
		\blocksensing{demander \ovalnum{Combien vaut x ?} et attendre}
		\blockvariable{mettre \selectmenu{x} à \ovalsensing{réponse}}
		\blockvariable{mettre \selectmenu{R} à \ovaloperator{\ovalvariable{x} * \ovalvariable{x}}}
		\blockvariable{ajouter \ovaloperator{\ovalvariable{ } * \ovalvariable{x}} à \ovalvariable{R}}
        \blockvariable{ajouter \ovalvariable{ } à \ovalvariable{R}}
        \blocklook{dire \ovaloperator{regrouper \ovalnum{L'aire du rectangle est } et \ovalvariable{ }} pendant \ovalnum{2} secondes}
	\end{scratch}

\item On a pressé la touche espace puis saisi le nombre 8. Que renvoie le programme ?

\item Quel nombre $x$ doit-on choisir pour que l'aire du rectangle soit égale à l'aire du carré ?\\
\textit{Toute trace de recherche, même non aboutie, sera prise en compte.}


\end{enumerate}
\medskip

