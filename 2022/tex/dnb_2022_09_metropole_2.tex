\textbf{{\large \textsc{Exercice 2}} \hfill 20 points}

\bigskip

Yanis vit en France métropolitaine. Il part cet été en Guadeloupe en vacances.

Il se renseigne quant aux locations de véhicules.

Une société de location de voitures à Pointe-à-Pitre propose les tarifs suivants pour unvéhicule  5 places de taille moyenne,  assurances non comprises :

\begin{itemize}
\item[$\bullet~~$]Tarif \og Affaire \fg{} : 0,50~\euro{} par kilomètre parcouru.
\item[$\bullet~~$]Tarif \og Voyage court \fg{} : un forfait de 120~\euro{} puis 20 centimes par 
kilomètre parcouru
\item[$\bullet~~$]Tarif \og Voyage long \fg : un forfait de 230~\euro, quel que soit le nombre de kilomètres effectués.
\end{itemize}

\medskip

\begin{enumerate}
\item Yanis a préparé son plan de route et il fera $280$~km. Il choisit le tarif
\og Affaire \fg. 

Combien va-t-il payer ?
\item S'il parcourt $450$~km, quelle offre est la plus avantageuse financièrement ?
\item Dans la suite, $x$ désigne le nombre de kilomètres parcourus en voiture.

On considère les trois fonctions  $l,\:m,\:n$ suivantes :

\[l(x) = 230 \qquad m(x) = 0,5x \qquad n(x) = 0,2x + 120\]

	\begin{enumerate}
		\item Associer, sans justifier, chacune de ces fonctions au tarif correspondant.
		\item Déterminer le nombre de kilomètres à parcourir pour que le tarif \og Voyage court \fg{} soit égal au tarif \og Affaire \fg.
	\end{enumerate}
\item 
	\begin{enumerate}
		\item Sur le graphique joint, tracer les courbes représentatives des fonctions $l,\:m$ et $n$.
		\item Déterminez graphiquement le nombre de kilomètres que devra atteindre Yanis  pour que le tarif \og Voyage long \fg soit le plus avantageux.
		
\emph{On laissera les traits de constructions apparents sur le graphique.}
	\end{enumerate}

		% \hspace*{-25mm}
		\psset{xunit=0.0325cm,yunit=0.05cm}
		\begin{pspicture}(600,260)
		\multido{\n=0+20}{31}{\psline[linewidth=0.15pt](\n,0)(\n,260)}
		\multido{\n=0+20}{14}{\psline[linewidth=0.15pt](0,\n)(600,\n)}
		\psaxes[linewidth=1.25pt,Dx=40,Dy=40]{->}(0,0)(0,0)(600,260)
		\uput[u](528,0){distance parcourue en km}
		\uput[r](0,255){Coût en euro}
		\end{pspicture}
\end{enumerate}

\bigskip

