
\medskip

\begin{enumerate}[itemsep=1em]
\item
  Fuite: 1 goutte /s\\
  En une journée: $1\times 60\times 60\times{24 = 86~400}$ gouttes.
\item
  En une semaine: $86400\times{7 = 604~800}$ gouttes\\
  $604~800\div 20 = 30~240~\text{mL} = 30,24~\text{L}$.
\item
  $V_{\text{vasque}} = \pi\times 20^{2}\times 15\approx 18~850 \text{cm}^{3}$ soit $ 18,85 \text{dm}^{3}$ ou $18,85~\text{L}$.
\item
  Le volume d'eau qui s'écoule est de 30,24~L, ce qui est supérieur au
  volume de la vasque (18,85~L).\\
  L'eau va donc déborder.
\item
  En 2004: 165~L par jour et par habitant.\\
  En 2018: 148~L par jour et par habitant.\\
  Pourcentage de diminution:
  \[\frac{165-148}{165} \approx 0,10  \text{ soit } 10~\text{\%}\]
\end{enumerate}





