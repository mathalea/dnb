
\medskip

La figure ci-dessous n'est pas à l'échelle.

\setlength\parindent{10mm}
\begin{itemize}
\item[$\bullet~~$] les points M, A et S sont alignés
\item[$\bullet~~$] les points M, T et H sont alignés
\item[$\bullet~~$] MH = 5 cm
\item[$\bullet~~$] MS = 13 cm
\item[$\bullet~~$] MT = 7 cm
\end{itemize}
\setlength\parindent{0mm}

\begin{center}
\psset{unit=1cm}
\begin{pspicture}(12.5,5.8)
%\psgrid
\pspolygon(0.2,0.2)(7,0.2)(7,5.2)(11.8,5.2)%ATHS
\psframe(7,0.2)(6.75,0.45)\psframe(7,5.2)(7.25,4.95)
\uput[dl](0.2,0.2){A} \uput[dr](7,0.2){T} \uput[ul](7,5.2){H} \uput[r](11.8,5.2){S}
\uput[ul](7,3.16){M}
\end{pspicture}
\end{center}

\smallskip

\begin{enumerate}
\item Démontrer que la longueur HS est égale à $12$~cm.
\item Calculer la longueur AT.
\item Calculer la mesure de l'angle $\widehat{\text{HMS}}$. On arrondira le résultat au degré près.
\item Parmi les transformations suivantes quelle est celle qui permet d'obtenir le triangle MAT à partir du triangle MHS ?

\emph{Dans cette question, aucune justification n'est attendue.}

Recopier la réponse sur la copie.


\begin{minipage}{2.35cm}
\begin{tabular}{|m{2.35cm}|}\hline
Une symétrie centrale\\ \hline
\end{tabular}
\end{minipage}\quad
\begin{minipage}{2.35cm}
\begin{tabular}{|m{2.35cm}|}\hline
Une symétrie axiale\\ \hline
\end{tabular}
\end{minipage}\quad
\begin{minipage}{2.15cm}
\begin{tabular}{|m{2.15cm}|}\hline
Une rotation~~~~~~~\\ \hline
\end{tabular}
\end{minipage}\quad
\begin{minipage}{2.35cm}
\begin{tabular}{|m{2.35cm}|}\hline
Une translation\\ \hline
\end{tabular}
\end{minipage}\quad
\begin{minipage}{2.35cm}
\begin{tabular}{|m{2.35cm}|}\hline
Une homothétie\\ \hline
\end{tabular}
\end{minipage}
%Une symétrie  Une }Une Une
%\end{tabularx}

\item Sachant que la longueur MT est 1,4 fois plus grande que la longueur HM, un élève affirme: \og L'aire du triangle MAT est $1,4$ fois plus grande que l'aire du triangle MHS. \fg

Cette affirmation est-elle vraie ? On rappelle que la réponse doit être justifiée.
\end{enumerate}

\bigskip

