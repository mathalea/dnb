\textbf{\large Exercice 5 \hfill 23 points}

\medskip

\begin{enumerate}
\item 
	\begin{enumerate}
		\item $135 = 6 \times 12,5 + 5p$, soit  $135 = 75 + 5p$, d'où en ajoutant $- 75$ à chaque membre :
		
	
$135 - 75 = 5p$ ou $60 = 5p$, c'est-à-dire $5 \times 12 = 5 \times p$, d'où $p = 12$~m.
		\item On a  5 hauteurs de $h$ pour un total de 32~m, soit : $5 \times h = 32$, d'où $h = \dfrac{32}{5} = \dfrac{64}{10} = 6,4$~(m).
		
La hauteur de chaque escalator est de 6,4~m.
	\end{enumerate}
\item 
	\begin{enumerate}
		\item Dans le triangle RST, rectangle en R on utilise le théorème de Pythagore, soit :
		
$\text{ST}^2= \text{SR}^2 + \text{RT}^2 = 12^2 +  6,4^2+  = 144 + 40,96  
=184,96$.

 D'où ST $=\sqrt{184,96} = 13,6$~m.
		\item Dans le triangle RST rectangle en R on a d'après la
trigonométrie :

$\cos \widehat{\text{RST}} = \dfrac{\text{SR}}{\text{ST}} = \dfrac{12}{13,6} \approx 0,882$.

La calculatrice donne $\widehat{\text{RST}} \approx 28,07$ soit 28\degres{} au degré près.
	\end{enumerate}
\item Script complété :

\begin{center}
\begin{scratch}[scale=1.75, num blocks]
\blockinit{quand \greenflag est cliqué}
\blockpen{effacer tout}
\blockmove{s'orienter à \ovalnum{90}}
\blockmove{aller à x: \ovalnum{-120} y: \ovalnum{-60}}
\blockpen{stylo en position d'écriture}
\blockrepeat{répéter \ovalnum{5} fois}
{
\blockmove{avancer de \ovalnum{12,5}}
\blockmove{tourner \turnleft{} de \ovalnum{28} degrés}
\blockmove{avancer de \ovalnum{13,6}}
\blockmove{s'orienter à \ovalnum{90}}
}
\blockmove{avancer de \ovalnum{12,5}}
\blockpen{relever le stylo}
\end{scratch}
\end{center}

\end{enumerate}


